%EVERY VARIABLE HAS IT'S OWN PAGE

    \setcounter{footnote}{0}

    %omit vertical space
    \vspace*{-1.8cm}
	\section{adem171 (Vater: höchster beruflicher Abschluss)}
	\label{section:adem171}



	% TABLE FOR VARIABLE DETAILS
  % '#' has to be escaped
    \vspace*{0.5cm}
    \noindent\textbf{Eigenschaften\footnote{Detailliertere Informationen zur Variable finden sich unter
		\url{https://metadata.fdz.dzhw.eu/\#!/de/variables/var-gra2009-ds1-adem171$}}}\\
	\begin{tabularx}{\hsize}{@{}lX}
	Datentyp: & numerisch \\
	Skalenniveau: & nominal \\
	Zugangswege: &
	  download-cuf, 
	  download-suf, 
	  remote-desktop-suf, 
	  onsite-suf
 \\
    \end{tabularx}



    %TABLE FOR QUESTION DETAILS
    %This has to be tested and has to be improved
    %rausfinden, ob einer Variable mehrere Fragen zugeordnet werden
    %dann evtl. nur die erste verwenden oder etwas anderes tun (Hinweis mehrere Fragen, auflisten mit Link)
				%TABLE FOR QUESTION DETAILS
				\vspace*{0.5cm}
                \noindent\textbf{Frage\footnote{Detailliertere Informationen zur Frage finden sich unter
		              \url{https://metadata.fdz.dzhw.eu/\#!/de/questions/que-gra2009-ins1-6.19$}}}\\
				\begin{tabularx}{\hsize}{@{}lX}
					Fragenummer: &
					  Fragebogen des DZHW-Absolventenpanels 2009 - erste Welle:
					  6.19
 \\
					%--
					Fragetext: & Welchen höchsten beruflichen Abschluss haben Ihre Eltern?\par  Vater\par  Promotion\par  Abschluss an einer Universität (einschl. Lehrerausbildung)\par  Abschluss an einer Fachhoch-/ Ingenieurschule, Handelsakademie\par  Abschluss an einer Fachschule (nur DDR) Abschluss an einer Meister-/ Technikerschule, Berufs- oder Fachakademie\par  Beruflich-betrieblicher Ausbildungsabschluss (z. B. Lehre, Facharbeiter/innen/ausbildung)\par  Beruflich-schulischer Ausbildungsabschluss (Berufsfach-, Handelsschule)\par  Keinen beruflichen Abschluss\par  Beruflicher Abschluss unbekannt \\
				\end{tabularx}





				%TABLE FOR THE NOMINAL / ORDINAL VALUES
        		\vspace*{0.5cm}
                \noindent\textbf{Häufigkeiten}

                \vspace*{-\baselineskip}
					%NUMERIC ELEMENTS NEED A HUGH SECOND COLOUMN AND A SMALL FIRST ONE
					\begin{filecontents}{\jobname-adem171}
					\begin{longtable}{lXrrr}
					\toprule
					\textbf{Wert} & \textbf{Label} & \textbf{Häufigkeit} & \textbf{Prozent(gültig)} & \textbf{Prozent} \\
					\endhead
					\midrule
					\multicolumn{5}{l}{\textbf{Gültige Werte}}\\
						%DIFFERENT OBSERVATIONS <=20

					1 &
				% TODO try size/length gt 0; take over for other passages
					\multicolumn{1}{X}{ Promotion   } &


					%721 &
					  \num{721} &
					%--
					  \num[round-mode=places,round-precision=2]{7.1} &
					    \num[round-mode=places,round-precision=2]{6.87} \\
							%????

					2 &
				% TODO try size/length gt 0; take over for other passages
					\multicolumn{1}{X}{ Universität   } &


					%2218 &
					  \num{2218} &
					%--
					  \num[round-mode=places,round-precision=2]{21.83} &
					    \num[round-mode=places,round-precision=2]{21.14} \\
							%????

					3 &
				% TODO try size/length gt 0; take over for other passages
					\multicolumn{1}{X}{ Fachhoch-/ Ingenieurschule, Handelsakademie   } &


					%1535 &
					  \num{1535} &
					%--
					  \num[round-mode=places,round-precision=2]{15.11} &
					    \num[round-mode=places,round-precision=2]{14.63} \\
							%????

					4 &
				% TODO try size/length gt 0; take over for other passages
					\multicolumn{1}{X}{ Fachschule (DDR)   } &


					%237 &
					  \num{237} &
					%--
					  \num[round-mode=places,round-precision=2]{2.33} &
					    \num[round-mode=places,round-precision=2]{2.26} \\
							%????

					5 &
				% TODO try size/length gt 0; take over for other passages
					\multicolumn{1}{X}{ Meister-/Technikerschule, Berufs-/Fachakademie   } &


					%1608 &
					  \num{1608} &
					%--
					  \num[round-mode=places,round-precision=2]{15.83} &
					    \num[round-mode=places,round-precision=2]{15.32} \\
							%????

					6 &
				% TODO try size/length gt 0; take over for other passages
					\multicolumn{1}{X}{ berufl.-betriebl. Ausbildung   } &


					%2874 &
					  \num{2874} &
					%--
					  \num[round-mode=places,round-precision=2]{28.29} &
					    \num[round-mode=places,round-precision=2]{27.39} \\
							%????

					7 &
				% TODO try size/length gt 0; take over for other passages
					\multicolumn{1}{X}{ berufl.-schul. Ausbildung   } &


					%444 &
					  \num{444} &
					%--
					  \num[round-mode=places,round-precision=2]{4.37} &
					    \num[round-mode=places,round-precision=2]{4.23} \\
							%????

					8 &
				% TODO try size/length gt 0; take over for other passages
					\multicolumn{1}{X}{ kein beruflicher Abschluss   } &


					%233 &
					  \num{233} &
					%--
					  \num[round-mode=places,round-precision=2]{2.29} &
					    \num[round-mode=places,round-precision=2]{2.22} \\
							%????

					9 &
				% TODO try size/length gt 0; take over for other passages
					\multicolumn{1}{X}{ beruflicher Abschluss unbekannt   } &


					%289 &
					  \num{289} &
					%--
					  \num[round-mode=places,round-precision=2]{2.84} &
					    \num[round-mode=places,round-precision=2]{2.75} \\
							%????
						%DIFFERENT OBSERVATIONS >20
					\midrule
					\multicolumn{2}{l}{Summe (gültig)} &
					  \textbf{\num{10159}} &
					\textbf{\num{100}} &
					  \textbf{\num[round-mode=places,round-precision=2]{96.81}} \\
					%--
					\multicolumn{5}{l}{\textbf{Fehlende Werte}}\\
							-998 &
							keine Angabe &
							  \num{335} &
							 - &
							  \num[round-mode=places,round-precision=2]{3.19} \\
					\midrule
					\multicolumn{2}{l}{\textbf{Summe (gesamt)}} &
				      \textbf{\num{10494}} &
				    \textbf{-} &
				    \textbf{\num{100}} \\
					\bottomrule
					\end{longtable}
					\end{filecontents}
					\LTXtable{\textwidth}{\jobname-adem171}
				\label{tableValues:adem171}
				\vspace*{-\baselineskip}
                    \begin{noten}
                	    \note{} Deskriptive Maßzahlen:
                	    Anzahl unterschiedlicher Beobachtungen: 9%
                	    ; 
                	      Modus ($h$): 6
                     \end{noten}

