%EVERY VARIABLE HAS IT'S OWN PAGE

    \setcounter{footnote}{0}

    %omit vertical space
    \vspace*{-1.8cm}
	\section{aocc21\_g2d (Beruf: KldB 2010 (3-stellig))}
	\label{section:aocc21_g2d}



	% TABLE FOR VARIABLE DETAILS
  % '#' has to be escaped
    \vspace*{0.5cm}
    \noindent\textbf{Eigenschaften\footnote{Detailliertere Informationen zur Variable finden sich unter
		\url{https://metadata.fdz.dzhw.eu/\#!/de/variables/var-gra2009-ds1-aocc21_g2d$}}}\\
	\begin{tabularx}{\hsize}{@{}lX}
	Datentyp: & numerisch \\
	Skalenniveau: & nominal \\
	Zugangswege: &
	  download-suf, 
	  remote-desktop-suf, 
	  onsite-suf
 \\
    \end{tabularx}



    %TABLE FOR QUESTION DETAILS
    %This has to be tested and has to be improved
    %rausfinden, ob einer Variable mehrere Fragen zugeordnet werden
    %dann evtl. nur die erste verwenden oder etwas anderes tun (Hinweis mehrere Fragen, auflisten mit Link)
				%TABLE FOR QUESTION DETAILS
				\vspace*{0.5cm}
                \noindent\textbf{Frage\footnote{Detailliertere Informationen zur Frage finden sich unter
		              \url{https://metadata.fdz.dzhw.eu/\#!/de/questions/que-gra2009-ins1-5.2$}}}\\
				\begin{tabularx}{\hsize}{@{}lX}
					Fragenummer: &
					  Fragebogen des DZHW-Absolventenpanels 2009 - erste Welle:
					  5.2
 \\
					%--
					Fragetext: & Bitte geben Sie Ihre genaue Berufsbezeichnung, Ihren Aufgabenbereich sowie typische Arbeitsschwerpunkte Ihrer derzeitigen bzw. – falls Sie zurzeit nicht erwerbstätig sind – letzten (Haupt-)Tätigkeit an. \\
				\end{tabularx}





				%TABLE FOR THE NOMINAL / ORDINAL VALUES
        		\vspace*{0.5cm}
                \noindent\textbf{Häufigkeiten}

                \vspace*{-\baselineskip}
					%NUMERIC ELEMENTS NEED A HUGH SECOND COLOUMN AND A SMALL FIRST ONE
					\begin{filecontents}{\jobname-aocc21_g2d}
					\begin{longtable}{lXrrr}
					\toprule
					\textbf{Wert} & \textbf{Label} & \textbf{Häufigkeit} & \textbf{Prozent(gültig)} & \textbf{Prozent} \\
					\endhead
					\midrule
					\multicolumn{5}{l}{\textbf{Gültige Werte}}\\
						%DIFFERENT OBSERVATIONS <=20
								11 & \multicolumn{1}{X}{Offiziere} & %1 &
								  \num{1} &
								%--
								  \num[round-mode=places,round-precision=2]{0.01} &
								  \num[round-mode=places,round-precision=2]{0.01} \\
								111 & \multicolumn{1}{X}{Landwirtschaft} & %41 &
								  \num{41} &
								%--
								  \num[round-mode=places,round-precision=2]{0.54} &
								  \num[round-mode=places,round-precision=2]{0.39} \\
								112 & \multicolumn{1}{X}{Tierwirtschaft} & %2 &
								  \num{2} &
								%--
								  \num[round-mode=places,round-precision=2]{0.03} &
								  \num[round-mode=places,round-precision=2]{0.02} \\
								113 & \multicolumn{1}{X}{Pferdewirtschaft} & %1 &
								  \num{1} &
								%--
								  \num[round-mode=places,round-precision=2]{0.01} &
								  \num[round-mode=places,round-precision=2]{0.01} \\
								115 & \multicolumn{1}{X}{Tierpflege} & %1 &
								  \num{1} &
								%--
								  \num[round-mode=places,round-precision=2]{0.01} &
								  \num[round-mode=places,round-precision=2]{0.01} \\
								116 & \multicolumn{1}{X}{Weinbau} & %2 &
								  \num{2} &
								%--
								  \num[round-mode=places,round-precision=2]{0.03} &
								  \num[round-mode=places,round-precision=2]{0.02} \\
								117 & \multicolumn{1}{X}{Forst-,Jagdwirtschaft, Landschaftspflege} & %49 &
								  \num{49} &
								%--
								  \num[round-mode=places,round-precision=2]{0.65} &
								  \num[round-mode=places,round-precision=2]{0.47} \\
								121 & \multicolumn{1}{X}{Gartenbau} & %75 &
								  \num{75} &
								%--
								  \num[round-mode=places,round-precision=2]{0.99} &
								  \num[round-mode=places,round-precision=2]{0.71} \\
								211 & \multicolumn{1}{X}{Berg-, Tagebau und Sprengtechnik} & %9 &
								  \num{9} &
								%--
								  \num[round-mode=places,round-precision=2]{0.12} &
								  \num[round-mode=places,round-precision=2]{0.09} \\
								212 & \multicolumn{1}{X}{Naturstein-,Mineral-,Baustoffherstell.} & %1 &
								  \num{1} &
								%--
								  \num[round-mode=places,round-precision=2]{0.01} &
								  \num[round-mode=places,round-precision=2]{0.01} \\
							... & ... & ... & ... & ... \\
								941 & \multicolumn{1}{X}{Musik-, Gesang-, Dirigententätigkeiten} & %10 &
								  \num{10} &
								%--
								  \num[round-mode=places,round-precision=2]{0.13} &
								  \num[round-mode=places,round-precision=2]{0.1} \\

								942 & \multicolumn{1}{X}{Schauspiel, Tanz und Bewegungskunst} & %8 &
								  \num{8} &
								%--
								  \num[round-mode=places,round-precision=2]{0.11} &
								  \num[round-mode=places,round-precision=2]{0.08} \\

								943 & \multicolumn{1}{X}{Moderation und Unterhaltung} & %3 &
								  \num{3} &
								%--
								  \num[round-mode=places,round-precision=2]{0.04} &
								  \num[round-mode=places,round-precision=2]{0.03} \\

								944 & \multicolumn{1}{X}{Theater-, Film- und Fernsehproduktion} & %12 &
								  \num{12} &
								%--
								  \num[round-mode=places,round-precision=2]{0.16} &
								  \num[round-mode=places,round-precision=2]{0.11} \\

								945 & \multicolumn{1}{X}{Veranstaltungs-, Kamera-, Tontechnik} & %6 &
								  \num{6} &
								%--
								  \num[round-mode=places,round-precision=2]{0.08} &
								  \num[round-mode=places,round-precision=2]{0.06} \\

								946 & \multicolumn{1}{X}{Bühnen- und Kostümbildnerei, Requisite} & %5 &
								  \num{5} &
								%--
								  \num[round-mode=places,round-precision=2]{0.07} &
								  \num[round-mode=places,round-precision=2]{0.05} \\

								947 & \multicolumn{1}{X}{Museumstechnik und -management} & %11 &
								  \num{11} &
								%--
								  \num[round-mode=places,round-precision=2]{0.15} &
								  \num[round-mode=places,round-precision=2]{0.1} \\

								99996 & \multicolumn{1}{X}{Nachhilfelehrer/in} & %38 &
								  \num{38} &
								%--
								  \num[round-mode=places,round-precision=2]{0.5} &
								  \num[round-mode=places,round-precision=2]{0.36} \\

								99997 & \multicolumn{1}{X}{studentische Hilfskraft} & %265 &
								  \num{265} &
								%--
								  \num[round-mode=places,round-precision=2]{3.51} &
								  \num[round-mode=places,round-precision=2]{2.53} \\

								99998 & \multicolumn{1}{X}{wissenschaftliche Hilfskraft} & %201 &
								  \num{201} &
								%--
								  \num[round-mode=places,round-precision=2]{2.66} &
								  \num[round-mode=places,round-precision=2]{1.92} \\

					\midrule
					\multicolumn{2}{l}{Summe (gültig)} &
					  \textbf{\num{7557}} &
					\textbf{\num{100}} &
					  \textbf{\num[round-mode=places,round-precision=2]{72.01}} \\
					%--
					\multicolumn{5}{l}{\textbf{Fehlende Werte}}\\
							-998 &
							keine Angabe &
							  \num{814} &
							 - &
							  \num[round-mode=places,round-precision=2]{7.76} \\
							-989 &
							filterbedingt fehlend &
							  \num{2088} &
							 - &
							  \num[round-mode=places,round-precision=2]{19.9} \\
							-966 &
							nicht bestimmbar &
							  \num{35} &
							 - &
							  \num[round-mode=places,round-precision=2]{0.33} \\
					\midrule
					\multicolumn{2}{l}{\textbf{Summe (gesamt)}} &
				      \textbf{\num{10494}} &
				    \textbf{-} &
				    \textbf{\num{100}} \\
					\bottomrule
					\end{longtable}
					\end{filecontents}
					\LTXtable{\textwidth}{\jobname-aocc21_g2d}
				\label{tableValues:aocc21_g2d}
				\vspace*{-\baselineskip}
                    \begin{noten}
                	    \note{} Deskriptive Maßzahlen:
                	    Anzahl unterschiedlicher Beobachtungen: 134%
                	    ; 
                	      Modus ($h$): 841
                     \end{noten}

