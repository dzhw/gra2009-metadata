%EVERY VARIABLE HAS IT'S OWN PAGE

    \setcounter{footnote}{0}

    %omit vertical space
    \vspace*{-1.8cm}
	\section{aski01y (wichtig für Beruf: Auswirkung auf Natur und Gesellschaft)}
	\label{section:aski01y}



	% TABLE FOR VARIABLE DETAILS
  % '#' has to be escaped
    \vspace*{0.5cm}
    \noindent\textbf{Eigenschaften\footnote{Detailliertere Informationen zur Variable finden sich unter
		\url{https://metadata.fdz.dzhw.eu/\#!/de/variables/var-gra2009-ds1-aski01y$}}}\\
	\begin{tabularx}{\hsize}{@{}lX}
	Datentyp: & numerisch \\
	Skalenniveau: & ordinal \\
	Zugangswege: &
	  download-cuf, 
	  download-suf, 
	  remote-desktop-suf, 
	  onsite-suf
 \\
    \end{tabularx}



    %TABLE FOR QUESTION DETAILS
    %This has to be tested and has to be improved
    %rausfinden, ob einer Variable mehrere Fragen zugeordnet werden
    %dann evtl. nur die erste verwenden oder etwas anderes tun (Hinweis mehrere Fragen, auflisten mit Link)
				%TABLE FOR QUESTION DETAILS
				\vspace*{0.5cm}
                \noindent\textbf{Frage\footnote{Detailliertere Informationen zur Frage finden sich unter
		              \url{https://metadata.fdz.dzhw.eu/\#!/de/questions/que-gra2009-ins1-1.19$}}}\\
				\begin{tabularx}{\hsize}{@{}lX}
					Fragenummer: &
					  Fragebogen des DZHW-Absolventenpanels 2009 - erste Welle:
					  1.19
 \\
					%--
					Fragetext: & Wie wichtig sind die folgenden Kenntnisse und Fähigkeiten für Ihre derzeitige (bzw., wenn Sie nicht berufstätig sind, voraussichtliche) berufliche Tätigkeit (linke Spalte)? In welchem Maße verfügten Sie bei Abschluss des Erststudiums über diese Kenntnisse und Fähigkeiten (rechte Spalte)?\par  Wichtigkeit für die berufliche Tätigkeit\par  Wissen über die Auswirkungen meiner Arbeit auf Natur und Gesellschaft \\
				\end{tabularx}





				%TABLE FOR THE NOMINAL / ORDINAL VALUES
        		\vspace*{0.5cm}
                \noindent\textbf{Häufigkeiten}

                \vspace*{-\baselineskip}
					%NUMERIC ELEMENTS NEED A HUGH SECOND COLOUMN AND A SMALL FIRST ONE
					\begin{filecontents}{\jobname-aski01y}
					\begin{longtable}{lXrrr}
					\toprule
					\textbf{Wert} & \textbf{Label} & \textbf{Häufigkeit} & \textbf{Prozent(gültig)} & \textbf{Prozent} \\
					\endhead
					\midrule
					\multicolumn{5}{l}{\textbf{Gültige Werte}}\\
						%DIFFERENT OBSERVATIONS <=20

					1 &
				% TODO try size/length gt 0; take over for other passages
					\multicolumn{1}{X}{ sehr wichtig   } &


					%1902 &
					  \num{1902} &
					%--
					  \num[round-mode=places,round-precision=2]{18.8} &
					    \num[round-mode=places,round-precision=2]{18.12} \\
							%????

					2 &
				% TODO try size/length gt 0; take over for other passages
					\multicolumn{1}{X}{ 2   } &


					%2628 &
					  \num{2628} &
					%--
					  \num[round-mode=places,round-precision=2]{25.97} &
					    \num[round-mode=places,round-precision=2]{25.04} \\
							%????

					3 &
				% TODO try size/length gt 0; take over for other passages
					\multicolumn{1}{X}{ 3   } &


					%2826 &
					  \num{2826} &
					%--
					  \num[round-mode=places,round-precision=2]{27.93} &
					    \num[round-mode=places,round-precision=2]{26.93} \\
							%????

					4 &
				% TODO try size/length gt 0; take over for other passages
					\multicolumn{1}{X}{ 4   } &


					%1804 &
					  \num{1804} &
					%--
					  \num[round-mode=places,round-precision=2]{17.83} &
					    \num[round-mode=places,round-precision=2]{17.19} \\
							%????

					5 &
				% TODO try size/length gt 0; take over for other passages
					\multicolumn{1}{X}{ unwichtig   } &


					%959 &
					  \num{959} &
					%--
					  \num[round-mode=places,round-precision=2]{9.48} &
					    \num[round-mode=places,round-precision=2]{9.14} \\
							%????
						%DIFFERENT OBSERVATIONS >20
					\midrule
					\multicolumn{2}{l}{Summe (gültig)} &
					  \textbf{\num{10119}} &
					\textbf{\num{100}} &
					  \textbf{\num[round-mode=places,round-precision=2]{96.43}} \\
					%--
					\multicolumn{5}{l}{\textbf{Fehlende Werte}}\\
							-998 &
							keine Angabe &
							  \num{375} &
							 - &
							  \num[round-mode=places,round-precision=2]{3.57} \\
					\midrule
					\multicolumn{2}{l}{\textbf{Summe (gesamt)}} &
				      \textbf{\num{10494}} &
				    \textbf{-} &
				    \textbf{\num{100}} \\
					\bottomrule
					\end{longtable}
					\end{filecontents}
					\LTXtable{\textwidth}{\jobname-aski01y}
				\label{tableValues:aski01y}
				\vspace*{-\baselineskip}
                    \begin{noten}
                	    \note{} Deskriptive Maßzahlen:
                	    Anzahl unterschiedlicher Beobachtungen: 5%
                	    ; 
                	      Minimum ($min$): 1; 
                	      Maximum ($max$): 5; 
                	      Median ($\tilde{x}$): 3; 
                	      Modus ($h$): 3
                     \end{noten}

