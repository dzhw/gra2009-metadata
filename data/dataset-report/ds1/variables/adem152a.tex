%EVERY VARIABLE HAS IT'S OWN PAGE

    \setcounter{footnote}{0}

    %omit vertical space
    \vspace*{-1.8cm}
	\section{adem152a (2. Kind: Geburt (Monat))}
	\label{section:adem152a}



	% TABLE FOR VARIABLE DETAILS
  % '#' has to be escaped
    \vspace*{0.5cm}
    \noindent\textbf{Eigenschaften\footnote{Detailliertere Informationen zur Variable finden sich unter
		\url{https://metadata.fdz.dzhw.eu/\#!/de/variables/var-gra2009-ds1-adem152a$}}}\\
	\begin{tabularx}{\hsize}{@{}lX}
	Datentyp: & numerisch \\
	Skalenniveau: & ordinal \\
	Zugangswege: &
	  download-cuf, 
	  download-suf, 
	  remote-desktop-suf, 
	  onsite-suf
 \\
    \end{tabularx}



    %TABLE FOR QUESTION DETAILS
    %This has to be tested and has to be improved
    %rausfinden, ob einer Variable mehrere Fragen zugeordnet werden
    %dann evtl. nur die erste verwenden oder etwas anderes tun (Hinweis mehrere Fragen, auflisten mit Link)
				%TABLE FOR QUESTION DETAILS
				\vspace*{0.5cm}
                \noindent\textbf{Frage\footnote{Detailliertere Informationen zur Frage finden sich unter
		              \url{https://metadata.fdz.dzhw.eu/\#!/de/questions/que-gra2009-ins1-6.17$}}}\\
				\begin{tabularx}{\hsize}{@{}lX}
					Fragenummer: &
					  Fragebogen des DZHW-Absolventenpanels 2009 - erste Welle:
					  6.17
 \\
					%--
					Fragetext: & Wann wurden Ihre Kinder geboren?\par  2. Kind\par  Monat \\
				\end{tabularx}





				%TABLE FOR THE NOMINAL / ORDINAL VALUES
        		\vspace*{0.5cm}
                \noindent\textbf{Häufigkeiten}

                \vspace*{-\baselineskip}
					%NUMERIC ELEMENTS NEED A HUGH SECOND COLOUMN AND A SMALL FIRST ONE
					\begin{filecontents}{\jobname-adem152a}
					\begin{longtable}{lXrrr}
					\toprule
					\textbf{Wert} & \textbf{Label} & \textbf{Häufigkeit} & \textbf{Prozent(gültig)} & \textbf{Prozent} \\
					\endhead
					\midrule
					\multicolumn{5}{l}{\textbf{Gültige Werte}}\\
						%DIFFERENT OBSERVATIONS <=20

					1 &
				% TODO try size/length gt 0; take over for other passages
					\multicolumn{1}{X}{ Januar   } &


					%24 &
					  \num{24} &
					%--
					  \num[round-mode=places,round-precision=2]{9.3} &
					    \num[round-mode=places,round-precision=2]{0.23} \\
							%????

					2 &
				% TODO try size/length gt 0; take over for other passages
					\multicolumn{1}{X}{ Februar   } &


					%31 &
					  \num{31} &
					%--
					  \num[round-mode=places,round-precision=2]{12.02} &
					    \num[round-mode=places,round-precision=2]{0.3} \\
							%????

					3 &
				% TODO try size/length gt 0; take over for other passages
					\multicolumn{1}{X}{ März   } &


					%28 &
					  \num{28} &
					%--
					  \num[round-mode=places,round-precision=2]{10.85} &
					    \num[round-mode=places,round-precision=2]{0.27} \\
							%????

					4 &
				% TODO try size/length gt 0; take over for other passages
					\multicolumn{1}{X}{ April   } &


					%26 &
					  \num{26} &
					%--
					  \num[round-mode=places,round-precision=2]{10.08} &
					    \num[round-mode=places,round-precision=2]{0.25} \\
							%????

					5 &
				% TODO try size/length gt 0; take over for other passages
					\multicolumn{1}{X}{ Mai   } &


					%21 &
					  \num{21} &
					%--
					  \num[round-mode=places,round-precision=2]{8.14} &
					    \num[round-mode=places,round-precision=2]{0.2} \\
							%????

					6 &
				% TODO try size/length gt 0; take over for other passages
					\multicolumn{1}{X}{ Juni   } &


					%20 &
					  \num{20} &
					%--
					  \num[round-mode=places,round-precision=2]{7.75} &
					    \num[round-mode=places,round-precision=2]{0.19} \\
							%????

					7 &
				% TODO try size/length gt 0; take over for other passages
					\multicolumn{1}{X}{ Juli   } &


					%20 &
					  \num{20} &
					%--
					  \num[round-mode=places,round-precision=2]{7.75} &
					    \num[round-mode=places,round-precision=2]{0.19} \\
							%????

					8 &
				% TODO try size/length gt 0; take over for other passages
					\multicolumn{1}{X}{ August   } &


					%25 &
					  \num{25} &
					%--
					  \num[round-mode=places,round-precision=2]{9.69} &
					    \num[round-mode=places,round-precision=2]{0.24} \\
							%????

					9 &
				% TODO try size/length gt 0; take over for other passages
					\multicolumn{1}{X}{ September   } &


					%16 &
					  \num{16} &
					%--
					  \num[round-mode=places,round-precision=2]{6.2} &
					    \num[round-mode=places,round-precision=2]{0.15} \\
							%????

					10 &
				% TODO try size/length gt 0; take over for other passages
					\multicolumn{1}{X}{ Oktober   } &


					%17 &
					  \num{17} &
					%--
					  \num[round-mode=places,round-precision=2]{6.59} &
					    \num[round-mode=places,round-precision=2]{0.16} \\
							%????

					11 &
				% TODO try size/length gt 0; take over for other passages
					\multicolumn{1}{X}{ November   } &


					%13 &
					  \num{13} &
					%--
					  \num[round-mode=places,round-precision=2]{5.04} &
					    \num[round-mode=places,round-precision=2]{0.12} \\
							%????

					12 &
				% TODO try size/length gt 0; take over for other passages
					\multicolumn{1}{X}{ Dezember   } &


					%17 &
					  \num{17} &
					%--
					  \num[round-mode=places,round-precision=2]{6.59} &
					    \num[round-mode=places,round-precision=2]{0.16} \\
							%????
						%DIFFERENT OBSERVATIONS >20
					\midrule
					\multicolumn{2}{l}{Summe (gültig)} &
					  \textbf{\num{258}} &
					\textbf{\num{100}} &
					  \textbf{\num[round-mode=places,round-precision=2]{2.46}} \\
					%--
					\multicolumn{5}{l}{\textbf{Fehlende Werte}}\\
							-998 &
							keine Angabe &
							  \num{618} &
							 - &
							  \num[round-mode=places,round-precision=2]{5.89} \\
							-989 &
							filterbedingt fehlend &
							  \num{9618} &
							 - &
							  \num[round-mode=places,round-precision=2]{91.65} \\
					\midrule
					\multicolumn{2}{l}{\textbf{Summe (gesamt)}} &
				      \textbf{\num{10494}} &
				    \textbf{-} &
				    \textbf{\num{100}} \\
					\bottomrule
					\end{longtable}
					\end{filecontents}
					\LTXtable{\textwidth}{\jobname-adem152a}
				\label{tableValues:adem152a}
				\vspace*{-\baselineskip}
                    \begin{noten}
                	    \note{} Deskriptive Maßzahlen:
                	    Anzahl unterschiedlicher Beobachtungen: 12%
                	    ; 
                	      Minimum ($min$): 1; 
                	      Maximum ($max$): 12; 
                	      Median ($\tilde{x}$): 5; 
                	      Modus ($h$): 2
                     \end{noten}

