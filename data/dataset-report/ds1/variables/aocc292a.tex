%EVERY VARIABLE HAS IT'S OWN PAGE

    \setcounter{footnote}{0}

    %omit vertical space
    \vspace*{-1.8cm}
	\section{aocc292a (letzte Stelle: Betriebsgröße)}
	\label{section:aocc292a}



	% TABLE FOR VARIABLE DETAILS
  % '#' has to be escaped
    \vspace*{0.5cm}
    \noindent\textbf{Eigenschaften\footnote{Detailliertere Informationen zur Variable finden sich unter
		\url{https://metadata.fdz.dzhw.eu/\#!/de/variables/var-gra2009-ds1-aocc292a$}}}\\
	\begin{tabularx}{\hsize}{@{}lX}
	Datentyp: & numerisch \\
	Skalenniveau: & nominal \\
	Zugangswege: &
	  download-cuf, 
	  download-suf, 
	  remote-desktop-suf, 
	  onsite-suf
 \\
    \end{tabularx}



    %TABLE FOR QUESTION DETAILS
    %This has to be tested and has to be improved
    %rausfinden, ob einer Variable mehrere Fragen zugeordnet werden
    %dann evtl. nur die erste verwenden oder etwas anderes tun (Hinweis mehrere Fragen, auflisten mit Link)
				%TABLE FOR QUESTION DETAILS
				\vspace*{0.5cm}
                \noindent\textbf{Frage\footnote{Detailliertere Informationen zur Frage finden sich unter
		              \url{https://metadata.fdz.dzhw.eu/\#!/de/questions/que-gra2009-ins1-5.9$}}}\\
				\begin{tabularx}{\hsize}{@{}lX}
					Fragenummer: &
					  Fragebogen des DZHW-Absolventenpanels 2009 - erste Welle:
					  5.9
 \\
					%--
					Fragetext: & Welcher der folgenden Betriebsgrößen ist Ihr Betrieb/Ihre Dienststelle zuzuordnen?\par  heutige Stelle\par  Über 1000 Mitarbeiter/innen\par  Über 500 bis 1000 Mitarbeiter/innen Über 100 bis 500 Mitarbeiter/innen\par  Über 20 bis 100 Mitarbeiter/innen\par  5 bis 20 Mitarbeiter/innen\par  Weniger als 5 Mitarbeiter/innen\par  Freischaffend, ohne Mitarbeiter/innen\par  Sonstiges \\
				\end{tabularx}





				%TABLE FOR THE NOMINAL / ORDINAL VALUES
        		\vspace*{0.5cm}
                \noindent\textbf{Häufigkeiten}

                \vspace*{-\baselineskip}
					%NUMERIC ELEMENTS NEED A HUGH SECOND COLOUMN AND A SMALL FIRST ONE
					\begin{filecontents}{\jobname-aocc292a}
					\begin{longtable}{lXrrr}
					\toprule
					\textbf{Wert} & \textbf{Label} & \textbf{Häufigkeit} & \textbf{Prozent(gültig)} & \textbf{Prozent} \\
					\endhead
					\midrule
					\multicolumn{5}{l}{\textbf{Gültige Werte}}\\
						%DIFFERENT OBSERVATIONS <=20

					1 &
				% TODO try size/length gt 0; take over for other passages
					\multicolumn{1}{X}{ über 1000 Mitarbeiter(innen)   } &


					%1612 &
					  \num{1612} &
					%--
					  \num[round-mode=places,round-precision=2]{22.75} &
					    \num[round-mode=places,round-precision=2]{15.36} \\
							%????

					2 &
				% TODO try size/length gt 0; take over for other passages
					\multicolumn{1}{X}{ 500 bis 1000 Mitarbeiter(innen)   } &


					%533 &
					  \num{533} &
					%--
					  \num[round-mode=places,round-precision=2]{7.52} &
					    \num[round-mode=places,round-precision=2]{5.08} \\
							%????

					3 &
				% TODO try size/length gt 0; take over for other passages
					\multicolumn{1}{X}{ 100 bis 500 Mitarbeiter(innen)   } &


					%1147 &
					  \num{1147} &
					%--
					  \num[round-mode=places,round-precision=2]{16.18} &
					    \num[round-mode=places,round-precision=2]{10.93} \\
							%????

					4 &
				% TODO try size/length gt 0; take over for other passages
					\multicolumn{1}{X}{ 20 bis 100 Mitarbeiter(innen)   } &


					%1660 &
					  \num{1660} &
					%--
					  \num[round-mode=places,round-precision=2]{23.42} &
					    \num[round-mode=places,round-precision=2]{15.82} \\
							%????

					5 &
				% TODO try size/length gt 0; take over for other passages
					\multicolumn{1}{X}{ 5 bis 20 Mitarbeiter(innen)   } &


					%1485 &
					  \num{1485} &
					%--
					  \num[round-mode=places,round-precision=2]{20.95} &
					    \num[round-mode=places,round-precision=2]{14.15} \\
							%????

					6 &
				% TODO try size/length gt 0; take over for other passages
					\multicolumn{1}{X}{ weniger als 5 Mitarbeiter(innen)   } &


					%389 &
					  \num{389} &
					%--
					  \num[round-mode=places,round-precision=2]{5.49} &
					    \num[round-mode=places,round-precision=2]{3.71} \\
							%????

					7 &
				% TODO try size/length gt 0; take over for other passages
					\multicolumn{1}{X}{ freischaffend, ohne Mitarbeiter(innen)   } &


					%184 &
					  \num{184} &
					%--
					  \num[round-mode=places,round-precision=2]{2.6} &
					    \num[round-mode=places,round-precision=2]{1.75} \\
							%????

					8 &
				% TODO try size/length gt 0; take over for other passages
					\multicolumn{1}{X}{ Sonstiges   } &


					%77 &
					  \num{77} &
					%--
					  \num[round-mode=places,round-precision=2]{1.09} &
					    \num[round-mode=places,round-precision=2]{0.73} \\
							%????
						%DIFFERENT OBSERVATIONS >20
					\midrule
					\multicolumn{2}{l}{Summe (gültig)} &
					  \textbf{\num{7087}} &
					\textbf{\num{100}} &
					  \textbf{\num[round-mode=places,round-precision=2]{67.53}} \\
					%--
					\multicolumn{5}{l}{\textbf{Fehlende Werte}}\\
							-998 &
							keine Angabe &
							  \num{1319} &
							 - &
							  \num[round-mode=places,round-precision=2]{12.57} \\
							-989 &
							filterbedingt fehlend &
							  \num{2088} &
							 - &
							  \num[round-mode=places,round-precision=2]{19.9} \\
					\midrule
					\multicolumn{2}{l}{\textbf{Summe (gesamt)}} &
				      \textbf{\num{10494}} &
				    \textbf{-} &
				    \textbf{\num{100}} \\
					\bottomrule
					\end{longtable}
					\end{filecontents}
					\LTXtable{\textwidth}{\jobname-aocc292a}
				\label{tableValues:aocc292a}
				\vspace*{-\baselineskip}
                    \begin{noten}
                	    \note{} Deskriptive Maßzahlen:
                	    Anzahl unterschiedlicher Beobachtungen: 8%
                	    ; 
                	      Modus ($h$): 4
                     \end{noten}

