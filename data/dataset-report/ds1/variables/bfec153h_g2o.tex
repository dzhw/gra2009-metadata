%EVERY VARIABLE HAS IT'S OWN PAGE

    \setcounter{footnote}{0}

    %omit vertical space
    \vspace*{-1.8cm}
	\section{bfec153h\_g2o (3. weitere akad. Qualifikation: Hochschule (NUTS2))}
	\label{section:bfec153h_g2o}



	% TABLE FOR VARIABLE DETAILS
  % '#' has to be escaped
    \vspace*{0.5cm}
    \noindent\textbf{Eigenschaften\footnote{Detailliertere Informationen zur Variable finden sich unter
		\url{https://metadata.fdz.dzhw.eu/\#!/de/variables/var-gra2009-ds1-bfec153h_g2o$}}}\\
	\begin{tabularx}{\hsize}{@{}lX}
	Datentyp: & string \\
	Skalenniveau: & nominal \\
	Zugangswege: &
	  onsite-suf
 \\
    \end{tabularx}



    %TABLE FOR QUESTION DETAILS
    %This has to be tested and has to be improved
    %rausfinden, ob einer Variable mehrere Fragen zugeordnet werden
    %dann evtl. nur die erste verwenden oder etwas anderes tun (Hinweis mehrere Fragen, auflisten mit Link)
				%TABLE FOR QUESTION DETAILS
				\vspace*{0.5cm}
                \noindent\textbf{Frage\footnote{Detailliertere Informationen zur Frage finden sich unter
		              \url{https://metadata.fdz.dzhw.eu/\#!/de/questions/que-gra2009-ins2-5.2$}}}\\
				\begin{tabularx}{\hsize}{@{}lX}
					Fragenummer: &
					  Fragebogen des DZHW-Absolventenpanels 2009 - zweite Welle, Hauptbefragung (PAPI):
					  5.2
 \\
					%--
					Fragetext: & Bitte tragen Sie diese längerfristigen Studienangebote, die Sie nach Ihrem Studienabschluss aus dem Jahr 2008/2009 begonnen, weitergeführt oder abgeschlossen haben (auch abgebrochene oder unterbrochene), in das folgende Tableau ein! \\
				\end{tabularx}





				%TABLE FOR THE NOMINAL / ORDINAL VALUES
        		\vspace*{0.5cm}
                \noindent\textbf{Häufigkeiten}

                \vspace*{-\baselineskip}
					%STRING ELEMENTS NEEDS A HUGH FIRST COLOUMN AND A SMALL SECOND ONE
					\begin{filecontents}{\jobname-bfec153h_g2o}
					\begin{longtable}{Xlrrr}
					\toprule
					\textbf{Wert} & \textbf{Label} & \textbf{Häufigkeit} & \textbf{Prozent (gültig)} & \textbf{Prozent} \\
					\endhead
					\midrule
					\multicolumn{5}{l}{\textbf{Gültige Werte}}\\
						%DIFFERENT OBSERVATIONS <=20

					\multicolumn{1}{X}{DE14 Tübingen} &
					- &
					\num{1} &
					\num[round-mode=places,round-precision=2]{10} &
					\num[round-mode=places,round-precision=2]{0.01} \\
					
					\multicolumn{1}{X}{DE21 Oberbayern} &
					- &
					\num{2} &
					\num[round-mode=places,round-precision=2]{20} &
					\num[round-mode=places,round-precision=2]{0.02} \\
					
					\multicolumn{1}{X}{DE23 Oberpfalz} &
					- &
					\num{1} &
					\num[round-mode=places,round-precision=2]{10} &
					\num[round-mode=places,round-precision=2]{0.01} \\
					
					\multicolumn{1}{X}{DE94 Weser-Ems} &
					- &
					\num{2} &
					\num[round-mode=places,round-precision=2]{20} &
					\num[round-mode=places,round-precision=2]{0.02} \\
					
					\multicolumn{1}{X}{DEA5 Arnsberg} &
					- &
					\num{2} &
					\num[round-mode=places,round-precision=2]{20} &
					\num[round-mode=places,round-precision=2]{0.02} \\
					
					\multicolumn{1}{X}{DEB3 Rheinhessen-Pfalz} &
					- &
					\num{1} &
					\num[round-mode=places,round-precision=2]{10} &
					\num[round-mode=places,round-precision=2]{0.01} \\
					
					\multicolumn{1}{X}{DEG0 Thüringen} &
					- &
					\num{1} &
					\num[round-mode=places,round-precision=2]{10} &
					\num[round-mode=places,round-precision=2]{0.01} \\
											%DIFFERENT OBSERVATIONS >20
					\midrule
						\multicolumn{2}{l}{Summe (gültig)} & \textbf{\num{10}} &
						\textbf{\num{100}} &
					    \textbf{\num[round-mode=places,round-precision=2]{0.1}} \\
					\multicolumn{5}{l}{\textbf{Fehlende Werte}}\\
							-989 & filterbedingt fehlend & \num{2662} & - & \num[round-mode=places,round-precision=2]{25.37} \\

							-995 & keine Teilnahme (Panel) & \num{5739} & - & \num[round-mode=places,round-precision=2]{54.69} \\

							-998 & keine Angabe & \num{2083} & - & \num[round-mode=places,round-precision=2]{19.85} \\

					\midrule
					\multicolumn{2}{l}{\textbf{Summe (gesamt)}} & \textbf{\num{10494}} & \textbf{-} & \textbf{\num{100}} \\
					\bottomrule
					\caption{Werte der Variable bfec153h\_g2o}
					\end{longtable}
					\end{filecontents}
					\LTXtable{\textwidth}{\jobname-bfec153h_g2o}

