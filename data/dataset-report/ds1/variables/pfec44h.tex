%EVERY VARIABLE HAS IT'S OWN PAGE

    \setcounter{footnote}{0}

    %omit vertical space
    \vspace*{-1.8cm}
	\section{pfec44h (Finanzierung Promotion: private Zuwendungen)}
	\label{section:pfec44h}



	% TABLE FOR VARIABLE DETAILS
  % '#' has to be escaped
    \vspace*{0.5cm}
    \noindent\textbf{Eigenschaften\footnote{Detailliertere Informationen zur Variable finden sich unter
		\url{https://metadata.fdz.dzhw.eu/\#!/de/variables/var-gra2009-ds1-pfec44h$}}}\\
	\begin{tabularx}{\hsize}{@{}lX}
	Datentyp: & numerisch \\
	Skalenniveau: & nominal \\
	Zugangswege: &
	  download-cuf, 
	  download-suf, 
	  remote-desktop-suf, 
	  onsite-suf
 \\
    \end{tabularx}



    %TABLE FOR QUESTION DETAILS
    %This has to be tested and has to be improved
    %rausfinden, ob einer Variable mehrere Fragen zugeordnet werden
    %dann evtl. nur die erste verwenden oder etwas anderes tun (Hinweis mehrere Fragen, auflisten mit Link)
				%TABLE FOR QUESTION DETAILS
				\vspace*{0.5cm}
                \noindent\textbf{Frage\footnote{Detailliertere Informationen zur Frage finden sich unter
		              \url{https://metadata.fdz.dzhw.eu/\#!/de/questions/que-gra2009-ins4-37$}}}\\
				\begin{tabularx}{\hsize}{@{}lX}
					Fragenummer: &
					  Fragebogen des DZHW-Absolventenpanels 2009 - zweite Welle, Vertiefungsbefragung Promotion:
					  37
 \\
					%--
					Fragetext: & Wie finanzierten Sie Ihre Promotion?,Wie finanzieren Sie Ihre Promotion?,Mehrfachnennungen sind möglich.,Aus privaten Zuwendungen (z.B. Eltern, Partner(in)) \\
				\end{tabularx}





				%TABLE FOR THE NOMINAL / ORDINAL VALUES
        		\vspace*{0.5cm}
                \noindent\textbf{Häufigkeiten}

                \vspace*{-\baselineskip}
					%NUMERIC ELEMENTS NEED A HUGH SECOND COLOUMN AND A SMALL FIRST ONE
					\begin{filecontents}{\jobname-pfec44h}
					\begin{longtable}{lXrrr}
					\toprule
					\textbf{Wert} & \textbf{Label} & \textbf{Häufigkeit} & \textbf{Prozent(gültig)} & \textbf{Prozent} \\
					\endhead
					\midrule
					\multicolumn{5}{l}{\textbf{Gültige Werte}}\\
						%DIFFERENT OBSERVATIONS <=20

					0 &
				% TODO try size/length gt 0; take over for other passages
					\multicolumn{1}{X}{ nicht genannt   } &


					%517 &
					  \num{517} &
					%--
					  \num[round-mode=places,round-precision=2]{80.53} &
					    \num[round-mode=places,round-precision=2]{4.93} \\
							%????

					1 &
				% TODO try size/length gt 0; take over for other passages
					\multicolumn{1}{X}{ genannt   } &


					%125 &
					  \num{125} &
					%--
					  \num[round-mode=places,round-precision=2]{19.47} &
					    \num[round-mode=places,round-precision=2]{1.19} \\
							%????
						%DIFFERENT OBSERVATIONS >20
					\midrule
					\multicolumn{2}{l}{Summe (gültig)} &
					  \textbf{\num{642}} &
					\textbf{\num{100}} &
					  \textbf{\num[round-mode=places,round-precision=2]{6.12}} \\
					%--
					\multicolumn{5}{l}{\textbf{Fehlende Werte}}\\
							-998 &
							keine Angabe &
							  \num{28} &
							 - &
							  \num[round-mode=places,round-precision=2]{0.27} \\
							-995 &
							keine Teilnahme (Panel) &
							  \num{9818} &
							 - &
							  \num[round-mode=places,round-precision=2]{93.56} \\
							-989 &
							filterbedingt fehlend &
							  \num{6} &
							 - &
							  \num[round-mode=places,round-precision=2]{0.06} \\
					\midrule
					\multicolumn{2}{l}{\textbf{Summe (gesamt)}} &
				      \textbf{\num{10494}} &
				    \textbf{-} &
				    \textbf{\num{100}} \\
					\bottomrule
					\end{longtable}
					\end{filecontents}
					\LTXtable{\textwidth}{\jobname-pfec44h}
				\label{tableValues:pfec44h}
				\vspace*{-\baselineskip}
                    \begin{noten}
                	    \note{} Deskriptive Maßzahlen:
                	    Anzahl unterschiedlicher Beobachtungen: 2%
                	    ; 
                	      Modus ($h$): 0
                     \end{noten}

