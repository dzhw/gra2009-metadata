%EVERY VARIABLE HAS IT'S OWN PAGE

    \setcounter{footnote}{0}

    %omit vertical space
    \vspace*{-1.8cm}
	\section{bfvt065d (mehrstündige berufl. Weiterbildung: Inhalt 2)}
	\label{section:bfvt065d}



	% TABLE FOR VARIABLE DETAILS
  % '#' has to be escaped
    \vspace*{0.5cm}
    \noindent\textbf{Eigenschaften\footnote{Detailliertere Informationen zur Variable finden sich unter
		\url{https://metadata.fdz.dzhw.eu/\#!/de/variables/var-gra2009-ds1-bfvt065d$}}}\\
	\begin{tabularx}{\hsize}{@{}lX}
	Datentyp: & numerisch \\
	Skalenniveau: & nominal \\
	Zugangswege: &
	  download-cuf, 
	  download-suf, 
	  remote-desktop-suf, 
	  onsite-suf
 \\
    \end{tabularx}



    %TABLE FOR QUESTION DETAILS
    %This has to be tested and has to be improved
    %rausfinden, ob einer Variable mehrere Fragen zugeordnet werden
    %dann evtl. nur die erste verwenden oder etwas anderes tun (Hinweis mehrere Fragen, auflisten mit Link)
				%TABLE FOR QUESTION DETAILS
				\vspace*{0.5cm}
                \noindent\textbf{Frage\footnote{Detailliertere Informationen zur Frage finden sich unter
		              \url{https://metadata.fdz.dzhw.eu/\#!/de/questions/que-gra2009-ins2-6.5$}}}\\
				\begin{tabularx}{\hsize}{@{}lX}
					Fragenummer: &
					  Fragebogen des DZHW-Absolventenpanels 2009 - zweite Welle, Hauptbefragung (PAPI):
					  6.5
 \\
					%--
					Fragetext: & Im Folgenden bitten wir Sie um Angaben zu beruflichen Fort- und Weiterbildungen der letzten 12 Monate. Bitte denken Sie dabei an alle Weiterbildungen, die Sie besucht haben und geben sie diese in der passenden Zeile an.\par  5. Fort- /oder Weiterbildung\par  Themen (Mehrfachnennung möglich)\par  Schlüssel s. Klappliste B) \\
				\end{tabularx}
				%TABLE FOR QUESTION DETAILS
				\vspace*{0.5cm}
                \noindent\textbf{Frage\footnote{Detailliertere Informationen zur Frage finden sich unter
		              \url{https://metadata.fdz.dzhw.eu/\#!/de/questions/que-gra2009-ins3-75$}}}\\
				\begin{tabularx}{\hsize}{@{}lX}
					Fragenummer: &
					  Fragebogen des DZHW-Absolventenpanels 2009 - zweite Welle, Hauptbefragung (CAWI):
					  75
 \\
					%--
					Fragetext: & Bitte tragen Sie hier die für Sie wichtigsten Themen bzw. Fachgebiete dieser Veranstaltungen ein. \\
				\end{tabularx}





				%TABLE FOR THE NOMINAL / ORDINAL VALUES
        		\vspace*{0.5cm}
                \noindent\textbf{Häufigkeiten}

                \vspace*{-\baselineskip}
					%NUMERIC ELEMENTS NEED A HUGH SECOND COLOUMN AND A SMALL FIRST ONE
					\begin{filecontents}{\jobname-bfvt065d}
					\begin{longtable}{lXrrr}
					\toprule
					\textbf{Wert} & \textbf{Label} & \textbf{Häufigkeit} & \textbf{Prozent(gültig)} & \textbf{Prozent} \\
					\endhead
					\midrule
					\multicolumn{5}{l}{\textbf{Gültige Werte}}\\
						%DIFFERENT OBSERVATIONS <=20
								1 & \multicolumn{1}{X}{ingenieurwissenschaftliche Themen} & %31 &
								  \num{31} &
								%--
								  \num[round-mode=places,round-precision=2]{3.88} &
								  \num[round-mode=places,round-precision=2]{0.3} \\
								2 & \multicolumn{1}{X}{naturwissenschaftliche Themen} & %42 &
								  \num{42} &
								%--
								  \num[round-mode=places,round-precision=2]{5.26} &
								  \num[round-mode=places,round-precision=2]{0.4} \\
								3 & \multicolumn{1}{X}{mathematische Gebiete/Statistik} & %19 &
								  \num{19} &
								%--
								  \num[round-mode=places,round-precision=2]{2.38} &
								  \num[round-mode=places,round-precision=2]{0.18} \\
								4 & \multicolumn{1}{X}{sozialwissenschaftliche Themen} & %33 &
								  \num{33} &
								%--
								  \num[round-mode=places,round-precision=2]{4.13} &
								  \num[round-mode=places,round-precision=2]{0.31} \\
								5 & \multicolumn{1}{X}{geisteswissenschtliche Themen} & %25 &
								  \num{25} &
								%--
								  \num[round-mode=places,round-precision=2]{3.13} &
								  \num[round-mode=places,round-precision=2]{0.24} \\
								6 & \multicolumn{1}{X}{pädagogische/psychologische Themen} & %127 &
								  \num{127} &
								%--
								  \num[round-mode=places,round-precision=2]{15.89} &
								  \num[round-mode=places,round-precision=2]{1.21} \\
								7 & \multicolumn{1}{X}{medizinische Spezialgebiete} & %49 &
								  \num{49} &
								%--
								  \num[round-mode=places,round-precision=2]{6.13} &
								  \num[round-mode=places,round-precision=2]{0.47} \\
								8 & \multicolumn{1}{X}{informationstechnisches Spezialwissen} & %33 &
								  \num{33} &
								%--
								  \num[round-mode=places,round-precision=2]{4.13} &
								  \num[round-mode=places,round-precision=2]{0.31} \\
								9 & \multicolumn{1}{X}{Managementwissen} & %41 &
								  \num{41} &
								%--
								  \num[round-mode=places,round-precision=2]{5.13} &
								  \num[round-mode=places,round-precision=2]{0.39} \\
								10 & \multicolumn{1}{X}{Wirtschaftskenntnisse} & %24 &
								  \num{24} &
								%--
								  \num[round-mode=places,round-precision=2]{3} &
								  \num[round-mode=places,round-precision=2]{0.23} \\
							... & ... & ... & ... & ... \\
								15 & \multicolumn{1}{X}{EDV-Anwendungen} & %94 &
								  \num{94} &
								%--
								  \num[round-mode=places,round-precision=2]{11.76} &
								  \num[round-mode=places,round-precision=2]{0.9} \\

								16 & \multicolumn{1}{X}{Fremdsprachen} & %23 &
								  \num{23} &
								%--
								  \num[round-mode=places,round-precision=2]{2.88} &
								  \num[round-mode=places,round-precision=2]{0.22} \\

								17 & \multicolumn{1}{X}{Mitarbeiterführung/Personalentwicklung} & %24 &
								  \num{24} &
								%--
								  \num[round-mode=places,round-precision=2]{3} &
								  \num[round-mode=places,round-precision=2]{0.23} \\

								18 & \multicolumn{1}{X}{Kommunikations-/Interaktionstraining} & %61 &
								  \num{61} &
								%--
								  \num[round-mode=places,round-precision=2]{7.63} &
								  \num[round-mode=places,round-precision=2]{0.58} \\

								19 & \multicolumn{1}{X}{internationale Beziehungen, Kulturkenntnisse, Landeskunde} & %12 &
								  \num{12} &
								%--
								  \num[round-mode=places,round-precision=2]{1.5} &
								  \num[round-mode=places,round-precision=2]{0.11} \\

								20 & \multicolumn{1}{X}{ökologische Themen} & %4 &
								  \num{4} &
								%--
								  \num[round-mode=places,round-precision=2]{0.5} &
								  \num[round-mode=places,round-precision=2]{0.04} \\

								21 & \multicolumn{1}{X}{berufsethische Themen} & %13 &
								  \num{13} &
								%--
								  \num[round-mode=places,round-precision=2]{1.63} &
								  \num[round-mode=places,round-precision=2]{0.12} \\

								22 & \multicolumn{1}{X}{Existenzgründung} & %5 &
								  \num{5} &
								%--
								  \num[round-mode=places,round-precision=2]{0.63} &
								  \num[round-mode=places,round-precision=2]{0.05} \\

								23 & \multicolumn{1}{X}{betriebliches Gesundheitswesen, Arbeitssicherheit} & %32 &
								  \num{32} &
								%--
								  \num[round-mode=places,round-precision=2]{4.01} &
								  \num[round-mode=places,round-precision=2]{0.3} \\

								24 & \multicolumn{1}{X}{Sonstige} & %25 &
								  \num{25} &
								%--
								  \num[round-mode=places,round-precision=2]{3.13} &
								  \num[round-mode=places,round-precision=2]{0.24} \\

					\midrule
					\multicolumn{2}{l}{Summe (gültig)} &
					  \textbf{\num{799}} &
					\textbf{\num{100}} &
					  \textbf{\num[round-mode=places,round-precision=2]{7.61}} \\
					%--
					\multicolumn{5}{l}{\textbf{Fehlende Werte}}\\
							-998 &
							keine Angabe &
							  \num{3956} &
							 - &
							  \num[round-mode=places,round-precision=2]{37.7} \\
							-995 &
							keine Teilnahme (Panel) &
							  \num{5739} &
							 - &
							  \num[round-mode=places,round-precision=2]{54.69} \\
					\midrule
					\multicolumn{2}{l}{\textbf{Summe (gesamt)}} &
				      \textbf{\num{10494}} &
				    \textbf{-} &
				    \textbf{\num{100}} \\
					\bottomrule
					\end{longtable}
					\end{filecontents}
					\LTXtable{\textwidth}{\jobname-bfvt065d}
				\label{tableValues:bfvt065d}
				\vspace*{-\baselineskip}
                    \begin{noten}
                	    \note{} Deskriptive Maßzahlen:
                	    Anzahl unterschiedlicher Beobachtungen: 24%
                	    ; 
                	      Modus ($h$): 6
                     \end{noten}

