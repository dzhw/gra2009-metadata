%EVERY VARIABLE HAS IT'S OWN PAGE

    \setcounter{footnote}{0}

    %omit vertical space
    \vspace*{-1.8cm}
	\section{bfvt09b (Bedarf Weiterbildung: Inhalt 2)}
	\label{section:bfvt09b}



	% TABLE FOR VARIABLE DETAILS
  % '#' has to be escaped
    \vspace*{0.5cm}
    \noindent\textbf{Eigenschaften\footnote{Detailliertere Informationen zur Variable finden sich unter
		\url{https://metadata.fdz.dzhw.eu/\#!/de/variables/var-gra2009-ds1-bfvt09b$}}}\\
	\begin{tabularx}{\hsize}{@{}lX}
	Datentyp: & numerisch \\
	Skalenniveau: & nominal \\
	Zugangswege: &
	  download-cuf, 
	  download-suf, 
	  remote-desktop-suf, 
	  onsite-suf
 \\
    \end{tabularx}



    %TABLE FOR QUESTION DETAILS
    %This has to be tested and has to be improved
    %rausfinden, ob einer Variable mehrere Fragen zugeordnet werden
    %dann evtl. nur die erste verwenden oder etwas anderes tun (Hinweis mehrere Fragen, auflisten mit Link)
				%TABLE FOR QUESTION DETAILS
				\vspace*{0.5cm}
                \noindent\textbf{Frage\footnote{Detailliertere Informationen zur Frage finden sich unter
		              \url{https://metadata.fdz.dzhw.eu/\#!/de/questions/que-gra2009-ins2-7.1$}}}\\
				\begin{tabularx}{\hsize}{@{}lX}
					Fragenummer: &
					  Fragebogen des DZHW-Absolventenpanels 2009 - zweite Welle, Hauptbefragung (PAPI):
					  7.1
 \\
					%--
					Fragetext: & Sehen Sie für sich persönlich generell (weiteren) Bedarf zur Teilnahme an Weiterbildung und Qualifizierung?; Wenn ja: Tragen Sie hier bitte die für Sie wichtigsten Themen bzw. Fachgebiete ein.\par  Thema \\
				\end{tabularx}
				%TABLE FOR QUESTION DETAILS
				\vspace*{0.5cm}
                \noindent\textbf{Frage\footnote{Detailliertere Informationen zur Frage finden sich unter
		              \url{https://metadata.fdz.dzhw.eu/\#!/de/questions/que-gra2009-ins3-80$}}}\\
				\begin{tabularx}{\hsize}{@{}lX}
					Fragenummer: &
					  Fragebogen des DZHW-Absolventenpanels 2009 - zweite Welle, Hauptbefragung (CAWI):
					  80
 \\
					%--
					Fragetext: & Wählen Sie bitte die für Sie wichtigsten Themen bzw. Fachgebiete aus \\
				\end{tabularx}





				%TABLE FOR THE NOMINAL / ORDINAL VALUES
        		\vspace*{0.5cm}
                \noindent\textbf{Häufigkeiten}

                \vspace*{-\baselineskip}
					%NUMERIC ELEMENTS NEED A HUGH SECOND COLOUMN AND A SMALL FIRST ONE
					\begin{filecontents}{\jobname-bfvt09b}
					\begin{longtable}{lXrrr}
					\toprule
					\textbf{Wert} & \textbf{Label} & \textbf{Häufigkeit} & \textbf{Prozent(gültig)} & \textbf{Prozent} \\
					\endhead
					\midrule
					\multicolumn{5}{l}{\textbf{Gültige Werte}}\\
						%DIFFERENT OBSERVATIONS <=20
								1 & \multicolumn{1}{X}{ingenieurwissenschaftliche Themen} & %74 &
								  \num{74} &
								%--
								  \num[round-mode=places,round-precision=2]{2.55} &
								  \num[round-mode=places,round-precision=2]{0.71} \\
								2 & \multicolumn{1}{X}{naturwissenschaftliche Themen} & %141 &
								  \num{141} &
								%--
								  \num[round-mode=places,round-precision=2]{4.86} &
								  \num[round-mode=places,round-precision=2]{1.34} \\
								3 & \multicolumn{1}{X}{mathematische Gebiete/Statistik} & %95 &
								  \num{95} &
								%--
								  \num[round-mode=places,round-precision=2]{3.28} &
								  \num[round-mode=places,round-precision=2]{0.91} \\
								4 & \multicolumn{1}{X}{sozialwissenschaftliche Themen} & %124 &
								  \num{124} &
								%--
								  \num[round-mode=places,round-precision=2]{4.28} &
								  \num[round-mode=places,round-precision=2]{1.18} \\
								5 & \multicolumn{1}{X}{geisteswissenschtliche Themen} & %90 &
								  \num{90} &
								%--
								  \num[round-mode=places,round-precision=2]{3.1} &
								  \num[round-mode=places,round-precision=2]{0.86} \\
								6 & \multicolumn{1}{X}{pädagogische/psychologische Themen} & %244 &
								  \num{244} &
								%--
								  \num[round-mode=places,round-precision=2]{8.41} &
								  \num[round-mode=places,round-precision=2]{2.33} \\
								7 & \multicolumn{1}{X}{medizinische Spezialgebiete} & %112 &
								  \num{112} &
								%--
								  \num[round-mode=places,round-precision=2]{3.86} &
								  \num[round-mode=places,round-precision=2]{1.07} \\
								8 & \multicolumn{1}{X}{informationstechnisches Spezialwissen} & %80 &
								  \num{80} &
								%--
								  \num[round-mode=places,round-precision=2]{2.76} &
								  \num[round-mode=places,round-precision=2]{0.76} \\
								9 & \multicolumn{1}{X}{Managementwissen} & %300 &
								  \num{300} &
								%--
								  \num[round-mode=places,round-precision=2]{10.34} &
								  \num[round-mode=places,round-precision=2]{2.86} \\
								10 & \multicolumn{1}{X}{Wirtschaftskenntnisse} & %218 &
								  \num{218} &
								%--
								  \num[round-mode=places,round-precision=2]{7.52} &
								  \num[round-mode=places,round-precision=2]{2.08} \\
							... & ... & ... & ... & ... \\
								15 & \multicolumn{1}{X}{EDV-Anwendungen} & %212 &
								  \num{212} &
								%--
								  \num[round-mode=places,round-precision=2]{7.31} &
								  \num[round-mode=places,round-precision=2]{2.02} \\

								16 & \multicolumn{1}{X}{Fremdsprachen} & %180 &
								  \num{180} &
								%--
								  \num[round-mode=places,round-precision=2]{6.21} &
								  \num[round-mode=places,round-precision=2]{1.72} \\

								17 & \multicolumn{1}{X}{Mitarbeiterführung/Personalentwicklung} & %261 &
								  \num{261} &
								%--
								  \num[round-mode=places,round-precision=2]{9} &
								  \num[round-mode=places,round-precision=2]{2.49} \\

								18 & \multicolumn{1}{X}{Kommunikations-/Interaktionstraining} & %295 &
								  \num{295} &
								%--
								  \num[round-mode=places,round-precision=2]{10.17} &
								  \num[round-mode=places,round-precision=2]{2.81} \\

								19 & \multicolumn{1}{X}{internationale Beziehungen, Kulturkenntnisse, Landeskunde} & %30 &
								  \num{30} &
								%--
								  \num[round-mode=places,round-precision=2]{1.03} &
								  \num[round-mode=places,round-precision=2]{0.29} \\

								20 & \multicolumn{1}{X}{ökologische Themen} & %20 &
								  \num{20} &
								%--
								  \num[round-mode=places,round-precision=2]{0.69} &
								  \num[round-mode=places,round-precision=2]{0.19} \\

								21 & \multicolumn{1}{X}{berufsethische Themen} & %22 &
								  \num{22} &
								%--
								  \num[round-mode=places,round-precision=2]{0.76} &
								  \num[round-mode=places,round-precision=2]{0.21} \\

								22 & \multicolumn{1}{X}{Existenzgründung} & %32 &
								  \num{32} &
								%--
								  \num[round-mode=places,round-precision=2]{1.1} &
								  \num[round-mode=places,round-precision=2]{0.3} \\

								23 & \multicolumn{1}{X}{betriebliches Gesundheitswesen, Arbeitssicherheit} & %30 &
								  \num{30} &
								%--
								  \num[round-mode=places,round-precision=2]{1.03} &
								  \num[round-mode=places,round-precision=2]{0.29} \\

								24 & \multicolumn{1}{X}{Sonstige} & %87 &
								  \num{87} &
								%--
								  \num[round-mode=places,round-precision=2]{3} &
								  \num[round-mode=places,round-precision=2]{0.83} \\

					\midrule
					\multicolumn{2}{l}{Summe (gültig)} &
					  \textbf{\num{2900}} &
					\textbf{\num{100}} &
					  \textbf{\num[round-mode=places,round-precision=2]{27.63}} \\
					%--
					\multicolumn{5}{l}{\textbf{Fehlende Werte}}\\
							-998 &
							keine Angabe &
							  \num{1309} &
							 - &
							  \num[round-mode=places,round-precision=2]{12.47} \\
							-995 &
							keine Teilnahme (Panel) &
							  \num{5739} &
							 - &
							  \num[round-mode=places,round-precision=2]{54.69} \\
							-988 &
							trifft nicht zu &
							  \num{546} &
							 - &
							  \num[round-mode=places,round-precision=2]{5.2} \\
					\midrule
					\multicolumn{2}{l}{\textbf{Summe (gesamt)}} &
				      \textbf{\num{10494}} &
				    \textbf{-} &
				    \textbf{\num{100}} \\
					\bottomrule
					\end{longtable}
					\end{filecontents}
					\LTXtable{\textwidth}{\jobname-bfvt09b}
				\label{tableValues:bfvt09b}
				\vspace*{-\baselineskip}
                    \begin{noten}
                	    \note{} Deskriptive Maßzahlen:
                	    Anzahl unterschiedlicher Beobachtungen: 24%
                	    ; 
                	      Modus ($h$): 9
                     \end{noten}

