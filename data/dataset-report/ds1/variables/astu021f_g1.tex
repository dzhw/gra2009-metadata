%EVERY VARIABLE HAS IT'S OWN PAGE

    \setcounter{footnote}{0}

    %omit vertical space
    \vspace*{-1.8cm}
	\section{astu021f\_g1 (1. Abschluss: Abschlussart)}
	\label{section:astu021f_g1}



	% TABLE FOR VARIABLE DETAILS
  % '#' has to be escaped
    \vspace*{0.5cm}
    \noindent\textbf{Eigenschaften\footnote{Detailliertere Informationen zur Variable finden sich unter
		\url{https://metadata.fdz.dzhw.eu/\#!/de/variables/var-gra2009-ds1-astu021f_g1$}}}\\
	\begin{tabularx}{\hsize}{@{}lX}
	Datentyp: & numerisch \\
	Skalenniveau: & nominal \\
	Zugangswege: &
	  download-cuf, 
	  download-suf, 
	  remote-desktop-suf, 
	  onsite-suf
 \\
    \end{tabularx}



    %TABLE FOR QUESTION DETAILS
    %This has to be tested and has to be improved
    %rausfinden, ob einer Variable mehrere Fragen zugeordnet werden
    %dann evtl. nur die erste verwenden oder etwas anderes tun (Hinweis mehrere Fragen, auflisten mit Link)
				%TABLE FOR QUESTION DETAILS
				\vspace*{0.5cm}
                \noindent\textbf{Frage\footnote{Detailliertere Informationen zur Frage finden sich unter
		              \url{https://metadata.fdz.dzhw.eu/\#!/de/questions/que-gra2009-ins1-1.2$}}}\\
				\begin{tabularx}{\hsize}{@{}lX}
					Fragenummer: &
					  Fragebogen des DZHW-Absolventenpanels 2009 - erste Welle:
					  1.2
 \\
					%--
					Fragetext: & Welche Studienabschlüsse haben Sie erlangt?\par  1. Abschluss\par  Angestrebte Abschlussart (z.B. Diplom, Bachelor, Staatsexamen) \\
				\end{tabularx}





				%TABLE FOR THE NOMINAL / ORDINAL VALUES
        		\vspace*{0.5cm}
                \noindent\textbf{Häufigkeiten}

                \vspace*{-\baselineskip}
					%NUMERIC ELEMENTS NEED A HUGH SECOND COLOUMN AND A SMALL FIRST ONE
					\begin{filecontents}{\jobname-astu021f_g1}
					\begin{longtable}{lXrrr}
					\toprule
					\textbf{Wert} & \textbf{Label} & \textbf{Häufigkeit} & \textbf{Prozent(gültig)} & \textbf{Prozent} \\
					\endhead
					\midrule
					\multicolumn{5}{l}{\textbf{Gültige Werte}}\\
						%DIFFERENT OBSERVATIONS <=20

					1 &
				% TODO try size/length gt 0; take over for other passages
					\multicolumn{1}{X}{ Diplom FH   } &


					%1350 &
					  \num{1350} &
					%--
					  \num[round-mode=places,round-precision=2]{12.86} &
					    \num[round-mode=places,round-precision=2]{12.86} \\
							%????

					2 &
				% TODO try size/length gt 0; take over for other passages
					\multicolumn{1}{X}{ Diplom Uni   } &


					%2087 &
					  \num{2087} &
					%--
					  \num[round-mode=places,round-precision=2]{19.89} &
					    \num[round-mode=places,round-precision=2]{19.89} \\
							%????

					3 &
				% TODO try size/length gt 0; take over for other passages
					\multicolumn{1}{X}{ Magister   } &


					%485 &
					  \num{485} &
					%--
					  \num[round-mode=places,round-precision=2]{4.62} &
					    \num[round-mode=places,round-precision=2]{4.62} \\
							%????

					4 &
				% TODO try size/length gt 0; take over for other passages
					\multicolumn{1}{X}{ Bachelor FH   } &


					%1960 &
					  \num{1960} &
					%--
					  \num[round-mode=places,round-precision=2]{18.68} &
					    \num[round-mode=places,round-precision=2]{18.68} \\
							%????

					5 &
				% TODO try size/length gt 0; take over for other passages
					\multicolumn{1}{X}{ Bachelor Uni   } &


					%2923 &
					  \num{2923} &
					%--
					  \num[round-mode=places,round-precision=2]{27.85} &
					    \num[round-mode=places,round-precision=2]{27.85} \\
							%????

					8 &
				% TODO try size/length gt 0; take over for other passages
					\multicolumn{1}{X}{ Staatsexamen (ohne LA)   } &


					%743 &
					  \num{743} &
					%--
					  \num[round-mode=places,round-precision=2]{7.08} &
					    \num[round-mode=places,round-precision=2]{7.08} \\
							%????

					9 &
				% TODO try size/length gt 0; take over for other passages
					\multicolumn{1}{X}{ LA Grund-/Hauptschule   } &


					%324 &
					  \num{324} &
					%--
					  \num[round-mode=places,round-precision=2]{3.09} &
					    \num[round-mode=places,round-precision=2]{3.09} \\
							%????

					10 &
				% TODO try size/length gt 0; take over for other passages
					\multicolumn{1}{X}{ LA Realschule   } &


					%189 &
					  \num{189} &
					%--
					  \num[round-mode=places,round-precision=2]{1.8} &
					    \num[round-mode=places,round-precision=2]{1.8} \\
							%????

					11 &
				% TODO try size/length gt 0; take over for other passages
					\multicolumn{1}{X}{ LA Gymnasium   } &


					%296 &
					  \num{296} &
					%--
					  \num[round-mode=places,round-precision=2]{2.82} &
					    \num[round-mode=places,round-precision=2]{2.82} \\
							%????

					12 &
				% TODO try size/length gt 0; take over for other passages
					\multicolumn{1}{X}{ LA Berufsschule   } &


					%63 &
					  \num{63} &
					%--
					  \num[round-mode=places,round-precision=2]{0.6} &
					    \num[round-mode=places,round-precision=2]{0.6} \\
							%????

					13 &
				% TODO try size/length gt 0; take over for other passages
					\multicolumn{1}{X}{ LA Sonderschule   } &


					%65 &
					  \num{65} &
					%--
					  \num[round-mode=places,round-precision=2]{0.62} &
					    \num[round-mode=places,round-precision=2]{0.62} \\
							%????

					16 &
				% TODO try size/length gt 0; take over for other passages
					\multicolumn{1}{X}{ kirchl. Abschluss   } &


					%8 &
					  \num{8} &
					%--
					  \num[round-mode=places,round-precision=2]{0.08} &
					    \num[round-mode=places,round-precision=2]{0.08} \\
							%????

					17 &
				% TODO try size/length gt 0; take over for other passages
					\multicolumn{1}{X}{ künstler. Abschluss   } &


					%1 &
					  \num{1} &
					%--
					  \num[round-mode=places,round-precision=2]{0.01} &
					    \num[round-mode=places,round-precision=2]{0.01} \\
							%????
						%DIFFERENT OBSERVATIONS >20
					\midrule
					\multicolumn{2}{l}{Summe (gültig)} &
					  \textbf{\num{10494}} &
					\textbf{\num{100}} &
					  \textbf{\num[round-mode=places,round-precision=2]{100}} \\
					%--
					\multicolumn{5}{l}{\textbf{Fehlende Werte}}\\
						& & 0 & 0 & 0 \\
					\midrule
					\multicolumn{2}{l}{\textbf{Summe (gesamt)}} &
				      \textbf{\num{10494}} &
				    \textbf{-} &
				    \textbf{\num{100}} \\
					\bottomrule
					\end{longtable}
					\end{filecontents}
					\LTXtable{\textwidth}{\jobname-astu021f_g1}
				\label{tableValues:astu021f_g1}
				\vspace*{-\baselineskip}
                    \begin{noten}
                	    \note{} Deskriptive Maßzahlen:
                	    Anzahl unterschiedlicher Beobachtungen: 13%
                	    ; 
                	      Modus ($h$): 5
                     \end{noten}

