%EVERY VARIABLE HAS IT'S OWN PAGE

    \setcounter{footnote}{0}

    %omit vertical space
    \vspace*{-1.8cm}
	\section{bfvt09d (Bedarf Weiterbildung: Inhalt 4)}
	\label{section:bfvt09d}



	% TABLE FOR VARIABLE DETAILS
  % '#' has to be escaped
    \vspace*{0.5cm}
    \noindent\textbf{Eigenschaften\footnote{Detailliertere Informationen zur Variable finden sich unter
		\url{https://metadata.fdz.dzhw.eu/\#!/de/variables/var-gra2009-ds1-bfvt09d$}}}\\
	\begin{tabularx}{\hsize}{@{}lX}
	Datentyp: & numerisch \\
	Skalenniveau: & nominal \\
	Zugangswege: &
	  download-cuf, 
	  download-suf, 
	  remote-desktop-suf, 
	  onsite-suf
 \\
    \end{tabularx}



    %TABLE FOR QUESTION DETAILS
    %This has to be tested and has to be improved
    %rausfinden, ob einer Variable mehrere Fragen zugeordnet werden
    %dann evtl. nur die erste verwenden oder etwas anderes tun (Hinweis mehrere Fragen, auflisten mit Link)
				%TABLE FOR QUESTION DETAILS
				\vspace*{0.5cm}
                \noindent\textbf{Frage\footnote{Detailliertere Informationen zur Frage finden sich unter
		              \url{https://metadata.fdz.dzhw.eu/\#!/de/questions/que-gra2009-ins2-7.1$}}}\\
				\begin{tabularx}{\hsize}{@{}lX}
					Fragenummer: &
					  Fragebogen des DZHW-Absolventenpanels 2009 - zweite Welle, Hauptbefragung (PAPI):
					  7.1
 \\
					%--
					Fragetext: & Sehen Sie für sich persönlich generell (weiteren) Bedarf zur Teilnahme an Weiterbildung und Qualifizierung?; Wenn ja: Tragen Sie hier bitte die für Sie wichtigsten Themen bzw. Fachgebiete ein.\par  Thema \\
				\end{tabularx}
				%TABLE FOR QUESTION DETAILS
				\vspace*{0.5cm}
                \noindent\textbf{Frage\footnote{Detailliertere Informationen zur Frage finden sich unter
		              \url{https://metadata.fdz.dzhw.eu/\#!/de/questions/que-gra2009-ins3-80$}}}\\
				\begin{tabularx}{\hsize}{@{}lX}
					Fragenummer: &
					  Fragebogen des DZHW-Absolventenpanels 2009 - zweite Welle, Hauptbefragung (CAWI):
					  80
 \\
					%--
					Fragetext: & Wählen Sie bitte die für Sie wichtigsten Themen bzw. Fachgebiete aus \\
				\end{tabularx}





				%TABLE FOR THE NOMINAL / ORDINAL VALUES
        		\vspace*{0.5cm}
                \noindent\textbf{Häufigkeiten}

                \vspace*{-\baselineskip}
					%NUMERIC ELEMENTS NEED A HUGH SECOND COLOUMN AND A SMALL FIRST ONE
					\begin{filecontents}{\jobname-bfvt09d}
					\begin{longtable}{lXrrr}
					\toprule
					\textbf{Wert} & \textbf{Label} & \textbf{Häufigkeit} & \textbf{Prozent(gültig)} & \textbf{Prozent} \\
					\endhead
					\midrule
					\multicolumn{5}{l}{\textbf{Gültige Werte}}\\
						%DIFFERENT OBSERVATIONS <=20
								1 & \multicolumn{1}{X}{ingenieurwissenschaftliche Themen} & %13 &
								  \num{13} &
								%--
								  \num[round-mode=places,round-precision=2]{1.07} &
								  \num[round-mode=places,round-precision=2]{0.12} \\
								2 & \multicolumn{1}{X}{naturwissenschaftliche Themen} & %12 &
								  \num{12} &
								%--
								  \num[round-mode=places,round-precision=2]{0.99} &
								  \num[round-mode=places,round-precision=2]{0.11} \\
								3 & \multicolumn{1}{X}{mathematische Gebiete/Statistik} & %15 &
								  \num{15} &
								%--
								  \num[round-mode=places,round-precision=2]{1.24} &
								  \num[round-mode=places,round-precision=2]{0.14} \\
								4 & \multicolumn{1}{X}{sozialwissenschaftliche Themen} & %25 &
								  \num{25} &
								%--
								  \num[round-mode=places,round-precision=2]{2.06} &
								  \num[round-mode=places,round-precision=2]{0.24} \\
								5 & \multicolumn{1}{X}{geisteswissenschtliche Themen} & %20 &
								  \num{20} &
								%--
								  \num[round-mode=places,round-precision=2]{1.65} &
								  \num[round-mode=places,round-precision=2]{0.19} \\
								6 & \multicolumn{1}{X}{pädagogische/psychologische Themen} & %39 &
								  \num{39} &
								%--
								  \num[round-mode=places,round-precision=2]{3.22} &
								  \num[round-mode=places,round-precision=2]{0.37} \\
								7 & \multicolumn{1}{X}{medizinische Spezialgebiete} & %19 &
								  \num{19} &
								%--
								  \num[round-mode=places,round-precision=2]{1.57} &
								  \num[round-mode=places,round-precision=2]{0.18} \\
								8 & \multicolumn{1}{X}{informationstechnisches Spezialwissen} & %26 &
								  \num{26} &
								%--
								  \num[round-mode=places,round-precision=2]{2.14} &
								  \num[round-mode=places,round-precision=2]{0.25} \\
								9 & \multicolumn{1}{X}{Managementwissen} & %63 &
								  \num{63} &
								%--
								  \num[round-mode=places,round-precision=2]{5.19} &
								  \num[round-mode=places,round-precision=2]{0.6} \\
								10 & \multicolumn{1}{X}{Wirtschaftskenntnisse} & %60 &
								  \num{60} &
								%--
								  \num[round-mode=places,round-precision=2]{4.95} &
								  \num[round-mode=places,round-precision=2]{0.57} \\
							... & ... & ... & ... & ... \\
								15 & \multicolumn{1}{X}{EDV-Anwendungen} & %102 &
								  \num{102} &
								%--
								  \num[round-mode=places,round-precision=2]{8.41} &
								  \num[round-mode=places,round-precision=2]{0.97} \\

								16 & \multicolumn{1}{X}{Fremdsprachen} & %125 &
								  \num{125} &
								%--
								  \num[round-mode=places,round-precision=2]{10.31} &
								  \num[round-mode=places,round-precision=2]{1.19} \\

								17 & \multicolumn{1}{X}{Mitarbeiterführung/Personalentwicklung} & %156 &
								  \num{156} &
								%--
								  \num[round-mode=places,round-precision=2]{12.86} &
								  \num[round-mode=places,round-precision=2]{1.49} \\

								18 & \multicolumn{1}{X}{Kommunikations-/Interaktionstraining} & %197 &
								  \num{197} &
								%--
								  \num[round-mode=places,round-precision=2]{16.24} &
								  \num[round-mode=places,round-precision=2]{1.88} \\

								19 & \multicolumn{1}{X}{internationale Beziehungen, Kulturkenntnisse, Landeskunde} & %43 &
								  \num{43} &
								%--
								  \num[round-mode=places,round-precision=2]{3.54} &
								  \num[round-mode=places,round-precision=2]{0.41} \\

								20 & \multicolumn{1}{X}{ökologische Themen} & %27 &
								  \num{27} &
								%--
								  \num[round-mode=places,round-precision=2]{2.23} &
								  \num[round-mode=places,round-precision=2]{0.26} \\

								21 & \multicolumn{1}{X}{berufsethische Themen} & %32 &
								  \num{32} &
								%--
								  \num[round-mode=places,round-precision=2]{2.64} &
								  \num[round-mode=places,round-precision=2]{0.3} \\

								22 & \multicolumn{1}{X}{Existenzgründung} & %27 &
								  \num{27} &
								%--
								  \num[round-mode=places,round-precision=2]{2.23} &
								  \num[round-mode=places,round-precision=2]{0.26} \\

								23 & \multicolumn{1}{X}{betriebliches Gesundheitswesen, Arbeitssicherheit} & %30 &
								  \num{30} &
								%--
								  \num[round-mode=places,round-precision=2]{2.47} &
								  \num[round-mode=places,round-precision=2]{0.29} \\

								24 & \multicolumn{1}{X}{Sonstige} & %62 &
								  \num{62} &
								%--
								  \num[round-mode=places,round-precision=2]{5.11} &
								  \num[round-mode=places,round-precision=2]{0.59} \\

					\midrule
					\multicolumn{2}{l}{Summe (gültig)} &
					  \textbf{\num{1213}} &
					\textbf{\num{100}} &
					  \textbf{\num[round-mode=places,round-precision=2]{11.56}} \\
					%--
					\multicolumn{5}{l}{\textbf{Fehlende Werte}}\\
							-998 &
							keine Angabe &
							  \num{2996} &
							 - &
							  \num[round-mode=places,round-precision=2]{28.55} \\
							-995 &
							keine Teilnahme (Panel) &
							  \num{5739} &
							 - &
							  \num[round-mode=places,round-precision=2]{54.69} \\
							-988 &
							trifft nicht zu &
							  \num{546} &
							 - &
							  \num[round-mode=places,round-precision=2]{5.2} \\
					\midrule
					\multicolumn{2}{l}{\textbf{Summe (gesamt)}} &
				      \textbf{\num{10494}} &
				    \textbf{-} &
				    \textbf{\num{100}} \\
					\bottomrule
					\end{longtable}
					\end{filecontents}
					\LTXtable{\textwidth}{\jobname-bfvt09d}
				\label{tableValues:bfvt09d}
				\vspace*{-\baselineskip}
                    \begin{noten}
                	    \note{} Deskriptive Maßzahlen:
                	    Anzahl unterschiedlicher Beobachtungen: 24%
                	    ; 
                	      Modus ($h$): 18
                     \end{noten}

