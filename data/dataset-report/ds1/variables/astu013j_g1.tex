%EVERY VARIABLE HAS IT'S OWN PAGE

    \setcounter{footnote}{0}

    %omit vertical space
    \vspace*{-1.8cm}
	\section{astu013j\_g1 (3. Studium: angestrebter Abschluss (2. Nebenfach))}
	\label{section:astu013j_g1}



	% TABLE FOR VARIABLE DETAILS
  % '#' has to be escaped
    \vspace*{0.5cm}
    \noindent\textbf{Eigenschaften\footnote{Detailliertere Informationen zur Variable finden sich unter
		\url{https://metadata.fdz.dzhw.eu/\#!/de/variables/var-gra2009-ds1-astu013j_g1$}}}\\
	\begin{tabularx}{\hsize}{@{}lX}
	Datentyp: & numerisch \\
	Skalenniveau: & nominal \\
	Zugangswege: &
	  download-cuf, 
	  download-suf, 
	  remote-desktop-suf, 
	  onsite-suf
 \\
    \end{tabularx}



    %TABLE FOR QUESTION DETAILS
    %This has to be tested and has to be improved
    %rausfinden, ob einer Variable mehrere Fragen zugeordnet werden
    %dann evtl. nur die erste verwenden oder etwas anderes tun (Hinweis mehrere Fragen, auflisten mit Link)
				%TABLE FOR QUESTION DETAILS
				\vspace*{0.5cm}
                \noindent\textbf{Frage\footnote{Detailliertere Informationen zur Frage finden sich unter
		              \url{https://metadata.fdz.dzhw.eu/\#!/de/questions/que-gra2009-ins1-1.1$}}}\\
				\begin{tabularx}{\hsize}{@{}lX}
					Fragenummer: &
					  Fragebogen des DZHW-Absolventenpanels 2009 - erste Welle:
					  1.1
 \\
					%--
					Fragetext: & Bitte tragen Sie in das folgende Tableau Ihren Studienverlauf ein.\par  Angestrebte Abschlussart (z.B. Diplom, Bachelor) \\
				\end{tabularx}





				%TABLE FOR THE NOMINAL / ORDINAL VALUES
        		\vspace*{0.5cm}
                \noindent\textbf{Häufigkeiten}

                \vspace*{-\baselineskip}
					%NUMERIC ELEMENTS NEED A HUGH SECOND COLOUMN AND A SMALL FIRST ONE
					\begin{filecontents}{\jobname-astu013j_g1}
					\begin{longtable}{lXrrr}
					\toprule
					\textbf{Wert} & \textbf{Label} & \textbf{Häufigkeit} & \textbf{Prozent(gültig)} & \textbf{Prozent} \\
					\endhead
					\midrule
					\multicolumn{5}{l}{\textbf{Gültige Werte}}\\
						%DIFFERENT OBSERVATIONS <=20

					3 &
				% TODO try size/length gt 0; take over for other passages
					\multicolumn{1}{X}{ Magister   } &


					%74 &
					  \num{74} &
					%--
					  \num[round-mode=places,round-precision=2]{64.35} &
					    \num[round-mode=places,round-precision=2]{0.71} \\
							%????

					4 &
				% TODO try size/length gt 0; take over for other passages
					\multicolumn{1}{X}{ Bachelor FH   } &


					%1 &
					  \num{1} &
					%--
					  \num[round-mode=places,round-precision=2]{0.87} &
					    \num[round-mode=places,round-precision=2]{0.01} \\
							%????

					5 &
				% TODO try size/length gt 0; take over for other passages
					\multicolumn{1}{X}{ Bachelor Uni   } &


					%6 &
					  \num{6} &
					%--
					  \num[round-mode=places,round-precision=2]{5.22} &
					    \num[round-mode=places,round-precision=2]{0.06} \\
							%????

					7 &
				% TODO try size/length gt 0; take over for other passages
					\multicolumn{1}{X}{ Master Uni   } &


					%4 &
					  \num{4} &
					%--
					  \num[round-mode=places,round-precision=2]{3.48} &
					    \num[round-mode=places,round-precision=2]{0.04} \\
							%????

					9 &
				% TODO try size/length gt 0; take over for other passages
					\multicolumn{1}{X}{ LA Grund-/Hauptschule   } &


					%7 &
					  \num{7} &
					%--
					  \num[round-mode=places,round-precision=2]{6.09} &
					    \num[round-mode=places,round-precision=2]{0.07} \\
							%????

					10 &
				% TODO try size/length gt 0; take over for other passages
					\multicolumn{1}{X}{ LA Realschule   } &


					%4 &
					  \num{4} &
					%--
					  \num[round-mode=places,round-precision=2]{3.48} &
					    \num[round-mode=places,round-precision=2]{0.04} \\
							%????

					11 &
				% TODO try size/length gt 0; take over for other passages
					\multicolumn{1}{X}{ LA Gymnasium   } &


					%6 &
					  \num{6} &
					%--
					  \num[round-mode=places,round-precision=2]{5.22} &
					    \num[round-mode=places,round-precision=2]{0.06} \\
							%????

					12 &
				% TODO try size/length gt 0; take over for other passages
					\multicolumn{1}{X}{ LA Berufsschule   } &


					%1 &
					  \num{1} &
					%--
					  \num[round-mode=places,round-precision=2]{0.87} &
					    \num[round-mode=places,round-precision=2]{0.01} \\
							%????

					13 &
				% TODO try size/length gt 0; take over for other passages
					\multicolumn{1}{X}{ LA Sonderschule   } &


					%1 &
					  \num{1} &
					%--
					  \num[round-mode=places,round-precision=2]{0.87} &
					    \num[round-mode=places,round-precision=2]{0.01} \\
							%????

					20 &
				% TODO try size/length gt 0; take over for other passages
					\multicolumn{1}{X}{ trad. Auslandsabschluss   } &


					%9 &
					  \num{9} &
					%--
					  \num[round-mode=places,round-precision=2]{7.83} &
					    \num[round-mode=places,round-precision=2]{0.09} \\
							%????

					27 &
				% TODO try size/length gt 0; take over for other passages
					\multicolumn{1}{X}{ Bachelor im Ausland   } &


					%2 &
					  \num{2} &
					%--
					  \num[round-mode=places,round-precision=2]{1.74} &
					    \num[round-mode=places,round-precision=2]{0.02} \\
							%????
						%DIFFERENT OBSERVATIONS >20
					\midrule
					\multicolumn{2}{l}{Summe (gültig)} &
					  \textbf{\num{115}} &
					\textbf{\num{100}} &
					  \textbf{\num[round-mode=places,round-precision=2]{1.1}} \\
					%--
					\multicolumn{5}{l}{\textbf{Fehlende Werte}}\\
							-998 &
							keine Angabe &
							  \num{10379} &
							 - &
							  \num[round-mode=places,round-precision=2]{98.9} \\
					\midrule
					\multicolumn{2}{l}{\textbf{Summe (gesamt)}} &
				      \textbf{\num{10494}} &
				    \textbf{-} &
				    \textbf{\num{100}} \\
					\bottomrule
					\end{longtable}
					\end{filecontents}
					\LTXtable{\textwidth}{\jobname-astu013j_g1}
				\label{tableValues:astu013j_g1}
				\vspace*{-\baselineskip}
                    \begin{noten}
                	    \note{} Deskriptive Maßzahlen:
                	    Anzahl unterschiedlicher Beobachtungen: 11%
                	    ; 
                	      Modus ($h$): 3
                     \end{noten}

