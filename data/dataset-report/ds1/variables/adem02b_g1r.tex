%EVERY VARIABLE HAS IT'S OWN PAGE

    \setcounter{footnote}{0}

    %omit vertical space
    \vspace*{-1.8cm}
	\section{adem02b\_g1r (Studienberechtigung: Anderer Bildungsweg, und zwar)}
	\label{section:adem02b_g1r}



	% TABLE FOR VARIABLE DETAILS
  % '#' has to be escaped
    \vspace*{0.5cm}
    \noindent\textbf{Eigenschaften\footnote{Detailliertere Informationen zur Variable finden sich unter
		\url{https://metadata.fdz.dzhw.eu/\#!/de/variables/var-gra2009-ds1-adem02b_g1r$}}}\\
	\begin{tabularx}{\hsize}{@{}lX}
	Datentyp: & numerisch \\
	Skalenniveau: & nominal \\
	Zugangswege: &
	  remote-desktop-suf, 
	  onsite-suf
 \\
    \end{tabularx}



    %TABLE FOR QUESTION DETAILS
    %This has to be tested and has to be improved
    %rausfinden, ob einer Variable mehrere Fragen zugeordnet werden
    %dann evtl. nur die erste verwenden oder etwas anderes tun (Hinweis mehrere Fragen, auflisten mit Link)
				%TABLE FOR QUESTION DETAILS
				\vspace*{0.5cm}
                \noindent\textbf{Frage\footnote{Detailliertere Informationen zur Frage finden sich unter
		              \url{https://metadata.fdz.dzhw.eu/\#!/de/questions/que-gra2009-ins1-6.2$}}}\\
				\begin{tabularx}{\hsize}{@{}lX}
					Fragenummer: &
					  Fragebogen des DZHW-Absolventenpanels 2009 - erste Welle:
					  6.2
 \\
					%--
					Fragetext: & Über welchen Bildungsweg haben Sie Ihre Studien- berechtigung erworben?\par  Anderer Bildungsweg, und zwar: \\
				\end{tabularx}





				%TABLE FOR THE NOMINAL / ORDINAL VALUES
        		\vspace*{0.5cm}
                \noindent\textbf{Häufigkeiten}

                \vspace*{-\baselineskip}
					%NUMERIC ELEMENTS NEED A HUGH SECOND COLOUMN AND A SMALL FIRST ONE
					\begin{filecontents}{\jobname-adem02b_g1r}
					\begin{longtable}{lXrrr}
					\toprule
					\textbf{Wert} & \textbf{Label} & \textbf{Häufigkeit} & \textbf{Prozent(gültig)} & \textbf{Prozent} \\
					\endhead
					\midrule
					\multicolumn{5}{l}{\textbf{Gültige Werte}}\\
						%DIFFERENT OBSERVATIONS <=20

					1 &
				% TODO try size/length gt 0; take over for other passages
					\multicolumn{1}{X}{ Vorbereitungskurs an einer FH (alte Länder)   } &


					%1 &
					  \num{1} &
					%--
					  \num[round-mode=places,round-precision=2]{0.27} &
					    \num[round-mode=places,round-precision=2]{0.01} \\
							%????

					2 &
				% TODO try size/length gt 0; take over for other passages
					\multicolumn{1}{X}{ Schule im Ausland   } &


					%196 &
					  \num{196} &
					%--
					  \num[round-mode=places,round-precision=2]{52.69} &
					    \num[round-mode=places,round-precision=2]{1.87} \\
							%????

					3 &
				% TODO try size/length gt 0; take over for other passages
					\multicolumn{1}{X}{ sonstige Schularten   } &


					%4 &
					  \num{4} &
					%--
					  \num[round-mode=places,round-precision=2]{1.08} &
					    \num[round-mode=places,round-precision=2]{0.04} \\
							%????

					5 &
				% TODO try size/length gt 0; take over for other passages
					\multicolumn{1}{X}{ Immaturen-/Begabtenprüfung   } &


					%12 &
					  \num{12} &
					%--
					  \num[round-mode=places,round-precision=2]{3.23} &
					    \num[round-mode=places,round-precision=2]{0.11} \\
							%????

					6 &
				% TODO try size/length gt 0; take over for other passages
					\multicolumn{1}{X}{ Ingenieur- oder Fachschule (DDR)   } &


					%3 &
					  \num{3} &
					%--
					  \num[round-mode=places,round-precision=2]{0.81} &
					    \num[round-mode=places,round-precision=2]{0.03} \\
							%????

					8 &
				% TODO try size/length gt 0; take over for other passages
					\multicolumn{1}{X}{ spezieller Lehrgang a.n.g.   } &


					%1 &
					  \num{1} &
					%--
					  \num[round-mode=places,round-precision=2]{0.27} &
					    \num[round-mode=places,round-precision=2]{0.01} \\
							%????

					10 &
				% TODO try size/length gt 0; take over for other passages
					\multicolumn{1}{X}{ Eignungsprüfung (Kunst-/Musikhochschule)   } &


					%3 &
					  \num{3} &
					%--
					  \num[round-mode=places,round-precision=2]{0.81} &
					    \num[round-mode=places,round-precision=2]{0.03} \\
							%????

					11 &
				% TODO try size/length gt 0; take over for other passages
					\multicolumn{1}{X}{ berufl. qual. Bewerber(in)   } &


					%70 &
					  \num{70} &
					%--
					  \num[round-mode=places,round-precision=2]{18.82} &
					    \num[round-mode=places,round-precision=2]{0.67} \\
							%????

					12 &
				% TODO try size/length gt 0; take over for other passages
					\multicolumn{1}{X}{ Studienkolleg, Feststellungsprüfung   } &


					%14 &
					  \num{14} &
					%--
					  \num[round-mode=places,round-precision=2]{3.76} &
					    \num[round-mode=places,round-precision=2]{0.13} \\
							%????

					13 &
				% TODO try size/length gt 0; take over for other passages
					\multicolumn{1}{X}{ Studium im Ausland   } &


					%43 &
					  \num{43} &
					%--
					  \num[round-mode=places,round-precision=2]{11.56} &
					    \num[round-mode=places,round-precision=2]{0.41} \\
							%????

					14 &
				% TODO try size/length gt 0; take over for other passages
					\multicolumn{1}{X}{ Berufsausbildung   } &


					%25 &
					  \num{25} &
					%--
					  \num[round-mode=places,round-precision=2]{6.72} &
					    \num[round-mode=places,round-precision=2]{0.24} \\
							%????
						%DIFFERENT OBSERVATIONS >20
					\midrule
					\multicolumn{2}{l}{Summe (gültig)} &
					  \textbf{\num{372}} &
					\textbf{\num{100}} &
					  \textbf{\num[round-mode=places,round-precision=2]{3.54}} \\
					%--
					\multicolumn{5}{l}{\textbf{Fehlende Werte}}\\
							-998 &
							keine Angabe &
							  \num{45} &
							 - &
							  \num[round-mode=places,round-precision=2]{0.43} \\
							-988 &
							trifft nicht zu &
							  \num{10077} &
							 - &
							  \num[round-mode=places,round-precision=2]{96.03} \\
					\midrule
					\multicolumn{2}{l}{\textbf{Summe (gesamt)}} &
				      \textbf{\num{10494}} &
				    \textbf{-} &
				    \textbf{\num{100}} \\
					\bottomrule
					\end{longtable}
					\end{filecontents}
					\LTXtable{\textwidth}{\jobname-adem02b_g1r}
				\label{tableValues:adem02b_g1r}
				\vspace*{-\baselineskip}
                    \begin{noten}
                	    \note{} Deskriptive Maßzahlen:
                	    Anzahl unterschiedlicher Beobachtungen: 11%
                	    ; 
                	      Modus ($h$): 2
                     \end{noten}

