%EVERY VARIABLE HAS IT'S OWN PAGE

    \setcounter{footnote}{0}

    %omit vertical space
    \vspace*{-1.8cm}
	\section{bocc10 (Form der Selbständigkeit)}
	\label{section:bocc10}



	% TABLE FOR VARIABLE DETAILS
  % '#' has to be escaped
    \vspace*{0.5cm}
    \noindent\textbf{Eigenschaften\footnote{Detailliertere Informationen zur Variable finden sich unter
		\url{https://metadata.fdz.dzhw.eu/\#!/de/variables/var-gra2009-ds1-bocc10$}}}\\
	\begin{tabularx}{\hsize}{@{}lX}
	Datentyp: & numerisch \\
	Skalenniveau: & nominal \\
	Zugangswege: &
	  download-cuf, 
	  download-suf, 
	  remote-desktop-suf, 
	  onsite-suf
 \\
    \end{tabularx}



    %TABLE FOR QUESTION DETAILS
    %This has to be tested and has to be improved
    %rausfinden, ob einer Variable mehrere Fragen zugeordnet werden
    %dann evtl. nur die erste verwenden oder etwas anderes tun (Hinweis mehrere Fragen, auflisten mit Link)
				%TABLE FOR QUESTION DETAILS
				\vspace*{0.5cm}
                \noindent\textbf{Frage\footnote{Detailliertere Informationen zur Frage finden sich unter
		              \url{https://metadata.fdz.dzhw.eu/\#!/de/questions/que-gra2009-ins2-4.8$}}}\\
				\begin{tabularx}{\hsize}{@{}lX}
					Fragenummer: &
					  Fragebogen des DZHW-Absolventenpanels 2009 - zweite Welle, Hauptbefragung (PAPI):
					  4.8
 \\
					%--
					Fragetext: & In welcher Form sind Sie als Selbständiger tätig bzw. beabsichtigen Sie tätig zu sein?\par  Als Freiberufler(in) durch Übernahme (z. B. einer Praxis) oder Eintritt (z. B. in eine Kanzlei)\par  Als Freiberufler(in) durch Gründung (z. B. einer Praxis)\par  Durch Übernahme einer Firma\par  Durch Gründung einer Firma\par  Als sonstige(r) Selbständige(r) (z. B. auf Basis von Werkverträgen oder Honoraren)\par  Das ist noch unklar \\
				\end{tabularx}
				%TABLE FOR QUESTION DETAILS
				\vspace*{0.5cm}
                \noindent\textbf{Frage\footnote{Detailliertere Informationen zur Frage finden sich unter
		              \url{https://metadata.fdz.dzhw.eu/\#!/de/questions/que-gra2009-ins3-23$}}}\\
				\begin{tabularx}{\hsize}{@{}lX}
					Fragenummer: &
					  Fragebogen des DZHW-Absolventenpanels 2009 - zweite Welle, Hauptbefragung (CAWI):
					  23
 \\
					%--
					Fragetext: & In welcher Form sind Sie als Selbständige(r) tätig bzw. beabsichtigen Sie tätig zu sein? \\
				\end{tabularx}





				%TABLE FOR THE NOMINAL / ORDINAL VALUES
        		\vspace*{0.5cm}
                \noindent\textbf{Häufigkeiten}

                \vspace*{-\baselineskip}
					%NUMERIC ELEMENTS NEED A HUGH SECOND COLOUMN AND A SMALL FIRST ONE
					\begin{filecontents}{\jobname-bocc10}
					\begin{longtable}{lXrrr}
					\toprule
					\textbf{Wert} & \textbf{Label} & \textbf{Häufigkeit} & \textbf{Prozent(gültig)} & \textbf{Prozent} \\
					\endhead
					\midrule
					\multicolumn{5}{l}{\textbf{Gültige Werte}}\\
						%DIFFERENT OBSERVATIONS <=20

					1 &
				% TODO try size/length gt 0; take over for other passages
					\multicolumn{1}{X}{ als Freiberufler(in) durch Übernahme oder Eintritt   } &


					%120 &
					  \num{120} &
					%--
					  \num[round-mode=places,round-precision=2]{15.5} &
					    \num[round-mode=places,round-precision=2]{1.14} \\
							%????

					2 &
				% TODO try size/length gt 0; take over for other passages
					\multicolumn{1}{X}{ als Freiberufler(in) durch Gründung   } &


					%178 &
					  \num{178} &
					%--
					  \num[round-mode=places,round-precision=2]{23} &
					    \num[round-mode=places,round-precision=2]{1.7} \\
							%????

					3 &
				% TODO try size/length gt 0; take over for other passages
					\multicolumn{1}{X}{ durch Übernahme einer Firma   } &


					%42 &
					  \num{42} &
					%--
					  \num[round-mode=places,round-precision=2]{5.43} &
					    \num[round-mode=places,round-precision=2]{0.4} \\
							%????

					4 &
				% TODO try size/length gt 0; take over for other passages
					\multicolumn{1}{X}{ durch Gründung einer Firma   } &


					%150 &
					  \num{150} &
					%--
					  \num[round-mode=places,round-precision=2]{19.38} &
					    \num[round-mode=places,round-precision=2]{1.43} \\
							%????

					5 &
				% TODO try size/length gt 0; take over for other passages
					\multicolumn{1}{X}{ als sonstige(r) Selbständige(r)   } &


					%201 &
					  \num{201} &
					%--
					  \num[round-mode=places,round-precision=2]{25.97} &
					    \num[round-mode=places,round-precision=2]{1.92} \\
							%????

					6 &
				% TODO try size/length gt 0; take over for other passages
					\multicolumn{1}{X}{ das ist noch unklar   } &


					%83 &
					  \num{83} &
					%--
					  \num[round-mode=places,round-precision=2]{10.72} &
					    \num[round-mode=places,round-precision=2]{0.79} \\
							%????
						%DIFFERENT OBSERVATIONS >20
					\midrule
					\multicolumn{2}{l}{Summe (gültig)} &
					  \textbf{\num{774}} &
					\textbf{\num{100}} &
					  \textbf{\num[round-mode=places,round-precision=2]{7.38}} \\
					%--
					\multicolumn{5}{l}{\textbf{Fehlende Werte}}\\
							-998 &
							keine Angabe &
							  \num{41} &
							 - &
							  \num[round-mode=places,round-precision=2]{0.39} \\
							-995 &
							keine Teilnahme (Panel) &
							  \num{5739} &
							 - &
							  \num[round-mode=places,round-precision=2]{54.69} \\
							-989 &
							filterbedingt fehlend &
							  \num{3940} &
							 - &
							  \num[round-mode=places,round-precision=2]{37.55} \\
					\midrule
					\multicolumn{2}{l}{\textbf{Summe (gesamt)}} &
				      \textbf{\num{10494}} &
				    \textbf{-} &
				    \textbf{\num{100}} \\
					\bottomrule
					\end{longtable}
					\end{filecontents}
					\LTXtable{\textwidth}{\jobname-bocc10}
				\label{tableValues:bocc10}
				\vspace*{-\baselineskip}
                    \begin{noten}
                	    \note{} Deskriptive Maßzahlen:
                	    Anzahl unterschiedlicher Beobachtungen: 6%
                	    ; 
                	      Modus ($h$): 5
                     \end{noten}

