%EVERY VARIABLE HAS IT'S OWN PAGE

    \setcounter{footnote}{0}

    %omit vertical space
    \vspace*{-1.8cm}
	\section{bfvt064d (eintägige berufl. Weiterbildung: Inhalt 2)}
	\label{section:bfvt064d}



	% TABLE FOR VARIABLE DETAILS
  % '#' has to be escaped
    \vspace*{0.5cm}
    \noindent\textbf{Eigenschaften\footnote{Detailliertere Informationen zur Variable finden sich unter
		\url{https://metadata.fdz.dzhw.eu/\#!/de/variables/var-gra2009-ds1-bfvt064d$}}}\\
	\begin{tabularx}{\hsize}{@{}lX}
	Datentyp: & numerisch \\
	Skalenniveau: & nominal \\
	Zugangswege: &
	  download-cuf, 
	  download-suf, 
	  remote-desktop-suf, 
	  onsite-suf
 \\
    \end{tabularx}



    %TABLE FOR QUESTION DETAILS
    %This has to be tested and has to be improved
    %rausfinden, ob einer Variable mehrere Fragen zugeordnet werden
    %dann evtl. nur die erste verwenden oder etwas anderes tun (Hinweis mehrere Fragen, auflisten mit Link)
				%TABLE FOR QUESTION DETAILS
				\vspace*{0.5cm}
                \noindent\textbf{Frage\footnote{Detailliertere Informationen zur Frage finden sich unter
		              \url{https://metadata.fdz.dzhw.eu/\#!/de/questions/que-gra2009-ins2-6.5$}}}\\
				\begin{tabularx}{\hsize}{@{}lX}
					Fragenummer: &
					  Fragebogen des DZHW-Absolventenpanels 2009 - zweite Welle, Hauptbefragung (PAPI):
					  6.5
 \\
					%--
					Fragetext: & Im Folgenden bitten wir Sie um Angaben zu beruflichen Fort- und Weiterbildungen der letzten 12 Monate. Bitte denken Sie dabei an alle Weiterbildungen, die Sie besucht haben und geben sie diese in der passenden Zeile an.\par  4. Fort- /oder Weiterbildung\par  Themen (Mehrfachnennung möglich)\par  Schlüssel s. Klappliste B) \\
				\end{tabularx}
				%TABLE FOR QUESTION DETAILS
				\vspace*{0.5cm}
                \noindent\textbf{Frage\footnote{Detailliertere Informationen zur Frage finden sich unter
		              \url{https://metadata.fdz.dzhw.eu/\#!/de/questions/que-gra2009-ins3-71$}}}\\
				\begin{tabularx}{\hsize}{@{}lX}
					Fragenummer: &
					  Fragebogen des DZHW-Absolventenpanels 2009 - zweite Welle, Hauptbefragung (CAWI):
					  71
 \\
					%--
					Fragetext: & Bitte tragen Sie hier die für Sie wichtigsten Themen bzw. Fachgebiete dieser Veranstaltungen ein. \\
				\end{tabularx}





				%TABLE FOR THE NOMINAL / ORDINAL VALUES
        		\vspace*{0.5cm}
                \noindent\textbf{Häufigkeiten}

                \vspace*{-\baselineskip}
					%NUMERIC ELEMENTS NEED A HUGH SECOND COLOUMN AND A SMALL FIRST ONE
					\begin{filecontents}{\jobname-bfvt064d}
					\begin{longtable}{lXrrr}
					\toprule
					\textbf{Wert} & \textbf{Label} & \textbf{Häufigkeit} & \textbf{Prozent(gültig)} & \textbf{Prozent} \\
					\endhead
					\midrule
					\multicolumn{5}{l}{\textbf{Gültige Werte}}\\
						%DIFFERENT OBSERVATIONS <=20
								1 & \multicolumn{1}{X}{ingenieurwissenschaftliche Themen} & %62 &
								  \num{62} &
								%--
								  \num[round-mode=places,round-precision=2]{4.92} &
								  \num[round-mode=places,round-precision=2]{0.59} \\
								2 & \multicolumn{1}{X}{naturwissenschaftliche Themen} & %61 &
								  \num{61} &
								%--
								  \num[round-mode=places,round-precision=2]{4.84} &
								  \num[round-mode=places,round-precision=2]{0.58} \\
								3 & \multicolumn{1}{X}{mathematische Gebiete/Statistik} & %31 &
								  \num{31} &
								%--
								  \num[round-mode=places,round-precision=2]{2.46} &
								  \num[round-mode=places,round-precision=2]{0.3} \\
								4 & \multicolumn{1}{X}{sozialwissenschaftliche Themen} & %57 &
								  \num{57} &
								%--
								  \num[round-mode=places,round-precision=2]{4.52} &
								  \num[round-mode=places,round-precision=2]{0.54} \\
								5 & \multicolumn{1}{X}{geisteswissenschtliche Themen} & %50 &
								  \num{50} &
								%--
								  \num[round-mode=places,round-precision=2]{3.97} &
								  \num[round-mode=places,round-precision=2]{0.48} \\
								6 & \multicolumn{1}{X}{pädagogische/psychologische Themen} & %156 &
								  \num{156} &
								%--
								  \num[round-mode=places,round-precision=2]{12.37} &
								  \num[round-mode=places,round-precision=2]{1.49} \\
								7 & \multicolumn{1}{X}{medizinische Spezialgebiete} & %60 &
								  \num{60} &
								%--
								  \num[round-mode=places,round-precision=2]{4.76} &
								  \num[round-mode=places,round-precision=2]{0.57} \\
								8 & \multicolumn{1}{X}{informationstechnisches Spezialwissen} & %55 &
								  \num{55} &
								%--
								  \num[round-mode=places,round-precision=2]{4.36} &
								  \num[round-mode=places,round-precision=2]{0.52} \\
								9 & \multicolumn{1}{X}{Managementwissen} & %80 &
								  \num{80} &
								%--
								  \num[round-mode=places,round-precision=2]{6.34} &
								  \num[round-mode=places,round-precision=2]{0.76} \\
								10 & \multicolumn{1}{X}{Wirtschaftskenntnisse} & %46 &
								  \num{46} &
								%--
								  \num[round-mode=places,round-precision=2]{3.65} &
								  \num[round-mode=places,round-precision=2]{0.44} \\
							... & ... & ... & ... & ... \\
								15 & \multicolumn{1}{X}{EDV-Anwendungen} & %112 &
								  \num{112} &
								%--
								  \num[round-mode=places,round-precision=2]{8.88} &
								  \num[round-mode=places,round-precision=2]{1.07} \\

								16 & \multicolumn{1}{X}{Fremdsprachen} & %34 &
								  \num{34} &
								%--
								  \num[round-mode=places,round-precision=2]{2.7} &
								  \num[round-mode=places,round-precision=2]{0.32} \\

								17 & \multicolumn{1}{X}{Mitarbeiterführung/Personalentwicklung} & %47 &
								  \num{47} &
								%--
								  \num[round-mode=places,round-precision=2]{3.73} &
								  \num[round-mode=places,round-precision=2]{0.45} \\

								18 & \multicolumn{1}{X}{Kommunikations-/Interaktionstraining} & %136 &
								  \num{136} &
								%--
								  \num[round-mode=places,round-precision=2]{10.79} &
								  \num[round-mode=places,round-precision=2]{1.3} \\

								19 & \multicolumn{1}{X}{internationale Beziehungen, Kulturkenntnisse, Landeskunde} & %14 &
								  \num{14} &
								%--
								  \num[round-mode=places,round-precision=2]{1.11} &
								  \num[round-mode=places,round-precision=2]{0.13} \\

								20 & \multicolumn{1}{X}{ökologische Themen} & %10 &
								  \num{10} &
								%--
								  \num[round-mode=places,round-precision=2]{0.79} &
								  \num[round-mode=places,round-precision=2]{0.1} \\

								21 & \multicolumn{1}{X}{berufsethische Themen} & %17 &
								  \num{17} &
								%--
								  \num[round-mode=places,round-precision=2]{1.35} &
								  \num[round-mode=places,round-precision=2]{0.16} \\

								22 & \multicolumn{1}{X}{Existenzgründung} & %7 &
								  \num{7} &
								%--
								  \num[round-mode=places,round-precision=2]{0.56} &
								  \num[round-mode=places,round-precision=2]{0.07} \\

								23 & \multicolumn{1}{X}{betriebliches Gesundheitswesen, Arbeitssicherheit} & %48 &
								  \num{48} &
								%--
								  \num[round-mode=places,round-precision=2]{3.81} &
								  \num[round-mode=places,round-precision=2]{0.46} \\

								24 & \multicolumn{1}{X}{Sonstige} & %26 &
								  \num{26} &
								%--
								  \num[round-mode=places,round-precision=2]{2.06} &
								  \num[round-mode=places,round-precision=2]{0.25} \\

					\midrule
					\multicolumn{2}{l}{Summe (gültig)} &
					  \textbf{\num{1261}} &
					\textbf{\num{100}} &
					  \textbf{\num[round-mode=places,round-precision=2]{12.02}} \\
					%--
					\multicolumn{5}{l}{\textbf{Fehlende Werte}}\\
							-998 &
							keine Angabe &
							  \num{3494} &
							 - &
							  \num[round-mode=places,round-precision=2]{33.3} \\
							-995 &
							keine Teilnahme (Panel) &
							  \num{5739} &
							 - &
							  \num[round-mode=places,round-precision=2]{54.69} \\
					\midrule
					\multicolumn{2}{l}{\textbf{Summe (gesamt)}} &
				      \textbf{\num{10494}} &
				    \textbf{-} &
				    \textbf{\num{100}} \\
					\bottomrule
					\end{longtable}
					\end{filecontents}
					\LTXtable{\textwidth}{\jobname-bfvt064d}
				\label{tableValues:bfvt064d}
				\vspace*{-\baselineskip}
                    \begin{noten}
                	    \note{} Deskriptive Maßzahlen:
                	    Anzahl unterschiedlicher Beobachtungen: 24%
                	    ; 
                	      Modus ($h$): 6
                     \end{noten}

