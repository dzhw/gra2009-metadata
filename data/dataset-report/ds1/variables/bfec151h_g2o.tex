%EVERY VARIABLE HAS IT'S OWN PAGE

    \setcounter{footnote}{0}

    %omit vertical space
    \vspace*{-1.8cm}
	\section{bfec151h\_g2o (1. weitere akad. Qualifikation: Hochschule (NUTS2))}
	\label{section:bfec151h_g2o}



	% TABLE FOR VARIABLE DETAILS
  % '#' has to be escaped
    \vspace*{0.5cm}
    \noindent\textbf{Eigenschaften\footnote{Detailliertere Informationen zur Variable finden sich unter
		\url{https://metadata.fdz.dzhw.eu/\#!/de/variables/var-gra2009-ds1-bfec151h_g2o$}}}\\
	\begin{tabularx}{\hsize}{@{}lX}
	Datentyp: & string \\
	Skalenniveau: & nominal \\
	Zugangswege: &
	  onsite-suf
 \\
    \end{tabularx}



    %TABLE FOR QUESTION DETAILS
    %This has to be tested and has to be improved
    %rausfinden, ob einer Variable mehrere Fragen zugeordnet werden
    %dann evtl. nur die erste verwenden oder etwas anderes tun (Hinweis mehrere Fragen, auflisten mit Link)
				%TABLE FOR QUESTION DETAILS
				\vspace*{0.5cm}
                \noindent\textbf{Frage\footnote{Detailliertere Informationen zur Frage finden sich unter
		              \url{https://metadata.fdz.dzhw.eu/\#!/de/questions/que-gra2009-ins2-5.2$}}}\\
				\begin{tabularx}{\hsize}{@{}lX}
					Fragenummer: &
					  Fragebogen des DZHW-Absolventenpanels 2009 - zweite Welle, Hauptbefragung (PAPI):
					  5.2
 \\
					%--
					Fragetext: & Bitte tragen Sie diese längerfristigen Studienangebote, die Sie nach Ihrem Studienabschluss aus dem Jahr 2008/2009 begonnen, weitergeführt oder abgeschlossen haben (auch abgebrochene oder unterbrochene), in das folgende Tableau ein! \\
				\end{tabularx}





				%TABLE FOR THE NOMINAL / ORDINAL VALUES
        		\vspace*{0.5cm}
                \noindent\textbf{Häufigkeiten}

                \vspace*{-\baselineskip}
					%STRING ELEMENTS NEEDS A HUGH FIRST COLOUMN AND A SMALL SECOND ONE
					\begin{filecontents}{\jobname-bfec151h_g2o}
					\begin{longtable}{Xlrrr}
					\toprule
					\textbf{Wert} & \textbf{Label} & \textbf{Häufigkeit} & \textbf{Prozent (gültig)} & \textbf{Prozent} \\
					\endhead
					\midrule
					\multicolumn{5}{l}{\textbf{Gültige Werte}}\\
						%DIFFERENT OBSERVATIONS <=20
								\multicolumn{1}{X}{DE11 Stuttgart} & - & \num{69} & \num[round-mode=places,round-precision=2]{4.35} & \num[round-mode=places,round-precision=2]{0.66} \\
								\multicolumn{1}{X}{DE12 Karlsruhe} & - & \num{61} & \num[round-mode=places,round-precision=2]{3.85} & \num[round-mode=places,round-precision=2]{0.58} \\
								\multicolumn{1}{X}{DE13 Freiburg} & - & \num{27} & \num[round-mode=places,round-precision=2]{1.7} & \num[round-mode=places,round-precision=2]{0.26} \\
								\multicolumn{1}{X}{DE14 Tübingen} & - & \num{33} & \num[round-mode=places,round-precision=2]{2.08} & \num[round-mode=places,round-precision=2]{0.31} \\
								\multicolumn{1}{X}{DE21 Oberbayern} & - & \num{100} & \num[round-mode=places,round-precision=2]{6.31} & \num[round-mode=places,round-precision=2]{0.95} \\
								\multicolumn{1}{X}{DE22 Niederbayern} & - & \num{14} & \num[round-mode=places,round-precision=2]{0.88} & \num[round-mode=places,round-precision=2]{0.13} \\
								\multicolumn{1}{X}{DE23 Oberpfalz} & - & \num{43} & \num[round-mode=places,round-precision=2]{2.71} & \num[round-mode=places,round-precision=2]{0.41} \\
								\multicolumn{1}{X}{DE24 Oberfranken} & - & \num{47} & \num[round-mode=places,round-precision=2]{2.97} & \num[round-mode=places,round-precision=2]{0.45} \\
								\multicolumn{1}{X}{DE25 Mittelfranken} & - & \num{35} & \num[round-mode=places,round-precision=2]{2.21} & \num[round-mode=places,round-precision=2]{0.33} \\
								\multicolumn{1}{X}{DE26 Unterfranken} & - & \num{4} & \num[round-mode=places,round-precision=2]{0.25} & \num[round-mode=places,round-precision=2]{0.04} \\
							... & ... & ... & ... & ... \\
								\multicolumn{1}{X}{DEB1 Koblenz} & - & \num{10} & \num[round-mode=places,round-precision=2]{0.63} & \num[round-mode=places,round-precision=2]{0.1} \\
								\multicolumn{1}{X}{DEB2 Trier} & - & \num{9} & \num[round-mode=places,round-precision=2]{0.57} & \num[round-mode=places,round-precision=2]{0.09} \\
								\multicolumn{1}{X}{DEB3 Rheinhessen-Pfalz} & - & \num{25} & \num[round-mode=places,round-precision=2]{1.58} & \num[round-mode=places,round-precision=2]{0.24} \\
								\multicolumn{1}{X}{DEC0 Saarland} & - & \num{6} & \num[round-mode=places,round-precision=2]{0.38} & \num[round-mode=places,round-precision=2]{0.06} \\
								\multicolumn{1}{X}{DED2 Dresden} & - & \num{37} & \num[round-mode=places,round-precision=2]{2.33} & \num[round-mode=places,round-precision=2]{0.35} \\
								\multicolumn{1}{X}{DED4 Chemnitz} & - & \num{42} & \num[round-mode=places,round-precision=2]{2.65} & \num[round-mode=places,round-precision=2]{0.4} \\
								\multicolumn{1}{X}{DED5 Leipzig} & - & \num{36} & \num[round-mode=places,round-precision=2]{2.27} & \num[round-mode=places,round-precision=2]{0.34} \\
								\multicolumn{1}{X}{DEE0 Sachsen-Anhalt} & - & \num{38} & \num[round-mode=places,round-precision=2]{2.4} & \num[round-mode=places,round-precision=2]{0.36} \\
								\multicolumn{1}{X}{DEF0 Schleswig-Holstein} & - & \num{45} & \num[round-mode=places,round-precision=2]{2.84} & \num[round-mode=places,round-precision=2]{0.43} \\
								\multicolumn{1}{X}{DEG0 Thüringen} & - & \num{108} & \num[round-mode=places,round-precision=2]{6.81} & \num[round-mode=places,round-precision=2]{1.03} \\
					\midrule
						\multicolumn{2}{l}{Summe (gültig)} & \textbf{\num{1585}} &
						\textbf{\num{100}} &
					    \textbf{\num[round-mode=places,round-precision=2]{15.1}} \\
					\multicolumn{5}{l}{\textbf{Fehlende Werte}}\\
							-966 & nicht bestimmbar & \num{126} & - & \num[round-mode=places,round-precision=2]{1.2} \\

							-968 & unplausibler Wert & \num{5} & - & \num[round-mode=places,round-precision=2]{0.05} \\

							-989 & filterbedingt fehlend & \num{2662} & - & \num[round-mode=places,round-precision=2]{25.37} \\

							-995 & keine Teilnahme (Panel) & \num{5739} & - & \num[round-mode=places,round-precision=2]{54.69} \\

							-998 & keine Angabe & \num{377} & - & \num[round-mode=places,round-precision=2]{3.59} \\

					\midrule
					\multicolumn{2}{l}{\textbf{Summe (gesamt)}} & \textbf{\num{10494}} & \textbf{-} & \textbf{\num{100}} \\
					\bottomrule
					\caption{Werte der Variable bfec151h\_g2o}
					\end{longtable}
					\end{filecontents}
					\LTXtable{\textwidth}{\jobname-bfec151h_g2o}

