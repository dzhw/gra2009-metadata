%EVERY VARIABLE HAS IT'S OWN PAGE

    \setcounter{footnote}{0}

    %omit vertical space
    \vspace*{-1.8cm}
	\section{aocc21\_g3 (Beruf: KldB 2010 (2-stellig))}
	\label{section:aocc21_g3}



	% TABLE FOR VARIABLE DETAILS
  % '#' has to be escaped
    \vspace*{0.5cm}
    \noindent\textbf{Eigenschaften\footnote{Detailliertere Informationen zur Variable finden sich unter
		\url{https://metadata.fdz.dzhw.eu/\#!/de/variables/var-gra2009-ds1-aocc21_g3$}}}\\
	\begin{tabularx}{\hsize}{@{}lX}
	Datentyp: & numerisch \\
	Skalenniveau: & nominal \\
	Zugangswege: &
	  download-cuf, 
	  download-suf, 
	  remote-desktop-suf, 
	  onsite-suf
 \\
    \end{tabularx}



    %TABLE FOR QUESTION DETAILS
    %This has to be tested and has to be improved
    %rausfinden, ob einer Variable mehrere Fragen zugeordnet werden
    %dann evtl. nur die erste verwenden oder etwas anderes tun (Hinweis mehrere Fragen, auflisten mit Link)
				%TABLE FOR QUESTION DETAILS
				\vspace*{0.5cm}
                \noindent\textbf{Frage\footnote{Detailliertere Informationen zur Frage finden sich unter
		              \url{https://metadata.fdz.dzhw.eu/\#!/de/questions/que-gra2009-ins1-5.2$}}}\\
				\begin{tabularx}{\hsize}{@{}lX}
					Fragenummer: &
					  Fragebogen des DZHW-Absolventenpanels 2009 - erste Welle:
					  5.2
 \\
					%--
					Fragetext: & Bitte geben Sie Ihre genaue Berufsbezeichnung, Ihren Aufgabenbereich sowie typische Arbeitsschwerpunkte Ihrer derzeitigen bzw. – falls Sie zurzeit nicht erwerbstätig sind – letzten (Haupt-)Tätigkeit an. \\
				\end{tabularx}





				%TABLE FOR THE NOMINAL / ORDINAL VALUES
        		\vspace*{0.5cm}
                \noindent\textbf{Häufigkeiten}

                \vspace*{-\baselineskip}
					%NUMERIC ELEMENTS NEED A HUGH SECOND COLOUMN AND A SMALL FIRST ONE
					\begin{filecontents}{\jobname-aocc21_g3}
					\begin{longtable}{lXrrr}
					\toprule
					\textbf{Wert} & \textbf{Label} & \textbf{Häufigkeit} & \textbf{Prozent(gültig)} & \textbf{Prozent} \\
					\endhead
					\midrule
					\multicolumn{5}{l}{\textbf{Gültige Werte}}\\
						%DIFFERENT OBSERVATIONS <=20
								1 & \multicolumn{1}{X}{Angehörige der regulären Streitkräfte} & %1 &
								  \num{1} &
								%--
								  \num[round-mode=places,round-precision=2]{0.01} &
								  \num[round-mode=places,round-precision=2]{0.01} \\
								11 & \multicolumn{1}{X}{Land-, Tier-, Forstwirtschaftsberufe} & %96 &
								  \num{96} &
								%--
								  \num[round-mode=places,round-precision=2]{1.27} &
								  \num[round-mode=places,round-precision=2]{0.91} \\
								12 & \multicolumn{1}{X}{Gartenbauberufe, Floristik} & %75 &
								  \num{75} &
								%--
								  \num[round-mode=places,round-precision=2]{0.99} &
								  \num[round-mode=places,round-precision=2]{0.71} \\
								21 & \multicolumn{1}{X}{Rohstoffgewinn,Glas-,Keramikverarbeitung} & %11 &
								  \num{11} &
								%--
								  \num[round-mode=places,round-precision=2]{0.15} &
								  \num[round-mode=places,round-precision=2]{0.1} \\
								22 & \multicolumn{1}{X}{Kunststoff- u. Holzherst.,-verarbeitung} & %14 &
								  \num{14} &
								%--
								  \num[round-mode=places,round-precision=2]{0.19} &
								  \num[round-mode=places,round-precision=2]{0.13} \\
								23 & \multicolumn{1}{X}{Papier-,Druckberufe, tech.Mediengestalt.} & %58 &
								  \num{58} &
								%--
								  \num[round-mode=places,round-precision=2]{0.77} &
								  \num[round-mode=places,round-precision=2]{0.55} \\
								24 & \multicolumn{1}{X}{Metallerzeugung,-bearbeitung, Metallbau} & %14 &
								  \num{14} &
								%--
								  \num[round-mode=places,round-precision=2]{0.19} &
								  \num[round-mode=places,round-precision=2]{0.13} \\
								25 & \multicolumn{1}{X}{Maschinen- und Fahrzeugtechnikberufe} & %114 &
								  \num{114} &
								%--
								  \num[round-mode=places,round-precision=2]{1.51} &
								  \num[round-mode=places,round-precision=2]{1.09} \\
								26 & \multicolumn{1}{X}{Mechatronik-, Energie- u. Elektroberufe} & %123 &
								  \num{123} &
								%--
								  \num[round-mode=places,round-precision=2]{1.63} &
								  \num[round-mode=places,round-precision=2]{1.17} \\
								27 & \multicolumn{1}{X}{Techn.Entwickl.Konstr.Produktionssteuer.} & %241 &
								  \num{241} &
								%--
								  \num[round-mode=places,round-precision=2]{3.19} &
								  \num[round-mode=places,round-precision=2]{2.3} \\
							... & ... & ... & ... & ... \\
								82 & \multicolumn{1}{X}{Nichtmed.Gesundheit,Körperpfl.,Medizint.} & %50 &
								  \num{50} &
								%--
								  \num[round-mode=places,round-precision=2]{0.66} &
								  \num[round-mode=places,round-precision=2]{0.48} \\

								83 & \multicolumn{1}{X}{Erziehung,soz.,hauswirt.Berufe,Theologie} & %654 &
								  \num{654} &
								%--
								  \num[round-mode=places,round-precision=2]{8.65} &
								  \num[round-mode=places,round-precision=2]{6.23} \\

								84 & \multicolumn{1}{X}{Lehrende und ausbildende Berufe} & %1610 &
								  \num{1610} &
								%--
								  \num[round-mode=places,round-precision=2]{21.3} &
								  \num[round-mode=places,round-precision=2]{15.34} \\

								91 & \multicolumn{1}{X}{Geistes-Gesellschafts-Wirtschaftswissen.} & %103 &
								  \num{103} &
								%--
								  \num[round-mode=places,round-precision=2]{1.36} &
								  \num[round-mode=places,round-precision=2]{0.98} \\

								92 & \multicolumn{1}{X}{Werbung,Marketing,kaufm,red.Medienberufe} & %387 &
								  \num{387} &
								%--
								  \num[round-mode=places,round-precision=2]{5.12} &
								  \num[round-mode=places,round-precision=2]{3.69} \\

								93 & \multicolumn{1}{X}{Produktdesign, Kunsthandwerk} & %56 &
								  \num{56} &
								%--
								  \num[round-mode=places,round-precision=2]{0.74} &
								  \num[round-mode=places,round-precision=2]{0.53} \\

								94 & \multicolumn{1}{X}{Darstellende, unterhaltende Berufe} & %55 &
								  \num{55} &
								%--
								  \num[round-mode=places,round-precision=2]{0.73} &
								  \num[round-mode=places,round-precision=2]{0.52} \\

								99996 & \multicolumn{1}{X}{Nachhilfelehrer/in} & %38 &
								  \num{38} &
								%--
								  \num[round-mode=places,round-precision=2]{0.5} &
								  \num[round-mode=places,round-precision=2]{0.36} \\

								99997 & \multicolumn{1}{X}{studentische Hilfskraft} & %265 &
								  \num{265} &
								%--
								  \num[round-mode=places,round-precision=2]{3.51} &
								  \num[round-mode=places,round-precision=2]{2.53} \\

								99998 & \multicolumn{1}{X}{wissenschaftliche Hilfskraft} & %201 &
								  \num{201} &
								%--
								  \num[round-mode=places,round-precision=2]{2.66} &
								  \num[round-mode=places,round-precision=2]{1.92} \\

					\midrule
					\multicolumn{2}{l}{Summe (gültig)} &
					  \textbf{\num{7557}} &
					\textbf{\num{100}} &
					  \textbf{\num[round-mode=places,round-precision=2]{72.01}} \\
					%--
					\multicolumn{5}{l}{\textbf{Fehlende Werte}}\\
							-998 &
							keine Angabe &
							  \num{814} &
							 - &
							  \num[round-mode=places,round-precision=2]{7.76} \\
							-989 &
							filterbedingt fehlend &
							  \num{2088} &
							 - &
							  \num[round-mode=places,round-precision=2]{19.9} \\
							-966 &
							nicht bestimmbar &
							  \num{35} &
							 - &
							  \num[round-mode=places,round-precision=2]{0.33} \\
					\midrule
					\multicolumn{2}{l}{\textbf{Summe (gesamt)}} &
				      \textbf{\num{10494}} &
				    \textbf{-} &
				    \textbf{\num{100}} \\
					\bottomrule
					\end{longtable}
					\end{filecontents}
					\LTXtable{\textwidth}{\jobname-aocc21_g3}
				\label{tableValues:aocc21_g3}
				\vspace*{-\baselineskip}
                    \begin{noten}
                	    \note{} Deskriptive Maßzahlen:
                	    Anzahl unterschiedlicher Beobachtungen: 40%
                	    ; 
                	      Modus ($h$): 84
                     \end{noten}

