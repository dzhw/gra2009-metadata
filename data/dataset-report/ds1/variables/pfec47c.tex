%EVERY VARIABLE HAS IT'S OWN PAGE

    \setcounter{footnote}{0}

    %omit vertical space
    \vspace*{-1.8cm}
	\section{pfec47c (Beurteilung Promotionsphase: flexible Zeiteinteilung)}
	\label{section:pfec47c}



	% TABLE FOR VARIABLE DETAILS
  % '#' has to be escaped
    \vspace*{0.5cm}
    \noindent\textbf{Eigenschaften\footnote{Detailliertere Informationen zur Variable finden sich unter
		\url{https://metadata.fdz.dzhw.eu/\#!/de/variables/var-gra2009-ds1-pfec47c$}}}\\
	\begin{tabularx}{\hsize}{@{}lX}
	Datentyp: & numerisch \\
	Skalenniveau: & ordinal \\
	Zugangswege: &
	  download-cuf, 
	  download-suf, 
	  remote-desktop-suf, 
	  onsite-suf
 \\
    \end{tabularx}



    %TABLE FOR QUESTION DETAILS
    %This has to be tested and has to be improved
    %rausfinden, ob einer Variable mehrere Fragen zugeordnet werden
    %dann evtl. nur die erste verwenden oder etwas anderes tun (Hinweis mehrere Fragen, auflisten mit Link)
				%TABLE FOR QUESTION DETAILS
				\vspace*{0.5cm}
                \noindent\textbf{Frage\footnote{Detailliertere Informationen zur Frage finden sich unter
		              \url{https://metadata.fdz.dzhw.eu/\#!/de/questions/que-gra2009-ins4-46$}}}\\
				\begin{tabularx}{\hsize}{@{}lX}
					Fragenummer: &
					  Fragebogen des DZHW-Absolventenpanels 2009 - zweite Welle, Vertiefungsbefragung Promotion:
					  46
 \\
					%--
					Fragetext: & Inwieweit treffen die folgenden Aussagen auf Ihre Promotionsphase zu?,Während meiner Promotionsphase…,trifft voll und ganz zu,trifft überhaupt nicht zu,...konnte ich mir die Zeit, die ich an meiner Promotion/Dissertation arbeitete, flexibel einteilen.,...kann ich mir die Zeit, die ich an meiner Promotion/Dissertation arbeite, flexibel einteilen. \\
				\end{tabularx}





				%TABLE FOR THE NOMINAL / ORDINAL VALUES
        		\vspace*{0.5cm}
                \noindent\textbf{Häufigkeiten}

                \vspace*{-\baselineskip}
					%NUMERIC ELEMENTS NEED A HUGH SECOND COLOUMN AND A SMALL FIRST ONE
					\begin{filecontents}{\jobname-pfec47c}
					\begin{longtable}{lXrrr}
					\toprule
					\textbf{Wert} & \textbf{Label} & \textbf{Häufigkeit} & \textbf{Prozent(gültig)} & \textbf{Prozent} \\
					\endhead
					\midrule
					\multicolumn{5}{l}{\textbf{Gültige Werte}}\\
						%DIFFERENT OBSERVATIONS <=20

					1 &
				% TODO try size/length gt 0; take over for other passages
					\multicolumn{1}{X}{ trifft voll und ganz zu   } &


					%242 &
					  \num{242} &
					%--
					  \num[round-mode=places,round-precision=2]{41.51} &
					    \num[round-mode=places,round-precision=2]{2.31} \\
							%????

					2 &
				% TODO try size/length gt 0; take over for other passages
					\multicolumn{1}{X}{ 2   } &


					%196 &
					  \num{196} &
					%--
					  \num[round-mode=places,round-precision=2]{33.62} &
					    \num[round-mode=places,round-precision=2]{1.87} \\
							%????

					3 &
				% TODO try size/length gt 0; take over for other passages
					\multicolumn{1}{X}{ 3   } &


					%86 &
					  \num{86} &
					%--
					  \num[round-mode=places,round-precision=2]{14.75} &
					    \num[round-mode=places,round-precision=2]{0.82} \\
							%????

					4 &
				% TODO try size/length gt 0; take over for other passages
					\multicolumn{1}{X}{ 4   } &


					%49 &
					  \num{49} &
					%--
					  \num[round-mode=places,round-precision=2]{8.4} &
					    \num[round-mode=places,round-precision=2]{0.47} \\
							%????

					5 &
				% TODO try size/length gt 0; take over for other passages
					\multicolumn{1}{X}{ trifft überhaupt nicht zu   } &


					%10 &
					  \num{10} &
					%--
					  \num[round-mode=places,round-precision=2]{1.72} &
					    \num[round-mode=places,round-precision=2]{0.1} \\
							%????
						%DIFFERENT OBSERVATIONS >20
					\midrule
					\multicolumn{2}{l}{Summe (gültig)} &
					  \textbf{\num{583}} &
					\textbf{\num{100}} &
					  \textbf{\num[round-mode=places,round-precision=2]{5.56}} \\
					%--
					\multicolumn{5}{l}{\textbf{Fehlende Werte}}\\
							-998 &
							keine Angabe &
							  \num{28} &
							 - &
							  \num[round-mode=places,round-precision=2]{0.27} \\
							-995 &
							keine Teilnahme (Panel) &
							  \num{9818} &
							 - &
							  \num[round-mode=places,round-precision=2]{93.56} \\
							-989 &
							filterbedingt fehlend &
							  \num{65} &
							 - &
							  \num[round-mode=places,round-precision=2]{0.62} \\
					\midrule
					\multicolumn{2}{l}{\textbf{Summe (gesamt)}} &
				      \textbf{\num{10494}} &
				    \textbf{-} &
				    \textbf{\num{100}} \\
					\bottomrule
					\end{longtable}
					\end{filecontents}
					\LTXtable{\textwidth}{\jobname-pfec47c}
				\label{tableValues:pfec47c}
				\vspace*{-\baselineskip}
                    \begin{noten}
                	    \note{} Deskriptive Maßzahlen:
                	    Anzahl unterschiedlicher Beobachtungen: 5%
                	    ; 
                	      Minimum ($min$): 1; 
                	      Maximum ($max$): 5; 
                	      Median ($\tilde{x}$): 2; 
                	      Modus ($h$): 1
                     \end{noten}

