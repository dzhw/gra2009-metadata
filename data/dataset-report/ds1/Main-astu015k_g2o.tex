%% LaTeX2e file `Main-astu015k_g2o'
%% generated by the `filecontents' environment
%% from source `Main' on 2019/02/28.
%%
     \begin{longtable}{Xlrrr}
     \toprule
     \textbf{Wert} & \textbf{Label} & \textbf{Häufigkeit} & \textbf{Prozent (gültig)} & \textbf{Prozent} \\
     \endhead
     \midrule
     \multicolumn{5}{l}{\textbf{Gültige Werte}}\\
      %DIFFERENT OBSERVATIONS <=20
        \multicolumn{1}{X}{DE11 Stuttgart} & - & \num{7} & \num[round-mode=places,round-precision=2]{7.87} & \num[round-mode=places,round-precision=2]{0.07} \\
        \multicolumn{1}{X}{DE12 Karlsruhe} & - & \num{3} & \num[round-mode=places,round-precision=2]{3.37} & \num[round-mode=places,round-precision=2]{0.03} \\
        \multicolumn{1}{X}{DE13 Freiburg} & - & \num{1} & \num[round-mode=places,round-precision=2]{1.12} & \num[round-mode=places,round-precision=2]{0.01} \\
        \multicolumn{1}{X}{DE14 Tübingen} & - & \num{5} & \num[round-mode=places,round-precision=2]{5.62} & \num[round-mode=places,round-precision=2]{0.05} \\
        \multicolumn{1}{X}{DE21 Oberbayern} & - & \num{14} & \num[round-mode=places,round-precision=2]{15.73} & \num[round-mode=places,round-precision=2]{0.13} \\
        \multicolumn{1}{X}{DE22 Niederbayern} & - & \num{3} & \num[round-mode=places,round-precision=2]{3.37} & \num[round-mode=places,round-precision=2]{0.03} \\
        \multicolumn{1}{X}{DE23 Oberpfalz} & - & \num{4} & \num[round-mode=places,round-precision=2]{4.49} & \num[round-mode=places,round-precision=2]{0.04} \\
        \multicolumn{1}{X}{DE24 Oberfranken} & - & \num{1} & \num[round-mode=places,round-precision=2]{1.12} & \num[round-mode=places,round-precision=2]{0.01} \\
        \multicolumn{1}{X}{DE25 Mittelfranken} & - & \num{3} & \num[round-mode=places,round-precision=2]{3.37} & \num[round-mode=places,round-precision=2]{0.03} \\
        \multicolumn{1}{X}{DE27 Schwaben} & - & \num{1} & \num[round-mode=places,round-precision=2]{1.12} & \num[round-mode=places,round-precision=2]{0.01} \\
       ... & ... & ... & ... & ... \\
        \multicolumn{1}{X}{DEA4 Detmold} & - & \num{3} & \num[round-mode=places,round-precision=2]{3.37} & \num[round-mode=places,round-precision=2]{0.03} \\
        \multicolumn{1}{X}{DEA5 Arnsberg} & - & \num{3} & \num[round-mode=places,round-precision=2]{3.37} & \num[round-mode=places,round-precision=2]{0.03} \\
        \multicolumn{1}{X}{DEB1 Koblenz} & - & \num{1} & \num[round-mode=places,round-precision=2]{1.12} & \num[round-mode=places,round-precision=2]{0.01} \\
        \multicolumn{1}{X}{DEB2 Trier} & - & \num{1} & \num[round-mode=places,round-precision=2]{1.12} & \num[round-mode=places,round-precision=2]{0.01} \\
        \multicolumn{1}{X}{DEB3 Rheinhessen-Pfalz} & - & \num{6} & \num[round-mode=places,round-precision=2]{6.74} & \num[round-mode=places,round-precision=2]{0.06} \\
        \multicolumn{1}{X}{DEC0 Saarland} & - & \num{1} & \num[round-mode=places,round-precision=2]{1.12} & \num[round-mode=places,round-precision=2]{0.01} \\
        \multicolumn{1}{X}{DED2 Dresden} & - & \num{1} & \num[round-mode=places,round-precision=2]{1.12} & \num[round-mode=places,round-precision=2]{0.01} \\
        \multicolumn{1}{X}{DED4 Chemnitz} & - & \num{2} & \num[round-mode=places,round-precision=2]{2.25} & \num[round-mode=places,round-precision=2]{0.02} \\
        \multicolumn{1}{X}{DEF0 Schleswig-Holstein} & - & \num{1} & \num[round-mode=places,round-precision=2]{1.12} & \num[round-mode=places,round-precision=2]{0.01} \\
        \multicolumn{1}{X}{DEG0 Thüringen} & - & \num{2} & \num[round-mode=places,round-precision=2]{2.25} & \num[round-mode=places,round-precision=2]{0.02} \\
     \midrule
      \multicolumn{2}{l}{Summe (gültig)} & \textbf{\num{89}} &
      \textbf{\num{100}} &
         \textbf{\num[round-mode=places,round-precision=2]{0.85}} \\
     \multicolumn{5}{l}{\textbf{Fehlende Werte}}\\
       -966 & nicht bestimmbar & \num{12} & - & \num[round-mode=places,round-precision=2]{0.11} \\

       -998 & keine Angabe & \num{10393} & - & \num[round-mode=places,round-precision=2]{99.04} \\

     \midrule
     \multicolumn{2}{l}{\textbf{Summe (gesamt)}} & \textbf{\num{10494}} & \textbf{-} & \textbf{\num{100}} \\
     \bottomrule
     \caption{Werte der Variable astu015k\_g2o}
     \end{longtable}
     
