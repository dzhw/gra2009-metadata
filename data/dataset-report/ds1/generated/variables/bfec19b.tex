%EVERY VARIABLE HAS IT'S OWN PAGE

    \setcounter{footnote}{0}

    %omit vertical space
    \vspace*{-1.8cm}
	\section{bfec19b (kurze Weiterbildung an Hochschule: Inhalt 2)}
	\label{section:bfec19b}



	%TABLE FOR VARIABLE DETAILS
    \vspace*{0.5cm}
    \noindent\textbf{Eigenschaften
	% '#' has to be escaped
	\footnote{Detailliertere Informationen zur Variable finden sich unter
		\url{https://metadata.fdz.dzhw.eu/\#!/de/variables/var-gra2009-ds1-bfec19b$}}}\\
	\begin{tabularx}{\hsize}{@{}lX}
	Datentyp: & numerisch \\
	Skalenniveau: & nominal \\
	Zugangswege: &
	  download-cuf, 
	  download-suf, 
	  remote-desktop-suf, 
	  onsite-suf
 \\
    \end{tabularx}



    %TABLE FOR QUESTION DETAILS
    %This has to be tested and has to be improved
    %rausfinden, ob einer Variable mehrere Fragen zugeordnet werden
    %dann evtl. nur die erste verwenden oder etwas anderes tun (Hinweis mehrere Fragen, auflisten mit Link)
				%TABLE FOR QUESTION DETAILS
				\vspace*{0.5cm}
                \noindent\textbf{Frage
	                \footnote{Detailliertere Informationen zur Frage finden sich unter
		              \url{https://metadata.fdz.dzhw.eu/\#!/de/questions/que-gra2009-ins2-5.6$}}}\\
				\begin{tabularx}{\hsize}{@{}lX}
					Fragenummer: &
					  Fragebogen des DZHW-Absolventenpanels 2009 - zweite Welle, Hauptbefragung (PAPI):
					  5.6
 \\
					%--
					Fragetext: & Bitte tragen Sie hier die für Sie wichtigsten Themen bzw. Fachgebiete dieser Veranstaltungen ein.\par  Kennziffer aus der Liste \\
				\end{tabularx}
				%TABLE FOR QUESTION DETAILS
				\vspace*{0.5cm}
                \noindent\textbf{Frage
	                \footnote{Detailliertere Informationen zur Frage finden sich unter
		              \url{https://metadata.fdz.dzhw.eu/\#!/de/questions/que-gra2009-ins3-52$}}}\\
				\begin{tabularx}{\hsize}{@{}lX}
					Fragenummer: &
					  Fragebogen des DZHW-Absolventenpanels 2009 - zweite Welle, Hauptbefragung (CAWI):
					  52
 \\
					%--
					Fragetext: & Bitte tragen Sie hier die für Sie wichtigsten Themen bzw. Fachgebiete dieser Veranstaltungen an. \\
				\end{tabularx}





				%TABLE FOR THE NOMINAL / ORDINAL VALUES
        		\vspace*{0.5cm}
                \noindent\textbf{Häufigkeiten}

                \vspace*{-\baselineskip}
					%NUMERIC ELEMENTS NEED A HUGH SECOND COLOUMN AND A SMALL FIRST ONE
					\begin{filecontents}{\jobname-bfec19b}
					\begin{longtable}{lXrrr}
					\toprule
					\textbf{Wert} & \textbf{Label} & \textbf{Häufigkeit} & \textbf{Prozent(gültig)} & \textbf{Prozent} \\
					\endhead
					\midrule
					\multicolumn{5}{l}{\textbf{Gültige Werte}}\\
						%DIFFERENT OBSERVATIONS <=20
								1 & \multicolumn{1}{X}{ingenieurwissenschaftliche Themen} & %34 &
								  \num{34} &
								%--
								  \num[round-mode=places,round-precision=2]{3,86} &
								  \num[round-mode=places,round-precision=2]{0,32} \\
								2 & \multicolumn{1}{X}{naturwissenschaftliche Themen} & %63 &
								  \num{63} &
								%--
								  \num[round-mode=places,round-precision=2]{7,15} &
								  \num[round-mode=places,round-precision=2]{0,6} \\
								3 & \multicolumn{1}{X}{mathematische Gebiete/Statistik} & %29 &
								  \num{29} &
								%--
								  \num[round-mode=places,round-precision=2]{3,29} &
								  \num[round-mode=places,round-precision=2]{0,28} \\
								4 & \multicolumn{1}{X}{sozialwissenschaftliche Themen} & %52 &
								  \num{52} &
								%--
								  \num[round-mode=places,round-precision=2]{5,9} &
								  \num[round-mode=places,round-precision=2]{0,5} \\
								5 & \multicolumn{1}{X}{geisteswissenschtliche Themen} & %50 &
								  \num{50} &
								%--
								  \num[round-mode=places,round-precision=2]{5,68} &
								  \num[round-mode=places,round-precision=2]{0,48} \\
								6 & \multicolumn{1}{X}{pädagogische/psychologische Themen} & %95 &
								  \num{95} &
								%--
								  \num[round-mode=places,round-precision=2]{10,78} &
								  \num[round-mode=places,round-precision=2]{0,91} \\
								7 & \multicolumn{1}{X}{medizinische Spezialgebiete} & %56 &
								  \num{56} &
								%--
								  \num[round-mode=places,round-precision=2]{6,36} &
								  \num[round-mode=places,round-precision=2]{0,53} \\
								8 & \multicolumn{1}{X}{informationstechnisches Spezialwissen} & %10 &
								  \num{10} &
								%--
								  \num[round-mode=places,round-precision=2]{1,14} &
								  \num[round-mode=places,round-precision=2]{0,1} \\
								9 & \multicolumn{1}{X}{Managementwissen} & %56 &
								  \num{56} &
								%--
								  \num[round-mode=places,round-precision=2]{6,36} &
								  \num[round-mode=places,round-precision=2]{0,53} \\
								10 & \multicolumn{1}{X}{Wirtschaftskenntnisse} & %45 &
								  \num{45} &
								%--
								  \num[round-mode=places,round-precision=2]{5,11} &
								  \num[round-mode=places,round-precision=2]{0,43} \\
							... & ... & ... & ... & ... \\
								15 & \multicolumn{1}{X}{EDV-Anwendungen} & %50 &
								  \num{50} &
								%--
								  \num[round-mode=places,round-precision=2]{5,68} &
								  \num[round-mode=places,round-precision=2]{0,48} \\

								16 & \multicolumn{1}{X}{Fremdsprachen} & %56 &
								  \num{56} &
								%--
								  \num[round-mode=places,round-precision=2]{6,36} &
								  \num[round-mode=places,round-precision=2]{0,53} \\

								17 & \multicolumn{1}{X}{Mitarbeiterführung/Personalentwicklung} & %38 &
								  \num{38} &
								%--
								  \num[round-mode=places,round-precision=2]{4,31} &
								  \num[round-mode=places,round-precision=2]{0,36} \\

								18 & \multicolumn{1}{X}{Kommunikations-/Interaktionstraining} & %97 &
								  \num{97} &
								%--
								  \num[round-mode=places,round-precision=2]{11,01} &
								  \num[round-mode=places,round-precision=2]{0,92} \\

								19 & \multicolumn{1}{X}{internationale Beziehungen, Kulturkenntnisse, Landeskunde} & %16 &
								  \num{16} &
								%--
								  \num[round-mode=places,round-precision=2]{1,82} &
								  \num[round-mode=places,round-precision=2]{0,15} \\

								20 & \multicolumn{1}{X}{ökologische Themen} & %7 &
								  \num{7} &
								%--
								  \num[round-mode=places,round-precision=2]{0,79} &
								  \num[round-mode=places,round-precision=2]{0,07} \\

								21 & \multicolumn{1}{X}{berufsethische Themen} & %5 &
								  \num{5} &
								%--
								  \num[round-mode=places,round-precision=2]{0,57} &
								  \num[round-mode=places,round-precision=2]{0,05} \\

								22 & \multicolumn{1}{X}{Existenzgründung} & %7 &
								  \num{7} &
								%--
								  \num[round-mode=places,round-precision=2]{0,79} &
								  \num[round-mode=places,round-precision=2]{0,07} \\

								23 & \multicolumn{1}{X}{betriebliches Gesundheitswesen, Arbeitssicherheit} & %12 &
								  \num{12} &
								%--
								  \num[round-mode=places,round-precision=2]{1,36} &
								  \num[round-mode=places,round-precision=2]{0,11} \\

								24 & \multicolumn{1}{X}{Sonstige} & %61 &
								  \num{61} &
								%--
								  \num[round-mode=places,round-precision=2]{6,92} &
								  \num[round-mode=places,round-precision=2]{0,58} \\

					\midrule
					\multicolumn{2}{l}{Summe (gültig)} &
					  \textbf{\num{881}} &
					\textbf{100} &
					  \textbf{\num[round-mode=places,round-precision=2]{8,4}} \\
					%--
					\multicolumn{5}{l}{\textbf{Fehlende Werte}}\\
							-998 &
							keine Angabe &
							  \num{413} &
							 - &
							  \num[round-mode=places,round-precision=2]{3,94} \\
							-995 &
							keine Teilnahme (Panel) &
							  \num{5739} &
							 - &
							  \num[round-mode=places,round-precision=2]{54,69} \\
							-989 &
							filterbedingt fehlend &
							  \num{3461} &
							 - &
							  \num[round-mode=places,round-precision=2]{32,98} \\
					\midrule
					\multicolumn{2}{l}{\textbf{Summe (gesamt)}} &
				      \textbf{\num{10494}} &
				    \textbf{-} &
				    \textbf{100} \\
					\bottomrule
					\end{longtable}
					\end{filecontents}
					\LTXtable{\textwidth}{\jobname-bfec19b}
				\label{tableValues:bfec19b}
				\vspace*{-\baselineskip}
                    \begin{noten}
                	    \note{} Deskritive Maßzahlen:
                	    Anzahl unterschiedlicher Beobachtungen: 24%
                	    ; 
                	      Modus ($h$): 18
                     \end{noten}


