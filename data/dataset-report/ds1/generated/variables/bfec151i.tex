%EVERY VARIABLE HAS IT'S OWN PAGE

    \setcounter{footnote}{0}

    %omit vertical space
    \vspace*{-1.8cm}
	\section{bfec151i (1. weitere akad. Qualifikation: Abschlussart)}
	\label{section:bfec151i}



	%TABLE FOR VARIABLE DETAILS
    \vspace*{0.5cm}
    \noindent\textbf{Eigenschaften
	% '#' has to be escaped
	\footnote{Detailliertere Informationen zur Variable finden sich unter
		\url{https://metadata.fdz.dzhw.eu/\#!/de/variables/var-gra2009-ds1-bfec151i$}}}\\
	\begin{tabularx}{\hsize}{@{}lX}
	Datentyp: & numerisch \\
	Skalenniveau: & nominal \\
	Zugangswege: &
	  download-cuf, 
	  download-suf, 
	  remote-desktop-suf, 
	  onsite-suf
 \\
    \end{tabularx}



    %TABLE FOR QUESTION DETAILS
    %This has to be tested and has to be improved
    %rausfinden, ob einer Variable mehrere Fragen zugeordnet werden
    %dann evtl. nur die erste verwenden oder etwas anderes tun (Hinweis mehrere Fragen, auflisten mit Link)
				%TABLE FOR QUESTION DETAILS
				\vspace*{0.5cm}
                \noindent\textbf{Frage
	                \footnote{Detailliertere Informationen zur Frage finden sich unter
		              \url{https://metadata.fdz.dzhw.eu/\#!/de/questions/que-gra2009-ins2-5.2$}}}\\
				\begin{tabularx}{\hsize}{@{}lX}
					Fragenummer: &
					  Fragebogen des DZHW-Absolventenpanels 2009 - zweite Welle, Hauptbefragung (PAPI):
					  5.2
 \\
					%--
					Fragetext: & Bitte tragen Sie diese längerfristigen Studienangebote, die Sie nach Ihrem Studienabschluss aus dem Jahr 2008/2009 begonnen, weitergeführt oder abgeschlossen haben (auch abgebrochene oder unterbrochene), in das folgende Tableau ein!\par  1. Studienangebot\par  Angestrebter oder erreichter Abschluss\par  Schlüssel siehe unten \\
				\end{tabularx}
				%TABLE FOR QUESTION DETAILS
				\vspace*{0.5cm}
                \noindent\textbf{Frage
	                \footnote{Detailliertere Informationen zur Frage finden sich unter
		              \url{https://metadata.fdz.dzhw.eu/\#!/de/questions/que-gra2009-ins3-47$}}}\\
				\begin{tabularx}{\hsize}{@{}lX}
					Fragenummer: &
					  Fragebogen des DZHW-Absolventenpanels 2009 - zweite Welle, Hauptbefragung (CAWI):
					  47
 \\
					%--
					Fragetext: & Bitte tragen Sie diese längerfristigen Studienangebote, die Sie nach Ihrem Studienabschluss aus dem Jahr 2008/2009 begonnen, weitergeführt oder abgeschlossen haben (auch abgebrochene oder unterbrochene), in das folgenden Tableau ein! \\
				\end{tabularx}





				%TABLE FOR THE NOMINAL / ORDINAL VALUES
        		\vspace*{0.5cm}
                \noindent\textbf{Häufigkeiten}

                \vspace*{-\baselineskip}
					%NUMERIC ELEMENTS NEED A HUGH SECOND COLOUMN AND A SMALL FIRST ONE
					\begin{filecontents}{\jobname-bfec151i}
					\begin{longtable}{lXrrr}
					\toprule
					\textbf{Wert} & \textbf{Label} & \textbf{Häufigkeit} & \textbf{Prozent(gültig)} & \textbf{Prozent} \\
					\endhead
					\midrule
					\multicolumn{5}{l}{\textbf{Gültige Werte}}\\
						%DIFFERENT OBSERVATIONS <=20

					1 &
				% TODO try size/length gt 0; take over for other passages
					\multicolumn{1}{X}{ kein Abschluss angestrebt   } &


					%30 &
					  \num{30} &
					%--
					  \num[round-mode=places,round-precision=2]{1,68} &
					    \num[round-mode=places,round-precision=2]{0,29} \\
							%????

					2 &
				% TODO try size/length gt 0; take over for other passages
					\multicolumn{1}{X}{ Master   } &


					%1498 &
					  \num{1498} &
					%--
					  \num[round-mode=places,round-precision=2]{84,11} &
					    \num[round-mode=places,round-precision=2]{14,27} \\
							%????

					3 &
				% TODO try size/length gt 0; take over for other passages
					\multicolumn{1}{X}{ Bachelor   } &


					%43 &
					  \num{43} &
					%--
					  \num[round-mode=places,round-precision=2]{2,41} &
					    \num[round-mode=places,round-precision=2]{0,41} \\
							%????

					4 &
				% TODO try size/length gt 0; take over for other passages
					\multicolumn{1}{X}{ Diplom / Magister   } &


					%54 &
					  \num{54} &
					%--
					  \num[round-mode=places,round-precision=2]{3,03} &
					    \num[round-mode=places,round-precision=2]{0,51} \\
							%????

					5 &
				% TODO try size/length gt 0; take over for other passages
					\multicolumn{1}{X}{ Staatsexamen   } &


					%50 &
					  \num{50} &
					%--
					  \num[round-mode=places,round-precision=2]{2,81} &
					    \num[round-mode=places,round-precision=2]{0,48} \\
							%????

					6 &
				% TODO try size/length gt 0; take over for other passages
					\multicolumn{1}{X}{ Zertifikat   } &


					%88 &
					  \num{88} &
					%--
					  \num[round-mode=places,round-precision=2]{4,94} &
					    \num[round-mode=places,round-precision=2]{0,84} \\
							%????

					7 &
				% TODO try size/length gt 0; take over for other passages
					\multicolumn{1}{X}{ sonstiger Abschluss   } &


					%18 &
					  \num{18} &
					%--
					  \num[round-mode=places,round-precision=2]{1,01} &
					    \num[round-mode=places,round-precision=2]{0,17} \\
							%????
						%DIFFERENT OBSERVATIONS >20
					\midrule
					\multicolumn{2}{l}{Summe (gültig)} &
					  \textbf{\num{1781}} &
					\textbf{100} &
					  \textbf{\num[round-mode=places,round-precision=2]{16,97}} \\
					%--
					\multicolumn{5}{l}{\textbf{Fehlende Werte}}\\
							-998 &
							keine Angabe &
							  \num{312} &
							 - &
							  \num[round-mode=places,round-precision=2]{2,97} \\
							-995 &
							keine Teilnahme (Panel) &
							  \num{5739} &
							 - &
							  \num[round-mode=places,round-precision=2]{54,69} \\
							-989 &
							filterbedingt fehlend &
							  \num{2662} &
							 - &
							  \num[round-mode=places,round-precision=2]{25,37} \\
					\midrule
					\multicolumn{2}{l}{\textbf{Summe (gesamt)}} &
				      \textbf{\num{10494}} &
				    \textbf{-} &
				    \textbf{100} \\
					\bottomrule
					\end{longtable}
					\end{filecontents}
					\LTXtable{\textwidth}{\jobname-bfec151i}
				\label{tableValues:bfec151i}
				\vspace*{-\baselineskip}
                    \begin{noten}
                	    \note{} Deskritive Maßzahlen:
                	    Anzahl unterschiedlicher Beobachtungen: 7%
                	    ; 
                	      Modus ($h$): 2
                     \end{noten}


