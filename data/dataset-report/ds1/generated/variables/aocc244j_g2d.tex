%EVERY VARIABLE HAS IT'S OWN PAGE

    \setcounter{footnote}{0}

    %omit vertical space
    \vspace*{-1.8cm}
	\section{aocc244j\_g2d (4. Tätigkeit: Arbeitsort (Bundes-/Ausland))}
	\label{section:aocc244j_g2d}



	%TABLE FOR VARIABLE DETAILS
    \vspace*{0.5cm}
    \noindent\textbf{Eigenschaften
	% '#' has to be escaped
	\footnote{Detailliertere Informationen zur Variable finden sich unter
		\url{https://metadata.fdz.dzhw.eu/\#!/de/variables/var-gra2009-ds1-aocc244j_g2d$}}}\\
	\begin{tabularx}{\hsize}{@{}lX}
	Datentyp: & numerisch \\
	Skalenniveau: & nominal \\
	Zugangswege: &
	  download-suf, 
	  remote-desktop-suf, 
	  onsite-suf
 \\
    \end{tabularx}



    %TABLE FOR QUESTION DETAILS
    %This has to be tested and has to be improved
    %rausfinden, ob einer Variable mehrere Fragen zugeordnet werden
    %dann evtl. nur die erste verwenden oder etwas anderes tun (Hinweis mehrere Fragen, auflisten mit Link)
				%TABLE FOR QUESTION DETAILS
				\vspace*{0.5cm}
                \noindent\textbf{Frage
	                \footnote{Detailliertere Informationen zur Frage finden sich unter
		              \url{https://metadata.fdz.dzhw.eu/\#!/de/questions/que-gra2009-ins1-5.4$}}}\\
				\begin{tabularx}{\hsize}{@{}lX}
					Fragenummer: &
					  Fragebogen des DZHW-Absolventenpanels 2009 - erste Welle:
					  5.4
 \\
					%--
					Fragetext: & Im Folgenden bitten wir Sie um eine Beschreibung der verschiedenen beruflichen Tätigkeiten, die Sie seit Ihrem Studienabschluss ausgeübt haben. \\
				\end{tabularx}





				%TABLE FOR THE NOMINAL / ORDINAL VALUES
        		\vspace*{0.5cm}
                \noindent\textbf{Häufigkeiten}

                \vspace*{-\baselineskip}
					%NUMERIC ELEMENTS NEED A HUGH SECOND COLOUMN AND A SMALL FIRST ONE
					\begin{filecontents}{\jobname-aocc244j_g2d}
					\begin{longtable}{lXrrr}
					\toprule
					\textbf{Wert} & \textbf{Label} & \textbf{Häufigkeit} & \textbf{Prozent(gültig)} & \textbf{Prozent} \\
					\endhead
					\midrule
					\multicolumn{5}{l}{\textbf{Gültige Werte}}\\
						%DIFFERENT OBSERVATIONS <=20

					1 &
				% TODO try size/length gt 0; take over for other passages
					\multicolumn{1}{X}{ Schleswig-Holstein   } &


					%4 &
					  \num{4} &
					%--
					  \num[round-mode=places,round-precision=2]{2,45} &
					    \num[round-mode=places,round-precision=2]{0,04} \\
							%????

					2 &
				% TODO try size/length gt 0; take over for other passages
					\multicolumn{1}{X}{ Hamburg   } &


					%4 &
					  \num{4} &
					%--
					  \num[round-mode=places,round-precision=2]{2,45} &
					    \num[round-mode=places,round-precision=2]{0,04} \\
							%????

					3 &
				% TODO try size/length gt 0; take over for other passages
					\multicolumn{1}{X}{ Niedersachsen   } &


					%11 &
					  \num{11} &
					%--
					  \num[round-mode=places,round-precision=2]{6,75} &
					    \num[round-mode=places,round-precision=2]{0,1} \\
							%????

					4 &
				% TODO try size/length gt 0; take over for other passages
					\multicolumn{1}{X}{ Bremen   } &


					%1 &
					  \num{1} &
					%--
					  \num[round-mode=places,round-precision=2]{0,61} &
					    \num[round-mode=places,round-precision=2]{0,01} \\
							%????

					5 &
				% TODO try size/length gt 0; take over for other passages
					\multicolumn{1}{X}{ Nordrhein-Westfalen   } &


					%21 &
					  \num{21} &
					%--
					  \num[round-mode=places,round-precision=2]{12,88} &
					    \num[round-mode=places,round-precision=2]{0,2} \\
							%????

					6 &
				% TODO try size/length gt 0; take over for other passages
					\multicolumn{1}{X}{ Hessen   } &


					%13 &
					  \num{13} &
					%--
					  \num[round-mode=places,round-precision=2]{7,98} &
					    \num[round-mode=places,round-precision=2]{0,12} \\
							%????

					7 &
				% TODO try size/length gt 0; take over for other passages
					\multicolumn{1}{X}{ Rheinland-Pfalz   } &


					%5 &
					  \num{5} &
					%--
					  \num[round-mode=places,round-precision=2]{3,07} &
					    \num[round-mode=places,round-precision=2]{0,05} \\
							%????

					8 &
				% TODO try size/length gt 0; take over for other passages
					\multicolumn{1}{X}{ Baden-Württemberg   } &


					%18 &
					  \num{18} &
					%--
					  \num[round-mode=places,round-precision=2]{11,04} &
					    \num[round-mode=places,round-precision=2]{0,17} \\
							%????

					9 &
				% TODO try size/length gt 0; take over for other passages
					\multicolumn{1}{X}{ Bayern   } &


					%29 &
					  \num{29} &
					%--
					  \num[round-mode=places,round-precision=2]{17,79} &
					    \num[round-mode=places,round-precision=2]{0,28} \\
							%????

					10 &
				% TODO try size/length gt 0; take over for other passages
					\multicolumn{1}{X}{ Saarland   } &


					%1 &
					  \num{1} &
					%--
					  \num[round-mode=places,round-precision=2]{0,61} &
					    \num[round-mode=places,round-precision=2]{0,01} \\
							%????

					11 &
				% TODO try size/length gt 0; take over for other passages
					\multicolumn{1}{X}{ Berlin   } &


					%19 &
					  \num{19} &
					%--
					  \num[round-mode=places,round-precision=2]{11,66} &
					    \num[round-mode=places,round-precision=2]{0,18} \\
							%????

					12 &
				% TODO try size/length gt 0; take over for other passages
					\multicolumn{1}{X}{ Brandenburg   } &


					%2 &
					  \num{2} &
					%--
					  \num[round-mode=places,round-precision=2]{1,23} &
					    \num[round-mode=places,round-precision=2]{0,02} \\
							%????

					13 &
				% TODO try size/length gt 0; take over for other passages
					\multicolumn{1}{X}{ Mecklenburg-Vorpommern   } &


					%2 &
					  \num{2} &
					%--
					  \num[round-mode=places,round-precision=2]{1,23} &
					    \num[round-mode=places,round-precision=2]{0,02} \\
							%????

					14 &
				% TODO try size/length gt 0; take over for other passages
					\multicolumn{1}{X}{ Sachsen   } &


					%13 &
					  \num{13} &
					%--
					  \num[round-mode=places,round-precision=2]{7,98} &
					    \num[round-mode=places,round-precision=2]{0,12} \\
							%????

					15 &
				% TODO try size/length gt 0; take over for other passages
					\multicolumn{1}{X}{ Sachsen-Anhalt   } &


					%2 &
					  \num{2} &
					%--
					  \num[round-mode=places,round-precision=2]{1,23} &
					    \num[round-mode=places,round-precision=2]{0,02} \\
							%????

					16 &
				% TODO try size/length gt 0; take over for other passages
					\multicolumn{1}{X}{ Thüringen   } &


					%10 &
					  \num{10} &
					%--
					  \num[round-mode=places,round-precision=2]{6,13} &
					    \num[round-mode=places,round-precision=2]{0,1} \\
							%????

					94 &
				% TODO try size/length gt 0; take over for other passages
					\multicolumn{1}{X}{ mehrere deutsche Bundesländer (alte und neue)   } &


					%1 &
					  \num{1} &
					%--
					  \num[round-mode=places,round-precision=2]{0,61} &
					    \num[round-mode=places,round-precision=2]{0,01} \\
							%????

					100 &
				% TODO try size/length gt 0; take over for other passages
					\multicolumn{1}{X}{ Ausland   } &


					%7 &
					  \num{7} &
					%--
					  \num[round-mode=places,round-precision=2]{4,29} &
					    \num[round-mode=places,round-precision=2]{0,07} \\
							%????
						%DIFFERENT OBSERVATIONS >20
					\midrule
					\multicolumn{2}{l}{Summe (gültig)} &
					  \textbf{\num{163}} &
					\textbf{100} &
					  \textbf{\num[round-mode=places,round-precision=2]{1,55}} \\
					%--
					\multicolumn{5}{l}{\textbf{Fehlende Werte}}\\
							-998 &
							keine Angabe &
							  \num{8243} &
							 - &
							  \num[round-mode=places,round-precision=2]{78,55} \\
							-989 &
							filterbedingt fehlend &
							  \num{2088} &
							 - &
							  \num[round-mode=places,round-precision=2]{19,9} \\
					\midrule
					\multicolumn{2}{l}{\textbf{Summe (gesamt)}} &
				      \textbf{\num{10494}} &
				    \textbf{-} &
				    \textbf{100} \\
					\bottomrule
					\end{longtable}
					\end{filecontents}
					\LTXtable{\textwidth}{\jobname-aocc244j_g2d}
				\label{tableValues:aocc244j_g2d}
				\vspace*{-\baselineskip}
                    \begin{noten}
                	    \note{} Deskritive Maßzahlen:
                	    Anzahl unterschiedlicher Beobachtungen: 18%
                	    ; 
                	      Modus ($h$): 9
                     \end{noten}


