%EVERY VARIABLE HAS IT'S OWN PAGE

    \setcounter{footnote}{0}

    %omit vertical space
    \vspace*{-1.8cm}
	\section{mres022e\_g1r (1. Wohnung: Ort (Bundesland/Land))}
	\label{section:mres022e_g1r}



	%TABLE FOR VARIABLE DETAILS
    \vspace*{0.5cm}
    \noindent\textbf{Eigenschaften
	% '#' has to be escaped
	\footnote{Detailliertere Informationen zur Variable finden sich unter
		\url{https://metadata.fdz.dzhw.eu/\#!/de/variables/var-gra2009-ds1-mres022e_g1r$}}}\\
	\begin{tabularx}{\hsize}{@{}lX}
	Datentyp: & numerisch \\
	Skalenniveau: & nominal \\
	Zugangswege: &
	  remote-desktop-suf, 
	  onsite-suf
 \\
    \end{tabularx}



    %TABLE FOR QUESTION DETAILS
    %This has to be tested and has to be improved
    %rausfinden, ob einer Variable mehrere Fragen zugeordnet werden
    %dann evtl. nur die erste verwenden oder etwas anderes tun (Hinweis mehrere Fragen, auflisten mit Link)
				%TABLE FOR QUESTION DETAILS
				\vspace*{0.5cm}
                \noindent\textbf{Frage
	                \footnote{Detailliertere Informationen zur Frage finden sich unter
		              \url{https://metadata.fdz.dzhw.eu/\#!/de/questions/que-gra2009-ins5-08.1$}}}\\
				\begin{tabularx}{\hsize}{@{}lX}
					Fragenummer: &
					  Fragebogen des DZHW-Absolventenpanels 2009 - zweite Welle, Vertiefungsbefragung Mobilität:
					  08.1
 \\
					%--
					Fragetext: & Nun bitten wir Sie, alle Wohnungen aufzulisten, in denen Sie seit dem Ende Ihres Studiums 2008/09 gelebt haben.,Uns interessiert dabei nur, wo Sie tatsächlich gelebt haben, nicht wo Sie ihren Wohnsitz gemeldet hatten. Denken Sie dabei bitte auch an Zweit- und Nebenwohnungen. Bitte nennen Sie uns nun die nächste Wohnung, in die Sie nach Ihrem Studienabschluss eingezogen sind.,Zeitraum (Monat/Jahr),Wohnort,Wohnten Sie die meiste Zeit(Mehrfachnennung möglich),Handelte es sich um,Bundesland bzw. Land (bei Ausland) \\
				\end{tabularx}





				%TABLE FOR THE NOMINAL / ORDINAL VALUES
        		\vspace*{0.5cm}
                \noindent\textbf{Häufigkeiten}

                \vspace*{-\baselineskip}
					%NUMERIC ELEMENTS NEED A HUGH SECOND COLOUMN AND A SMALL FIRST ONE
					\begin{filecontents}{\jobname-mres022e_g1r}
					\begin{longtable}{lXrrr}
					\toprule
					\textbf{Wert} & \textbf{Label} & \textbf{Häufigkeit} & \textbf{Prozent(gültig)} & \textbf{Prozent} \\
					\endhead
					\midrule
					\multicolumn{5}{l}{\textbf{Gültige Werte}}\\
						%DIFFERENT OBSERVATIONS <=20
								1 & \multicolumn{1}{X}{Schleswig-Holstein} & %35 &
								  \num{35} &
								%--
								  \num[round-mode=places,round-precision=2]{2,14} &
								  \num[round-mode=places,round-precision=2]{0,33} \\
								2 & \multicolumn{1}{X}{Hamburg} & %69 &
								  \num{69} &
								%--
								  \num[round-mode=places,round-precision=2]{4,23} &
								  \num[round-mode=places,round-precision=2]{0,66} \\
								3 & \multicolumn{1}{X}{Niedersachsen} & %102 &
								  \num{102} &
								%--
								  \num[round-mode=places,round-precision=2]{6,25} &
								  \num[round-mode=places,round-precision=2]{0,97} \\
								4 & \multicolumn{1}{X}{Bremen} & %14 &
								  \num{14} &
								%--
								  \num[round-mode=places,round-precision=2]{0,86} &
								  \num[round-mode=places,round-precision=2]{0,13} \\
								5 & \multicolumn{1}{X}{Nordrhein-Westfalen} & %220 &
								  \num{220} &
								%--
								  \num[round-mode=places,round-precision=2]{13,48} &
								  \num[round-mode=places,round-precision=2]{2,1} \\
								6 & \multicolumn{1}{X}{Hessen} & %88 &
								  \num{88} &
								%--
								  \num[round-mode=places,round-precision=2]{5,39} &
								  \num[round-mode=places,round-precision=2]{0,84} \\
								7 & \multicolumn{1}{X}{Rheinland-Pfalz} & %67 &
								  \num{67} &
								%--
								  \num[round-mode=places,round-precision=2]{4,11} &
								  \num[round-mode=places,round-precision=2]{0,64} \\
								8 & \multicolumn{1}{X}{Baden-Württemberg} & %196 &
								  \num{196} &
								%--
								  \num[round-mode=places,round-precision=2]{12,01} &
								  \num[round-mode=places,round-precision=2]{1,87} \\
								9 & \multicolumn{1}{X}{Bayern} & %245 &
								  \num{245} &
								%--
								  \num[round-mode=places,round-precision=2]{15,01} &
								  \num[round-mode=places,round-precision=2]{2,33} \\
								10 & \multicolumn{1}{X}{Saarland} & %12 &
								  \num{12} &
								%--
								  \num[round-mode=places,round-precision=2]{0,74} &
								  \num[round-mode=places,round-precision=2]{0,11} \\
							... & ... & ... & ... & ... \\
								451 & \multicolumn{1}{X}{Libanon} & %1 &
								  \num{1} &
								%--
								  \num[round-mode=places,round-precision=2]{0,06} &
								  \num[round-mode=places,round-precision=2]{0,01} \\

								467 & \multicolumn{1}{X}{Republik Korea, auch Süd-Korea} & %1 &
								  \num{1} &
								%--
								  \num[round-mode=places,round-precision=2]{0,06} &
								  \num[round-mode=places,round-precision=2]{0,01} \\

								469 & \multicolumn{1}{X}{Vereinigte Arabische Emirate} & %1 &
								  \num{1} &
								%--
								  \num[round-mode=places,round-precision=2]{0,06} &
								  \num[round-mode=places,round-precision=2]{0,01} \\

								472 & \multicolumn{1}{X}{Saudi-Arabien} & %2 &
								  \num{2} &
								%--
								  \num[round-mode=places,round-precision=2]{0,12} &
								  \num[round-mode=places,round-precision=2]{0,02} \\

								474 & \multicolumn{1}{X}{Singapur} & %2 &
								  \num{2} &
								%--
								  \num[round-mode=places,round-precision=2]{0,12} &
								  \num[round-mode=places,round-precision=2]{0,02} \\

								475 & \multicolumn{1}{X}{Arabische Republik Syrien} & %1 &
								  \num{1} &
								%--
								  \num[round-mode=places,round-precision=2]{0,06} &
								  \num[round-mode=places,round-precision=2]{0,01} \\

								479 & \multicolumn{1}{X}{China} & %4 &
								  \num{4} &
								%--
								  \num[round-mode=places,round-precision=2]{0,25} &
								  \num[round-mode=places,round-precision=2]{0,04} \\

								523 & \multicolumn{1}{X}{Australien} & %5 &
								  \num{5} &
								%--
								  \num[round-mode=places,round-precision=2]{0,31} &
								  \num[round-mode=places,round-precision=2]{0,05} \\

								536 & \multicolumn{1}{X}{Neuseeland} & %1 &
								  \num{1} &
								%--
								  \num[round-mode=places,round-precision=2]{0,06} &
								  \num[round-mode=places,round-precision=2]{0,01} \\

								996 & \multicolumn{1}{X}{international} & %1 &
								  \num{1} &
								%--
								  \num[round-mode=places,round-precision=2]{0,06} &
								  \num[round-mode=places,round-precision=2]{0,01} \\

					\midrule
					\multicolumn{2}{l}{Summe (gültig)} &
					  \textbf{\num{1632}} &
					\textbf{100} &
					  \textbf{\num[round-mode=places,round-precision=2]{15,55}} \\
					%--
					\multicolumn{5}{l}{\textbf{Fehlende Werte}}\\
							-998 &
							keine Angabe &
							  \num{330} &
							 - &
							  \num[round-mode=places,round-precision=2]{3,14} \\
							-995 &
							keine Teilnahme (Panel) &
							  \num{8029} &
							 - &
							  \num[round-mode=places,round-precision=2]{76,51} \\
							-989 &
							filterbedingt fehlend &
							  \num{503} &
							 - &
							  \num[round-mode=places,round-precision=2]{4,79} \\
					\midrule
					\multicolumn{2}{l}{\textbf{Summe (gesamt)}} &
				      \textbf{\num{10494}} &
				    \textbf{-} &
				    \textbf{100} \\
					\bottomrule
					\end{longtable}
					\end{filecontents}
					\LTXtable{\textwidth}{\jobname-mres022e_g1r}
				\label{tableValues:mres022e_g1r}
				\vspace*{-\baselineskip}
                    \begin{noten}
                	    \note{} Deskritive Maßzahlen:
                	    Anzahl unterschiedlicher Beobachtungen: 64%
                	    ; 
                	      Modus ($h$): 9
                     \end{noten}


