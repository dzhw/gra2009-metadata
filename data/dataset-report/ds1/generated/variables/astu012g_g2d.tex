%EVERY VARIABLE HAS IT'S OWN PAGE

    \setcounter{footnote}{0}

    %omit vertical space
    \vspace*{-1.8cm}
	\section{astu012g\_g2d (2. Studium: 1. Nebenfach (Studienbereiche))}
	\label{section:astu012g_g2d}



	%TABLE FOR VARIABLE DETAILS
    \vspace*{0.5cm}
    \noindent\textbf{Eigenschaften
	% '#' has to be escaped
	\footnote{Detailliertere Informationen zur Variable finden sich unter
		\url{https://metadata.fdz.dzhw.eu/\#!/de/variables/var-gra2009-ds1-astu012g_g2d$}}}\\
	\begin{tabularx}{\hsize}{@{}lX}
	Datentyp: & numerisch \\
	Skalenniveau: & nominal \\
	Zugangswege: &
	  download-suf, 
	  remote-desktop-suf, 
	  onsite-suf
 \\
    \end{tabularx}



    %TABLE FOR QUESTION DETAILS
    %This has to be tested and has to be improved
    %rausfinden, ob einer Variable mehrere Fragen zugeordnet werden
    %dann evtl. nur die erste verwenden oder etwas anderes tun (Hinweis mehrere Fragen, auflisten mit Link)
				%TABLE FOR QUESTION DETAILS
				\vspace*{0.5cm}
                \noindent\textbf{Frage
	                \footnote{Detailliertere Informationen zur Frage finden sich unter
		              \url{https://metadata.fdz.dzhw.eu/\#!/de/questions/que-gra2009-ins1-1.1$}}}\\
				\begin{tabularx}{\hsize}{@{}lX}
					Fragenummer: &
					  Fragebogen des DZHW-Absolventenpanels 2009 - erste Welle:
					  1.1
 \\
					%--
					Fragetext: & Bitte tragen Sie in das folgende Tableau Ihren Studienverlauf ein. \\
				\end{tabularx}





				%TABLE FOR THE NOMINAL / ORDINAL VALUES
        		\vspace*{0.5cm}
                \noindent\textbf{Häufigkeiten}

                \vspace*{-\baselineskip}
					%NUMERIC ELEMENTS NEED A HUGH SECOND COLOUMN AND A SMALL FIRST ONE
					\begin{filecontents}{\jobname-astu012g_g2d}
					\begin{longtable}{lXrrr}
					\toprule
					\textbf{Wert} & \textbf{Label} & \textbf{Häufigkeit} & \textbf{Prozent(gültig)} & \textbf{Prozent} \\
					\endhead
					\midrule
					\multicolumn{5}{l}{\textbf{Gültige Werte}}\\
						%DIFFERENT OBSERVATIONS <=20
								1 & \multicolumn{1}{X}{Sprach- und Kulturwissenschaften allgemein} & %23 &
								  \num{23} &
								%--
								  \num[round-mode=places,round-precision=2]{2,12} &
								  \num[round-mode=places,round-precision=2]{0,22} \\
								2 & \multicolumn{1}{X}{Evang. Theologie, -Religionslehre} & %29 &
								  \num{29} &
								%--
								  \num[round-mode=places,round-precision=2]{2,67} &
								  \num[round-mode=places,round-precision=2]{0,28} \\
								3 & \multicolumn{1}{X}{Kath. Theologie, -Religionslehre} & %14 &
								  \num{14} &
								%--
								  \num[round-mode=places,round-precision=2]{1,29} &
								  \num[round-mode=places,round-precision=2]{0,13} \\
								4 & \multicolumn{1}{X}{Philosophie} & %51 &
								  \num{51} &
								%--
								  \num[round-mode=places,round-precision=2]{4,69} &
								  \num[round-mode=places,round-precision=2]{0,49} \\
								5 & \multicolumn{1}{X}{Geschichte} & %111 &
								  \num{111} &
								%--
								  \num[round-mode=places,round-precision=2]{10,21} &
								  \num[round-mode=places,round-precision=2]{1,06} \\
								7 & \multicolumn{1}{X}{Allgemeine und vergleichende Literatur- und Sprachwissenschaft} & %16 &
								  \num{16} &
								%--
								  \num[round-mode=places,round-precision=2]{1,47} &
								  \num[round-mode=places,round-precision=2]{0,15} \\
								8 & \multicolumn{1}{X}{Altphilologie (klass. Philologie), Neugriechisch} & %4 &
								  \num{4} &
								%--
								  \num[round-mode=places,round-precision=2]{0,37} &
								  \num[round-mode=places,round-precision=2]{0,04} \\
								9 & \multicolumn{1}{X}{Germanistik (Deutsch, germanische Sprachen ohne Anglistik)} & %118 &
								  \num{118} &
								%--
								  \num[round-mode=places,round-precision=2]{10,86} &
								  \num[round-mode=places,round-precision=2]{1,12} \\
								10 & \multicolumn{1}{X}{Anglistik, Amerikanistik} & %94 &
								  \num{94} &
								%--
								  \num[round-mode=places,round-precision=2]{8,65} &
								  \num[round-mode=places,round-precision=2]{0,9} \\
								11 & \multicolumn{1}{X}{Romanistik} & %46 &
								  \num{46} &
								%--
								  \num[round-mode=places,round-precision=2]{4,23} &
								  \num[round-mode=places,round-precision=2]{0,44} \\
							... & ... & ... & ... & ... \\
								63 & \multicolumn{1}{X}{Maschinenbau/Verfahrenstechnik} & %10 &
								  \num{10} &
								%--
								  \num[round-mode=places,round-precision=2]{0,92} &
								  \num[round-mode=places,round-precision=2]{0,1} \\

								64 & \multicolumn{1}{X}{Elektrotechnik} & %3 &
								  \num{3} &
								%--
								  \num[round-mode=places,round-precision=2]{0,28} &
								  \num[round-mode=places,round-precision=2]{0,03} \\

								66 & \multicolumn{1}{X}{Architektur, Innenarchitektur} & %1 &
								  \num{1} &
								%--
								  \num[round-mode=places,round-precision=2]{0,09} &
								  \num[round-mode=places,round-precision=2]{0,01} \\

								67 & \multicolumn{1}{X}{Raumplanung} & %3 &
								  \num{3} &
								%--
								  \num[round-mode=places,round-precision=2]{0,28} &
								  \num[round-mode=places,round-precision=2]{0,03} \\

								68 & \multicolumn{1}{X}{Bauingenieurwesen} & %1 &
								  \num{1} &
								%--
								  \num[round-mode=places,round-precision=2]{0,09} &
								  \num[round-mode=places,round-precision=2]{0,01} \\

								74 & \multicolumn{1}{X}{Kunst, Kunstwissenschaft allgemein} & %27 &
								  \num{27} &
								%--
								  \num[round-mode=places,round-precision=2]{2,48} &
								  \num[round-mode=places,round-precision=2]{0,26} \\

								75 & \multicolumn{1}{X}{Bildende Kunst} & %1 &
								  \num{1} &
								%--
								  \num[round-mode=places,round-precision=2]{0,09} &
								  \num[round-mode=places,round-precision=2]{0,01} \\

								76 & \multicolumn{1}{X}{Gestaltung} & %6 &
								  \num{6} &
								%--
								  \num[round-mode=places,round-precision=2]{0,55} &
								  \num[round-mode=places,round-precision=2]{0,06} \\

								77 & \multicolumn{1}{X}{Darstellende Kunst, Film und Fernsehen, Theaterwissenschaft} & %2 &
								  \num{2} &
								%--
								  \num[round-mode=places,round-precision=2]{0,18} &
								  \num[round-mode=places,round-precision=2]{0,02} \\

								78 & \multicolumn{1}{X}{Musik, Musikwissenschaft} & %17 &
								  \num{17} &
								%--
								  \num[round-mode=places,round-precision=2]{1,56} &
								  \num[round-mode=places,round-precision=2]{0,16} \\

					\midrule
					\multicolumn{2}{l}{Summe (gültig)} &
					  \textbf{\num{1087}} &
					\textbf{100} &
					  \textbf{\num[round-mode=places,round-precision=2]{10,36}} \\
					%--
					\multicolumn{5}{l}{\textbf{Fehlende Werte}}\\
							-998 &
							keine Angabe &
							  \num{9407} &
							 - &
							  \num[round-mode=places,round-precision=2]{89,64} \\
					\midrule
					\multicolumn{2}{l}{\textbf{Summe (gesamt)}} &
				      \textbf{\num{10494}} &
				    \textbf{-} &
				    \textbf{100} \\
					\bottomrule
					\end{longtable}
					\end{filecontents}
					\LTXtable{\textwidth}{\jobname-astu012g_g2d}
				\label{tableValues:astu012g_g2d}
				\vspace*{-\baselineskip}
                    \begin{noten}
                	    \note{} Deskritive Maßzahlen:
                	    Anzahl unterschiedlicher Beobachtungen: 47%
                	    ; 
                	      Modus ($h$): 9
                     \end{noten}


