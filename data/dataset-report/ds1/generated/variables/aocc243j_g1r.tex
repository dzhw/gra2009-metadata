%EVERY VARIABLE HAS IT'S OWN PAGE

    \setcounter{footnote}{0}

    %omit vertical space
    \vspace*{-1.8cm}
	\section{aocc243j\_g1r (3. Tätigkeit: Arbeitsort (Bundesland/Land))}
	\label{section:aocc243j_g1r}



	%TABLE FOR VARIABLE DETAILS
    \vspace*{0.5cm}
    \noindent\textbf{Eigenschaften
	% '#' has to be escaped
	\footnote{Detailliertere Informationen zur Variable finden sich unter
		\url{https://metadata.fdz.dzhw.eu/\#!/de/variables/var-gra2009-ds1-aocc243j_g1r$}}}\\
	\begin{tabularx}{\hsize}{@{}lX}
	Datentyp: & numerisch \\
	Skalenniveau: & nominal \\
	Zugangswege: &
	  remote-desktop-suf, 
	  onsite-suf
 \\
    \end{tabularx}



    %TABLE FOR QUESTION DETAILS
    %This has to be tested and has to be improved
    %rausfinden, ob einer Variable mehrere Fragen zugeordnet werden
    %dann evtl. nur die erste verwenden oder etwas anderes tun (Hinweis mehrere Fragen, auflisten mit Link)
				%TABLE FOR QUESTION DETAILS
				\vspace*{0.5cm}
                \noindent\textbf{Frage
	                \footnote{Detailliertere Informationen zur Frage finden sich unter
		              \url{https://metadata.fdz.dzhw.eu/\#!/de/questions/que-gra2009-ins1-5.4$}}}\\
				\begin{tabularx}{\hsize}{@{}lX}
					Fragenummer: &
					  Fragebogen des DZHW-Absolventenpanels 2009 - erste Welle:
					  5.4
 \\
					%--
					Fragetext: & Im Folgenden bitten wir Sie um eine Beschreibung der verschiedenen beruflichen Tätigkeiten, die Sie seit Ihrem Studienabschluss ausgeübt haben.\par  3. Erwerbstätigkeit\par  Arbeitsort\par  Bundesland bzw. Land (bei Ausland) \\
				\end{tabularx}





				%TABLE FOR THE NOMINAL / ORDINAL VALUES
        		\vspace*{0.5cm}
                \noindent\textbf{Häufigkeiten}

                \vspace*{-\baselineskip}
					%NUMERIC ELEMENTS NEED A HUGH SECOND COLOUMN AND A SMALL FIRST ONE
					\begin{filecontents}{\jobname-aocc243j_g1r}
					\begin{longtable}{lXrrr}
					\toprule
					\textbf{Wert} & \textbf{Label} & \textbf{Häufigkeit} & \textbf{Prozent(gültig)} & \textbf{Prozent} \\
					\endhead
					\midrule
					\multicolumn{5}{l}{\textbf{Gültige Werte}}\\
						%DIFFERENT OBSERVATIONS <=20
								1 & \multicolumn{1}{X}{Schleswig-Holstein} & %19 &
								  \num{19} &
								%--
								  \num[round-mode=places,round-precision=2]{2,57} &
								  \num[round-mode=places,round-precision=2]{0,18} \\
								2 & \multicolumn{1}{X}{Hamburg} & %32 &
								  \num{32} &
								%--
								  \num[round-mode=places,round-precision=2]{4,34} &
								  \num[round-mode=places,round-precision=2]{0,3} \\
								3 & \multicolumn{1}{X}{Niedersachsen} & %50 &
								  \num{50} &
								%--
								  \num[round-mode=places,round-precision=2]{6,78} &
								  \num[round-mode=places,round-precision=2]{0,48} \\
								4 & \multicolumn{1}{X}{Bremen} & %6 &
								  \num{6} &
								%--
								  \num[round-mode=places,round-precision=2]{0,81} &
								  \num[round-mode=places,round-precision=2]{0,06} \\
								5 & \multicolumn{1}{X}{Nordrhein-Westfalen} & %103 &
								  \num{103} &
								%--
								  \num[round-mode=places,round-precision=2]{13,96} &
								  \num[round-mode=places,round-precision=2]{0,98} \\
								6 & \multicolumn{1}{X}{Hessen} & %62 &
								  \num{62} &
								%--
								  \num[round-mode=places,round-precision=2]{8,4} &
								  \num[round-mode=places,round-precision=2]{0,59} \\
								7 & \multicolumn{1}{X}{Rheinland-Pfalz} & %34 &
								  \num{34} &
								%--
								  \num[round-mode=places,round-precision=2]{4,61} &
								  \num[round-mode=places,round-precision=2]{0,32} \\
								8 & \multicolumn{1}{X}{Baden-Württemberg} & %92 &
								  \num{92} &
								%--
								  \num[round-mode=places,round-precision=2]{12,47} &
								  \num[round-mode=places,round-precision=2]{0,88} \\
								9 & \multicolumn{1}{X}{Bayern} & %96 &
								  \num{96} &
								%--
								  \num[round-mode=places,round-precision=2]{13,01} &
								  \num[round-mode=places,round-precision=2]{0,91} \\
								10 & \multicolumn{1}{X}{Saarland} & %8 &
								  \num{8} &
								%--
								  \num[round-mode=places,round-precision=2]{1,08} &
								  \num[round-mode=places,round-precision=2]{0,08} \\
							... & ... & ... & ... & ... \\
								56 & \multicolumn{1}{X}{Kanada} & %1 &
								  \num{1} &
								%--
								  \num[round-mode=places,round-precision=2]{0,14} &
								  \num[round-mode=places,round-precision=2]{0,01} \\

								58 & \multicolumn{1}{X}{Brasilien} & %1 &
								  \num{1} &
								%--
								  \num[round-mode=places,round-precision=2]{0,14} &
								  \num[round-mode=places,round-precision=2]{0,01} \\

								60 & \multicolumn{1}{X}{Südamerika ohne Brasilien} & %1 &
								  \num{1} &
								%--
								  \num[round-mode=places,round-precision=2]{0,14} &
								  \num[round-mode=places,round-precision=2]{0,01} \\

								61 & \multicolumn{1}{X}{Rumänien} & %1 &
								  \num{1} &
								%--
								  \num[round-mode=places,round-precision=2]{0,14} &
								  \num[round-mode=places,round-precision=2]{0,01} \\

								67 & \multicolumn{1}{X}{naher und mittlerer Osten (z.B. Saudi-Arabien, Syrien, V.A.E., Irak, Jordanien)} & %1 &
								  \num{1} &
								%--
								  \num[round-mode=places,round-precision=2]{0,14} &
								  \num[round-mode=places,round-precision=2]{0,01} \\

								69 & \multicolumn{1}{X}{China, Volksrepublik} & %1 &
								  \num{1} &
								%--
								  \num[round-mode=places,round-precision=2]{0,14} &
								  \num[round-mode=places,round-precision=2]{0,01} \\

								80 & \multicolumn{1}{X}{Australien} & %3 &
								  \num{3} &
								%--
								  \num[round-mode=places,round-precision=2]{0,41} &
								  \num[round-mode=places,round-precision=2]{0,03} \\

								88 & \multicolumn{1}{X}{Kamerun} & %1 &
								  \num{1} &
								%--
								  \num[round-mode=places,round-precision=2]{0,14} &
								  \num[round-mode=places,round-precision=2]{0,01} \\

								94 & \multicolumn{1}{X}{mehrere deutsche Bundesländer (alte und neue)} & %2 &
								  \num{2} &
								%--
								  \num[round-mode=places,round-precision=2]{0,27} &
								  \num[round-mode=places,round-precision=2]{0,02} \\

								95 & \multicolumn{1}{X}{Deutschland und Ausland} & %1 &
								  \num{1} &
								%--
								  \num[round-mode=places,round-precision=2]{0,14} &
								  \num[round-mode=places,round-precision=2]{0,01} \\

					\midrule
					\multicolumn{2}{l}{Summe (gültig)} &
					  \textbf{\num{738}} &
					\textbf{100} &
					  \textbf{\num[round-mode=places,round-precision=2]{7,03}} \\
					%--
					\multicolumn{5}{l}{\textbf{Fehlende Werte}}\\
							-998 &
							keine Angabe &
							  \num{7668} &
							 - &
							  \num[round-mode=places,round-precision=2]{73,07} \\
							-989 &
							filterbedingt fehlend &
							  \num{2088} &
							 - &
							  \num[round-mode=places,round-precision=2]{19,9} \\
					\midrule
					\multicolumn{2}{l}{\textbf{Summe (gesamt)}} &
				      \textbf{\num{10494}} &
				    \textbf{-} &
				    \textbf{100} \\
					\bottomrule
					\end{longtable}
					\end{filecontents}
					\LTXtable{\textwidth}{\jobname-aocc243j_g1r}
				\label{tableValues:aocc243j_g1r}
				\vspace*{-\baselineskip}
                    \begin{noten}
                	    \note{} Deskritive Maßzahlen:
                	    Anzahl unterschiedlicher Beobachtungen: 39%
                	    ; 
                	      Modus ($h$): 5
                     \end{noten}


