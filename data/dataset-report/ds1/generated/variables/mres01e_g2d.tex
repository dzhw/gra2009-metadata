%EVERY VARIABLE HAS IT'S OWN PAGE

    \setcounter{footnote}{0}

    %omit vertical space
    \vspace*{-1.8cm}
	\section{mres01e\_g2d (Wohnung Studium: Ort (Bundes-/Ausland))}
	\label{section:mres01e_g2d}



	%TABLE FOR VARIABLE DETAILS
    \vspace*{0.5cm}
    \noindent\textbf{Eigenschaften
	% '#' has to be escaped
	\footnote{Detailliertere Informationen zur Variable finden sich unter
		\url{https://metadata.fdz.dzhw.eu/\#!/de/variables/var-gra2009-ds1-mres01e_g2d$}}}\\
	\begin{tabularx}{\hsize}{@{}lX}
	Datentyp: & numerisch \\
	Skalenniveau: & nominal \\
	Zugangswege: &
	  download-suf, 
	  remote-desktop-suf, 
	  onsite-suf
 \\
    \end{tabularx}



    %TABLE FOR QUESTION DETAILS
    %This has to be tested and has to be improved
    %rausfinden, ob einer Variable mehrere Fragen zugeordnet werden
    %dann evtl. nur die erste verwenden oder etwas anderes tun (Hinweis mehrere Fragen, auflisten mit Link)
				%TABLE FOR QUESTION DETAILS
				\vspace*{0.5cm}
                \noindent\textbf{Frage
	                \footnote{Detailliertere Informationen zur Frage finden sich unter
		              \url{https://metadata.fdz.dzhw.eu/\#!/de/questions/que-gra2009-ins5-07.1$}}}\\
				\begin{tabularx}{\hsize}{@{}lX}
					Fragenummer: &
					  Fragebogen des DZHW-Absolventenpanels 2009 - zweite Welle, Vertiefungsbefragung Mobilität:
					  07.1
 \\
					%--
					Fragetext: & Zunächst bitten wir Sie uns dabei mitzuteilen, wo und wie Sie direkt während Ihres Studienabschlusses 2008/09 gewohnt haben \\
				\end{tabularx}





				%TABLE FOR THE NOMINAL / ORDINAL VALUES
        		\vspace*{0.5cm}
                \noindent\textbf{Häufigkeiten}

                \vspace*{-\baselineskip}
					%NUMERIC ELEMENTS NEED A HUGH SECOND COLOUMN AND A SMALL FIRST ONE
					\begin{filecontents}{\jobname-mres01e_g2d}
					\begin{longtable}{lXrrr}
					\toprule
					\textbf{Wert} & \textbf{Label} & \textbf{Häufigkeit} & \textbf{Prozent(gültig)} & \textbf{Prozent} \\
					\endhead
					\midrule
					\multicolumn{5}{l}{\textbf{Gültige Werte}}\\
						%DIFFERENT OBSERVATIONS <=20

					1 &
				% TODO try size/length gt 0; take over for other passages
					\multicolumn{1}{X}{ Schleswig-Holstein   } &


					%45 &
					  \num{45} &
					%--
					  \num[round-mode=places,round-precision=2]{2,15} &
					    \num[round-mode=places,round-precision=2]{0,43} \\
							%????

					2 &
				% TODO try size/length gt 0; take over for other passages
					\multicolumn{1}{X}{ Hamburg   } &


					%69 &
					  \num{69} &
					%--
					  \num[round-mode=places,round-precision=2]{3,3} &
					    \num[round-mode=places,round-precision=2]{0,66} \\
							%????

					3 &
				% TODO try size/length gt 0; take over for other passages
					\multicolumn{1}{X}{ Niedersachsen   } &


					%156 &
					  \num{156} &
					%--
					  \num[round-mode=places,round-precision=2]{7,45} &
					    \num[round-mode=places,round-precision=2]{1,49} \\
							%????

					4 &
				% TODO try size/length gt 0; take over for other passages
					\multicolumn{1}{X}{ Bremen   } &


					%17 &
					  \num{17} &
					%--
					  \num[round-mode=places,round-precision=2]{0,81} &
					    \num[round-mode=places,round-precision=2]{0,16} \\
							%????

					5 &
				% TODO try size/length gt 0; take over for other passages
					\multicolumn{1}{X}{ Nordrhein-Westfalen   } &


					%303 &
					  \num{303} &
					%--
					  \num[round-mode=places,round-precision=2]{14,47} &
					    \num[round-mode=places,round-precision=2]{2,89} \\
							%????

					6 &
				% TODO try size/length gt 0; take over for other passages
					\multicolumn{1}{X}{ Hessen   } &


					%115 &
					  \num{115} &
					%--
					  \num[round-mode=places,round-precision=2]{5,49} &
					    \num[round-mode=places,round-precision=2]{1,1} \\
							%????

					7 &
				% TODO try size/length gt 0; take over for other passages
					\multicolumn{1}{X}{ Rheinland-Pfalz   } &


					%93 &
					  \num{93} &
					%--
					  \num[round-mode=places,round-precision=2]{4,44} &
					    \num[round-mode=places,round-precision=2]{0,89} \\
							%????

					8 &
				% TODO try size/length gt 0; take over for other passages
					\multicolumn{1}{X}{ Baden-Württemberg   } &


					%260 &
					  \num{260} &
					%--
					  \num[round-mode=places,round-precision=2]{12,42} &
					    \num[round-mode=places,round-precision=2]{2,48} \\
							%????

					9 &
				% TODO try size/length gt 0; take over for other passages
					\multicolumn{1}{X}{ Bayern   } &


					%373 &
					  \num{373} &
					%--
					  \num[round-mode=places,round-precision=2]{17,81} &
					    \num[round-mode=places,round-precision=2]{3,55} \\
							%????

					10 &
				% TODO try size/length gt 0; take over for other passages
					\multicolumn{1}{X}{ Saarland   } &


					%12 &
					  \num{12} &
					%--
					  \num[round-mode=places,round-precision=2]{0,57} &
					    \num[round-mode=places,round-precision=2]{0,11} \\
							%????

					11 &
				% TODO try size/length gt 0; take over for other passages
					\multicolumn{1}{X}{ Berlin   } &


					%136 &
					  \num{136} &
					%--
					  \num[round-mode=places,round-precision=2]{6,49} &
					    \num[round-mode=places,round-precision=2]{1,3} \\
							%????

					12 &
				% TODO try size/length gt 0; take over for other passages
					\multicolumn{1}{X}{ Brandenburg   } &


					%27 &
					  \num{27} &
					%--
					  \num[round-mode=places,round-precision=2]{1,29} &
					    \num[round-mode=places,round-precision=2]{0,26} \\
							%????

					13 &
				% TODO try size/length gt 0; take over for other passages
					\multicolumn{1}{X}{ Mecklenburg-Vorpommern   } &


					%42 &
					  \num{42} &
					%--
					  \num[round-mode=places,round-precision=2]{2,01} &
					    \num[round-mode=places,round-precision=2]{0,4} \\
							%????

					14 &
				% TODO try size/length gt 0; take over for other passages
					\multicolumn{1}{X}{ Sachsen   } &


					%205 &
					  \num{205} &
					%--
					  \num[round-mode=places,round-precision=2]{9,79} &
					    \num[round-mode=places,round-precision=2]{1,95} \\
							%????

					15 &
				% TODO try size/length gt 0; take over for other passages
					\multicolumn{1}{X}{ Sachsen-Anhalt   } &


					%35 &
					  \num{35} &
					%--
					  \num[round-mode=places,round-precision=2]{1,67} &
					    \num[round-mode=places,round-precision=2]{0,33} \\
							%????

					16 &
				% TODO try size/length gt 0; take over for other passages
					\multicolumn{1}{X}{ Thüringen   } &


					%145 &
					  \num{145} &
					%--
					  \num[round-mode=places,round-precision=2]{6,92} &
					    \num[round-mode=places,round-precision=2]{1,38} \\
							%????

					93 &
				% TODO try size/length gt 0; take over for other passages
					\multicolumn{1}{X}{ Deutschland ohne nähere Angabe   } &


					%25 &
					  \num{25} &
					%--
					  \num[round-mode=places,round-precision=2]{1,19} &
					    \num[round-mode=places,round-precision=2]{0,24} \\
							%????

					100 &
				% TODO try size/length gt 0; take over for other passages
					\multicolumn{1}{X}{ Ausland   } &


					%36 &
					  \num{36} &
					%--
					  \num[round-mode=places,round-precision=2]{1,72} &
					    \num[round-mode=places,round-precision=2]{0,34} \\
							%????
						%DIFFERENT OBSERVATIONS >20
					\midrule
					\multicolumn{2}{l}{Summe (gültig)} &
					  \textbf{\num{2094}} &
					\textbf{100} &
					  \textbf{\num[round-mode=places,round-precision=2]{19,95}} \\
					%--
					\multicolumn{5}{l}{\textbf{Fehlende Werte}}\\
							-998 &
							keine Angabe &
							  \num{371} &
							 - &
							  \num[round-mode=places,round-precision=2]{3,54} \\
							-995 &
							keine Teilnahme (Panel) &
							  \num{8029} &
							 - &
							  \num[round-mode=places,round-precision=2]{76,51} \\
					\midrule
					\multicolumn{2}{l}{\textbf{Summe (gesamt)}} &
				      \textbf{\num{10494}} &
				    \textbf{-} &
				    \textbf{100} \\
					\bottomrule
					\end{longtable}
					\end{filecontents}
					\LTXtable{\textwidth}{\jobname-mres01e_g2d}
				\label{tableValues:mres01e_g2d}
				\vspace*{-\baselineskip}
                    \begin{noten}
                	    \note{} Deskritive Maßzahlen:
                	    Anzahl unterschiedlicher Beobachtungen: 18%
                	    ; 
                	      Modus ($h$): 9
                     \end{noten}


