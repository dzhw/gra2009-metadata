%EVERY VARIABLE HAS IT'S OWN PAGE

    \setcounter{footnote}{0}

    %omit vertical space
    \vspace*{-1.8cm}
	\section{aocc15h\_g1r (Motiv Praktikum: Sonstiges, und zwar)}
	\label{section:aocc15h_g1r}



	%TABLE FOR VARIABLE DETAILS
    \vspace*{0.5cm}
    \noindent\textbf{Eigenschaften
	% '#' has to be escaped
	\footnote{Detailliertere Informationen zur Variable finden sich unter
		\url{https://metadata.fdz.dzhw.eu/\#!/de/variables/var-gra2009-ds1-aocc15h_g1r$}}}\\
	\begin{tabularx}{\hsize}{@{}lX}
	Datentyp: & numerisch \\
	Skalenniveau: & nominal \\
	Zugangswege: &
	  remote-desktop-suf, 
	  onsite-suf
 \\
    \end{tabularx}



    %TABLE FOR QUESTION DETAILS
    %This has to be tested and has to be improved
    %rausfinden, ob einer Variable mehrere Fragen zugeordnet werden
    %dann evtl. nur die erste verwenden oder etwas anderes tun (Hinweis mehrere Fragen, auflisten mit Link)
				%TABLE FOR QUESTION DETAILS
				\vspace*{0.5cm}
                \noindent\textbf{Frage
	                \footnote{Detailliertere Informationen zur Frage finden sich unter
		              \url{https://metadata.fdz.dzhw.eu/\#!/de/questions/que-gra2009-ins1-4.14$}}}\\
				\begin{tabularx}{\hsize}{@{}lX}
					Fragenummer: &
					  Fragebogen des DZHW-Absolventenpanels 2009 - erste Welle:
					  4.14
 \\
					%--
					Fragetext: & Was hat Sie bewogen, nach dem Studienabschluss ein Praktikum aufzunehmen?\par  Sonstiges, und zwar: \\
				\end{tabularx}





				%TABLE FOR THE NOMINAL / ORDINAL VALUES
        		\vspace*{0.5cm}
                \noindent\textbf{Häufigkeiten}

                \vspace*{-\baselineskip}
					%NUMERIC ELEMENTS NEED A HUGH SECOND COLOUMN AND A SMALL FIRST ONE
					\begin{filecontents}{\jobname-aocc15h_g1r}
					\begin{longtable}{lXrrr}
					\toprule
					\textbf{Wert} & \textbf{Label} & \textbf{Häufigkeit} & \textbf{Prozent(gültig)} & \textbf{Prozent} \\
					\endhead
					\midrule
					\multicolumn{5}{l}{\textbf{Gültige Werte}}\\
						%DIFFERENT OBSERVATIONS <=20

					1 &
				% TODO try size/length gt 0; take over for other passages
					\multicolumn{1}{X}{ Überbrückung   } &


					%127 &
					  \num{127} &
					%--
					  \num[round-mode=places,round-precision=2]{27,73} &
					    \num[round-mode=places,round-precision=2]{1,21} \\
							%????

					2 &
				% TODO try size/length gt 0; take over for other passages
					\multicolumn{1}{X}{ Auslandserfahrung   } &


					%89 &
					  \num{89} &
					%--
					  \num[round-mode=places,round-precision=2]{19,43} &
					    \num[round-mode=places,round-precision=2]{0,85} \\
							%????

					3 &
				% TODO try size/length gt 0; take over for other passages
					\multicolumn{1}{X}{ Orientierung   } &


					%48 &
					  \num{48} &
					%--
					  \num[round-mode=places,round-precision=2]{10,48} &
					    \num[round-mode=places,round-precision=2]{0,46} \\
							%????

					4 &
				% TODO try size/length gt 0; take over for other passages
					\multicolumn{1}{X}{ wurde verlangt   } &


					%12 &
					  \num{12} &
					%--
					  \num[round-mode=places,round-precision=2]{2,62} &
					    \num[round-mode=places,round-precision=2]{0,11} \\
							%????

					5 &
				% TODO try size/length gt 0; take over for other passages
					\multicolumn{1}{X}{ im Folgestudium   } &


					%118 &
					  \num{118} &
					%--
					  \num[round-mode=places,round-precision=2]{25,76} &
					    \num[round-mode=places,round-precision=2]{1,12} \\
							%????

					6 &
				% TODO try size/length gt 0; take over for other passages
					\multicolumn{1}{X}{ Fortsetzung Studienkontakte   } &


					%6 &
					  \num{6} &
					%--
					  \num[round-mode=places,round-precision=2]{1,31} &
					    \num[round-mode=places,round-precision=2]{0,06} \\
							%????

					9 &
				% TODO try size/length gt 0; take over for other passages
					\multicolumn{1}{X}{ Sonstiges   } &


					%58 &
					  \num{58} &
					%--
					  \num[round-mode=places,round-precision=2]{12,66} &
					    \num[round-mode=places,round-precision=2]{0,55} \\
							%????
						%DIFFERENT OBSERVATIONS >20
					\midrule
					\multicolumn{2}{l}{Summe (gültig)} &
					  \textbf{\num{458}} &
					\textbf{100} &
					  \textbf{\num[round-mode=places,round-precision=2]{4,36}} \\
					%--
					\multicolumn{5}{l}{\textbf{Fehlende Werte}}\\
							-998 &
							keine Angabe &
							  \num{257} &
							 - &
							  \num[round-mode=places,round-precision=2]{2,45} \\
							-989 &
							filterbedingt fehlend &
							  \num{8569} &
							 - &
							  \num[round-mode=places,round-precision=2]{81,66} \\
							-988 &
							trifft nicht zu &
							  \num{1210} &
							 - &
							  \num[round-mode=places,round-precision=2]{11,53} \\
					\midrule
					\multicolumn{2}{l}{\textbf{Summe (gesamt)}} &
				      \textbf{\num{10494}} &
				    \textbf{-} &
				    \textbf{100} \\
					\bottomrule
					\end{longtable}
					\end{filecontents}
					\LTXtable{\textwidth}{\jobname-aocc15h_g1r}
				\label{tableValues:aocc15h_g1r}
				\vspace*{-\baselineskip}
                    \begin{noten}
                	    \note{} Deskritive Maßzahlen:
                	    Anzahl unterschiedlicher Beobachtungen: 7%
                	    ; 
                	      Modus ($h$): 1
                     \end{noten}


