%EVERY VARIABLE HAS IT'S OWN PAGE

    \setcounter{footnote}{0}

    %omit vertical space
    \vspace*{-1.8cm}
	\section{astu023g\_g2o (3. Abschluss: Hochschule (NUTS2))}
	\label{section:astu023g_g2o}



	%TABLE FOR VARIABLE DETAILS
    \vspace*{0.5cm}
    \noindent\textbf{Eigenschaften
	% '#' has to be escaped
	\footnote{Detailliertere Informationen zur Variable finden sich unter
		\url{https://metadata.fdz.dzhw.eu/\#!/de/variables/var-gra2009-ds1-astu023g_g2o$}}}\\
	\begin{tabularx}{\hsize}{@{}lX}
	Datentyp: & string \\
	Skalenniveau: & nominal \\
	Zugangswege: &
	  onsite-suf
 \\
    \end{tabularx}



    %TABLE FOR QUESTION DETAILS
    %This has to be tested and has to be improved
    %rausfinden, ob einer Variable mehrere Fragen zugeordnet werden
    %dann evtl. nur die erste verwenden oder etwas anderes tun (Hinweis mehrere Fragen, auflisten mit Link)
				%TABLE FOR QUESTION DETAILS
				\vspace*{0.5cm}
                \noindent\textbf{Frage
	                \footnote{Detailliertere Informationen zur Frage finden sich unter
		              \url{https://metadata.fdz.dzhw.eu/\#!/de/questions/que-gra2009-ins1-1.2$}}}\\
				\begin{tabularx}{\hsize}{@{}lX}
					Fragenummer: &
					  Fragebogen des DZHW-Absolventenpanels 2009 - erste Welle:
					  1.2
 \\
					%--
					Fragetext: & Welche Studienabschlüsse haben Sie erlangt? \\
				\end{tabularx}





				%TABLE FOR THE NOMINAL / ORDINAL VALUES
        		\vspace*{0.5cm}
                \noindent\textbf{Häufigkeiten}

                \vspace*{-\baselineskip}
					%STRING ELEMENTS NEEDS A HUGH FIRST COLOUMN AND A SMALL SECOND ONE
					\begin{filecontents}{\jobname-astu023g_g2o}
					\begin{longtable}{Xlrrr}
					\toprule
					\textbf{Wert} & \textbf{Label} & \textbf{Häufigkeit} & \textbf{Prozent (gültig)} & \textbf{Prozent} \\
					\endhead
					\midrule
					\multicolumn{5}{l}{\textbf{Gültige Werte}}\\
						%DIFFERENT OBSERVATIONS <=20

					\multicolumn{1}{X}{DE11 Stuttgart} &
					- &
					2 &
					9,52 &
					0,02 \\
					
					\multicolumn{1}{X}{DE14 Tübingen} &
					- &
					1 &
					4,76 &
					0,01 \\
					
					\multicolumn{1}{X}{DE21 Oberbayern} &
					- &
					2 &
					9,52 &
					0,02 \\
					
					\multicolumn{1}{X}{DE23 Oberpfalz} &
					- &
					1 &
					4,76 &
					0,01 \\
					
					\multicolumn{1}{X}{DE80 Mecklenburg-Vorpommern} &
					- &
					6 &
					28,57 &
					0,06 \\
					
					\multicolumn{1}{X}{DE92 Hannover} &
					- &
					1 &
					4,76 &
					0,01 \\
					
					\multicolumn{1}{X}{DE94 Weser-Ems} &
					- &
					2 &
					9,52 &
					0,02 \\
					
					\multicolumn{1}{X}{DEA2 Köln} &
					- &
					1 &
					4,76 &
					0,01 \\
					
					\multicolumn{1}{X}{DEB3 Rheinhessen-Pfalz} &
					- &
					1 &
					4,76 &
					0,01 \\
					
					\multicolumn{1}{X}{DED2 Dresden} &
					- &
					1 &
					4,76 &
					0,01 \\
					
					\multicolumn{1}{X}{DEG0 Thüringen} &
					- &
					3 &
					14,29 &
					0,03 \\
											%DIFFERENT OBSERVATIONS >20
					\midrule
						\multicolumn{2}{l}{Summe (gültig)} & 21 &
						\textbf{100} &
					    0,2 \\
					\multicolumn{5}{l}{\textbf{Fehlende Werte}}\\
							-966 & nicht bestimmbar & 9 & - & 0,09 \\

							-998 & keine Angabe & 10464 & - & 99,71 \\

					\midrule
					\multicolumn{2}{l}{\textbf{Summe (gesamt)}} & \textbf{10494} & \textbf{-} & \textbf{100} \\
					\bottomrule
					\caption{Werte der Variable astu023g\_g2o}
					\end{longtable}
					\end{filecontents}
					\LTXtable{\textwidth}{\jobname-astu023g_g2o}


