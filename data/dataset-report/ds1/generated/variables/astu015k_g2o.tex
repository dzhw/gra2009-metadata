%EVERY VARIABLE HAS IT'S OWN PAGE

    \setcounter{footnote}{0}

    %omit vertical space
    \vspace*{-1.8cm}
	\section{astu015k\_g2o (5. Studium: Hochschule (NUTS2))}
	\label{section:astu015k_g2o}



	%TABLE FOR VARIABLE DETAILS
    \vspace*{0.5cm}
    \noindent\textbf{Eigenschaften
	% '#' has to be escaped
	\footnote{Detailliertere Informationen zur Variable finden sich unter
		\url{https://metadata.fdz.dzhw.eu/\#!/de/variables/var-gra2009-ds1-astu015k_g2o$}}}\\
	\begin{tabularx}{\hsize}{@{}lX}
	Datentyp: & string \\
	Skalenniveau: & nominal \\
	Zugangswege: &
	  onsite-suf
 \\
    \end{tabularx}



    %TABLE FOR QUESTION DETAILS
    %This has to be tested and has to be improved
    %rausfinden, ob einer Variable mehrere Fragen zugeordnet werden
    %dann evtl. nur die erste verwenden oder etwas anderes tun (Hinweis mehrere Fragen, auflisten mit Link)
				%TABLE FOR QUESTION DETAILS
				\vspace*{0.5cm}
                \noindent\textbf{Frage
	                \footnote{Detailliertere Informationen zur Frage finden sich unter
		              \url{https://metadata.fdz.dzhw.eu/\#!/de/questions/que-gra2009-ins1-1.1$}}}\\
				\begin{tabularx}{\hsize}{@{}lX}
					Fragenummer: &
					  Fragebogen des DZHW-Absolventenpanels 2009 - erste Welle:
					  1.1
 \\
					%--
					Fragetext: & Bitte tragen Sie in das folgende Tableau Ihren Studienverlauf ein. \\
				\end{tabularx}





				%TABLE FOR THE NOMINAL / ORDINAL VALUES
        		\vspace*{0.5cm}
                \noindent\textbf{Häufigkeiten}

                \vspace*{-\baselineskip}
					%STRING ELEMENTS NEEDS A HUGH FIRST COLOUMN AND A SMALL SECOND ONE
					\begin{filecontents}{\jobname-astu015k_g2o}
					\begin{longtable}{Xlrrr}
					\toprule
					\textbf{Wert} & \textbf{Label} & \textbf{Häufigkeit} & \textbf{Prozent (gültig)} & \textbf{Prozent} \\
					\endhead
					\midrule
					\multicolumn{5}{l}{\textbf{Gültige Werte}}\\
						%DIFFERENT OBSERVATIONS <=20
								\multicolumn{1}{X}{DE11 Stuttgart} & - & 7 & 7,87 & 0,07 \\
								\multicolumn{1}{X}{DE12 Karlsruhe} & - & 3 & 3,37 & 0,03 \\
								\multicolumn{1}{X}{DE13 Freiburg} & - & 1 & 1,12 & 0,01 \\
								\multicolumn{1}{X}{DE14 Tübingen} & - & 5 & 5,62 & 0,05 \\
								\multicolumn{1}{X}{DE21 Oberbayern} & - & 14 & 15,73 & 0,13 \\
								\multicolumn{1}{X}{DE22 Niederbayern} & - & 3 & 3,37 & 0,03 \\
								\multicolumn{1}{X}{DE23 Oberpfalz} & - & 4 & 4,49 & 0,04 \\
								\multicolumn{1}{X}{DE24 Oberfranken} & - & 1 & 1,12 & 0,01 \\
								\multicolumn{1}{X}{DE25 Mittelfranken} & - & 3 & 3,37 & 0,03 \\
								\multicolumn{1}{X}{DE27 Schwaben} & - & 1 & 1,12 & 0,01 \\
							... & ... & ... & ... & ... \\
								\multicolumn{1}{X}{DEA4 Detmold} & - & 3 & 3,37 & 0,03 \\
								\multicolumn{1}{X}{DEA5 Arnsberg} & - & 3 & 3,37 & 0,03 \\
								\multicolumn{1}{X}{DEB1 Koblenz} & - & 1 & 1,12 & 0,01 \\
								\multicolumn{1}{X}{DEB2 Trier} & - & 1 & 1,12 & 0,01 \\
								\multicolumn{1}{X}{DEB3 Rheinhessen-Pfalz} & - & 6 & 6,74 & 0,06 \\
								\multicolumn{1}{X}{DEC0 Saarland} & - & 1 & 1,12 & 0,01 \\
								\multicolumn{1}{X}{DED2 Dresden} & - & 1 & 1,12 & 0,01 \\
								\multicolumn{1}{X}{DED4 Chemnitz} & - & 2 & 2,25 & 0,02 \\
								\multicolumn{1}{X}{DEF0 Schleswig-Holstein} & - & 1 & 1,12 & 0,01 \\
								\multicolumn{1}{X}{DEG0 Thüringen} & - & 2 & 2,25 & 0,02 \\
					\midrule
						\multicolumn{2}{l}{Summe (gültig)} & 89 &
						\textbf{100} &
					    0,85 \\
					\multicolumn{5}{l}{\textbf{Fehlende Werte}}\\
							-966 & nicht bestimmbar & 12 & - & 0,11 \\

							-998 & keine Angabe & 10393 & - & 99,04 \\

					\midrule
					\multicolumn{2}{l}{\textbf{Summe (gesamt)}} & \textbf{10494} & \textbf{-} & \textbf{100} \\
					\bottomrule
					\caption{Werte der Variable astu015k\_g2o}
					\end{longtable}
					\end{filecontents}
					\LTXtable{\textwidth}{\jobname-astu015k_g2o}


