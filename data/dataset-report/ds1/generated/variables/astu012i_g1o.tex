%EVERY VARIABLE HAS IT'S OWN PAGE

    \setcounter{footnote}{0}

    %omit vertical space
    \vspace*{-1.8cm}
	\section{astu012i\_g1o (2. Studium: 2. Nebenfach)}
	\label{section:astu012i_g1o}



	%TABLE FOR VARIABLE DETAILS
    \vspace*{0.5cm}
    \noindent\textbf{Eigenschaften
	% '#' has to be escaped
	\footnote{Detailliertere Informationen zur Variable finden sich unter
		\url{https://metadata.fdz.dzhw.eu/\#!/de/variables/var-gra2009-ds1-astu012i_g1o$}}}\\
	\begin{tabularx}{\hsize}{@{}lX}
	Datentyp: & numerisch \\
	Skalenniveau: & nominal \\
	Zugangswege: &
	  onsite-suf
 \\
    \end{tabularx}



    %TABLE FOR QUESTION DETAILS
    %This has to be tested and has to be improved
    %rausfinden, ob einer Variable mehrere Fragen zugeordnet werden
    %dann evtl. nur die erste verwenden oder etwas anderes tun (Hinweis mehrere Fragen, auflisten mit Link)
				%TABLE FOR QUESTION DETAILS
				\vspace*{0.5cm}
                \noindent\textbf{Frage
	                \footnote{Detailliertere Informationen zur Frage finden sich unter
		              \url{https://metadata.fdz.dzhw.eu/\#!/de/questions/que-gra2009-ins1-1.1$}}}\\
				\begin{tabularx}{\hsize}{@{}lX}
					Fragenummer: &
					  Fragebogen des DZHW-Absolventenpanels 2009 - erste Welle:
					  1.1
 \\
					%--
					Fragetext: & Bitte tragen Sie in das folgende Tableau Ihren Studienverlauf ein.\par  Studienfach (ggf 2. Hauptfach oder Nebenfächer) \\
				\end{tabularx}





				%TABLE FOR THE NOMINAL / ORDINAL VALUES
        		\vspace*{0.5cm}
                \noindent\textbf{Häufigkeiten}

                \vspace*{-\baselineskip}
					%NUMERIC ELEMENTS NEED A HUGH SECOND COLOUMN AND A SMALL FIRST ONE
					\begin{filecontents}{\jobname-astu012i_g1o}
					\begin{longtable}{lXrrr}
					\toprule
					\textbf{Wert} & \textbf{Label} & \textbf{Häufigkeit} & \textbf{Prozent(gültig)} & \textbf{Prozent} \\
					\endhead
					\midrule
					\multicolumn{5}{l}{\textbf{Gültige Werte}}\\
						%DIFFERENT OBSERVATIONS <=20
								4 & \multicolumn{1}{X}{Interdisziplinäre Studien (Schwerp. Sprach- und Kulturwissenschaften)} & %5 &
								  \num{5} &
								%--
								  \num[round-mode=places,round-precision=2]{1,72} &
								  \num[round-mode=places,round-precision=2]{0,05} \\
								8 & \multicolumn{1}{X}{Anglistik/Englisch} & %13 &
								  \num{13} &
								%--
								  \num[round-mode=places,round-precision=2]{4,47} &
								  \num[round-mode=places,round-precision=2]{0,12} \\
								9 & \multicolumn{1}{X}{Anthropologie (Humanbiologie)} & %1 &
								  \num{1} &
								%--
								  \num[round-mode=places,round-precision=2]{0,34} &
								  \num[round-mode=places,round-precision=2]{0,01} \\
								12 & \multicolumn{1}{X}{Archäologie} & %3 &
								  \num{3} &
								%--
								  \num[round-mode=places,round-precision=2]{1,03} &
								  \num[round-mode=places,round-precision=2]{0,03} \\
								21 & \multicolumn{1}{X}{Betriebswirtschaftslehre} & %4 &
								  \num{4} &
								%--
								  \num[round-mode=places,round-precision=2]{1,37} &
								  \num[round-mode=places,round-precision=2]{0,04} \\
								22 & \multicolumn{1}{X}{Bibliothekswissenschaft/-wesen} & %1 &
								  \num{1} &
								%--
								  \num[round-mode=places,round-precision=2]{0,34} &
								  \num[round-mode=places,round-precision=2]{0,01} \\
								24 & \multicolumn{1}{X}{Europäische Ethnologie u. Kulturwissenschaft} & %1 &
								  \num{1} &
								%--
								  \num[round-mode=places,round-precision=2]{0,34} &
								  \num[round-mode=places,round-precision=2]{0,01} \\
								26 & \multicolumn{1}{X}{Biologie} & %4 &
								  \num{4} &
								%--
								  \num[round-mode=places,round-precision=2]{1,37} &
								  \num[round-mode=places,round-precision=2]{0,04} \\
								29 & \multicolumn{1}{X}{Sportwissenschaft} & %2 &
								  \num{2} &
								%--
								  \num[round-mode=places,round-precision=2]{0,69} &
								  \num[round-mode=places,round-precision=2]{0,02} \\
								30 & \multicolumn{1}{X}{Interdisziplinäre Studien (Schwerpunkt Rechts-, Wirtschafts- und Sozialwissenschaften)} & %2 &
								  \num{2} &
								%--
								  \num[round-mode=places,round-precision=2]{0,69} &
								  \num[round-mode=places,round-precision=2]{0,02} \\
							... & ... & ... & ... & ... \\
								199 & \multicolumn{1}{X}{Lernbereich Technik} & %1 &
								  \num{1} &
								%--
								  \num[round-mode=places,round-precision=2]{0,34} &
								  \num[round-mode=places,round-precision=2]{0,01} \\

								254 & \multicolumn{1}{X}{Sachunterricht (einschl. Schulgarten)} & %2 &
								  \num{2} &
								%--
								  \num[round-mode=places,round-precision=2]{0,69} &
								  \num[round-mode=places,round-precision=2]{0,02} \\

								271 & \multicolumn{1}{X}{Deutsch für Ausländer} & %1 &
								  \num{1} &
								%--
								  \num[round-mode=places,round-precision=2]{0,34} &
								  \num[round-mode=places,round-precision=2]{0,01} \\

								272 & \multicolumn{1}{X}{Alte Geschichte} & %1 &
								  \num{1} &
								%--
								  \num[round-mode=places,round-precision=2]{0,34} &
								  \num[round-mode=places,round-precision=2]{0,01} \\

								273 & \multicolumn{1}{X}{Mittlere und neuere Geschichte} & %4 &
								  \num{4} &
								%--
								  \num[round-mode=places,round-precision=2]{1,37} &
								  \num[round-mode=places,round-precision=2]{0,04} \\

								284 & \multicolumn{1}{X}{Angewandte Sprachwissenschaft} & %2 &
								  \num{2} &
								%--
								  \num[round-mode=places,round-precision=2]{0,69} &
								  \num[round-mode=places,round-precision=2]{0,02} \\

								302 & \multicolumn{1}{X}{Medienwissenschaft} & %2 &
								  \num{2} &
								%--
								  \num[round-mode=places,round-precision=2]{0,69} &
								  \num[round-mode=places,round-precision=2]{0,02} \\

								303 & \multicolumn{1}{X}{Kommunikationswissenschaft/Publizistik} & %9 &
								  \num{9} &
								%--
								  \num[round-mode=places,round-precision=2]{3,09} &
								  \num[round-mode=places,round-precision=2]{0,09} \\

								321 & \multicolumn{1}{X}{Erwachsenenbildung und außerschulische Jugendbildung} & %1 &
								  \num{1} &
								%--
								  \num[round-mode=places,round-precision=2]{0,34} &
								  \num[round-mode=places,round-precision=2]{0,01} \\

								361 & \multicolumn{1}{X}{Schulpädagogik} & %1 &
								  \num{1} &
								%--
								  \num[round-mode=places,round-precision=2]{0,34} &
								  \num[round-mode=places,round-precision=2]{0,01} \\

					\midrule
					\multicolumn{2}{l}{Summe (gültig)} &
					  \textbf{\num{291}} &
					\textbf{100} &
					  \textbf{\num[round-mode=places,round-precision=2]{2,77}} \\
					%--
					\multicolumn{5}{l}{\textbf{Fehlende Werte}}\\
							-998 &
							keine Angabe &
							  \num{10203} &
							 - &
							  \num[round-mode=places,round-precision=2]{97,23} \\
					\midrule
					\multicolumn{2}{l}{\textbf{Summe (gesamt)}} &
				      \textbf{\num{10494}} &
				    \textbf{-} &
				    \textbf{100} \\
					\bottomrule
					\end{longtable}
					\end{filecontents}
					\LTXtable{\textwidth}{\jobname-astu012i_g1o}
				\label{tableValues:astu012i_g1o}
				\vspace*{-\baselineskip}
                    \begin{noten}
                	    \note{} Deskritive Maßzahlen:
                	    Anzahl unterschiedlicher Beobachtungen: 70%
                	    ; 
                	      Modus ($h$): 67
                     \end{noten}


