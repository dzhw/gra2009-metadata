%EVERY VARIABLE HAS IT'S OWN PAGE

    \setcounter{footnote}{0}

    %omit vertical space
    \vspace*{-1.8cm}
	\section{aocc244a (4. Tätigkeit: Beginn (Monat))}
	\label{section:aocc244a}



	%TABLE FOR VARIABLE DETAILS
    \vspace*{0.5cm}
    \noindent\textbf{Eigenschaften
	% '#' has to be escaped
	\footnote{Detailliertere Informationen zur Variable finden sich unter
		\url{https://metadata.fdz.dzhw.eu/\#!/de/variables/var-gra2009-ds1-aocc244a$}}}\\
	\begin{tabularx}{\hsize}{@{}lX}
	Datentyp: & numerisch \\
	Skalenniveau: & ordinal \\
	Zugangswege: &
	  download-cuf, 
	  download-suf, 
	  remote-desktop-suf, 
	  onsite-suf
 \\
    \end{tabularx}



    %TABLE FOR QUESTION DETAILS
    %This has to be tested and has to be improved
    %rausfinden, ob einer Variable mehrere Fragen zugeordnet werden
    %dann evtl. nur die erste verwenden oder etwas anderes tun (Hinweis mehrere Fragen, auflisten mit Link)
				%TABLE FOR QUESTION DETAILS
				\vspace*{0.5cm}
                \noindent\textbf{Frage
	                \footnote{Detailliertere Informationen zur Frage finden sich unter
		              \url{https://metadata.fdz.dzhw.eu/\#!/de/questions/que-gra2009-ins1-5.4$}}}\\
				\begin{tabularx}{\hsize}{@{}lX}
					Fragenummer: &
					  Fragebogen des DZHW-Absolventenpanels 2009 - erste Welle:
					  5.4
 \\
					%--
					Fragetext: & Im Folgenden bitten wir Sie um eine Beschreibung der verschiedenen beruflichen Tätigkeiten, die Sie seit Ihrem Studienabschluss ausgeübt haben.\par  4. Erwerbstätigkeit\par  Zeitraum (Monat/ Jahr)\par  von:\par  Monat \\
				\end{tabularx}





				%TABLE FOR THE NOMINAL / ORDINAL VALUES
        		\vspace*{0.5cm}
                \noindent\textbf{Häufigkeiten}

                \vspace*{-\baselineskip}
					%NUMERIC ELEMENTS NEED A HUGH SECOND COLOUMN AND A SMALL FIRST ONE
					\begin{filecontents}{\jobname-aocc244a}
					\begin{longtable}{lXrrr}
					\toprule
					\textbf{Wert} & \textbf{Label} & \textbf{Häufigkeit} & \textbf{Prozent(gültig)} & \textbf{Prozent} \\
					\endhead
					\midrule
					\multicolumn{5}{l}{\textbf{Gültige Werte}}\\
						%DIFFERENT OBSERVATIONS <=20

					1 &
				% TODO try size/length gt 0; take over for other passages
					\multicolumn{1}{X}{ Januar   } &


					%15 &
					  \num{15} &
					%--
					  \num[round-mode=places,round-precision=2]{8,88} &
					    \num[round-mode=places,round-precision=2]{0,14} \\
							%????

					2 &
				% TODO try size/length gt 0; take over for other passages
					\multicolumn{1}{X}{ Februar   } &


					%23 &
					  \num{23} &
					%--
					  \num[round-mode=places,round-precision=2]{13,61} &
					    \num[round-mode=places,round-precision=2]{0,22} \\
							%????

					3 &
				% TODO try size/length gt 0; take over for other passages
					\multicolumn{1}{X}{ März   } &


					%17 &
					  \num{17} &
					%--
					  \num[round-mode=places,round-precision=2]{10,06} &
					    \num[round-mode=places,round-precision=2]{0,16} \\
							%????

					4 &
				% TODO try size/length gt 0; take over for other passages
					\multicolumn{1}{X}{ April   } &


					%25 &
					  \num{25} &
					%--
					  \num[round-mode=places,round-precision=2]{14,79} &
					    \num[round-mode=places,round-precision=2]{0,24} \\
							%????

					5 &
				% TODO try size/length gt 0; take over for other passages
					\multicolumn{1}{X}{ Mai   } &


					%20 &
					  \num{20} &
					%--
					  \num[round-mode=places,round-precision=2]{11,83} &
					    \num[round-mode=places,round-precision=2]{0,19} \\
							%????

					6 &
				% TODO try size/length gt 0; take over for other passages
					\multicolumn{1}{X}{ Juni   } &


					%12 &
					  \num{12} &
					%--
					  \num[round-mode=places,round-precision=2]{7,1} &
					    \num[round-mode=places,round-precision=2]{0,11} \\
							%????

					7 &
				% TODO try size/length gt 0; take over for other passages
					\multicolumn{1}{X}{ Juli   } &


					%9 &
					  \num{9} &
					%--
					  \num[round-mode=places,round-precision=2]{5,33} &
					    \num[round-mode=places,round-precision=2]{0,09} \\
							%????

					8 &
				% TODO try size/length gt 0; take over for other passages
					\multicolumn{1}{X}{ August   } &


					%13 &
					  \num{13} &
					%--
					  \num[round-mode=places,round-precision=2]{7,69} &
					    \num[round-mode=places,round-precision=2]{0,12} \\
							%????

					9 &
				% TODO try size/length gt 0; take over for other passages
					\multicolumn{1}{X}{ September   } &


					%16 &
					  \num{16} &
					%--
					  \num[round-mode=places,round-precision=2]{9,47} &
					    \num[round-mode=places,round-precision=2]{0,15} \\
							%????

					10 &
				% TODO try size/length gt 0; take over for other passages
					\multicolumn{1}{X}{ Oktober   } &


					%8 &
					  \num{8} &
					%--
					  \num[round-mode=places,round-precision=2]{4,73} &
					    \num[round-mode=places,round-precision=2]{0,08} \\
							%????

					11 &
				% TODO try size/length gt 0; take over for other passages
					\multicolumn{1}{X}{ November   } &


					%6 &
					  \num{6} &
					%--
					  \num[round-mode=places,round-precision=2]{3,55} &
					    \num[round-mode=places,round-precision=2]{0,06} \\
							%????

					12 &
				% TODO try size/length gt 0; take over for other passages
					\multicolumn{1}{X}{ Dezember   } &


					%5 &
					  \num{5} &
					%--
					  \num[round-mode=places,round-precision=2]{2,96} &
					    \num[round-mode=places,round-precision=2]{0,05} \\
							%????
						%DIFFERENT OBSERVATIONS >20
					\midrule
					\multicolumn{2}{l}{Summe (gültig)} &
					  \textbf{\num{169}} &
					\textbf{100} &
					  \textbf{\num[round-mode=places,round-precision=2]{1,61}} \\
					%--
					\multicolumn{5}{l}{\textbf{Fehlende Werte}}\\
							-998 &
							keine Angabe &
							  \num{8237} &
							 - &
							  \num[round-mode=places,round-precision=2]{78,49} \\
							-989 &
							filterbedingt fehlend &
							  \num{2088} &
							 - &
							  \num[round-mode=places,round-precision=2]{19,9} \\
					\midrule
					\multicolumn{2}{l}{\textbf{Summe (gesamt)}} &
				      \textbf{\num{10494}} &
				    \textbf{-} &
				    \textbf{100} \\
					\bottomrule
					\end{longtable}
					\end{filecontents}
					\LTXtable{\textwidth}{\jobname-aocc244a}
				\label{tableValues:aocc244a}
				\vspace*{-\baselineskip}
                    \begin{noten}
                	    \note{} Deskritive Maßzahlen:
                	    Anzahl unterschiedlicher Beobachtungen: 12%
                	    ; 
                	      Minimum ($min$): 1; 
                	      Maximum ($max$): 12; 
                	      Median ($\tilde{x}$): 5; 
                	      Modus ($h$): 4
                     \end{noten}


