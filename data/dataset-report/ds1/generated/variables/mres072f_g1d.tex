%EVERY VARIABLE HAS IT'S OWN PAGE

    \setcounter{footnote}{0}

    %omit vertical space
    \vspace*{-1.8cm}
	\section{mres072f\_g1d (6. Wohnung: Ort (NUTS2))}
	\label{section:mres072f_g1d}



	%TABLE FOR VARIABLE DETAILS
    \vspace*{0.5cm}
    \noindent\textbf{Eigenschaften
	% '#' has to be escaped
	\footnote{Detailliertere Informationen zur Variable finden sich unter
		\url{https://metadata.fdz.dzhw.eu/\#!/de/variables/var-gra2009-ds1-mres072f_g1d$}}}\\
	\begin{tabularx}{\hsize}{@{}lX}
	Datentyp: & string \\
	Skalenniveau: & nominal \\
	Zugangswege: &
	  download-suf, 
	  remote-desktop-suf, 
	  onsite-suf
 \\
    \end{tabularx}



    %TABLE FOR QUESTION DETAILS
    %This has to be tested and has to be improved
    %rausfinden, ob einer Variable mehrere Fragen zugeordnet werden
    %dann evtl. nur die erste verwenden oder etwas anderes tun (Hinweis mehrere Fragen, auflisten mit Link)
				%TABLE FOR QUESTION DETAILS
				\vspace*{0.5cm}
                \noindent\textbf{Frage
	                \footnote{Detailliertere Informationen zur Frage finden sich unter
		              \url{https://metadata.fdz.dzhw.eu/\#!/de/questions/que-gra2009-ins5-23.1$}}}\\
				\begin{tabularx}{\hsize}{@{}lX}
					Fragenummer: &
					  Fragebogen des DZHW-Absolventenpanels 2009 - zweite Welle, Vertiefungsbefragung Mobilität:
					  23.1
 \\
					%--
					Fragetext: & Bitte nennen Sie uns nun die nächste Wohnung, in die Sie nach Ihrem Studienabschluss 2008/2009 eingezogen sind. \\
				\end{tabularx}





				%TABLE FOR THE NOMINAL / ORDINAL VALUES
        		\vspace*{0.5cm}
                \noindent\textbf{Häufigkeiten}

                \vspace*{-\baselineskip}
					%STRING ELEMENTS NEEDS A HUGH FIRST COLOUMN AND A SMALL SECOND ONE
					\begin{filecontents}{\jobname-mres072f_g1d}
					\begin{longtable}{Xlrrr}
					\toprule
					\textbf{Wert} & \textbf{Label} & \textbf{Häufigkeit} & \textbf{Prozent (gültig)} & \textbf{Prozent} \\
					\endhead
					\midrule
					\multicolumn{5}{l}{\textbf{Gültige Werte}}\\
						%DIFFERENT OBSERVATIONS <=20
								\multicolumn{1}{X}{DE11 Stuttgart} & - & 2 & 4,76 & 0,02 \\
								\multicolumn{1}{X}{DE14 Tübingen} & - & 2 & 4,76 & 0,02 \\
								\multicolumn{1}{X}{DE21 Oberbayern} & - & 5 & 11,9 & 0,05 \\
								\multicolumn{1}{X}{DE22 Niederbayern} & - & 1 & 2,38 & 0,01 \\
								\multicolumn{1}{X}{DE24 Oberfranken} & - & 1 & 2,38 & 0,01 \\
								\multicolumn{1}{X}{DE26 Unterfranken} & - & 1 & 2,38 & 0,01 \\
								\multicolumn{1}{X}{DE30 Berlin} & - & 2 & 4,76 & 0,02 \\
								\multicolumn{1}{X}{DE40 Brandenburg} & - & 1 & 2,38 & 0,01 \\
								\multicolumn{1}{X}{DE50 Bremen} & - & 2 & 4,76 & 0,02 \\
								\multicolumn{1}{X}{DE60 Hamburg} & - & 3 & 7,14 & 0,03 \\
							... & ... & ... & ... & ... \\
								\multicolumn{1}{X}{DE92 Hannover} & - & 2 & 4,76 & 0,02 \\
								\multicolumn{1}{X}{DE94 Weser-Ems} & - & 1 & 2,38 & 0,01 \\
								\multicolumn{1}{X}{DEA1 Düsseldorf} & - & 2 & 4,76 & 0,02 \\
								\multicolumn{1}{X}{DEA2 Köln} & - & 3 & 7,14 & 0,03 \\
								\multicolumn{1}{X}{DEA3 Münster} & - & 1 & 2,38 & 0,01 \\
								\multicolumn{1}{X}{DEA5 Arnsberg} & - & 1 & 2,38 & 0,01 \\
								\multicolumn{1}{X}{DEB2 Trier} & - & 1 & 2,38 & 0,01 \\
								\multicolumn{1}{X}{DED5 Leipzig} & - & 2 & 4,76 & 0,02 \\
								\multicolumn{1}{X}{DEF0 Schleswig-Holstein} & - & 3 & 7,14 & 0,03 \\
								\multicolumn{1}{X}{DEG0 Thüringen} & - & 2 & 4,76 & 0,02 \\
					\midrule
						\multicolumn{2}{l}{Summe (gültig)} & 42 &
						\textbf{100} &
					    0,4 \\
					\multicolumn{5}{l}{\textbf{Fehlende Werte}}\\
							-989 & filterbedingt fehlend & 2406 & - & 22,93 \\

							-995 & keine Teilnahme (Panel) & 8029 & - & 76,51 \\

							-998 & keine Angabe & 17 & - & 0,16 \\

					\midrule
					\multicolumn{2}{l}{\textbf{Summe (gesamt)}} & \textbf{10494} & \textbf{-} & \textbf{100} \\
					\bottomrule
					\caption{Werte der Variable mres072f\_g1d}
					\end{longtable}
					\end{filecontents}
					\LTXtable{\textwidth}{\jobname-mres072f_g1d}


