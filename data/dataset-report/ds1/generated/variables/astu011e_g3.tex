%EVERY VARIABLE HAS IT'S OWN PAGE

    \setcounter{footnote}{0}

    %omit vertical space
    \vspace*{-1.8cm}
	\section{astu011e\_g3 (1. Studium: Hauptfach (Fächergruppen))}
	\label{section:astu011e_g3}



	%TABLE FOR VARIABLE DETAILS
    \vspace*{0.5cm}
    \noindent\textbf{Eigenschaften
	% '#' has to be escaped
	\footnote{Detailliertere Informationen zur Variable finden sich unter
		\url{https://metadata.fdz.dzhw.eu/\#!/de/variables/var-gra2009-ds1-astu011e_g3$}}}\\
	\begin{tabularx}{\hsize}{@{}lX}
	Datentyp: & numerisch \\
	Skalenniveau: & nominal \\
	Zugangswege: &
	  download-cuf, 
	  download-suf, 
	  remote-desktop-suf, 
	  onsite-suf
 \\
    \end{tabularx}



    %TABLE FOR QUESTION DETAILS
    %This has to be tested and has to be improved
    %rausfinden, ob einer Variable mehrere Fragen zugeordnet werden
    %dann evtl. nur die erste verwenden oder etwas anderes tun (Hinweis mehrere Fragen, auflisten mit Link)
				%TABLE FOR QUESTION DETAILS
				\vspace*{0.5cm}
                \noindent\textbf{Frage
	                \footnote{Detailliertere Informationen zur Frage finden sich unter
		              \url{https://metadata.fdz.dzhw.eu/\#!/de/questions/que-gra2009-ins1-1.1$}}}\\
				\begin{tabularx}{\hsize}{@{}lX}
					Fragenummer: &
					  Fragebogen des DZHW-Absolventenpanels 2009 - erste Welle:
					  1.1
 \\
					%--
					Fragetext: & Bitte tragen Sie in das folgende Tableau Ihren Studienverlauf ein. \\
				\end{tabularx}





				%TABLE FOR THE NOMINAL / ORDINAL VALUES
        		\vspace*{0.5cm}
                \noindent\textbf{Häufigkeiten}

                \vspace*{-\baselineskip}
					%NUMERIC ELEMENTS NEED A HUGH SECOND COLOUMN AND A SMALL FIRST ONE
					\begin{filecontents}{\jobname-astu011e_g3}
					\begin{longtable}{lXrrr}
					\toprule
					\textbf{Wert} & \textbf{Label} & \textbf{Häufigkeit} & \textbf{Prozent(gültig)} & \textbf{Prozent} \\
					\endhead
					\midrule
					\multicolumn{5}{l}{\textbf{Gültige Werte}}\\
						%DIFFERENT OBSERVATIONS <=20

					1 &
				% TODO try size/length gt 0; take over for other passages
					\multicolumn{1}{X}{ Sprach- und Kulturwissenschaften   } &


					%2262 &
					  \num{2262} &
					%--
					  \num[round-mode=places,round-precision=2]{21,56} &
					    \num[round-mode=places,round-precision=2]{21,56} \\
							%????

					2 &
				% TODO try size/length gt 0; take over for other passages
					\multicolumn{1}{X}{ Sport   } &


					%67 &
					  \num{67} &
					%--
					  \num[round-mode=places,round-precision=2]{0,64} &
					    \num[round-mode=places,round-precision=2]{0,64} \\
							%????

					3 &
				% TODO try size/length gt 0; take over for other passages
					\multicolumn{1}{X}{ Rechts-, Wirtschafts- und Sozialwissenschaften   } &


					%3507 &
					  \num{3507} &
					%--
					  \num[round-mode=places,round-precision=2]{33,42} &
					    \num[round-mode=places,round-precision=2]{33,42} \\
							%????

					4 &
				% TODO try size/length gt 0; take over for other passages
					\multicolumn{1}{X}{ Mathematik, Naturwissenschaften   } &


					%1917 &
					  \num{1917} &
					%--
					  \num[round-mode=places,round-precision=2]{18,27} &
					    \num[round-mode=places,round-precision=2]{18,27} \\
							%????

					5 &
				% TODO try size/length gt 0; take over for other passages
					\multicolumn{1}{X}{ Humanmedizin/Gesundheitswissenschaften   } &


					%527 &
					  \num{527} &
					%--
					  \num[round-mode=places,round-precision=2]{5,02} &
					    \num[round-mode=places,round-precision=2]{5,02} \\
							%????

					6 &
				% TODO try size/length gt 0; take over for other passages
					\multicolumn{1}{X}{ Veterinärmedizin   } &


					%84 &
					  \num{84} &
					%--
					  \num[round-mode=places,round-precision=2]{0,8} &
					    \num[round-mode=places,round-precision=2]{0,8} \\
							%????

					7 &
				% TODO try size/length gt 0; take over for other passages
					\multicolumn{1}{X}{ Agrar-, Forst-, und Ernährungswissenschaften   } &


					%407 &
					  \num{407} &
					%--
					  \num[round-mode=places,round-precision=2]{3,88} &
					    \num[round-mode=places,round-precision=2]{3,88} \\
							%????

					8 &
				% TODO try size/length gt 0; take over for other passages
					\multicolumn{1}{X}{ Ingenieurwissenschaften   } &


					%1479 &
					  \num{1479} &
					%--
					  \num[round-mode=places,round-precision=2]{14,09} &
					    \num[round-mode=places,round-precision=2]{14,09} \\
							%????

					9 &
				% TODO try size/length gt 0; take over for other passages
					\multicolumn{1}{X}{ Kunst, Kunstwissenschaft   } &


					%244 &
					  \num{244} &
					%--
					  \num[round-mode=places,round-precision=2]{2,33} &
					    \num[round-mode=places,round-precision=2]{2,33} \\
							%????
						%DIFFERENT OBSERVATIONS >20
					\midrule
					\multicolumn{2}{l}{Summe (gültig)} &
					  \textbf{\num{10494}} &
					\textbf{100} &
					  \textbf{\num[round-mode=places,round-precision=2]{100}} \\
					%--
					\multicolumn{5}{l}{\textbf{Fehlende Werte}}\\
						& & 0 & 0 & 0 \\
					\midrule
					\multicolumn{2}{l}{\textbf{Summe (gesamt)}} &
				      \textbf{\num{10494}} &
				    \textbf{-} &
				    \textbf{100} \\
					\bottomrule
					\end{longtable}
					\end{filecontents}
					\LTXtable{\textwidth}{\jobname-astu011e_g3}
				\label{tableValues:astu011e_g3}
				\vspace*{-\baselineskip}
                    \begin{noten}
                	    \note{} Deskritive Maßzahlen:
                	    Anzahl unterschiedlicher Beobachtungen: 9%
                	    ; 
                	      Modus ($h$): 3
                     \end{noten}


