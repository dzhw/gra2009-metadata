%EVERY VARIABLE HAS IT'S OWN PAGE

    \setcounter{footnote}{0}

    %omit vertical space
    \vspace*{-1.8cm}
	\section{bocc50a (Selbstständigkeit: Anzahl Mitarbeiter(innen))}
	\label{section:bocc50a}



	%TABLE FOR VARIABLE DETAILS
    \vspace*{0.5cm}
    \noindent\textbf{Eigenschaften
	% '#' has to be escaped
	\footnote{Detailliertere Informationen zur Variable finden sich unter
		\url{https://metadata.fdz.dzhw.eu/\#!/de/variables/var-gra2009-ds1-bocc50a$}}}\\
	\begin{tabularx}{\hsize}{@{}lX}
	Datentyp: & numerisch \\
	Skalenniveau: & nominal \\
	Zugangswege: &
	  download-cuf, 
	  download-suf, 
	  remote-desktop-suf, 
	  onsite-suf
 \\
    \end{tabularx}



    %TABLE FOR QUESTION DETAILS
    %This has to be tested and has to be improved
    %rausfinden, ob einer Variable mehrere Fragen zugeordnet werden
    %dann evtl. nur die erste verwenden oder etwas anderes tun (Hinweis mehrere Fragen, auflisten mit Link)
				%TABLE FOR QUESTION DETAILS
				\vspace*{0.5cm}
                \noindent\textbf{Frage
	                \footnote{Detailliertere Informationen zur Frage finden sich unter
		              \url{https://metadata.fdz.dzhw.eu/\#!/de/questions/que-gra2009-ins2-4.7$}}}\\
				\begin{tabularx}{\hsize}{@{}lX}
					Fragenummer: &
					  Fragebogen des DZHW-Absolventenpanels 2009 - zweite Welle, Hauptbefragung (PAPI):
					  4.7
 \\
					%--
					Fragetext: & Beschäftigen Sie fest angestellte Mitarbeiter(innen)?\par  500 und mehr Mitarbeiter(innen)\par  250 bis 499 Mitarbeiter(innen)\par  100 bis 249 Mitarbeiter(innen)\par  50 bis 99 Mitarbeiter(innen)\par  20 bis 49 Mitarbeiter(innen)\par  10 bis 19 Mitarbeiter(innen)\par  5 bis 9 Mitarbeiter(innen)\par  Unter 5 Mitarbeiter(innen)\par  Freischaffend, ohne Mitarbeiter(innen)\par  Sonstiges \\
				\end{tabularx}
				%TABLE FOR QUESTION DETAILS
				\vspace*{0.5cm}
                \noindent\textbf{Frage
	                \footnote{Detailliertere Informationen zur Frage finden sich unter
		              \url{https://metadata.fdz.dzhw.eu/\#!/de/questions/que-gra2009-ins3-22$}}}\\
				\begin{tabularx}{\hsize}{@{}lX}
					Fragenummer: &
					  Fragebogen des DZHW-Absolventenpanels 2009 - zweite Welle, Hauptbefragung (CAWI):
					  22
 \\
					%--
					Fragetext: & Beschäftigen Sie fest angestellte Mitarbeiter(innen)? \\
				\end{tabularx}





				%TABLE FOR THE NOMINAL / ORDINAL VALUES
        		\vspace*{0.5cm}
                \noindent\textbf{Häufigkeiten}

                \vspace*{-\baselineskip}
					%NUMERIC ELEMENTS NEED A HUGH SECOND COLOUMN AND A SMALL FIRST ONE
					\begin{filecontents}{\jobname-bocc50a}
					\begin{longtable}{lXrrr}
					\toprule
					\textbf{Wert} & \textbf{Label} & \textbf{Häufigkeit} & \textbf{Prozent(gültig)} & \textbf{Prozent} \\
					\endhead
					\midrule
					\multicolumn{5}{l}{\textbf{Gültige Werte}}\\
						%DIFFERENT OBSERVATIONS <=20

					1 &
				% TODO try size/length gt 0; take over for other passages
					\multicolumn{1}{X}{ 500 und mehr Mitarbeiter(innen)   } &


					%2 &
					  \num{2} &
					%--
					  \num[round-mode=places,round-precision=2]{0,62} &
					    \num[round-mode=places,round-precision=2]{0,02} \\
							%????

					5 &
				% TODO try size/length gt 0; take over for other passages
					\multicolumn{1}{X}{ 20 bis 49 Mitarbeiter(innen)   } &


					%5 &
					  \num{5} &
					%--
					  \num[round-mode=places,round-precision=2]{1,55} &
					    \num[round-mode=places,round-precision=2]{0,05} \\
							%????

					6 &
				% TODO try size/length gt 0; take over for other passages
					\multicolumn{1}{X}{ 10 bis 19 Mitarbeiter(innen)   } &


					%13 &
					  \num{13} &
					%--
					  \num[round-mode=places,round-precision=2]{4,02} &
					    \num[round-mode=places,round-precision=2]{0,12} \\
							%????

					7 &
				% TODO try size/length gt 0; take over for other passages
					\multicolumn{1}{X}{ 5 bis 9 Mitarbeiter(innen)   } &


					%12 &
					  \num{12} &
					%--
					  \num[round-mode=places,round-precision=2]{3,72} &
					    \num[round-mode=places,round-precision=2]{0,11} \\
							%????

					8 &
				% TODO try size/length gt 0; take over for other passages
					\multicolumn{1}{X}{ unter 5 Mitarbeiter(innen)   } &


					%44 &
					  \num{44} &
					%--
					  \num[round-mode=places,round-precision=2]{13,62} &
					    \num[round-mode=places,round-precision=2]{0,42} \\
							%????

					9 &
				% TODO try size/length gt 0; take over for other passages
					\multicolumn{1}{X}{ freischaffend   } &


					%245 &
					  \num{245} &
					%--
					  \num[round-mode=places,round-precision=2]{75,85} &
					    \num[round-mode=places,round-precision=2]{2,33} \\
							%????

					10 &
				% TODO try size/length gt 0; take over for other passages
					\multicolumn{1}{X}{ Sonstiges   } &


					%2 &
					  \num{2} &
					%--
					  \num[round-mode=places,round-precision=2]{0,62} &
					    \num[round-mode=places,round-precision=2]{0,02} \\
							%????
						%DIFFERENT OBSERVATIONS >20
					\midrule
					\multicolumn{2}{l}{Summe (gültig)} &
					  \textbf{\num{323}} &
					\textbf{100} &
					  \textbf{\num[round-mode=places,round-precision=2]{3,08}} \\
					%--
					\multicolumn{5}{l}{\textbf{Fehlende Werte}}\\
							-998 &
							keine Angabe &
							  \num{43} &
							 - &
							  \num[round-mode=places,round-precision=2]{0,41} \\
							-995 &
							keine Teilnahme (Panel) &
							  \num{5739} &
							 - &
							  \num[round-mode=places,round-precision=2]{54,69} \\
							-989 &
							filterbedingt fehlend &
							  \num{4389} &
							 - &
							  \num[round-mode=places,round-precision=2]{41,82} \\
					\midrule
					\multicolumn{2}{l}{\textbf{Summe (gesamt)}} &
				      \textbf{\num{10494}} &
				    \textbf{-} &
				    \textbf{100} \\
					\bottomrule
					\end{longtable}
					\end{filecontents}
					\LTXtable{\textwidth}{\jobname-bocc50a}
				\label{tableValues:bocc50a}
				\vspace*{-\baselineskip}
                    \begin{noten}
                	    \note{} Deskritive Maßzahlen:
                	    Anzahl unterschiedlicher Beobachtungen: 7%
                	    ; 
                	      Modus ($h$): 9
                     \end{noten}


