%EVERY VARIABLE HAS IT'S OWN PAGE

    \setcounter{footnote}{0}

    %omit vertical space
    \vspace*{-1.8cm}
	\section{afec021l\_g2o (1. weitere akad. Qualifikation: 1. Hochschule (NUTS2))}
	\label{section:afec021l_g2o}



	%TABLE FOR VARIABLE DETAILS
    \vspace*{0.5cm}
    \noindent\textbf{Eigenschaften
	% '#' has to be escaped
	\footnote{Detailliertere Informationen zur Variable finden sich unter
		\url{https://metadata.fdz.dzhw.eu/\#!/de/variables/var-gra2009-ds1-afec021l_g2o$}}}\\
	\begin{tabularx}{\hsize}{@{}lX}
	Datentyp: & string \\
	Skalenniveau: & nominal \\
	Zugangswege: &
	  onsite-suf
 \\
    \end{tabularx}



    %TABLE FOR QUESTION DETAILS
    %This has to be tested and has to be improved
    %rausfinden, ob einer Variable mehrere Fragen zugeordnet werden
    %dann evtl. nur die erste verwenden oder etwas anderes tun (Hinweis mehrere Fragen, auflisten mit Link)
				%TABLE FOR QUESTION DETAILS
				\vspace*{0.5cm}
                \noindent\textbf{Frage
	                \footnote{Detailliertere Informationen zur Frage finden sich unter
		              \url{https://metadata.fdz.dzhw.eu/\#!/de/questions/que-gra2009-ins1-2.1$}}}\\
				\begin{tabularx}{\hsize}{@{}lX}
					Fragenummer: &
					  Fragebogen des DZHW-Absolventenpanels 2009 - erste Welle:
					  2.1
 \\
					%--
					Fragetext: & Bitte tragen Sie alle weiteren akademischen Qualifizierungen, die Sie begonnen, abgeschlossen oder abgebrochen haben oder die Sie beabsichtigen, in das folgende Tableau ein. \\
				\end{tabularx}





				%TABLE FOR THE NOMINAL / ORDINAL VALUES
        		\vspace*{0.5cm}
                \noindent\textbf{Häufigkeiten}

                \vspace*{-\baselineskip}
					%STRING ELEMENTS NEEDS A HUGH FIRST COLOUMN AND A SMALL SECOND ONE
					\begin{filecontents}{\jobname-afec021l_g2o}
					\begin{longtable}{Xlrrr}
					\toprule
					\textbf{Wert} & \textbf{Label} & \textbf{Häufigkeit} & \textbf{Prozent (gültig)} & \textbf{Prozent} \\
					\endhead
					\midrule
					\multicolumn{5}{l}{\textbf{Gültige Werte}}\\
						%DIFFERENT OBSERVATIONS <=20
								\multicolumn{1}{X}{DE11 Stuttgart} & - & 187 & 3,65 & 1,78 \\
								\multicolumn{1}{X}{DE12 Karlsruhe} & - & 149 & 2,91 & 1,42 \\
								\multicolumn{1}{X}{DE13 Freiburg} & - & 76 & 1,48 & 0,72 \\
								\multicolumn{1}{X}{DE14 Tübingen} & - & 116 & 2,27 & 1,11 \\
								\multicolumn{1}{X}{DE21 Oberbayern} & - & 393 & 7,67 & 3,74 \\
								\multicolumn{1}{X}{DE22 Niederbayern} & - & 53 & 1,03 & 0,51 \\
								\multicolumn{1}{X}{DE23 Oberpfalz} & - & 107 & 2,09 & 1,02 \\
								\multicolumn{1}{X}{DE24 Oberfranken} & - & 97 & 1,89 & 0,92 \\
								\multicolumn{1}{X}{DE25 Mittelfranken} & - & 114 & 2,23 & 1,09 \\
								\multicolumn{1}{X}{DE26 Unterfranken} & - & 15 & 0,29 & 0,14 \\
							... & ... & ... & ... & ... \\
								\multicolumn{1}{X}{DEB1 Koblenz} & - & 51 & 1 & 0,49 \\
								\multicolumn{1}{X}{DEB2 Trier} & - & 35 & 0,68 & 0,33 \\
								\multicolumn{1}{X}{DEB3 Rheinhessen-Pfalz} & - & 94 & 1,84 & 0,9 \\
								\multicolumn{1}{X}{DEC0 Saarland} & - & 35 & 0,68 & 0,33 \\
								\multicolumn{1}{X}{DED2 Dresden} & - & 122 & 2,38 & 1,16 \\
								\multicolumn{1}{X}{DED4 Chemnitz} & - & 104 & 2,03 & 0,99 \\
								\multicolumn{1}{X}{DED5 Leipzig} & - & 91 & 1,78 & 0,87 \\
								\multicolumn{1}{X}{DEE0 Sachsen-Anhalt} & - & 125 & 2,44 & 1,19 \\
								\multicolumn{1}{X}{DEF0 Schleswig-Holstein} & - & 145 & 2,83 & 1,38 \\
								\multicolumn{1}{X}{DEG0 Thüringen} & - & 307 & 5,99 & 2,93 \\
					\midrule
						\multicolumn{2}{l}{Summe (gültig)} & 5121 &
						\textbf{100} &
					    48,8 \\
					\multicolumn{5}{l}{\textbf{Fehlende Werte}}\\
							-966 & nicht bestimmbar & 823 & - & 7,84 \\

							-989 & filterbedingt fehlend & 4527 & - & 43,14 \\

							-998 & keine Angabe & 23 & - & 0,22 \\

					\midrule
					\multicolumn{2}{l}{\textbf{Summe (gesamt)}} & \textbf{10494} & \textbf{-} & \textbf{100} \\
					\bottomrule
					\caption{Werte der Variable afec021l\_g2o}
					\end{longtable}
					\end{filecontents}
					\LTXtable{\textwidth}{\jobname-afec021l_g2o}


