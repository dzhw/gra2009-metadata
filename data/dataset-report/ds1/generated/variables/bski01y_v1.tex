%EVERY VARIABLE HAS IT'S OWN PAGE

    \setcounter{footnote}{0}

    %omit vertical space
    \vspace*{-1.8cm}
	\section{bski01y\_v1 (wichtig für Beruf: Auswirkung auf Natur und Gesellschaft)}
	\label{section:bski01y_v1}



	%TABLE FOR VARIABLE DETAILS
    \vspace*{0.5cm}
    \noindent\textbf{Eigenschaften
	% '#' has to be escaped
	\footnote{Detailliertere Informationen zur Variable finden sich unter
		\url{https://metadata.fdz.dzhw.eu/\#!/de/variables/var-gra2009-ds1-bski01y_v1$}}}\\
	\begin{tabularx}{\hsize}{@{}lX}
	Datentyp: & numerisch \\
	Skalenniveau: & ordinal \\
	Zugangswege: &
	  download-cuf, 
	  download-suf, 
	  remote-desktop-suf, 
	  onsite-suf
 \\
    \end{tabularx}



    %TABLE FOR QUESTION DETAILS
    %This has to be tested and has to be improved
    %rausfinden, ob einer Variable mehrere Fragen zugeordnet werden
    %dann evtl. nur die erste verwenden oder etwas anderes tun (Hinweis mehrere Fragen, auflisten mit Link)
				%TABLE FOR QUESTION DETAILS
				\vspace*{0.5cm}
                \noindent\textbf{Frage
	                \footnote{Detailliertere Informationen zur Frage finden sich unter
		              \url{https://metadata.fdz.dzhw.eu/\#!/de/questions/que-gra2009-ins2-1.6$}}}\\
				\begin{tabularx}{\hsize}{@{}lX}
					Fragenummer: &
					  Fragebogen des DZHW-Absolventenpanels 2009 - zweite Welle, Hauptbefragung (PAPI):
					  1.6
 \\
					%--
					Fragetext: & Wie wichtig sind die folgenden Kenntnisse und Fähigkeiten für Ihre derzeitige (bzw. letzte, wenn Sie nicht berufstätig sind) berufliche Tätigkeit?\par  Wissen über die Auswirkungen meiner Arbeit auf Natur und Gesellschaft \\
				\end{tabularx}
				%TABLE FOR QUESTION DETAILS
				\vspace*{0.5cm}
                \noindent\textbf{Frage
	                \footnote{Detailliertere Informationen zur Frage finden sich unter
		              \url{https://metadata.fdz.dzhw.eu/\#!/de/questions/que-gra2009-ins3-07$}}}\\
				\begin{tabularx}{\hsize}{@{}lX}
					Fragenummer: &
					  Fragebogen des DZHW-Absolventenpanels 2009 - zweite Welle, Hauptbefragung (CAWI):
					  07
 \\
					%--
					Fragetext: & Wie wichtig sind die folgenden Kenntnisse und Fähigkeiten für Ihre derzeitige (bzw. letzte, wenn Sie nicht berufstätig sind) berufliche Tätigkeit? \\
				\end{tabularx}





				%TABLE FOR THE NOMINAL / ORDINAL VALUES
        		\vspace*{0.5cm}
                \noindent\textbf{Häufigkeiten}

                \vspace*{-\baselineskip}
					%NUMERIC ELEMENTS NEED A HUGH SECOND COLOUMN AND A SMALL FIRST ONE
					\begin{filecontents}{\jobname-bski01y_v1}
					\begin{longtable}{lXrrr}
					\toprule
					\textbf{Wert} & \textbf{Label} & \textbf{Häufigkeit} & \textbf{Prozent(gültig)} & \textbf{Prozent} \\
					\endhead
					\midrule
					\multicolumn{5}{l}{\textbf{Gültige Werte}}\\
						%DIFFERENT OBSERVATIONS <=20

					1 &
				% TODO try size/length gt 0; take over for other passages
					\multicolumn{1}{X}{ in hohem Maße   } &


					%568 &
					  \num{568} &
					%--
					  \num[round-mode=places,round-precision=2]{12,03} &
					    \num[round-mode=places,round-precision=2]{5,41} \\
							%????

					2 &
				% TODO try size/length gt 0; take over for other passages
					\multicolumn{1}{X}{ 2   } &


					%1022 &
					  \num{1022} &
					%--
					  \num[round-mode=places,round-precision=2]{21,64} &
					    \num[round-mode=places,round-precision=2]{9,74} \\
							%????

					3 &
				% TODO try size/length gt 0; take over for other passages
					\multicolumn{1}{X}{ 3   } &


					%1271 &
					  \num{1271} &
					%--
					  \num[round-mode=places,round-precision=2]{26,91} &
					    \num[round-mode=places,round-precision=2]{12,11} \\
							%????

					4 &
				% TODO try size/length gt 0; take over for other passages
					\multicolumn{1}{X}{ 4   } &


					%1143 &
					  \num{1143} &
					%--
					  \num[round-mode=places,round-precision=2]{24,2} &
					    \num[round-mode=places,round-precision=2]{10,89} \\
							%????

					5 &
				% TODO try size/length gt 0; take over for other passages
					\multicolumn{1}{X}{ überhaupt nicht   } &


					%719 &
					  \num{719} &
					%--
					  \num[round-mode=places,round-precision=2]{15,22} &
					    \num[round-mode=places,round-precision=2]{6,85} \\
							%????
						%DIFFERENT OBSERVATIONS >20
					\midrule
					\multicolumn{2}{l}{Summe (gültig)} &
					  \textbf{\num{4723}} &
					\textbf{100} &
					  \textbf{\num[round-mode=places,round-precision=2]{45,01}} \\
					%--
					\multicolumn{5}{l}{\textbf{Fehlende Werte}}\\
							-998 &
							keine Angabe &
							  \num{32} &
							 - &
							  \num[round-mode=places,round-precision=2]{0,3} \\
							-995 &
							keine Teilnahme (Panel) &
							  \num{5739} &
							 - &
							  \num[round-mode=places,round-precision=2]{54,69} \\
					\midrule
					\multicolumn{2}{l}{\textbf{Summe (gesamt)}} &
				      \textbf{\num{10494}} &
				    \textbf{-} &
				    \textbf{100} \\
					\bottomrule
					\end{longtable}
					\end{filecontents}
					\LTXtable{\textwidth}{\jobname-bski01y_v1}
				\label{tableValues:bski01y_v1}
				\vspace*{-\baselineskip}
                    \begin{noten}
                	    \note{} Deskritive Maßzahlen:
                	    Anzahl unterschiedlicher Beobachtungen: 5%
                	    ; 
                	      Minimum ($min$): 1; 
                	      Maximum ($max$): 5; 
                	      Median ($\tilde{x}$): 3; 
                	      Modus ($h$): 3
                     \end{noten}


