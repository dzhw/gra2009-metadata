%EVERY VARIABLE HAS IT'S OWN PAGE

    \setcounter{footnote}{0}

    %omit vertical space
    \vspace*{-1.8cm}
	\section{bfvt065e (mehrstündige berufl. Weiterbildung: Inhalt 3)}
	\label{section:bfvt065e}



	%TABLE FOR VARIABLE DETAILS
    \vspace*{0.5cm}
    \noindent\textbf{Eigenschaften
	% '#' has to be escaped
	\footnote{Detailliertere Informationen zur Variable finden sich unter
		\url{https://metadata.fdz.dzhw.eu/\#!/de/variables/var-gra2009-ds1-bfvt065e$}}}\\
	\begin{tabularx}{\hsize}{@{}lX}
	Datentyp: & numerisch \\
	Skalenniveau: & nominal \\
	Zugangswege: &
	  download-cuf, 
	  download-suf, 
	  remote-desktop-suf, 
	  onsite-suf
 \\
    \end{tabularx}



    %TABLE FOR QUESTION DETAILS
    %This has to be tested and has to be improved
    %rausfinden, ob einer Variable mehrere Fragen zugeordnet werden
    %dann evtl. nur die erste verwenden oder etwas anderes tun (Hinweis mehrere Fragen, auflisten mit Link)
				%TABLE FOR QUESTION DETAILS
				\vspace*{0.5cm}
                \noindent\textbf{Frage
	                \footnote{Detailliertere Informationen zur Frage finden sich unter
		              \url{https://metadata.fdz.dzhw.eu/\#!/de/questions/que-gra2009-ins2-6.5$}}}\\
				\begin{tabularx}{\hsize}{@{}lX}
					Fragenummer: &
					  Fragebogen des DZHW-Absolventenpanels 2009 - zweite Welle, Hauptbefragung (PAPI):
					  6.5
 \\
					%--
					Fragetext: & Im Folgenden bitten wir Sie um Angaben zu beruflichen Fort- und Weiterbildungen der letzten 12 Monate. Bitte denken Sie dabei an alle Weiterbildungen, die Sie besucht haben und geben sie diese in der passenden Zeile an.\par  5. Fort- /oder Weiterbildung\par  Themen (Mehrfachnennung möglich)\par  Schlüssel s. Klappliste B) \\
				\end{tabularx}
				%TABLE FOR QUESTION DETAILS
				\vspace*{0.5cm}
                \noindent\textbf{Frage
	                \footnote{Detailliertere Informationen zur Frage finden sich unter
		              \url{https://metadata.fdz.dzhw.eu/\#!/de/questions/que-gra2009-ins3-75$}}}\\
				\begin{tabularx}{\hsize}{@{}lX}
					Fragenummer: &
					  Fragebogen des DZHW-Absolventenpanels 2009 - zweite Welle, Hauptbefragung (CAWI):
					  75
 \\
					%--
					Fragetext: & Bitte tragen Sie hier die für Sie wichtigsten Themen bzw. Fachgebiete dieser Veranstaltungen ein. \\
				\end{tabularx}





				%TABLE FOR THE NOMINAL / ORDINAL VALUES
        		\vspace*{0.5cm}
                \noindent\textbf{Häufigkeiten}

                \vspace*{-\baselineskip}
					%NUMERIC ELEMENTS NEED A HUGH SECOND COLOUMN AND A SMALL FIRST ONE
					\begin{filecontents}{\jobname-bfvt065e}
					\begin{longtable}{lXrrr}
					\toprule
					\textbf{Wert} & \textbf{Label} & \textbf{Häufigkeit} & \textbf{Prozent(gültig)} & \textbf{Prozent} \\
					\endhead
					\midrule
					\multicolumn{5}{l}{\textbf{Gültige Werte}}\\
						%DIFFERENT OBSERVATIONS <=20
								1 & \multicolumn{1}{X}{ingenieurwissenschaftliche Themen} & %15 &
								  \num{15} &
								%--
								  \num[round-mode=places,round-precision=2]{3,33} &
								  \num[round-mode=places,round-precision=2]{0,14} \\
								2 & \multicolumn{1}{X}{naturwissenschaftliche Themen} & %10 &
								  \num{10} &
								%--
								  \num[round-mode=places,round-precision=2]{2,22} &
								  \num[round-mode=places,round-precision=2]{0,1} \\
								3 & \multicolumn{1}{X}{mathematische Gebiete/Statistik} & %5 &
								  \num{5} &
								%--
								  \num[round-mode=places,round-precision=2]{1,11} &
								  \num[round-mode=places,round-precision=2]{0,05} \\
								4 & \multicolumn{1}{X}{sozialwissenschaftliche Themen} & %20 &
								  \num{20} &
								%--
								  \num[round-mode=places,round-precision=2]{4,43} &
								  \num[round-mode=places,round-precision=2]{0,19} \\
								5 & \multicolumn{1}{X}{geisteswissenschtliche Themen} & %9 &
								  \num{9} &
								%--
								  \num[round-mode=places,round-precision=2]{2} &
								  \num[round-mode=places,round-precision=2]{0,09} \\
								6 & \multicolumn{1}{X}{pädagogische/psychologische Themen} & %62 &
								  \num{62} &
								%--
								  \num[round-mode=places,round-precision=2]{13,75} &
								  \num[round-mode=places,round-precision=2]{0,59} \\
								7 & \multicolumn{1}{X}{medizinische Spezialgebiete} & %31 &
								  \num{31} &
								%--
								  \num[round-mode=places,round-precision=2]{6,87} &
								  \num[round-mode=places,round-precision=2]{0,3} \\
								8 & \multicolumn{1}{X}{informationstechnisches Spezialwissen} & %18 &
								  \num{18} &
								%--
								  \num[round-mode=places,round-precision=2]{3,99} &
								  \num[round-mode=places,round-precision=2]{0,17} \\
								9 & \multicolumn{1}{X}{Managementwissen} & %17 &
								  \num{17} &
								%--
								  \num[round-mode=places,round-precision=2]{3,77} &
								  \num[round-mode=places,round-precision=2]{0,16} \\
								10 & \multicolumn{1}{X}{Wirtschaftskenntnisse} & %17 &
								  \num{17} &
								%--
								  \num[round-mode=places,round-precision=2]{3,77} &
								  \num[round-mode=places,round-precision=2]{0,16} \\
							... & ... & ... & ... & ... \\
								14 & \multicolumn{1}{X}{Vetriebsschulungen} & %11 &
								  \num{11} &
								%--
								  \num[round-mode=places,round-precision=2]{2,44} &
								  \num[round-mode=places,round-precision=2]{0,1} \\

								15 & \multicolumn{1}{X}{EDV-Anwendungen} & %49 &
								  \num{49} &
								%--
								  \num[round-mode=places,round-precision=2]{10,86} &
								  \num[round-mode=places,round-precision=2]{0,47} \\

								16 & \multicolumn{1}{X}{Fremdsprachen} & %16 &
								  \num{16} &
								%--
								  \num[round-mode=places,round-precision=2]{3,55} &
								  \num[round-mode=places,round-precision=2]{0,15} \\

								17 & \multicolumn{1}{X}{Mitarbeiterführung/Personalentwicklung} & %14 &
								  \num{14} &
								%--
								  \num[round-mode=places,round-precision=2]{3,1} &
								  \num[round-mode=places,round-precision=2]{0,13} \\

								18 & \multicolumn{1}{X}{Kommunikations-/Interaktionstraining} & %44 &
								  \num{44} &
								%--
								  \num[round-mode=places,round-precision=2]{9,76} &
								  \num[round-mode=places,round-precision=2]{0,42} \\

								19 & \multicolumn{1}{X}{internationale Beziehungen, Kulturkenntnisse, Landeskunde} & %10 &
								  \num{10} &
								%--
								  \num[round-mode=places,round-precision=2]{2,22} &
								  \num[round-mode=places,round-precision=2]{0,1} \\

								20 & \multicolumn{1}{X}{ökologische Themen} & %2 &
								  \num{2} &
								%--
								  \num[round-mode=places,round-precision=2]{0,44} &
								  \num[round-mode=places,round-precision=2]{0,02} \\

								21 & \multicolumn{1}{X}{berufsethische Themen} & %10 &
								  \num{10} &
								%--
								  \num[round-mode=places,round-precision=2]{2,22} &
								  \num[round-mode=places,round-precision=2]{0,1} \\

								23 & \multicolumn{1}{X}{betriebliches Gesundheitswesen, Arbeitssicherheit} & %28 &
								  \num{28} &
								%--
								  \num[round-mode=places,round-precision=2]{6,21} &
								  \num[round-mode=places,round-precision=2]{0,27} \\

								24 & \multicolumn{1}{X}{Sonstige} & %15 &
								  \num{15} &
								%--
								  \num[round-mode=places,round-precision=2]{3,33} &
								  \num[round-mode=places,round-precision=2]{0,14} \\

					\midrule
					\multicolumn{2}{l}{Summe (gültig)} &
					  \textbf{\num{451}} &
					\textbf{100} &
					  \textbf{\num[round-mode=places,round-precision=2]{4,3}} \\
					%--
					\multicolumn{5}{l}{\textbf{Fehlende Werte}}\\
							-998 &
							keine Angabe &
							  \num{4304} &
							 - &
							  \num[round-mode=places,round-precision=2]{41,01} \\
							-995 &
							keine Teilnahme (Panel) &
							  \num{5739} &
							 - &
							  \num[round-mode=places,round-precision=2]{54,69} \\
					\midrule
					\multicolumn{2}{l}{\textbf{Summe (gesamt)}} &
				      \textbf{\num{10494}} &
				    \textbf{-} &
				    \textbf{100} \\
					\bottomrule
					\end{longtable}
					\end{filecontents}
					\LTXtable{\textwidth}{\jobname-bfvt065e}
				\label{tableValues:bfvt065e}
				\vspace*{-\baselineskip}
                    \begin{noten}
                	    \note{} Deskritive Maßzahlen:
                	    Anzahl unterschiedlicher Beobachtungen: 23%
                	    ; 
                	      Modus ($h$): 6
                     \end{noten}


