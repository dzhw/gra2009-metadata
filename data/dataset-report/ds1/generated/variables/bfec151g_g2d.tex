%EVERY VARIABLE HAS IT'S OWN PAGE

    \setcounter{footnote}{0}

    %omit vertical space
    \vspace*{-1.8cm}
	\section{bfec151g\_g2d (1. weitere akad. Qualifikation: Studienfach (Studienbereiche))}
	\label{section:bfec151g_g2d}



	%TABLE FOR VARIABLE DETAILS
    \vspace*{0.5cm}
    \noindent\textbf{Eigenschaften
	% '#' has to be escaped
	\footnote{Detailliertere Informationen zur Variable finden sich unter
		\url{https://metadata.fdz.dzhw.eu/\#!/de/variables/var-gra2009-ds1-bfec151g_g2d$}}}\\
	\begin{tabularx}{\hsize}{@{}lX}
	Datentyp: & numerisch \\
	Skalenniveau: & nominal \\
	Zugangswege: &
	  download-suf, 
	  remote-desktop-suf, 
	  onsite-suf
 \\
    \end{tabularx}



    %TABLE FOR QUESTION DETAILS
    %This has to be tested and has to be improved
    %rausfinden, ob einer Variable mehrere Fragen zugeordnet werden
    %dann evtl. nur die erste verwenden oder etwas anderes tun (Hinweis mehrere Fragen, auflisten mit Link)
				%TABLE FOR QUESTION DETAILS
				\vspace*{0.5cm}
                \noindent\textbf{Frage
	                \footnote{Detailliertere Informationen zur Frage finden sich unter
		              \url{https://metadata.fdz.dzhw.eu/\#!/de/questions/que-gra2009-ins2-5.2$}}}\\
				\begin{tabularx}{\hsize}{@{}lX}
					Fragenummer: &
					  Fragebogen des DZHW-Absolventenpanels 2009 - zweite Welle, Hauptbefragung (PAPI):
					  5.2
 \\
					%--
					Fragetext: & Bitte tragen Sie diese längerfristigen Studienangebote, die Sie nach Ihrem Studienabschluss aus dem Jahr 2008/2009 begonnen, weitergeführt oder abgeschlossen haben (auch abgebrochene oder unterbrochene), in das folgende Tableau ein! \\
				\end{tabularx}





				%TABLE FOR THE NOMINAL / ORDINAL VALUES
        		\vspace*{0.5cm}
                \noindent\textbf{Häufigkeiten}

                \vspace*{-\baselineskip}
					%NUMERIC ELEMENTS NEED A HUGH SECOND COLOUMN AND A SMALL FIRST ONE
					\begin{filecontents}{\jobname-bfec151g_g2d}
					\begin{longtable}{lXrrr}
					\toprule
					\textbf{Wert} & \textbf{Label} & \textbf{Häufigkeit} & \textbf{Prozent(gültig)} & \textbf{Prozent} \\
					\endhead
					\midrule
					\multicolumn{5}{l}{\textbf{Gültige Werte}}\\
						%DIFFERENT OBSERVATIONS <=20
								1 & \multicolumn{1}{X}{Sprach- und Kulturwissenschaften allgemein} & %52 &
								  \num{52} &
								%--
								  \num[round-mode=places,round-precision=2]{2,94} &
								  \num[round-mode=places,round-precision=2]{0,5} \\
								2 & \multicolumn{1}{X}{Evang. Theologie, -Religionslehre} & %7 &
								  \num{7} &
								%--
								  \num[round-mode=places,round-precision=2]{0,4} &
								  \num[round-mode=places,round-precision=2]{0,07} \\
								3 & \multicolumn{1}{X}{Kath. Theologie, -Religionslehre} & %3 &
								  \num{3} &
								%--
								  \num[round-mode=places,round-precision=2]{0,17} &
								  \num[round-mode=places,round-precision=2]{0,03} \\
								4 & \multicolumn{1}{X}{Philosophie} & %6 &
								  \num{6} &
								%--
								  \num[round-mode=places,round-precision=2]{0,34} &
								  \num[round-mode=places,round-precision=2]{0,06} \\
								5 & \multicolumn{1}{X}{Geschichte} & %17 &
								  \num{17} &
								%--
								  \num[round-mode=places,round-precision=2]{0,96} &
								  \num[round-mode=places,round-precision=2]{0,16} \\
								6 & \multicolumn{1}{X}{Bibliothekswissenschaft, Dokumentation} & %2 &
								  \num{2} &
								%--
								  \num[round-mode=places,round-precision=2]{0,11} &
								  \num[round-mode=places,round-precision=2]{0,02} \\
								7 & \multicolumn{1}{X}{Allgemeine und vergleichende Literatur- und Sprachwissenschaft} & %15 &
								  \num{15} &
								%--
								  \num[round-mode=places,round-precision=2]{0,85} &
								  \num[round-mode=places,round-precision=2]{0,14} \\
								8 & \multicolumn{1}{X}{Altphilologie (klass. Philologie), Neugriechisch} & %3 &
								  \num{3} &
								%--
								  \num[round-mode=places,round-precision=2]{0,17} &
								  \num[round-mode=places,round-precision=2]{0,03} \\
								9 & \multicolumn{1}{X}{Germanistik (Deutsch, germanische Sprachen ohne Anglistik)} & %26 &
								  \num{26} &
								%--
								  \num[round-mode=places,round-precision=2]{1,47} &
								  \num[round-mode=places,round-precision=2]{0,25} \\
								10 & \multicolumn{1}{X}{Anglistik, Amerikanistik} & %19 &
								  \num{19} &
								%--
								  \num[round-mode=places,round-precision=2]{1,08} &
								  \num[round-mode=places,round-precision=2]{0,18} \\
							... & ... & ... & ... & ... \\
								66 & \multicolumn{1}{X}{Architektur, Innenarchitektur} & %29 &
								  \num{29} &
								%--
								  \num[round-mode=places,round-precision=2]{1,64} &
								  \num[round-mode=places,round-precision=2]{0,28} \\

								67 & \multicolumn{1}{X}{Raumplanung} & %21 &
								  \num{21} &
								%--
								  \num[round-mode=places,round-precision=2]{1,19} &
								  \num[round-mode=places,round-precision=2]{0,2} \\

								68 & \multicolumn{1}{X}{Bauingenieurwesen} & %37 &
								  \num{37} &
								%--
								  \num[round-mode=places,round-precision=2]{2,1} &
								  \num[round-mode=places,round-precision=2]{0,35} \\

								69 & \multicolumn{1}{X}{Vermessungswesen} & %8 &
								  \num{8} &
								%--
								  \num[round-mode=places,round-precision=2]{0,45} &
								  \num[round-mode=places,round-precision=2]{0,08} \\

								74 & \multicolumn{1}{X}{Kunst, Kunstwissenschaft allgemein} & %9 &
								  \num{9} &
								%--
								  \num[round-mode=places,round-precision=2]{0,51} &
								  \num[round-mode=places,round-precision=2]{0,09} \\

								75 & \multicolumn{1}{X}{Bildende Kunst} & %3 &
								  \num{3} &
								%--
								  \num[round-mode=places,round-precision=2]{0,17} &
								  \num[round-mode=places,round-precision=2]{0,03} \\

								76 & \multicolumn{1}{X}{Gestaltung} & %6 &
								  \num{6} &
								%--
								  \num[round-mode=places,round-precision=2]{0,34} &
								  \num[round-mode=places,round-precision=2]{0,06} \\

								77 & \multicolumn{1}{X}{Darstellende Kunst, Film und Fernsehen, Theaterwissenschaft} & %1 &
								  \num{1} &
								%--
								  \num[round-mode=places,round-precision=2]{0,06} &
								  \num[round-mode=places,round-precision=2]{0,01} \\

								78 & \multicolumn{1}{X}{Musik, Musikwissenschaft} & %7 &
								  \num{7} &
								%--
								  \num[round-mode=places,round-precision=2]{0,4} &
								  \num[round-mode=places,round-precision=2]{0,07} \\

								83 & \multicolumn{1}{X}{Außerhalb der Studienbereichsgliederung} & %50 &
								  \num{50} &
								%--
								  \num[round-mode=places,round-precision=2]{2,83} &
								  \num[round-mode=places,round-precision=2]{0,48} \\

					\midrule
					\multicolumn{2}{l}{Summe (gültig)} &
					  \textbf{\num{1766}} &
					\textbf{100} &
					  \textbf{\num[round-mode=places,round-precision=2]{16,83}} \\
					%--
					\multicolumn{5}{l}{\textbf{Fehlende Werte}}\\
							-998 &
							keine Angabe &
							  \num{327} &
							 - &
							  \num[round-mode=places,round-precision=2]{3,12} \\
							-995 &
							keine Teilnahme (Panel) &
							  \num{5739} &
							 - &
							  \num[round-mode=places,round-precision=2]{54,69} \\
							-989 &
							filterbedingt fehlend &
							  \num{2662} &
							 - &
							  \num[round-mode=places,round-precision=2]{25,37} \\
					\midrule
					\multicolumn{2}{l}{\textbf{Summe (gesamt)}} &
				      \textbf{\num{10494}} &
				    \textbf{-} &
				    \textbf{100} \\
					\bottomrule
					\end{longtable}
					\end{filecontents}
					\LTXtable{\textwidth}{\jobname-bfec151g_g2d}
				\label{tableValues:bfec151g_g2d}
				\vspace*{-\baselineskip}
                    \begin{noten}
                	    \note{} Deskritive Maßzahlen:
                	    Anzahl unterschiedlicher Beobachtungen: 58%
                	    ; 
                	      Modus ($h$): 30
                     \end{noten}


