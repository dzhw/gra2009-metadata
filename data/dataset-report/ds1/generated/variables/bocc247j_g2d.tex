%EVERY VARIABLE HAS IT'S OWN PAGE

    \setcounter{footnote}{0}

    %omit vertical space
    \vspace*{-1.8cm}
	\section{bocc247j\_g2d (7. Tätigkeit: Arbeitsort (Bundes-/Ausland))}
	\label{section:bocc247j_g2d}



	%TABLE FOR VARIABLE DETAILS
    \vspace*{0.5cm}
    \noindent\textbf{Eigenschaften
	% '#' has to be escaped
	\footnote{Detailliertere Informationen zur Variable finden sich unter
		\url{https://metadata.fdz.dzhw.eu/\#!/de/variables/var-gra2009-ds1-bocc247j_g2d$}}}\\
	\begin{tabularx}{\hsize}{@{}lX}
	Datentyp: & numerisch \\
	Skalenniveau: & nominal \\
	Zugangswege: &
	  download-suf, 
	  remote-desktop-suf, 
	  onsite-suf
 \\
    \end{tabularx}



    %TABLE FOR QUESTION DETAILS
    %This has to be tested and has to be improved
    %rausfinden, ob einer Variable mehrere Fragen zugeordnet werden
    %dann evtl. nur die erste verwenden oder etwas anderes tun (Hinweis mehrere Fragen, auflisten mit Link)
				%TABLE FOR QUESTION DETAILS
				\vspace*{0.5cm}
                \noindent\textbf{Frage
	                \footnote{Detailliertere Informationen zur Frage finden sich unter
		              \url{https://metadata.fdz.dzhw.eu/\#!/de/questions/que-gra2009-ins2-4.5$}}}\\
				\begin{tabularx}{\hsize}{@{}lX}
					Fragenummer: &
					  Fragebogen des DZHW-Absolventenpanels 2009 - zweite Welle, Hauptbefragung (PAPI):
					  4.5
 \\
					%--
					Fragetext: & Im Folgenden bitten wir Sie um eine nähere Beschreibung der verschiedenen beruflichen Tätigkeiten, die Sie im Jahr 2010 und danach ausgeübt haben. Bitte geben Sie auch Tätigkeiten an, die Sie bereits vorher begonnen haben, wenn diese in das Jahr 2010 hineinreichen. \\
				\end{tabularx}





				%TABLE FOR THE NOMINAL / ORDINAL VALUES
        		\vspace*{0.5cm}
                \noindent\textbf{Häufigkeiten}

                \vspace*{-\baselineskip}
					%NUMERIC ELEMENTS NEED A HUGH SECOND COLOUMN AND A SMALL FIRST ONE
					\begin{filecontents}{\jobname-bocc247j_g2d}
					\begin{longtable}{lXrrr}
					\toprule
					\textbf{Wert} & \textbf{Label} & \textbf{Häufigkeit} & \textbf{Prozent(gültig)} & \textbf{Prozent} \\
					\endhead
					\midrule
					\multicolumn{5}{l}{\textbf{Gültige Werte}}\\
						%DIFFERENT OBSERVATIONS <=20

					1 &
				% TODO try size/length gt 0; take over for other passages
					\multicolumn{1}{X}{ Schleswig-Holstein   } &


					%3 &
					  \num{3} &
					%--
					  \num[round-mode=places,round-precision=2]{5,77} &
					    \num[round-mode=places,round-precision=2]{0,03} \\
							%????

					2 &
				% TODO try size/length gt 0; take over for other passages
					\multicolumn{1}{X}{ Hamburg   } &


					%1 &
					  \num{1} &
					%--
					  \num[round-mode=places,round-precision=2]{1,92} &
					    \num[round-mode=places,round-precision=2]{0,01} \\
							%????

					3 &
				% TODO try size/length gt 0; take over for other passages
					\multicolumn{1}{X}{ Niedersachsen   } &


					%2 &
					  \num{2} &
					%--
					  \num[round-mode=places,round-precision=2]{3,85} &
					    \num[round-mode=places,round-precision=2]{0,02} \\
							%????

					5 &
				% TODO try size/length gt 0; take over for other passages
					\multicolumn{1}{X}{ Nordrhein-Westfalen   } &


					%10 &
					  \num{10} &
					%--
					  \num[round-mode=places,round-precision=2]{19,23} &
					    \num[round-mode=places,round-precision=2]{0,1} \\
							%????

					6 &
				% TODO try size/length gt 0; take over for other passages
					\multicolumn{1}{X}{ Hessen   } &


					%3 &
					  \num{3} &
					%--
					  \num[round-mode=places,round-precision=2]{5,77} &
					    \num[round-mode=places,round-precision=2]{0,03} \\
							%????

					7 &
				% TODO try size/length gt 0; take over for other passages
					\multicolumn{1}{X}{ Rheinland-Pfalz   } &


					%3 &
					  \num{3} &
					%--
					  \num[round-mode=places,round-precision=2]{5,77} &
					    \num[round-mode=places,round-precision=2]{0,03} \\
							%????

					8 &
				% TODO try size/length gt 0; take over for other passages
					\multicolumn{1}{X}{ Baden-Württemberg   } &


					%3 &
					  \num{3} &
					%--
					  \num[round-mode=places,round-precision=2]{5,77} &
					    \num[round-mode=places,round-precision=2]{0,03} \\
							%????

					9 &
				% TODO try size/length gt 0; take over for other passages
					\multicolumn{1}{X}{ Bayern   } &


					%5 &
					  \num{5} &
					%--
					  \num[round-mode=places,round-precision=2]{9,62} &
					    \num[round-mode=places,round-precision=2]{0,05} \\
							%????

					11 &
				% TODO try size/length gt 0; take over for other passages
					\multicolumn{1}{X}{ Berlin   } &


					%5 &
					  \num{5} &
					%--
					  \num[round-mode=places,round-precision=2]{9,62} &
					    \num[round-mode=places,round-precision=2]{0,05} \\
							%????

					12 &
				% TODO try size/length gt 0; take over for other passages
					\multicolumn{1}{X}{ Brandenburg   } &


					%1 &
					  \num{1} &
					%--
					  \num[round-mode=places,round-precision=2]{1,92} &
					    \num[round-mode=places,round-precision=2]{0,01} \\
							%????

					13 &
				% TODO try size/length gt 0; take over for other passages
					\multicolumn{1}{X}{ Mecklenburg-Vorpommern   } &


					%2 &
					  \num{2} &
					%--
					  \num[round-mode=places,round-precision=2]{3,85} &
					    \num[round-mode=places,round-precision=2]{0,02} \\
							%????

					14 &
				% TODO try size/length gt 0; take over for other passages
					\multicolumn{1}{X}{ Sachsen   } &


					%9 &
					  \num{9} &
					%--
					  \num[round-mode=places,round-precision=2]{17,31} &
					    \num[round-mode=places,round-precision=2]{0,09} \\
							%????

					16 &
				% TODO try size/length gt 0; take over for other passages
					\multicolumn{1}{X}{ Thüringen   } &


					%2 &
					  \num{2} &
					%--
					  \num[round-mode=places,round-precision=2]{3,85} &
					    \num[round-mode=places,round-precision=2]{0,02} \\
							%????

					100 &
				% TODO try size/length gt 0; take over for other passages
					\multicolumn{1}{X}{ Ausland   } &


					%3 &
					  \num{3} &
					%--
					  \num[round-mode=places,round-precision=2]{5,77} &
					    \num[round-mode=places,round-precision=2]{0,03} \\
							%????
						%DIFFERENT OBSERVATIONS >20
					\midrule
					\multicolumn{2}{l}{Summe (gültig)} &
					  \textbf{\num{52}} &
					\textbf{100} &
					  \textbf{\num[round-mode=places,round-precision=2]{0,5}} \\
					%--
					\multicolumn{5}{l}{\textbf{Fehlende Werte}}\\
							-998 &
							keine Angabe &
							  \num{4672} &
							 - &
							  \num[round-mode=places,round-precision=2]{44,52} \\
							-995 &
							keine Teilnahme (Panel) &
							  \num{5739} &
							 - &
							  \num[round-mode=places,round-precision=2]{54,69} \\
							-989 &
							filterbedingt fehlend &
							  \num{31} &
							 - &
							  \num[round-mode=places,round-precision=2]{0,3} \\
					\midrule
					\multicolumn{2}{l}{\textbf{Summe (gesamt)}} &
				      \textbf{\num{10494}} &
				    \textbf{-} &
				    \textbf{100} \\
					\bottomrule
					\end{longtable}
					\end{filecontents}
					\LTXtable{\textwidth}{\jobname-bocc247j_g2d}
				\label{tableValues:bocc247j_g2d}
				\vspace*{-\baselineskip}
                    \begin{noten}
                	    \note{} Deskritive Maßzahlen:
                	    Anzahl unterschiedlicher Beobachtungen: 14%
                	    ; 
                	      Modus ($h$): 5
                     \end{noten}


