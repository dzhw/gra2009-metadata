%EVERY VARIABLE HAS IT'S OWN PAGE

    \setcounter{footnote}{0}

    %omit vertical space
    \vspace*{-1.8cm}
	\section{bocc244k\_g1v1d (4. Tätigkeit: Arbeitsort (NUTS2))}
	\label{section:bocc244k_g1v1d}



	%TABLE FOR VARIABLE DETAILS
    \vspace*{0.5cm}
    \noindent\textbf{Eigenschaften
	% '#' has to be escaped
	\footnote{Detailliertere Informationen zur Variable finden sich unter
		\url{https://metadata.fdz.dzhw.eu/\#!/de/variables/var-gra2009-ds1-bocc244k_g1v1d$}}}\\
	\begin{tabularx}{\hsize}{@{}lX}
	Datentyp: & string \\
	Skalenniveau: & nominal \\
	Zugangswege: &
	  download-suf, 
	  remote-desktop-suf, 
	  onsite-suf
 \\
    \end{tabularx}



    %TABLE FOR QUESTION DETAILS
    %This has to be tested and has to be improved
    %rausfinden, ob einer Variable mehrere Fragen zugeordnet werden
    %dann evtl. nur die erste verwenden oder etwas anderes tun (Hinweis mehrere Fragen, auflisten mit Link)
				%TABLE FOR QUESTION DETAILS
				\vspace*{0.5cm}
                \noindent\textbf{Frage
	                \footnote{Detailliertere Informationen zur Frage finden sich unter
		              \url{https://metadata.fdz.dzhw.eu/\#!/de/questions/que-gra2009-ins2-4.5$}}}\\
				\begin{tabularx}{\hsize}{@{}lX}
					Fragenummer: &
					  Fragebogen des DZHW-Absolventenpanels 2009 - zweite Welle, Hauptbefragung (PAPI):
					  4.5
 \\
					%--
					Fragetext: & Im Folgenden bitten wir Sie um eine nähere Beschreibung der verschiedenen beruflichen Tätigkeiten, die Sie im Jahr 2010 und danach ausgeübt haben. Bitte geben Sie auch Tätigkeiten an, die Sie bereits vorher begonnen haben, wenn diese in das Jahr 2010 hineinreichen. \\
				\end{tabularx}





				%TABLE FOR THE NOMINAL / ORDINAL VALUES
        		\vspace*{0.5cm}
                \noindent\textbf{Häufigkeiten}

                \vspace*{-\baselineskip}
					%STRING ELEMENTS NEEDS A HUGH FIRST COLOUMN AND A SMALL SECOND ONE
					\begin{filecontents}{\jobname-bocc244k_g1v1d}
					\begin{longtable}{Xlrrr}
					\toprule
					\textbf{Wert} & \textbf{Label} & \textbf{Häufigkeit} & \textbf{Prozent (gültig)} & \textbf{Prozent} \\
					\endhead
					\midrule
					\multicolumn{5}{l}{\textbf{Gültige Werte}}\\
						%DIFFERENT OBSERVATIONS <=20
								\multicolumn{1}{X}{DE11 Stuttgart} & - & 39 & 7,69 & 0,37 \\
								\multicolumn{1}{X}{DE12 Karlsruhe} & - & 8 & 1,58 & 0,08 \\
								\multicolumn{1}{X}{DE13 Freiburg} & - & 8 & 1,58 & 0,08 \\
								\multicolumn{1}{X}{DE14 Tübingen} & - & 11 & 2,17 & 0,1 \\
								\multicolumn{1}{X}{DE21 Oberbayern} & - & 48 & 9,47 & 0,46 \\
								\multicolumn{1}{X}{DE22 Niederbayern} & - & 4 & 0,79 & 0,04 \\
								\multicolumn{1}{X}{DE24 Oberfranken} & - & 4 & 0,79 & 0,04 \\
								\multicolumn{1}{X}{DE25 Mittelfranken} & - & 10 & 1,97 & 0,1 \\
								\multicolumn{1}{X}{DE26 Unterfranken} & - & 1 & 0,2 & 0,01 \\
								\multicolumn{1}{X}{DE27 Schwaben} & - & 6 & 1,18 & 0,06 \\
							... & ... & ... & ... & ... \\
								\multicolumn{1}{X}{DEB1 Koblenz} & - & 12 & 2,37 & 0,11 \\
								\multicolumn{1}{X}{DEB2 Trier} & - & 5 & 0,99 & 0,05 \\
								\multicolumn{1}{X}{DEB3 Rheinhessen-Pfalz} & - & 6 & 1,18 & 0,06 \\
								\multicolumn{1}{X}{DEC0 Saarland} & - & 3 & 0,59 & 0,03 \\
								\multicolumn{1}{X}{DED2 Dresden} & - & 27 & 5,33 & 0,26 \\
								\multicolumn{1}{X}{DED4 Chemnitz} & - & 10 & 1,97 & 0,1 \\
								\multicolumn{1}{X}{DED5 Leipzig} & - & 5 & 0,99 & 0,05 \\
								\multicolumn{1}{X}{DEE0 Sachsen-Anhalt} & - & 6 & 1,18 & 0,06 \\
								\multicolumn{1}{X}{DEF0 Schleswig-Holstein} & - & 17 & 3,35 & 0,16 \\
								\multicolumn{1}{X}{DEG0 Thüringen} & - & 36 & 7,1 & 0,34 \\
					\midrule
						\multicolumn{2}{l}{Summe (gültig)} & 507 &
						\textbf{100} &
					    4,83 \\
					\multicolumn{5}{l}{\textbf{Fehlende Werte}}\\
							-966 & nicht bestimmbar & 49 & - & 0,47 \\

							-968 & unplausibler Wert & 8 & - & 0,08 \\

							-989 & filterbedingt fehlend & 31 & - & 0,3 \\

							-995 & keine Teilnahme (Panel) & 5739 & - & 54,69 \\

							-998 & keine Angabe & 4160 & - & 39,64 \\

					\midrule
					\multicolumn{2}{l}{\textbf{Summe (gesamt)}} & \textbf{10494} & \textbf{-} & \textbf{100} \\
					\bottomrule
					\caption{Werte der Variable bocc244k\_g1v1d}
					\end{longtable}
					\end{filecontents}
					\LTXtable{\textwidth}{\jobname-bocc244k_g1v1d}


