%EVERY VARIABLE HAS IT'S OWN PAGE

    \setcounter{footnote}{0}

    %omit vertical space
    \vspace*{-1.8cm}
	\section{bocc22a\_v1a (Beruf: Aufgabenbereich 1)}
	\label{section:bocc22a_v1a}



	%TABLE FOR VARIABLE DETAILS
    \vspace*{0.5cm}
    \noindent\textbf{Eigenschaften
	% '#' has to be escaped
	\footnote{Detailliertere Informationen zur Variable finden sich unter
		\url{https://metadata.fdz.dzhw.eu/\#!/de/variables/var-gra2009-ds1-bocc22a_v1a$}}}\\
	\begin{tabularx}{\hsize}{@{}lX}
	Datentyp: & string \\
	Skalenniveau: & nominal \\
	Zugangswege: &
	  not-accessible
 \\
    \end{tabularx}



    %TABLE FOR QUESTION DETAILS
    %This has to be tested and has to be improved
    %rausfinden, ob einer Variable mehrere Fragen zugeordnet werden
    %dann evtl. nur die erste verwenden oder etwas anderes tun (Hinweis mehrere Fragen, auflisten mit Link)
				%TABLE FOR QUESTION DETAILS
				\vspace*{0.5cm}
                \noindent\textbf{Frage
	                \footnote{Detailliertere Informationen zur Frage finden sich unter
		              \url{https://metadata.fdz.dzhw.eu/\#!/de/questions/que-gra2009-ins2-4.9$}}}\\
				\begin{tabularx}{\hsize}{@{}lX}
					Fragenummer: &
					  Fragebogen des DZHW-Absolventenpanels 2009 - zweite Welle, Hauptbefragung (PAPI):
					  4.9
 \\
					%--
					Fragetext: & Bitte nennen Sie Ihre Berufsbezeichnung, Ihren Aufgabenbereich sowie typische Arbeitsschwerpunkte Ihrer beruflichen Tätigkeit. Aufgabenbereich (z. B. Management, Finanzcontrolling, Qualitätswesen, Personal, Logistik, Software): \\
				\end{tabularx}
				%TABLE FOR QUESTION DETAILS
				\vspace*{0.5cm}
                \noindent\textbf{Frage
	                \footnote{Detailliertere Informationen zur Frage finden sich unter
		              \url{https://metadata.fdz.dzhw.eu/\#!/de/questions/que-gra2009-ins3-24$}}}\\
				\begin{tabularx}{\hsize}{@{}lX}
					Fragenummer: &
					  Fragebogen des DZHW-Absolventenpanels 2009 - zweite Welle, Hauptbefragung (CAWI):
					  24
 \\
					%--
					Fragetext: & Bitte nennen Sie Ihre Berufsbezeichnung, Ihren Aufgabenbereich sowie typische Arbeitsschwerpunkte Ihrer beruflichen Tätigkeit. \\
				\end{tabularx}





