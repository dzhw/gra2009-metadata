%EVERY VARIABLE HAS IT'S OWN PAGE

    \setcounter{footnote}{0}

    %omit vertical space
    \vspace*{-1.8cm}
	\section{adem11a\_g1d (Wohnsitz: NUTS2)}
	\label{section:adem11a_g1d}



	%TABLE FOR VARIABLE DETAILS
    \vspace*{0.5cm}
    \noindent\textbf{Eigenschaften
	% '#' has to be escaped
	\footnote{Detailliertere Informationen zur Variable finden sich unter
		\url{https://metadata.fdz.dzhw.eu/\#!/de/variables/var-gra2009-ds1-adem11a_g1d$}}}\\
	\begin{tabularx}{\hsize}{@{}lX}
	Datentyp: & string \\
	Skalenniveau: & nominal \\
	Zugangswege: &
	  download-suf, 
	  remote-desktop-suf, 
	  onsite-suf
 \\
    \end{tabularx}



    %TABLE FOR QUESTION DETAILS
    %This has to be tested and has to be improved
    %rausfinden, ob einer Variable mehrere Fragen zugeordnet werden
    %dann evtl. nur die erste verwenden oder etwas anderes tun (Hinweis mehrere Fragen, auflisten mit Link)
				%TABLE FOR QUESTION DETAILS
				\vspace*{0.5cm}
                \noindent\textbf{Frage
	                \footnote{Detailliertere Informationen zur Frage finden sich unter
		              \url{https://metadata.fdz.dzhw.eu/\#!/de/questions/que-gra2009-ins1-6.13$}}}\\
				\begin{tabularx}{\hsize}{@{}lX}
					Fragenummer: &
					  Fragebogen des DZHW-Absolventenpanels 2009 - erste Welle:
					  6.13
 \\
					%--
					Fragetext: & Bitte geben Sie Ihren Hauptwohnsitz an. \\
				\end{tabularx}





				%TABLE FOR THE NOMINAL / ORDINAL VALUES
        		\vspace*{0.5cm}
                \noindent\textbf{Häufigkeiten}

                \vspace*{-\baselineskip}
					%STRING ELEMENTS NEEDS A HUGH FIRST COLOUMN AND A SMALL SECOND ONE
					\begin{filecontents}{\jobname-adem11a_g1d}
					\begin{longtable}{Xlrrr}
					\toprule
					\textbf{Wert} & \textbf{Label} & \textbf{Häufigkeit} & \textbf{Prozent (gültig)} & \textbf{Prozent} \\
					\endhead
					\midrule
					\multicolumn{5}{l}{\textbf{Gültige Werte}}\\
						%DIFFERENT OBSERVATIONS <=20
								\multicolumn{1}{X}{DE11 Stuttgart} & - & 581 & 6,52 & 5,54 \\
								\multicolumn{1}{X}{DE12 Karlsruhe} & - & 184 & 2,07 & 1,75 \\
								\multicolumn{1}{X}{DE13 Freiburg} & - & 174 & 1,95 & 1,66 \\
								\multicolumn{1}{X}{DE14 Tübingen} & - & 205 & 2,3 & 1,95 \\
								\multicolumn{1}{X}{DE21 Oberbayern} & - & 789 & 8,86 & 7,52 \\
								\multicolumn{1}{X}{DE22 Niederbayern} & - & 98 & 1,1 & 0,93 \\
								\multicolumn{1}{X}{DE23 Oberpfalz} & - & 42 & 0,47 & 0,4 \\
								\multicolumn{1}{X}{DE24 Oberfranken} & - & 55 & 0,62 & 0,52 \\
								\multicolumn{1}{X}{DE25 Mittelfranken} & - & 155 & 1,74 & 1,48 \\
								\multicolumn{1}{X}{DE26 Unterfranken} & - & 44 & 0,49 & 0,42 \\
							... & ... & ... & ... & ... \\
								\multicolumn{1}{X}{DEB1 Koblenz} & - & 132 & 1,48 & 1,26 \\
								\multicolumn{1}{X}{DEB2 Trier} & - & 81 & 0,91 & 0,77 \\
								\multicolumn{1}{X}{DEB3 Rheinhessen-Pfalz} & - & 134 & 1,5 & 1,28 \\
								\multicolumn{1}{X}{DEC0 Saarland} & - & 68 & 0,76 & 0,65 \\
								\multicolumn{1}{X}{DED2 Dresden} & - & 380 & 4,27 & 3,62 \\
								\multicolumn{1}{X}{DED4 Chemnitz} & - & 215 & 2,41 & 2,05 \\
								\multicolumn{1}{X}{DED5 Leipzig} & - & 129 & 1,45 & 1,23 \\
								\multicolumn{1}{X}{DEE0 Sachsen-Anhalt} & - & 182 & 2,04 & 1,73 \\
								\multicolumn{1}{X}{DEF0 Schleswig-Holstein} & - & 255 & 2,86 & 2,43 \\
								\multicolumn{1}{X}{DEG0 Thüringen} & - & 450 & 5,05 & 4,29 \\
					\midrule
						\multicolumn{2}{l}{Summe (gültig)} & 8909 &
						\textbf{100} &
					    84,9 \\
					\multicolumn{5}{l}{\textbf{Fehlende Werte}}\\
							-966 & nicht bestimmbar & 1281 & - & 12,21 \\

							-968 & unplausibler Wert & 15 & - & 0,14 \\

							-998 & keine Angabe & 289 & - & 2,75 \\

					\midrule
					\multicolumn{2}{l}{\textbf{Summe (gesamt)}} & \textbf{10494} & \textbf{-} & \textbf{100} \\
					\bottomrule
					\caption{Werte der Variable adem11a\_g1d}
					\end{longtable}
					\end{filecontents}
					\LTXtable{\textwidth}{\jobname-adem11a_g1d}


