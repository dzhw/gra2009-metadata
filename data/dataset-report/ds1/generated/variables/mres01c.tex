%EVERY VARIABLE HAS IT'S OWN PAGE

    \setcounter{footnote}{0}

    %omit vertical space
    \vspace*{-1.8cm}
	\section{mres01c (Wohnung Studium: Auszug (Monat))}
	\label{section:mres01c}



	%TABLE FOR VARIABLE DETAILS
    \vspace*{0.5cm}
    \noindent\textbf{Eigenschaften
	% '#' has to be escaped
	\footnote{Detailliertere Informationen zur Variable finden sich unter
		\url{https://metadata.fdz.dzhw.eu/\#!/de/variables/var-gra2009-ds1-mres01c$}}}\\
	\begin{tabularx}{\hsize}{@{}lX}
	Datentyp: & numerisch \\
	Skalenniveau: & ordinal \\
	Zugangswege: &
	  download-cuf, 
	  download-suf, 
	  remote-desktop-suf, 
	  onsite-suf
 \\
    \end{tabularx}



    %TABLE FOR QUESTION DETAILS
    %This has to be tested and has to be improved
    %rausfinden, ob einer Variable mehrere Fragen zugeordnet werden
    %dann evtl. nur die erste verwenden oder etwas anderes tun (Hinweis mehrere Fragen, auflisten mit Link)
				%TABLE FOR QUESTION DETAILS
				\vspace*{0.5cm}
                \noindent\textbf{Frage
	                \footnote{Detailliertere Informationen zur Frage finden sich unter
		              \url{https://metadata.fdz.dzhw.eu/\#!/de/questions/que-gra2009-ins5-07.1$}}}\\
				\begin{tabularx}{\hsize}{@{}lX}
					Fragenummer: &
					  Fragebogen des DZHW-Absolventenpanels 2009 - zweite Welle, Vertiefungsbefragung Mobilität:
					  07.1
 \\
					%--
					Fragetext: & Um Ihre Wohnsituation besser nachvollziehen zu können, bitten wir Sie im Folgenden um einige Angaben zu Ihren Wohnungen der letzten Jahre. Zunächst bitten wir Sie uns dabei mitzuteilen, wo und wie Sie direkt während Ihres Studienabschlusses 2008/09 gewohnt haben,Zeitraum (Monat/Jahr),Wohnort,Wohnten Sie (Mehrfachnennung möglich),Handelte es sich um,bis: \\
				\end{tabularx}





				%TABLE FOR THE NOMINAL / ORDINAL VALUES
        		\vspace*{0.5cm}
                \noindent\textbf{Häufigkeiten}

                \vspace*{-\baselineskip}
					%NUMERIC ELEMENTS NEED A HUGH SECOND COLOUMN AND A SMALL FIRST ONE
					\begin{filecontents}{\jobname-mres01c}
					\begin{longtable}{lXrrr}
					\toprule
					\textbf{Wert} & \textbf{Label} & \textbf{Häufigkeit} & \textbf{Prozent(gültig)} & \textbf{Prozent} \\
					\endhead
					\midrule
					\multicolumn{5}{l}{\textbf{Gültige Werte}}\\
						%DIFFERENT OBSERVATIONS <=20

					1 &
				% TODO try size/length gt 0; take over for other passages
					\multicolumn{1}{X}{ Januar   } &


					%104 &
					  \num{104} &
					%--
					  \num[round-mode=places,round-precision=2]{4,71} &
					    \num[round-mode=places,round-precision=2]{0,99} \\
							%????

					2 &
				% TODO try size/length gt 0; take over for other passages
					\multicolumn{1}{X}{ Februar   } &


					%152 &
					  \num{152} &
					%--
					  \num[round-mode=places,round-precision=2]{6,88} &
					    \num[round-mode=places,round-precision=2]{1,45} \\
							%????

					3 &
				% TODO try size/length gt 0; take over for other passages
					\multicolumn{1}{X}{ März   } &


					%184 &
					  \num{184} &
					%--
					  \num[round-mode=places,round-precision=2]{8,33} &
					    \num[round-mode=places,round-precision=2]{1,75} \\
							%????

					4 &
				% TODO try size/length gt 0; take over for other passages
					\multicolumn{1}{X}{ April   } &


					%133 &
					  \num{133} &
					%--
					  \num[round-mode=places,round-precision=2]{6,02} &
					    \num[round-mode=places,round-precision=2]{1,27} \\
							%????

					5 &
				% TODO try size/length gt 0; take over for other passages
					\multicolumn{1}{X}{ Mai   } &


					%93 &
					  \num{93} &
					%--
					  \num[round-mode=places,round-precision=2]{4,21} &
					    \num[round-mode=places,round-precision=2]{0,89} \\
							%????

					6 &
				% TODO try size/length gt 0; take over for other passages
					\multicolumn{1}{X}{ Juni   } &


					%146 &
					  \num{146} &
					%--
					  \num[round-mode=places,round-precision=2]{6,61} &
					    \num[round-mode=places,round-precision=2]{1,39} \\
							%????

					7 &
				% TODO try size/length gt 0; take over for other passages
					\multicolumn{1}{X}{ Juli   } &


					%321 &
					  \num{321} &
					%--
					  \num[round-mode=places,round-precision=2]{14,54} &
					    \num[round-mode=places,round-precision=2]{3,06} \\
							%????

					8 &
				% TODO try size/length gt 0; take over for other passages
					\multicolumn{1}{X}{ August   } &


					%291 &
					  \num{291} &
					%--
					  \num[round-mode=places,round-precision=2]{13,18} &
					    \num[round-mode=places,round-precision=2]{2,77} \\
							%????

					9 &
				% TODO try size/length gt 0; take over for other passages
					\multicolumn{1}{X}{ September   } &


					%299 &
					  \num{299} &
					%--
					  \num[round-mode=places,round-precision=2]{13,54} &
					    \num[round-mode=places,round-precision=2]{2,85} \\
							%????

					10 &
				% TODO try size/length gt 0; take over for other passages
					\multicolumn{1}{X}{ Oktober   } &


					%185 &
					  \num{185} &
					%--
					  \num[round-mode=places,round-precision=2]{8,38} &
					    \num[round-mode=places,round-precision=2]{1,76} \\
							%????

					11 &
				% TODO try size/length gt 0; take over for other passages
					\multicolumn{1}{X}{ November   } &


					%81 &
					  \num{81} &
					%--
					  \num[round-mode=places,round-precision=2]{3,67} &
					    \num[round-mode=places,round-precision=2]{0,77} \\
							%????

					12 &
				% TODO try size/length gt 0; take over for other passages
					\multicolumn{1}{X}{ Dezember   } &


					%219 &
					  \num{219} &
					%--
					  \num[round-mode=places,round-precision=2]{9,92} &
					    \num[round-mode=places,round-precision=2]{2,09} \\
							%????
						%DIFFERENT OBSERVATIONS >20
					\midrule
					\multicolumn{2}{l}{Summe (gültig)} &
					  \textbf{\num{2208}} &
					\textbf{100} &
					  \textbf{\num[round-mode=places,round-precision=2]{21,04}} \\
					%--
					\multicolumn{5}{l}{\textbf{Fehlende Werte}}\\
							-998 &
							keine Angabe &
							  \num{257} &
							 - &
							  \num[round-mode=places,round-precision=2]{2,45} \\
							-995 &
							keine Teilnahme (Panel) &
							  \num{8029} &
							 - &
							  \num[round-mode=places,round-precision=2]{76,51} \\
					\midrule
					\multicolumn{2}{l}{\textbf{Summe (gesamt)}} &
				      \textbf{\num{10494}} &
				    \textbf{-} &
				    \textbf{100} \\
					\bottomrule
					\end{longtable}
					\end{filecontents}
					\LTXtable{\textwidth}{\jobname-mres01c}
				\label{tableValues:mres01c}
				\vspace*{-\baselineskip}
                    \begin{noten}
                	    \note{} Deskritive Maßzahlen:
                	    Anzahl unterschiedlicher Beobachtungen: 12%
                	    ; 
                	      Minimum ($min$): 1; 
                	      Maximum ($max$): 12; 
                	      Median ($\tilde{x}$): 7; 
                	      Modus ($h$): 7
                     \end{noten}


