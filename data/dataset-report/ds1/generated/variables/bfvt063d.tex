%EVERY VARIABLE HAS IT'S OWN PAGE

    \setcounter{footnote}{0}

    %omit vertical space
    \vspace*{-1.8cm}
	\section{bfvt063d (mehrtägige berufl. Weiterbildung: Inhalt 2)}
	\label{section:bfvt063d}



	%TABLE FOR VARIABLE DETAILS
    \vspace*{0.5cm}
    \noindent\textbf{Eigenschaften
	% '#' has to be escaped
	\footnote{Detailliertere Informationen zur Variable finden sich unter
		\url{https://metadata.fdz.dzhw.eu/\#!/de/variables/var-gra2009-ds1-bfvt063d$}}}\\
	\begin{tabularx}{\hsize}{@{}lX}
	Datentyp: & numerisch \\
	Skalenniveau: & nominal \\
	Zugangswege: &
	  download-cuf, 
	  download-suf, 
	  remote-desktop-suf, 
	  onsite-suf
 \\
    \end{tabularx}



    %TABLE FOR QUESTION DETAILS
    %This has to be tested and has to be improved
    %rausfinden, ob einer Variable mehrere Fragen zugeordnet werden
    %dann evtl. nur die erste verwenden oder etwas anderes tun (Hinweis mehrere Fragen, auflisten mit Link)
				%TABLE FOR QUESTION DETAILS
				\vspace*{0.5cm}
                \noindent\textbf{Frage
	                \footnote{Detailliertere Informationen zur Frage finden sich unter
		              \url{https://metadata.fdz.dzhw.eu/\#!/de/questions/que-gra2009-ins2-6.5$}}}\\
				\begin{tabularx}{\hsize}{@{}lX}
					Fragenummer: &
					  Fragebogen des DZHW-Absolventenpanels 2009 - zweite Welle, Hauptbefragung (PAPI):
					  6.5
 \\
					%--
					Fragetext: & Im Folgenden bitten wir Sie um Angaben zu beruflichen Fort- und Weiterbildungen der letzten 12 Monate. Bitte denken Sie dabei an alle Weiterbildungen, die Sie besucht haben und geben sie diese in der passenden Zeile an.\par  3. Fort- /oder Weiterbildung\par  Themen (Mehrfachnennung möglich)\par  Schlüssel s. Klappliste B) \\
				\end{tabularx}
				%TABLE FOR QUESTION DETAILS
				\vspace*{0.5cm}
                \noindent\textbf{Frage
	                \footnote{Detailliertere Informationen zur Frage finden sich unter
		              \url{https://metadata.fdz.dzhw.eu/\#!/de/questions/que-gra2009-ins3-67$}}}\\
				\begin{tabularx}{\hsize}{@{}lX}
					Fragenummer: &
					  Fragebogen des DZHW-Absolventenpanels 2009 - zweite Welle, Hauptbefragung (CAWI):
					  67
 \\
					%--
					Fragetext: & Bitte tragen Sie hier die für Sie wichtigsten Themen bzw. Fachgebiete dieser Veranstaltungen ein. \\
				\end{tabularx}





				%TABLE FOR THE NOMINAL / ORDINAL VALUES
        		\vspace*{0.5cm}
                \noindent\textbf{Häufigkeiten}

                \vspace*{-\baselineskip}
					%NUMERIC ELEMENTS NEED A HUGH SECOND COLOUMN AND A SMALL FIRST ONE
					\begin{filecontents}{\jobname-bfvt063d}
					\begin{longtable}{lXrrr}
					\toprule
					\textbf{Wert} & \textbf{Label} & \textbf{Häufigkeit} & \textbf{Prozent(gültig)} & \textbf{Prozent} \\
					\endhead
					\midrule
					\multicolumn{5}{l}{\textbf{Gültige Werte}}\\
						%DIFFERENT OBSERVATIONS <=20
								1 & \multicolumn{1}{X}{ingenieurwissenschaftliche Themen} & %54 &
								  \num{54} &
								%--
								  \num[round-mode=places,round-precision=2]{5,01} &
								  \num[round-mode=places,round-precision=2]{0,51} \\
								2 & \multicolumn{1}{X}{naturwissenschaftliche Themen} & %43 &
								  \num{43} &
								%--
								  \num[round-mode=places,round-precision=2]{3,99} &
								  \num[round-mode=places,round-precision=2]{0,41} \\
								3 & \multicolumn{1}{X}{mathematische Gebiete/Statistik} & %22 &
								  \num{22} &
								%--
								  \num[round-mode=places,round-precision=2]{2,04} &
								  \num[round-mode=places,round-precision=2]{0,21} \\
								4 & \multicolumn{1}{X}{sozialwissenschaftliche Themen} & %49 &
								  \num{49} &
								%--
								  \num[round-mode=places,round-precision=2]{4,55} &
								  \num[round-mode=places,round-precision=2]{0,47} \\
								5 & \multicolumn{1}{X}{geisteswissenschtliche Themen} & %33 &
								  \num{33} &
								%--
								  \num[round-mode=places,round-precision=2]{3,06} &
								  \num[round-mode=places,round-precision=2]{0,31} \\
								6 & \multicolumn{1}{X}{pädagogische/psychologische Themen} & %127 &
								  \num{127} &
								%--
								  \num[round-mode=places,round-precision=2]{11,79} &
								  \num[round-mode=places,round-precision=2]{1,21} \\
								7 & \multicolumn{1}{X}{medizinische Spezialgebiete} & %85 &
								  \num{85} &
								%--
								  \num[round-mode=places,round-precision=2]{7,89} &
								  \num[round-mode=places,round-precision=2]{0,81} \\
								8 & \multicolumn{1}{X}{informationstechnisches Spezialwissen} & %38 &
								  \num{38} &
								%--
								  \num[round-mode=places,round-precision=2]{3,53} &
								  \num[round-mode=places,round-precision=2]{0,36} \\
								9 & \multicolumn{1}{X}{Managementwissen} & %74 &
								  \num{74} &
								%--
								  \num[round-mode=places,round-precision=2]{6,87} &
								  \num[round-mode=places,round-precision=2]{0,71} \\
								10 & \multicolumn{1}{X}{Wirtschaftskenntnisse} & %36 &
								  \num{36} &
								%--
								  \num[round-mode=places,round-precision=2]{3,34} &
								  \num[round-mode=places,round-precision=2]{0,34} \\
							... & ... & ... & ... & ... \\
								15 & \multicolumn{1}{X}{EDV-Anwendungen} & %80 &
								  \num{80} &
								%--
								  \num[round-mode=places,round-precision=2]{7,43} &
								  \num[round-mode=places,round-precision=2]{0,76} \\

								16 & \multicolumn{1}{X}{Fremdsprachen} & %30 &
								  \num{30} &
								%--
								  \num[round-mode=places,round-precision=2]{2,79} &
								  \num[round-mode=places,round-precision=2]{0,29} \\

								17 & \multicolumn{1}{X}{Mitarbeiterführung/Personalentwicklung} & %66 &
								  \num{66} &
								%--
								  \num[round-mode=places,round-precision=2]{6,13} &
								  \num[round-mode=places,round-precision=2]{0,63} \\

								18 & \multicolumn{1}{X}{Kommunikations-/Interaktionstraining} & %165 &
								  \num{165} &
								%--
								  \num[round-mode=places,round-precision=2]{15,32} &
								  \num[round-mode=places,round-precision=2]{1,57} \\

								19 & \multicolumn{1}{X}{internationale Beziehungen, Kulturkenntnisse, Landeskunde} & %17 &
								  \num{17} &
								%--
								  \num[round-mode=places,round-precision=2]{1,58} &
								  \num[round-mode=places,round-precision=2]{0,16} \\

								20 & \multicolumn{1}{X}{ökologische Themen} & %3 &
								  \num{3} &
								%--
								  \num[round-mode=places,round-precision=2]{0,28} &
								  \num[round-mode=places,round-precision=2]{0,03} \\

								21 & \multicolumn{1}{X}{berufsethische Themen} & %10 &
								  \num{10} &
								%--
								  \num[round-mode=places,round-precision=2]{0,93} &
								  \num[round-mode=places,round-precision=2]{0,1} \\

								22 & \multicolumn{1}{X}{Existenzgründung} & %6 &
								  \num{6} &
								%--
								  \num[round-mode=places,round-precision=2]{0,56} &
								  \num[round-mode=places,round-precision=2]{0,06} \\

								23 & \multicolumn{1}{X}{betriebliches Gesundheitswesen, Arbeitssicherheit} & %22 &
								  \num{22} &
								%--
								  \num[round-mode=places,round-precision=2]{2,04} &
								  \num[round-mode=places,round-precision=2]{0,21} \\

								24 & \multicolumn{1}{X}{Sonstige} & %21 &
								  \num{21} &
								%--
								  \num[round-mode=places,round-precision=2]{1,95} &
								  \num[round-mode=places,round-precision=2]{0,2} \\

					\midrule
					\multicolumn{2}{l}{Summe (gültig)} &
					  \textbf{\num{1077}} &
					\textbf{100} &
					  \textbf{\num[round-mode=places,round-precision=2]{10,26}} \\
					%--
					\multicolumn{5}{l}{\textbf{Fehlende Werte}}\\
							-998 &
							keine Angabe &
							  \num{3678} &
							 - &
							  \num[round-mode=places,round-precision=2]{35,05} \\
							-995 &
							keine Teilnahme (Panel) &
							  \num{5739} &
							 - &
							  \num[round-mode=places,round-precision=2]{54,69} \\
					\midrule
					\multicolumn{2}{l}{\textbf{Summe (gesamt)}} &
				      \textbf{\num{10494}} &
				    \textbf{-} &
				    \textbf{100} \\
					\bottomrule
					\end{longtable}
					\end{filecontents}
					\LTXtable{\textwidth}{\jobname-bfvt063d}
				\label{tableValues:bfvt063d}
				\vspace*{-\baselineskip}
                    \begin{noten}
                	    \note{} Deskritive Maßzahlen:
                	    Anzahl unterschiedlicher Beobachtungen: 24%
                	    ; 
                	      Modus ($h$): 18
                     \end{noten}


