%EVERY VARIABLE HAS IT'S OWN PAGE

    \setcounter{footnote}{0}

    %omit vertical space
    \vspace*{-1.8cm}
	\section{astu012e\_g2d (2. Studium: Hauptfach (Studienbereiche))}
	\label{section:astu012e_g2d}



	%TABLE FOR VARIABLE DETAILS
    \vspace*{0.5cm}
    \noindent\textbf{Eigenschaften
	% '#' has to be escaped
	\footnote{Detailliertere Informationen zur Variable finden sich unter
		\url{https://metadata.fdz.dzhw.eu/\#!/de/variables/var-gra2009-ds1-astu012e_g2d$}}}\\
	\begin{tabularx}{\hsize}{@{}lX}
	Datentyp: & numerisch \\
	Skalenniveau: & nominal \\
	Zugangswege: &
	  download-suf, 
	  remote-desktop-suf, 
	  onsite-suf
 \\
    \end{tabularx}



    %TABLE FOR QUESTION DETAILS
    %This has to be tested and has to be improved
    %rausfinden, ob einer Variable mehrere Fragen zugeordnet werden
    %dann evtl. nur die erste verwenden oder etwas anderes tun (Hinweis mehrere Fragen, auflisten mit Link)
				%TABLE FOR QUESTION DETAILS
				\vspace*{0.5cm}
                \noindent\textbf{Frage
	                \footnote{Detailliertere Informationen zur Frage finden sich unter
		              \url{https://metadata.fdz.dzhw.eu/\#!/de/questions/que-gra2009-ins1-1.1$}}}\\
				\begin{tabularx}{\hsize}{@{}lX}
					Fragenummer: &
					  Fragebogen des DZHW-Absolventenpanels 2009 - erste Welle:
					  1.1
 \\
					%--
					Fragetext: & Bitte tragen Sie in das folgende Tableau Ihren Studienverlauf ein. \\
				\end{tabularx}





				%TABLE FOR THE NOMINAL / ORDINAL VALUES
        		\vspace*{0.5cm}
                \noindent\textbf{Häufigkeiten}

                \vspace*{-\baselineskip}
					%NUMERIC ELEMENTS NEED A HUGH SECOND COLOUMN AND A SMALL FIRST ONE
					\begin{filecontents}{\jobname-astu012e_g2d}
					\begin{longtable}{lXrrr}
					\toprule
					\textbf{Wert} & \textbf{Label} & \textbf{Häufigkeit} & \textbf{Prozent(gültig)} & \textbf{Prozent} \\
					\endhead
					\midrule
					\multicolumn{5}{l}{\textbf{Gültige Werte}}\\
						%DIFFERENT OBSERVATIONS <=20
								1 & \multicolumn{1}{X}{Sprach- und Kulturwissenschaften allgemein} & %152 &
								  \num{152} &
								%--
								  \num[round-mode=places,round-precision=2]{3,01} &
								  \num[round-mode=places,round-precision=2]{1,45} \\
								2 & \multicolumn{1}{X}{Evang. Theologie, -Religionslehre} & %40 &
								  \num{40} &
								%--
								  \num[round-mode=places,round-precision=2]{0,79} &
								  \num[round-mode=places,round-precision=2]{0,38} \\
								3 & \multicolumn{1}{X}{Kath. Theologie, -Religionslehre} & %25 &
								  \num{25} &
								%--
								  \num[round-mode=places,round-precision=2]{0,49} &
								  \num[round-mode=places,round-precision=2]{0,24} \\
								4 & \multicolumn{1}{X}{Philosophie} & %31 &
								  \num{31} &
								%--
								  \num[round-mode=places,round-precision=2]{0,61} &
								  \num[round-mode=places,round-precision=2]{0,3} \\
								5 & \multicolumn{1}{X}{Geschichte} & %123 &
								  \num{123} &
								%--
								  \num[round-mode=places,round-precision=2]{2,43} &
								  \num[round-mode=places,round-precision=2]{1,17} \\
								6 & \multicolumn{1}{X}{Bibliothekswissenschaft, Dokumentation} & %10 &
								  \num{10} &
								%--
								  \num[round-mode=places,round-precision=2]{0,2} &
								  \num[round-mode=places,round-precision=2]{0,1} \\
								7 & \multicolumn{1}{X}{Allgemeine und vergleichende Literatur- und Sprachwissenschaft} & %39 &
								  \num{39} &
								%--
								  \num[round-mode=places,round-precision=2]{0,77} &
								  \num[round-mode=places,round-precision=2]{0,37} \\
								8 & \multicolumn{1}{X}{Altphilologie (klass. Philologie), Neugriechisch} & %4 &
								  \num{4} &
								%--
								  \num[round-mode=places,round-precision=2]{0,08} &
								  \num[round-mode=places,round-precision=2]{0,04} \\
								9 & \multicolumn{1}{X}{Germanistik (Deutsch, germanische Sprachen ohne Anglistik)} & %202 &
								  \num{202} &
								%--
								  \num[round-mode=places,round-precision=2]{4} &
								  \num[round-mode=places,round-precision=2]{1,92} \\
								10 & \multicolumn{1}{X}{Anglistik, Amerikanistik} & %116 &
								  \num{116} &
								%--
								  \num[round-mode=places,round-precision=2]{2,29} &
								  \num[round-mode=places,round-precision=2]{1,11} \\
							... & ... & ... & ... & ... \\
								66 & \multicolumn{1}{X}{Architektur, Innenarchitektur} & %108 &
								  \num{108} &
								%--
								  \num[round-mode=places,round-precision=2]{2,14} &
								  \num[round-mode=places,round-precision=2]{1,03} \\

								67 & \multicolumn{1}{X}{Raumplanung} & %16 &
								  \num{16} &
								%--
								  \num[round-mode=places,round-precision=2]{0,32} &
								  \num[round-mode=places,round-precision=2]{0,15} \\

								68 & \multicolumn{1}{X}{Bauingenieurwesen} & %78 &
								  \num{78} &
								%--
								  \num[round-mode=places,round-precision=2]{1,54} &
								  \num[round-mode=places,round-precision=2]{0,74} \\

								69 & \multicolumn{1}{X}{Vermessungswesen} & %24 &
								  \num{24} &
								%--
								  \num[round-mode=places,round-precision=2]{0,47} &
								  \num[round-mode=places,round-precision=2]{0,23} \\

								74 & \multicolumn{1}{X}{Kunst, Kunstwissenschaft allgemein} & %29 &
								  \num{29} &
								%--
								  \num[round-mode=places,round-precision=2]{0,57} &
								  \num[round-mode=places,round-precision=2]{0,28} \\

								75 & \multicolumn{1}{X}{Bildende Kunst} & %8 &
								  \num{8} &
								%--
								  \num[round-mode=places,round-precision=2]{0,16} &
								  \num[round-mode=places,round-precision=2]{0,08} \\

								76 & \multicolumn{1}{X}{Gestaltung} & %24 &
								  \num{24} &
								%--
								  \num[round-mode=places,round-precision=2]{0,47} &
								  \num[round-mode=places,round-precision=2]{0,23} \\

								77 & \multicolumn{1}{X}{Darstellende Kunst, Film und Fernsehen, Theaterwissenschaft} & %9 &
								  \num{9} &
								%--
								  \num[round-mode=places,round-precision=2]{0,18} &
								  \num[round-mode=places,round-precision=2]{0,09} \\

								78 & \multicolumn{1}{X}{Musik, Musikwissenschaft} & %41 &
								  \num{41} &
								%--
								  \num[round-mode=places,round-precision=2]{0,81} &
								  \num[round-mode=places,round-precision=2]{0,39} \\

								83 & \multicolumn{1}{X}{Außerhalb der Studienbereichsgliederung} & %4 &
								  \num{4} &
								%--
								  \num[round-mode=places,round-precision=2]{0,08} &
								  \num[round-mode=places,round-precision=2]{0,04} \\

					\midrule
					\multicolumn{2}{l}{Summe (gültig)} &
					  \textbf{\num{5056}} &
					\textbf{100} &
					  \textbf{\num[round-mode=places,round-precision=2]{48,18}} \\
					%--
					\multicolumn{5}{l}{\textbf{Fehlende Werte}}\\
							-998 &
							keine Angabe &
							  \num{5438} &
							 - &
							  \num[round-mode=places,round-precision=2]{51,82} \\
					\midrule
					\multicolumn{2}{l}{\textbf{Summe (gesamt)}} &
				      \textbf{\num{10494}} &
				    \textbf{-} &
				    \textbf{100} \\
					\bottomrule
					\end{longtable}
					\end{filecontents}
					\LTXtable{\textwidth}{\jobname-astu012e_g2d}
				\label{tableValues:astu012e_g2d}
				\vspace*{-\baselineskip}
                    \begin{noten}
                	    \note{} Deskritive Maßzahlen:
                	    Anzahl unterschiedlicher Beobachtungen: 59%
                	    ; 
                	      Modus ($h$): 30
                     \end{noten}


