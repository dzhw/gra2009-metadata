%EVERY VARIABLE HAS IT'S OWN PAGE

    \setcounter{footnote}{0}

    %omit vertical space
    \vspace*{-1.8cm}
	\section{bfec151g\_g1o (1. weitere akad. Qualifikation: Studienfach)}
	\label{section:bfec151g_g1o}



	%TABLE FOR VARIABLE DETAILS
    \vspace*{0.5cm}
    \noindent\textbf{Eigenschaften
	% '#' has to be escaped
	\footnote{Detailliertere Informationen zur Variable finden sich unter
		\url{https://metadata.fdz.dzhw.eu/\#!/de/variables/var-gra2009-ds1-bfec151g_g1o$}}}\\
	\begin{tabularx}{\hsize}{@{}lX}
	Datentyp: & numerisch \\
	Skalenniveau: & nominal \\
	Zugangswege: &
	  onsite-suf
 \\
    \end{tabularx}



    %TABLE FOR QUESTION DETAILS
    %This has to be tested and has to be improved
    %rausfinden, ob einer Variable mehrere Fragen zugeordnet werden
    %dann evtl. nur die erste verwenden oder etwas anderes tun (Hinweis mehrere Fragen, auflisten mit Link)
				%TABLE FOR QUESTION DETAILS
				\vspace*{0.5cm}
                \noindent\textbf{Frage
	                \footnote{Detailliertere Informationen zur Frage finden sich unter
		              \url{https://metadata.fdz.dzhw.eu/\#!/de/questions/que-gra2009-ins2-5.2$}}}\\
				\begin{tabularx}{\hsize}{@{}lX}
					Fragenummer: &
					  Fragebogen des DZHW-Absolventenpanels 2009 - zweite Welle, Hauptbefragung (PAPI):
					  5.2
 \\
					%--
					Fragetext: & Bitte tragen Sie diese längerfristigen Studienangebote, die Sie nach Ihrem Studienabschluss aus dem Jahr 2008/2009 begonnen, weitergeführt oder abgeschlossen haben (auch abgebrochene oder unterbrochene), in das folgende Tableau ein!\par  1. Studienangebot\par  Studienfach/ Fachgebiet \\
				\end{tabularx}
				%TABLE FOR QUESTION DETAILS
				\vspace*{0.5cm}
                \noindent\textbf{Frage
	                \footnote{Detailliertere Informationen zur Frage finden sich unter
		              \url{https://metadata.fdz.dzhw.eu/\#!/de/questions/que-gra2009-ins3-47$}}}\\
				\begin{tabularx}{\hsize}{@{}lX}
					Fragenummer: &
					  Fragebogen des DZHW-Absolventenpanels 2009 - zweite Welle, Hauptbefragung (CAWI):
					  47
 \\
					%--
					Fragetext: & Bitte tragen Sie diese längerfristigen Studienangebote, die Sie nach Ihrem Studienabschluss aus dem Jahr 2008/2009 begonnen, weitergeführt oder abgeschlossen haben (auch abgebrochene oder unterbrochene), in das folgenden Tableau ein! \\
				\end{tabularx}





				%TABLE FOR THE NOMINAL / ORDINAL VALUES
        		\vspace*{0.5cm}
                \noindent\textbf{Häufigkeiten}

                \vspace*{-\baselineskip}
					%NUMERIC ELEMENTS NEED A HUGH SECOND COLOUMN AND A SMALL FIRST ONE
					\begin{filecontents}{\jobname-bfec151g_g1o}
					\begin{longtable}{lXrrr}
					\toprule
					\textbf{Wert} & \textbf{Label} & \textbf{Häufigkeit} & \textbf{Prozent(gültig)} & \textbf{Prozent} \\
					\endhead
					\midrule
					\multicolumn{5}{l}{\textbf{Gültige Werte}}\\
						%DIFFERENT OBSERVATIONS <=20
								3 & \multicolumn{1}{X}{Agrarwissenschaft/Landwirtschaft} & %11 &
								  \num{11} &
								%--
								  \num[round-mode=places,round-precision=2]{0,62} &
								  \num[round-mode=places,round-precision=2]{0,1} \\
								4 & \multicolumn{1}{X}{Interdisziplinäre Studien (Schwerp. Sprach- und Kulturwissenschaften)} & %39 &
								  \num{39} &
								%--
								  \num[round-mode=places,round-precision=2]{2,21} &
								  \num[round-mode=places,round-precision=2]{0,37} \\
								5 & \multicolumn{1}{X}{Klassische Philologie} & %1 &
								  \num{1} &
								%--
								  \num[round-mode=places,round-precision=2]{0,06} &
								  \num[round-mode=places,round-precision=2]{0,01} \\
								6 & \multicolumn{1}{X}{Amerikanistik/Amerikakunde} & %2 &
								  \num{2} &
								%--
								  \num[round-mode=places,round-precision=2]{0,11} &
								  \num[round-mode=places,round-precision=2]{0,02} \\
								8 & \multicolumn{1}{X}{Anglistik/Englisch} & %17 &
								  \num{17} &
								%--
								  \num[round-mode=places,round-precision=2]{0,96} &
								  \num[round-mode=places,round-precision=2]{0,16} \\
								11 & \multicolumn{1}{X}{Arbeitslehre/Wirtschaftslehre} & %1 &
								  \num{1} &
								%--
								  \num[round-mode=places,round-precision=2]{0,06} &
								  \num[round-mode=places,round-precision=2]{0,01} \\
								13 & \multicolumn{1}{X}{Architektur} & %20 &
								  \num{20} &
								%--
								  \num[round-mode=places,round-precision=2]{1,13} &
								  \num[round-mode=places,round-precision=2]{0,19} \\
								14 & \multicolumn{1}{X}{Astronomie, Astrophysik} & %1 &
								  \num{1} &
								%--
								  \num[round-mode=places,round-precision=2]{0,06} &
								  \num[round-mode=places,round-precision=2]{0,01} \\
								17 & \multicolumn{1}{X}{Bauingenieurwesen/Ingenieurbau} & %37 &
								  \num{37} &
								%--
								  \num[round-mode=places,round-precision=2]{2,1} &
								  \num[round-mode=places,round-precision=2]{0,35} \\
								21 & \multicolumn{1}{X}{Betriebswirtschaftslehre} & %212 &
								  \num{212} &
								%--
								  \num[round-mode=places,round-precision=2]{12} &
								  \num[round-mode=places,round-precision=2]{2,02} \\
							... & ... & ... & ... & ... \\
								321 & \multicolumn{1}{X}{Erwachsenenbildung und außerschulische Jugendbildung} & %11 &
								  \num{11} &
								%--
								  \num[round-mode=places,round-precision=2]{0,62} &
								  \num[round-mode=places,round-precision=2]{0,1} \\

								333 & \multicolumn{1}{X}{Haushaltswissenschaft} & %2 &
								  \num{2} &
								%--
								  \num[round-mode=places,round-precision=2]{0,11} &
								  \num[round-mode=places,round-precision=2]{0,02} \\

								353 & \multicolumn{1}{X}{Pflanzenproduktion} & %1 &
								  \num{1} &
								%--
								  \num[round-mode=places,round-precision=2]{0,06} &
								  \num[round-mode=places,round-precision=2]{0,01} \\

								361 & \multicolumn{1}{X}{Schulpädagogik} & %2 &
								  \num{2} &
								%--
								  \num[round-mode=places,round-precision=2]{0,11} &
								  \num[round-mode=places,round-precision=2]{0,02} \\

								380 & \multicolumn{1}{X}{Mechatronik} & %3 &
								  \num{3} &
								%--
								  \num[round-mode=places,round-precision=2]{0,17} &
								  \num[round-mode=places,round-precision=2]{0,03} \\

								457 & \multicolumn{1}{X}{Umwelttechnik einschl. Recycling} & %5 &
								  \num{5} &
								%--
								  \num[round-mode=places,round-precision=2]{0,28} &
								  \num[round-mode=places,round-precision=2]{0,05} \\

								458 & \multicolumn{1}{X}{Umweltschutz} & %1 &
								  \num{1} &
								%--
								  \num[round-mode=places,round-precision=2]{0,06} &
								  \num[round-mode=places,round-precision=2]{0,01} \\

								464 & \multicolumn{1}{X}{Facility Management} & %4 &
								  \num{4} &
								%--
								  \num[round-mode=places,round-precision=2]{0,23} &
								  \num[round-mode=places,round-precision=2]{0,04} \\

								544 & \multicolumn{1}{X}{Evang. Religionspädagogik, kirchliche Bildungsarbeit} & %3 &
								  \num{3} &
								%--
								  \num[round-mode=places,round-precision=2]{0,17} &
								  \num[round-mode=places,round-precision=2]{0,03} \\

								545 & \multicolumn{1}{X}{Kath. Religionspädagogik, kirchliche Bildungsarbeit} & %2 &
								  \num{2} &
								%--
								  \num[round-mode=places,round-precision=2]{0,11} &
								  \num[round-mode=places,round-precision=2]{0,02} \\

					\midrule
					\multicolumn{2}{l}{Summe (gültig)} &
					  \textbf{\num{1766}} &
					\textbf{100} &
					  \textbf{\num[round-mode=places,round-precision=2]{16,83}} \\
					%--
					\multicolumn{5}{l}{\textbf{Fehlende Werte}}\\
							-998 &
							keine Angabe &
							  \num{327} &
							 - &
							  \num[round-mode=places,round-precision=2]{3,12} \\
							-995 &
							keine Teilnahme (Panel) &
							  \num{5739} &
							 - &
							  \num[round-mode=places,round-precision=2]{54,69} \\
							-989 &
							filterbedingt fehlend &
							  \num{2662} &
							 - &
							  \num[round-mode=places,round-precision=2]{25,37} \\
					\midrule
					\multicolumn{2}{l}{\textbf{Summe (gesamt)}} &
				      \textbf{\num{10494}} &
				    \textbf{-} &
				    \textbf{100} \\
					\bottomrule
					\end{longtable}
					\end{filecontents}
					\LTXtable{\textwidth}{\jobname-bfec151g_g1o}
				\label{tableValues:bfec151g_g1o}
				\vspace*{-\baselineskip}
                    \begin{noten}
                	    \note{} Deskritive Maßzahlen:
                	    Anzahl unterschiedlicher Beobachtungen: 153%
                	    ; 
                	      Modus ($h$): 21
                     \end{noten}


