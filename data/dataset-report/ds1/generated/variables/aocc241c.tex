%EVERY VARIABLE HAS IT'S OWN PAGE

    \setcounter{footnote}{0}

    %omit vertical space
    \vspace*{-1.8cm}
	\section{aocc241c (1. Tätigkeit: Ende (Monat))}
	\label{section:aocc241c}



	%TABLE FOR VARIABLE DETAILS
    \vspace*{0.5cm}
    \noindent\textbf{Eigenschaften
	% '#' has to be escaped
	\footnote{Detailliertere Informationen zur Variable finden sich unter
		\url{https://metadata.fdz.dzhw.eu/\#!/de/variables/var-gra2009-ds1-aocc241c$}}}\\
	\begin{tabularx}{\hsize}{@{}lX}
	Datentyp: & numerisch \\
	Skalenniveau: & ordinal \\
	Zugangswege: &
	  download-cuf, 
	  download-suf, 
	  remote-desktop-suf, 
	  onsite-suf
 \\
    \end{tabularx}



    %TABLE FOR QUESTION DETAILS
    %This has to be tested and has to be improved
    %rausfinden, ob einer Variable mehrere Fragen zugeordnet werden
    %dann evtl. nur die erste verwenden oder etwas anderes tun (Hinweis mehrere Fragen, auflisten mit Link)
				%TABLE FOR QUESTION DETAILS
				\vspace*{0.5cm}
                \noindent\textbf{Frage
	                \footnote{Detailliertere Informationen zur Frage finden sich unter
		              \url{https://metadata.fdz.dzhw.eu/\#!/de/questions/que-gra2009-ins1-5.4$}}}\\
				\begin{tabularx}{\hsize}{@{}lX}
					Fragenummer: &
					  Fragebogen des DZHW-Absolventenpanels 2009 - erste Welle:
					  5.4
 \\
					%--
					Fragetext: & Im Folgenden bitten wir Sie um eine Beschreibung der verschiedenen beruflichen Tätigkeiten, die Sie seit Ihrem Studienabschluss ausgeübt haben.\par  1. Erwerbstätigkeit\par  Zeitraum (Monat/ Jahr)\par  bis:\par  Monat \\
				\end{tabularx}





				%TABLE FOR THE NOMINAL / ORDINAL VALUES
        		\vspace*{0.5cm}
                \noindent\textbf{Häufigkeiten}

                \vspace*{-\baselineskip}
					%NUMERIC ELEMENTS NEED A HUGH SECOND COLOUMN AND A SMALL FIRST ONE
					\begin{filecontents}{\jobname-aocc241c}
					\begin{longtable}{lXrrr}
					\toprule
					\textbf{Wert} & \textbf{Label} & \textbf{Häufigkeit} & \textbf{Prozent(gültig)} & \textbf{Prozent} \\
					\endhead
					\midrule
					\multicolumn{5}{l}{\textbf{Gültige Werte}}\\
						%DIFFERENT OBSERVATIONS <=20

					1 &
				% TODO try size/length gt 0; take over for other passages
					\multicolumn{1}{X}{ Januar   } &


					%318 &
					  \num{318} &
					%--
					  \num[round-mode=places,round-precision=2]{9,64} &
					    \num[round-mode=places,round-precision=2]{3,03} \\
							%????

					2 &
				% TODO try size/length gt 0; take over for other passages
					\multicolumn{1}{X}{ Februar   } &


					%276 &
					  \num{276} &
					%--
					  \num[round-mode=places,round-precision=2]{8,37} &
					    \num[round-mode=places,round-precision=2]{2,63} \\
							%????

					3 &
				% TODO try size/length gt 0; take over for other passages
					\multicolumn{1}{X}{ März   } &


					%342 &
					  \num{342} &
					%--
					  \num[round-mode=places,round-precision=2]{10,37} &
					    \num[round-mode=places,round-precision=2]{3,26} \\
							%????

					4 &
				% TODO try size/length gt 0; take over for other passages
					\multicolumn{1}{X}{ April   } &


					%259 &
					  \num{259} &
					%--
					  \num[round-mode=places,round-precision=2]{7,85} &
					    \num[round-mode=places,round-precision=2]{2,47} \\
							%????

					5 &
				% TODO try size/length gt 0; take over for other passages
					\multicolumn{1}{X}{ Mai   } &


					%172 &
					  \num{172} &
					%--
					  \num[round-mode=places,round-precision=2]{5,22} &
					    \num[round-mode=places,round-precision=2]{1,64} \\
							%????

					6 &
				% TODO try size/length gt 0; take over for other passages
					\multicolumn{1}{X}{ Juni   } &


					%209 &
					  \num{209} &
					%--
					  \num[round-mode=places,round-precision=2]{6,34} &
					    \num[round-mode=places,round-precision=2]{1,99} \\
							%????

					7 &
				% TODO try size/length gt 0; take over for other passages
					\multicolumn{1}{X}{ Juli   } &


					%270 &
					  \num{270} &
					%--
					  \num[round-mode=places,round-precision=2]{8,19} &
					    \num[round-mode=places,round-precision=2]{2,57} \\
							%????

					8 &
				% TODO try size/length gt 0; take over for other passages
					\multicolumn{1}{X}{ August   } &


					%290 &
					  \num{290} &
					%--
					  \num[round-mode=places,round-precision=2]{8,79} &
					    \num[round-mode=places,round-precision=2]{2,76} \\
							%????

					9 &
				% TODO try size/length gt 0; take over for other passages
					\multicolumn{1}{X}{ September   } &


					%421 &
					  \num{421} &
					%--
					  \num[round-mode=places,round-precision=2]{12,77} &
					    \num[round-mode=places,round-precision=2]{4,01} \\
							%????

					10 &
				% TODO try size/length gt 0; take over for other passages
					\multicolumn{1}{X}{ Oktober   } &


					%209 &
					  \num{209} &
					%--
					  \num[round-mode=places,round-precision=2]{6,34} &
					    \num[round-mode=places,round-precision=2]{1,99} \\
							%????

					11 &
				% TODO try size/length gt 0; take over for other passages
					\multicolumn{1}{X}{ November   } &


					%136 &
					  \num{136} &
					%--
					  \num[round-mode=places,round-precision=2]{4,12} &
					    \num[round-mode=places,round-precision=2]{1,3} \\
							%????

					12 &
				% TODO try size/length gt 0; take over for other passages
					\multicolumn{1}{X}{ Dezember   } &


					%396 &
					  \num{396} &
					%--
					  \num[round-mode=places,round-precision=2]{12,01} &
					    \num[round-mode=places,round-precision=2]{3,77} \\
							%????
						%DIFFERENT OBSERVATIONS >20
					\midrule
					\multicolumn{2}{l}{Summe (gültig)} &
					  \textbf{\num{3298}} &
					\textbf{100} &
					  \textbf{\num[round-mode=places,round-precision=2]{31,43}} \\
					%--
					\multicolumn{5}{l}{\textbf{Fehlende Werte}}\\
							-998 &
							keine Angabe &
							  \num{5108} &
							 - &
							  \num[round-mode=places,round-precision=2]{48,68} \\
							-989 &
							filterbedingt fehlend &
							  \num{2088} &
							 - &
							  \num[round-mode=places,round-precision=2]{19,9} \\
					\midrule
					\multicolumn{2}{l}{\textbf{Summe (gesamt)}} &
				      \textbf{\num{10494}} &
				    \textbf{-} &
				    \textbf{100} \\
					\bottomrule
					\end{longtable}
					\end{filecontents}
					\LTXtable{\textwidth}{\jobname-aocc241c}
				\label{tableValues:aocc241c}
				\vspace*{-\baselineskip}
                    \begin{noten}
                	    \note{} Deskritive Maßzahlen:
                	    Anzahl unterschiedlicher Beobachtungen: 12%
                	    ; 
                	      Minimum ($min$): 1; 
                	      Maximum ($max$): 12; 
                	      Median ($\tilde{x}$): 7; 
                	      Modus ($h$): 9
                     \end{noten}


