%EVERY VARIABLE HAS IT'S OWN PAGE

    \setcounter{footnote}{0}

    %omit vertical space
    \vspace*{-1.8cm}
	\section{bfec151h\_g3r (1. weitere akad. Qualifikation: Hochschule (Bundes-/Ausland))}
	\label{section:bfec151h_g3r}



	%TABLE FOR VARIABLE DETAILS
    \vspace*{0.5cm}
    \noindent\textbf{Eigenschaften
	% '#' has to be escaped
	\footnote{Detailliertere Informationen zur Variable finden sich unter
		\url{https://metadata.fdz.dzhw.eu/\#!/de/variables/var-gra2009-ds1-bfec151h_g3r$}}}\\
	\begin{tabularx}{\hsize}{@{}lX}
	Datentyp: & numerisch \\
	Skalenniveau: & nominal \\
	Zugangswege: &
	  remote-desktop-suf, 
	  onsite-suf
 \\
    \end{tabularx}



    %TABLE FOR QUESTION DETAILS
    %This has to be tested and has to be improved
    %rausfinden, ob einer Variable mehrere Fragen zugeordnet werden
    %dann evtl. nur die erste verwenden oder etwas anderes tun (Hinweis mehrere Fragen, auflisten mit Link)
				%TABLE FOR QUESTION DETAILS
				\vspace*{0.5cm}
                \noindent\textbf{Frage
	                \footnote{Detailliertere Informationen zur Frage finden sich unter
		              \url{https://metadata.fdz.dzhw.eu/\#!/de/questions/que-gra2009-ins2-5.2$}}}\\
				\begin{tabularx}{\hsize}{@{}lX}
					Fragenummer: &
					  Fragebogen des DZHW-Absolventenpanels 2009 - zweite Welle, Hauptbefragung (PAPI):
					  5.2
 \\
					%--
					Fragetext: & Bitte tragen Sie diese längerfristigen Studienangebote, die Sie nach Ihrem Studienabschluss aus dem Jahr 2008/2009 begonnen, weitergeführt oder abgeschlossen haben (auch abgebrochene oder unterbrochene), in das folgende Tableau ein! \\
				\end{tabularx}





				%TABLE FOR THE NOMINAL / ORDINAL VALUES
        		\vspace*{0.5cm}
                \noindent\textbf{Häufigkeiten}

                \vspace*{-\baselineskip}
					%NUMERIC ELEMENTS NEED A HUGH SECOND COLOUMN AND A SMALL FIRST ONE
					\begin{filecontents}{\jobname-bfec151h_g3r}
					\begin{longtable}{lXrrr}
					\toprule
					\textbf{Wert} & \textbf{Label} & \textbf{Häufigkeit} & \textbf{Prozent(gültig)} & \textbf{Prozent} \\
					\endhead
					\midrule
					\multicolumn{5}{l}{\textbf{Gültige Werte}}\\
						%DIFFERENT OBSERVATIONS <=20

					1 &
				% TODO try size/length gt 0; take over for other passages
					\multicolumn{1}{X}{ Schleswig-Holstein   } &


					%45 &
					  \num{45} &
					%--
					  \num[round-mode=places,round-precision=2]{2,63} &
					    \num[round-mode=places,round-precision=2]{0,43} \\
							%????

					2 &
				% TODO try size/length gt 0; take over for other passages
					\multicolumn{1}{X}{ Hamburg   } &


					%45 &
					  \num{45} &
					%--
					  \num[round-mode=places,round-precision=2]{2,63} &
					    \num[round-mode=places,round-precision=2]{0,43} \\
							%????

					3 &
				% TODO try size/length gt 0; take over for other passages
					\multicolumn{1}{X}{ Niedersachsen   } &


					%152 &
					  \num{152} &
					%--
					  \num[round-mode=places,round-precision=2]{8,88} &
					    \num[round-mode=places,round-precision=2]{1,45} \\
							%????

					4 &
				% TODO try size/length gt 0; take over for other passages
					\multicolumn{1}{X}{ Bremen   } &


					%17 &
					  \num{17} &
					%--
					  \num[round-mode=places,round-precision=2]{0,99} &
					    \num[round-mode=places,round-precision=2]{0,16} \\
							%????

					5 &
				% TODO try size/length gt 0; take over for other passages
					\multicolumn{1}{X}{ Nordrhein-Westfalen   } &


					%273 &
					  \num{273} &
					%--
					  \num[round-mode=places,round-precision=2]{15,96} &
					    \num[round-mode=places,round-precision=2]{2,6} \\
							%????

					6 &
				% TODO try size/length gt 0; take over for other passages
					\multicolumn{1}{X}{ Hessen   } &


					%94 &
					  \num{94} &
					%--
					  \num[round-mode=places,round-precision=2]{5,49} &
					    \num[round-mode=places,round-precision=2]{0,9} \\
							%????

					7 &
				% TODO try size/length gt 0; take over for other passages
					\multicolumn{1}{X}{ Rheinland-Pfalz   } &


					%44 &
					  \num{44} &
					%--
					  \num[round-mode=places,round-precision=2]{2,57} &
					    \num[round-mode=places,round-precision=2]{0,42} \\
							%????

					8 &
				% TODO try size/length gt 0; take over for other passages
					\multicolumn{1}{X}{ Baden-Württemberg   } &


					%190 &
					  \num{190} &
					%--
					  \num[round-mode=places,round-precision=2]{11,1} &
					    \num[round-mode=places,round-precision=2]{1,81} \\
							%????

					9 &
				% TODO try size/length gt 0; take over for other passages
					\multicolumn{1}{X}{ Bayern   } &


					%258 &
					  \num{258} &
					%--
					  \num[round-mode=places,round-precision=2]{15,08} &
					    \num[round-mode=places,round-precision=2]{2,46} \\
							%????

					10 &
				% TODO try size/length gt 0; take over for other passages
					\multicolumn{1}{X}{ Saarland   } &


					%6 &
					  \num{6} &
					%--
					  \num[round-mode=places,round-precision=2]{0,35} &
					    \num[round-mode=places,round-precision=2]{0,06} \\
							%????

					11 &
				% TODO try size/length gt 0; take over for other passages
					\multicolumn{1}{X}{ Berlin   } &


					%123 &
					  \num{123} &
					%--
					  \num[round-mode=places,round-precision=2]{7,19} &
					    \num[round-mode=places,round-precision=2]{1,17} \\
							%????

					12 &
				% TODO try size/length gt 0; take over for other passages
					\multicolumn{1}{X}{ Brandenburg   } &


					%39 &
					  \num{39} &
					%--
					  \num[round-mode=places,round-precision=2]{2,28} &
					    \num[round-mode=places,round-precision=2]{0,37} \\
							%????

					13 &
				% TODO try size/length gt 0; take over for other passages
					\multicolumn{1}{X}{ Mecklenburg-Vorpommern   } &


					%38 &
					  \num{38} &
					%--
					  \num[round-mode=places,round-precision=2]{2,22} &
					    \num[round-mode=places,round-precision=2]{0,36} \\
							%????

					14 &
				% TODO try size/length gt 0; take over for other passages
					\multicolumn{1}{X}{ Sachsen   } &


					%115 &
					  \num{115} &
					%--
					  \num[round-mode=places,round-precision=2]{6,72} &
					    \num[round-mode=places,round-precision=2]{1,1} \\
							%????

					15 &
				% TODO try size/length gt 0; take over for other passages
					\multicolumn{1}{X}{ Sachsen-Anhalt   } &


					%38 &
					  \num{38} &
					%--
					  \num[round-mode=places,round-precision=2]{2,22} &
					    \num[round-mode=places,round-precision=2]{0,36} \\
							%????

					16 &
				% TODO try size/length gt 0; take over for other passages
					\multicolumn{1}{X}{ Thüringen   } &


					%108 &
					  \num{108} &
					%--
					  \num[round-mode=places,round-precision=2]{6,31} &
					    \num[round-mode=places,round-precision=2]{1,03} \\
							%????

					21 &
				% TODO try size/length gt 0; take over for other passages
					\multicolumn{1}{X}{ Deutschland ohne nähere Angabe   } &


					%10 &
					  \num{10} &
					%--
					  \num[round-mode=places,round-precision=2]{0,58} &
					    \num[round-mode=places,round-precision=2]{0,1} \\
							%????

					22 &
				% TODO try size/length gt 0; take over for other passages
					\multicolumn{1}{X}{ Ausland   } &


					%116 &
					  \num{116} &
					%--
					  \num[round-mode=places,round-precision=2]{6,78} &
					    \num[round-mode=places,round-precision=2]{1,11} \\
							%????
						%DIFFERENT OBSERVATIONS >20
					\midrule
					\multicolumn{2}{l}{Summe (gültig)} &
					  \textbf{\num{1711}} &
					\textbf{100} &
					  \textbf{\num[round-mode=places,round-precision=2]{16,3}} \\
					%--
					\multicolumn{5}{l}{\textbf{Fehlende Werte}}\\
							-998 &
							keine Angabe &
							  \num{377} &
							 - &
							  \num[round-mode=places,round-precision=2]{3,59} \\
							-995 &
							keine Teilnahme (Panel) &
							  \num{5739} &
							 - &
							  \num[round-mode=places,round-precision=2]{54,69} \\
							-989 &
							filterbedingt fehlend &
							  \num{2662} &
							 - &
							  \num[round-mode=places,round-precision=2]{25,37} \\
							-968 &
							unplausibler Wert &
							  \num{5} &
							 - &
							  \num[round-mode=places,round-precision=2]{0,05} \\
					\midrule
					\multicolumn{2}{l}{\textbf{Summe (gesamt)}} &
				      \textbf{\num{10494}} &
				    \textbf{-} &
				    \textbf{100} \\
					\bottomrule
					\end{longtable}
					\end{filecontents}
					\LTXtable{\textwidth}{\jobname-bfec151h_g3r}
				\label{tableValues:bfec151h_g3r}
				\vspace*{-\baselineskip}
                    \begin{noten}
                	    \note{} Deskritive Maßzahlen:
                	    Anzahl unterschiedlicher Beobachtungen: 18%
                	    ; 
                	      Modus ($h$): 5
                     \end{noten}


