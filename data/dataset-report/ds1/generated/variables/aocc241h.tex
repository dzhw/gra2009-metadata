%EVERY VARIABLE HAS IT'S OWN PAGE

    \setcounter{footnote}{0}

    %omit vertical space
    \vspace*{-1.8cm}
	\section{aocc241h (1. Tätigkeit: Stunden pro Woche)}
	\label{section:aocc241h}



	%TABLE FOR VARIABLE DETAILS
    \vspace*{0.5cm}
    \noindent\textbf{Eigenschaften
	% '#' has to be escaped
	\footnote{Detailliertere Informationen zur Variable finden sich unter
		\url{https://metadata.fdz.dzhw.eu/\#!/de/variables/var-gra2009-ds1-aocc241h$}}}\\
	\begin{tabularx}{\hsize}{@{}lX}
	Datentyp: & numerisch \\
	Skalenniveau: & verhältnis \\
	Zugangswege: &
	  download-cuf, 
	  download-suf, 
	  remote-desktop-suf, 
	  onsite-suf
 \\
    \end{tabularx}



    %TABLE FOR QUESTION DETAILS
    %This has to be tested and has to be improved
    %rausfinden, ob einer Variable mehrere Fragen zugeordnet werden
    %dann evtl. nur die erste verwenden oder etwas anderes tun (Hinweis mehrere Fragen, auflisten mit Link)
				%TABLE FOR QUESTION DETAILS
				\vspace*{0.5cm}
                \noindent\textbf{Frage
	                \footnote{Detailliertere Informationen zur Frage finden sich unter
		              \url{https://metadata.fdz.dzhw.eu/\#!/de/questions/que-gra2009-ins1-5.4$}}}\\
				\begin{tabularx}{\hsize}{@{}lX}
					Fragenummer: &
					  Fragebogen des DZHW-Absolventenpanels 2009 - erste Welle:
					  5.4
 \\
					%--
					Fragetext: & Im Folgenden bitten wir Sie um eine Beschreibung der verschiedenen beruflichen Tätigkeiten, die Sie seit Ihrem Studienabschluss ausgeübt haben.\par  1. Erwerbstätigkeit\par  Arbeitszeit (ggf. laut Arbeitstag)\par  ohne fest vereinbarte Arbeitszeit mit ca. (…) Std./Woche \\
				\end{tabularx}





				%TABLE FOR THE NOMINAL / ORDINAL VALUES
        		\vspace*{0.5cm}
                \noindent\textbf{Häufigkeiten}

                \vspace*{-\baselineskip}
					%NUMERIC ELEMENTS NEED A HUGH SECOND COLOUMN AND A SMALL FIRST ONE
					\begin{filecontents}{\jobname-aocc241h}
					\begin{longtable}{lXrrr}
					\toprule
					\textbf{Wert} & \textbf{Label} & \textbf{Häufigkeit} & \textbf{Prozent(gültig)} & \textbf{Prozent} \\
					\endhead
					\midrule
					\multicolumn{5}{l}{\textbf{Gültige Werte}}\\
						%DIFFERENT OBSERVATIONS <=20
								1 & \multicolumn{1}{X}{-} & %9 &
								  \num{9} &
								%--
								  \num[round-mode=places,round-precision=2]{0,13} &
								  \num[round-mode=places,round-precision=2]{0,09} \\
								2 & \multicolumn{1}{X}{-} & %38 &
								  \num{38} &
								%--
								  \num[round-mode=places,round-precision=2]{0,57} &
								  \num[round-mode=places,round-precision=2]{0,36} \\
								3 & \multicolumn{1}{X}{-} & %37 &
								  \num{37} &
								%--
								  \num[round-mode=places,round-precision=2]{0,55} &
								  \num[round-mode=places,round-precision=2]{0,35} \\
								4 & \multicolumn{1}{X}{-} & %76 &
								  \num{76} &
								%--
								  \num[round-mode=places,round-precision=2]{1,13} &
								  \num[round-mode=places,round-precision=2]{0,72} \\
								5 & \multicolumn{1}{X}{-} & %129 &
								  \num{129} &
								%--
								  \num[round-mode=places,round-precision=2]{1,92} &
								  \num[round-mode=places,round-precision=2]{1,23} \\
								6 & \multicolumn{1}{X}{-} & %107 &
								  \num{107} &
								%--
								  \num[round-mode=places,round-precision=2]{1,59} &
								  \num[round-mode=places,round-precision=2]{1,02} \\
								7 & \multicolumn{1}{X}{-} & %34 &
								  \num{34} &
								%--
								  \num[round-mode=places,round-precision=2]{0,51} &
								  \num[round-mode=places,round-precision=2]{0,32} \\
								8 & \multicolumn{1}{X}{-} & %204 &
								  \num{204} &
								%--
								  \num[round-mode=places,round-precision=2]{3,04} &
								  \num[round-mode=places,round-precision=2]{1,94} \\
								9 & \multicolumn{1}{X}{-} & %37 &
								  \num{37} &
								%--
								  \num[round-mode=places,round-precision=2]{0,55} &
								  \num[round-mode=places,round-precision=2]{0,35} \\
								10 & \multicolumn{1}{X}{-} & %495 &
								  \num{495} &
								%--
								  \num[round-mode=places,round-precision=2]{7,37} &
								  \num[round-mode=places,round-precision=2]{4,72} \\
							... & ... & ... & ... & ... \\
								46 & \multicolumn{1}{X}{-} & %7 &
								  \num{7} &
								%--
								  \num[round-mode=places,round-precision=2]{0,1} &
								  \num[round-mode=places,round-precision=2]{0,07} \\

								48 & \multicolumn{1}{X}{-} & %32 &
								  \num{32} &
								%--
								  \num[round-mode=places,round-precision=2]{0,48} &
								  \num[round-mode=places,round-precision=2]{0,3} \\

								50 & \multicolumn{1}{X}{-} & %72 &
								  \num{72} &
								%--
								  \num[round-mode=places,round-precision=2]{1,07} &
								  \num[round-mode=places,round-precision=2]{0,69} \\

								52 & \multicolumn{1}{X}{-} & %1 &
								  \num{1} &
								%--
								  \num[round-mode=places,round-precision=2]{0,01} &
								  \num[round-mode=places,round-precision=2]{0,01} \\

								55 & \multicolumn{1}{X}{-} & %15 &
								  \num{15} &
								%--
								  \num[round-mode=places,round-precision=2]{0,22} &
								  \num[round-mode=places,round-precision=2]{0,14} \\

								56 & \multicolumn{1}{X}{-} & %4 &
								  \num{4} &
								%--
								  \num[round-mode=places,round-precision=2]{0,06} &
								  \num[round-mode=places,round-precision=2]{0,04} \\

								60 & \multicolumn{1}{X}{-} & %40 &
								  \num{40} &
								%--
								  \num[round-mode=places,round-precision=2]{0,6} &
								  \num[round-mode=places,round-precision=2]{0,38} \\

								65 & \multicolumn{1}{X}{-} & %4 &
								  \num{4} &
								%--
								  \num[round-mode=places,round-precision=2]{0,06} &
								  \num[round-mode=places,round-precision=2]{0,04} \\

								70 & \multicolumn{1}{X}{-} & %7 &
								  \num{7} &
								%--
								  \num[round-mode=places,round-precision=2]{0,1} &
								  \num[round-mode=places,round-precision=2]{0,07} \\

								80 & \multicolumn{1}{X}{-} & %2 &
								  \num{2} &
								%--
								  \num[round-mode=places,round-precision=2]{0,03} &
								  \num[round-mode=places,round-precision=2]{0,02} \\

					\midrule
					\multicolumn{2}{l}{Summe (gültig)} &
					  \textbf{\num{6720}} &
					\textbf{100} &
					  \textbf{\num[round-mode=places,round-precision=2]{64,04}} \\
					%--
					\multicolumn{5}{l}{\textbf{Fehlende Werte}}\\
							-998 &
							keine Angabe &
							  \num{1686} &
							 - &
							  \num[round-mode=places,round-precision=2]{16,07} \\
							-989 &
							filterbedingt fehlend &
							  \num{2088} &
							 - &
							  \num[round-mode=places,round-precision=2]{19,9} \\
					\midrule
					\multicolumn{2}{l}{\textbf{Summe (gesamt)}} &
				      \textbf{\num{10494}} &
				    \textbf{-} &
				    \textbf{100} \\
					\bottomrule
					\end{longtable}
					\end{filecontents}
					\LTXtable{\textwidth}{\jobname-aocc241h}
				\label{tableValues:aocc241h}
				\vspace*{-\baselineskip}
                    \begin{noten}
                	    \note{} Deskritive Maßzahlen:
                	    Anzahl unterschiedlicher Beobachtungen: 55%
                	    ; 
                	      Minimum ($min$): 1; 
                	      Maximum ($max$): 80; 
                	      arithmetisches Mittel ($\bar{x}$): \num[round-mode=places,round-precision=2]{28,6092}; 
                	      Median ($\tilde{x}$): 35; 
                	      Modus ($h$): 40; 
                	      Standardabweichung ($s$): \num[round-mode=places,round-precision=2]{13,5462}; 
                	      Schiefe ($v$): \num[round-mode=places,round-precision=2]{-0,3306}; 
                	      Wölbung ($w$): \num[round-mode=places,round-precision=2]{1,9439}
                     \end{noten}


