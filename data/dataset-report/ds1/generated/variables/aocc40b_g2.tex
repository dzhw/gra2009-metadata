%EVERY VARIABLE HAS IT'S OWN PAGE

    \setcounter{footnote}{0}

    %omit vertical space
    \vspace*{-1.8cm}
	\section{aocc40b\_g2 (Ausbildung vor Studienbeginn: Beruf (KldB 1992 2-Steller))}
	\label{section:aocc40b_g2}



	%TABLE FOR VARIABLE DETAILS
    \vspace*{0.5cm}
    \noindent\textbf{Eigenschaften
	% '#' has to be escaped
	\footnote{Detailliertere Informationen zur Variable finden sich unter
		\url{https://metadata.fdz.dzhw.eu/\#!/de/variables/var-gra2009-ds1-aocc40b_g2$}}}\\
	\begin{tabularx}{\hsize}{@{}lX}
	Datentyp: & numerisch \\
	Skalenniveau: & nominal \\
	Zugangswege: &
	  download-cuf, 
	  download-suf, 
	  remote-desktop-suf, 
	  onsite-suf
 \\
    \end{tabularx}



    %TABLE FOR QUESTION DETAILS
    %This has to be tested and has to be improved
    %rausfinden, ob einer Variable mehrere Fragen zugeordnet werden
    %dann evtl. nur die erste verwenden oder etwas anderes tun (Hinweis mehrere Fragen, auflisten mit Link)
				%TABLE FOR QUESTION DETAILS
				\vspace*{0.5cm}
                \noindent\textbf{Frage
	                \footnote{Detailliertere Informationen zur Frage finden sich unter
		              \url{https://metadata.fdz.dzhw.eu/\#!/de/questions/que-gra2009-ins1-6.6$}}}\\
				\begin{tabularx}{\hsize}{@{}lX}
					Fragenummer: &
					  Fragebogen des DZHW-Absolventenpanels 2009 - erste Welle:
					  6.6
 \\
					%--
					Fragetext: & Haben Sie vor dem Erststudium eine berufliche Ausbildung abgeschlossen? \\
				\end{tabularx}





				%TABLE FOR THE NOMINAL / ORDINAL VALUES
        		\vspace*{0.5cm}
                \noindent\textbf{Häufigkeiten}

                \vspace*{-\baselineskip}
					%NUMERIC ELEMENTS NEED A HUGH SECOND COLOUMN AND A SMALL FIRST ONE
					\begin{filecontents}{\jobname-aocc40b_g2}
					\begin{longtable}{lXrrr}
					\toprule
					\textbf{Wert} & \textbf{Label} & \textbf{Häufigkeit} & \textbf{Prozent(gültig)} & \textbf{Prozent} \\
					\endhead
					\midrule
					\multicolumn{5}{l}{\textbf{Gültige Werte}}\\
						%DIFFERENT OBSERVATIONS <=20
								1 & \multicolumn{1}{X}{Landwirtschaftliche Berufe} & %10 &
								  \num{10} &
								%--
								  \num[round-mode=places,round-precision=2]{0,4} &
								  \num[round-mode=places,round-precision=2]{0,1} \\
								2 & \multicolumn{1}{X}{Tierwirtschaftliche Berufe} & %7 &
								  \num{7} &
								%--
								  \num[round-mode=places,round-precision=2]{0,28} &
								  \num[round-mode=places,round-precision=2]{0,07} \\
								3 & \multicolumn{1}{X}{Verwaltungs-, Beratungs- und technische Fachkräfte in der Land- und Tierwirtschaft} & %3 &
								  \num{3} &
								%--
								  \num[round-mode=places,round-precision=2]{0,12} &
								  \num[round-mode=places,round-precision=2]{0,03} \\
								5 & \multicolumn{1}{X}{Gartenbauberufe} & %37 &
								  \num{37} &
								%--
								  \num[round-mode=places,round-precision=2]{1,48} &
								  \num[round-mode=places,round-precision=2]{0,35} \\
								6 & \multicolumn{1}{X}{Forst-, Jagdberufe} & %5 &
								  \num{5} &
								%--
								  \num[round-mode=places,round-precision=2]{0,2} &
								  \num[round-mode=places,round-precision=2]{0,05} \\
								7 & \multicolumn{1}{X}{Bergleute} & %1 &
								  \num{1} &
								%--
								  \num[round-mode=places,round-precision=2]{0,04} &
								  \num[round-mode=places,round-precision=2]{0,01} \\
								10 & \multicolumn{1}{X}{Steinbearbeiter/Steinbearbeiterinnen} & %1 &
								  \num{1} &
								%--
								  \num[round-mode=places,round-precision=2]{0,04} &
								  \num[round-mode=places,round-precision=2]{0,01} \\
								12 & \multicolumn{1}{X}{Keramiker/Keramikerinnen} & %1 &
								  \num{1} &
								%--
								  \num[round-mode=places,round-precision=2]{0,04} &
								  \num[round-mode=places,round-precision=2]{0,01} \\
								14 & \multicolumn{1}{X}{Chemieberufe} & %7 &
								  \num{7} &
								%--
								  \num[round-mode=places,round-precision=2]{0,28} &
								  \num[round-mode=places,round-precision=2]{0,07} \\
								17 & \multicolumn{1}{X}{Druck- und Druckweiterverarbeitungsberufe} & %5 &
								  \num{5} &
								%--
								  \num[round-mode=places,round-precision=2]{0,2} &
								  \num[round-mode=places,round-precision=2]{0,05} \\
							... & ... & ... & ... & ... \\
								83 & \multicolumn{1}{X}{Künstlerische und zugeordnete Berufe} & %75 &
								  \num{75} &
								%--
								  \num[round-mode=places,round-precision=2]{3} &
								  \num[round-mode=places,round-precision=2]{0,71} \\

								84 & \multicolumn{1}{X}{Ärzte/Ärztinnen, Apotheker/Apothekerinnen} & %2 &
								  \num{2} &
								%--
								  \num[round-mode=places,round-precision=2]{0,08} &
								  \num[round-mode=places,round-precision=2]{0,02} \\

								85 & \multicolumn{1}{X}{Übrige Gesundheitsdienstberufe} & %316 &
								  \num{316} &
								%--
								  \num[round-mode=places,round-precision=2]{12,66} &
								  \num[round-mode=places,round-precision=2]{3,01} \\

								86 & \multicolumn{1}{X}{Soziale Berufe} & %135 &
								  \num{135} &
								%--
								  \num[round-mode=places,round-precision=2]{5,41} &
								  \num[round-mode=places,round-precision=2]{1,29} \\

								87 & \multicolumn{1}{X}{Lehrer/Lehrerinnen} & %14 &
								  \num{14} &
								%--
								  \num[round-mode=places,round-precision=2]{0,56} &
								  \num[round-mode=places,round-precision=2]{0,13} \\

								88 & \multicolumn{1}{X}{Geistes- und naturwissenschaftliche Berufe, a.n.g.} & %8 &
								  \num{8} &
								%--
								  \num[round-mode=places,round-precision=2]{0,32} &
								  \num[round-mode=places,round-precision=2]{0,08} \\

								90 & \multicolumn{1}{X}{Berufe in der Körperpflege} & %9 &
								  \num{9} &
								%--
								  \num[round-mode=places,round-precision=2]{0,36} &
								  \num[round-mode=places,round-precision=2]{0,09} \\

								91 & \multicolumn{1}{X}{Hotel- und Gaststättenberufe} & %23 &
								  \num{23} &
								%--
								  \num[round-mode=places,round-precision=2]{0,92} &
								  \num[round-mode=places,round-precision=2]{0,22} \\

								92 & \multicolumn{1}{X}{Haus- und ernährungswirtschaftliche Berufe} & %6 &
								  \num{6} &
								%--
								  \num[round-mode=places,round-precision=2]{0,24} &
								  \num[round-mode=places,round-precision=2]{0,06} \\

								93 & \multicolumn{1}{X}{Reinigungs- und Entsorgungsberufe} & %4 &
								  \num{4} &
								%--
								  \num[round-mode=places,round-precision=2]{0,16} &
								  \num[round-mode=places,round-precision=2]{0,04} \\

					\midrule
					\multicolumn{2}{l}{Summe (gültig)} &
					  \textbf{\num{2497}} &
					\textbf{100} &
					  \textbf{\num[round-mode=places,round-precision=2]{23,79}} \\
					%--
					\multicolumn{5}{l}{\textbf{Fehlende Werte}}\\
							-998 &
							keine Angabe &
							  \num{104} &
							 - &
							  \num[round-mode=places,round-precision=2]{0,99} \\
							-988 &
							trifft nicht zu &
							  \num{7877} &
							 - &
							  \num[round-mode=places,round-precision=2]{75,06} \\
							-966 &
							nicht bestimmbar &
							  \num{16} &
							 - &
							  \num[round-mode=places,round-precision=2]{0,15} \\
					\midrule
					\multicolumn{2}{l}{\textbf{Summe (gesamt)}} &
				      \textbf{\num{10494}} &
				    \textbf{-} &
				    \textbf{100} \\
					\bottomrule
					\end{longtable}
					\end{filecontents}
					\LTXtable{\textwidth}{\jobname-aocc40b_g2}
				\label{tableValues:aocc40b_g2}
				\vspace*{-\baselineskip}
                    \begin{noten}
                	    \note{} Deskritive Maßzahlen:
                	    Anzahl unterschiedlicher Beobachtungen: 66%
                	    ; 
                	      Modus ($h$): 78
                     \end{noten}


