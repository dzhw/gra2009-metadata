%EVERY VARIABLE HAS IT'S OWN PAGE

    \setcounter{footnote}{0}

    %omit vertical space
    \vspace*{-1.8cm}
	\section{astu03b\_g1 (Lehramt: Schulform)}
	\label{section:astu03b_g1}



	%TABLE FOR VARIABLE DETAILS
    \vspace*{0.5cm}
    \noindent\textbf{Eigenschaften
	% '#' has to be escaped
	\footnote{Detailliertere Informationen zur Variable finden sich unter
		\url{https://metadata.fdz.dzhw.eu/\#!/de/variables/var-gra2009-ds1-astu03b_g1$}}}\\
	\begin{tabularx}{\hsize}{@{}lX}
	Datentyp: & numerisch \\
	Skalenniveau: & nominal \\
	Zugangswege: &
	  download-cuf, 
	  download-suf, 
	  remote-desktop-suf, 
	  onsite-suf
 \\
    \end{tabularx}



    %TABLE FOR QUESTION DETAILS
    %This has to be tested and has to be improved
    %rausfinden, ob einer Variable mehrere Fragen zugeordnet werden
    %dann evtl. nur die erste verwenden oder etwas anderes tun (Hinweis mehrere Fragen, auflisten mit Link)
				%TABLE FOR QUESTION DETAILS
				\vspace*{0.5cm}
                \noindent\textbf{Frage
	                \footnote{Detailliertere Informationen zur Frage finden sich unter
		              \url{https://metadata.fdz.dzhw.eu/\#!/de/questions/que-gra2009-ins1-1.3$}}}\\
				\begin{tabularx}{\hsize}{@{}lX}
					Fragenummer: &
					  Fragebogen des DZHW-Absolventenpanels 2009 - erste Welle:
					  1.3
 \\
					%--
					Fragetext: & Streben Sie ein Lehramt an?\par  Ja, und zwar mit folgender Ausrichtung: (z.B.: Grund-/Hauptschule, Gymnasium, Berufsschule, Sek I etc.) \\
				\end{tabularx}





				%TABLE FOR THE NOMINAL / ORDINAL VALUES
        		\vspace*{0.5cm}
                \noindent\textbf{Häufigkeiten}

                \vspace*{-\baselineskip}
					%NUMERIC ELEMENTS NEED A HUGH SECOND COLOUMN AND A SMALL FIRST ONE
					\begin{filecontents}{\jobname-astu03b_g1}
					\begin{longtable}{lXrrr}
					\toprule
					\textbf{Wert} & \textbf{Label} & \textbf{Häufigkeit} & \textbf{Prozent(gültig)} & \textbf{Prozent} \\
					\endhead
					\midrule
					\multicolumn{5}{l}{\textbf{Gültige Werte}}\\
						%DIFFERENT OBSERVATIONS <=20

					1 &
				% TODO try size/length gt 0; take over for other passages
					\multicolumn{1}{X}{ LA Grund-/Hauptschule   } &


					%436 &
					  \num{436} &
					%--
					  \num[round-mode=places,round-precision=2]{32,11} &
					    \num[round-mode=places,round-precision=2]{4,15} \\
							%????

					2 &
				% TODO try size/length gt 0; take over for other passages
					\multicolumn{1}{X}{ LA Realschule   } &


					%264 &
					  \num{264} &
					%--
					  \num[round-mode=places,round-precision=2]{19,44} &
					    \num[round-mode=places,round-precision=2]{2,52} \\
							%????

					3 &
				% TODO try size/length gt 0; take over for other passages
					\multicolumn{1}{X}{ LA Gymnasium   } &


					%402 &
					  \num{402} &
					%--
					  \num[round-mode=places,round-precision=2]{29,6} &
					    \num[round-mode=places,round-precision=2]{3,83} \\
							%????

					4 &
				% TODO try size/length gt 0; take over for other passages
					\multicolumn{1}{X}{ LA Berufsschule   } &


					%155 &
					  \num{155} &
					%--
					  \num[round-mode=places,round-precision=2]{11,41} &
					    \num[round-mode=places,round-precision=2]{1,48} \\
							%????

					5 &
				% TODO try size/length gt 0; take over for other passages
					\multicolumn{1}{X}{ LA Sonderschule   } &


					%99 &
					  \num{99} &
					%--
					  \num[round-mode=places,round-precision=2]{7,29} &
					    \num[round-mode=places,round-precision=2]{0,94} \\
							%????

					6 &
				% TODO try size/length gt 0; take over for other passages
					\multicolumn{1}{X}{ LA Sonstige   } &


					%2 &
					  \num{2} &
					%--
					  \num[round-mode=places,round-precision=2]{0,15} &
					    \num[round-mode=places,round-precision=2]{0,02} \\
							%????
						%DIFFERENT OBSERVATIONS >20
					\midrule
					\multicolumn{2}{l}{Summe (gültig)} &
					  \textbf{\num{1358}} &
					\textbf{100} &
					  \textbf{\num[round-mode=places,round-precision=2]{12,94}} \\
					%--
					\multicolumn{5}{l}{\textbf{Fehlende Werte}}\\
							-998 &
							keine Angabe &
							  \num{3} &
							 - &
							  \num[round-mode=places,round-precision=2]{0,03} \\
							-988 &
							trifft nicht zu &
							  \num{9133} &
							 - &
							  \num[round-mode=places,round-precision=2]{87,03} \\
					\midrule
					\multicolumn{2}{l}{\textbf{Summe (gesamt)}} &
				      \textbf{\num{10494}} &
				    \textbf{-} &
				    \textbf{100} \\
					\bottomrule
					\end{longtable}
					\end{filecontents}
					\LTXtable{\textwidth}{\jobname-astu03b_g1}
				\label{tableValues:astu03b_g1}
				\vspace*{-\baselineskip}
                    \begin{noten}
                	    \note{} Deskritive Maßzahlen:
                	    Anzahl unterschiedlicher Beobachtungen: 6%
                	    ; 
                	      Modus ($h$): 1
                     \end{noten}


