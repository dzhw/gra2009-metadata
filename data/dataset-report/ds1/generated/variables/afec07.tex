%EVERY VARIABLE HAS IT'S OWN PAGE

    \setcounter{footnote}{0}

    %omit vertical space
    \vspace*{-1.8cm}
	\section{afec07 (2. Ausbildungsphase nach Studium)}
	\label{section:afec07}



	%TABLE FOR VARIABLE DETAILS
    \vspace*{0.5cm}
    \noindent\textbf{Eigenschaften
	% '#' has to be escaped
	\footnote{Detailliertere Informationen zur Variable finden sich unter
		\url{https://metadata.fdz.dzhw.eu/\#!/de/variables/var-gra2009-ds1-afec07$}}}\\
	\begin{tabularx}{\hsize}{@{}lX}
	Datentyp: & numerisch \\
	Skalenniveau: & nominal \\
	Zugangswege: &
	  download-cuf, 
	  download-suf, 
	  remote-desktop-suf, 
	  onsite-suf
 \\
    \end{tabularx}



    %TABLE FOR QUESTION DETAILS
    %This has to be tested and has to be improved
    %rausfinden, ob einer Variable mehrere Fragen zugeordnet werden
    %dann evtl. nur die erste verwenden oder etwas anderes tun (Hinweis mehrere Fragen, auflisten mit Link)
				%TABLE FOR QUESTION DETAILS
				\vspace*{0.5cm}
                \noindent\textbf{Frage
	                \footnote{Detailliertere Informationen zur Frage finden sich unter
		              \url{https://metadata.fdz.dzhw.eu/\#!/de/questions/que-gra2009-ins1-3.1$}}}\\
				\begin{tabularx}{\hsize}{@{}lX}
					Fragenummer: &
					  Fragebogen des DZHW-Absolventenpanels 2009 - erste Welle:
					  3.1
 \\
					%--
					Fragetext: & Ist im Anschluss an Ihr Studium eine zweite praktische Ausbildungsstufe vorgesehen (integraler Ausbildungsbestandteil wie z. B. Referendariat, Vikariat, Anerkennungs-/Berufspraktikum)?\par  Nein\par  Ja, aber ich möchte sie nicht absolvieren\par  Ja, aber ich habe noch nicht damit begonnen\par  Ja, ich habe schon damit begonnen\par  Ja, ich habe sie schon abgeschlossen\par  Ja, aber ich habe sie abgebrochen \\
				\end{tabularx}





				%TABLE FOR THE NOMINAL / ORDINAL VALUES
        		\vspace*{0.5cm}
                \noindent\textbf{Häufigkeiten}

                \vspace*{-\baselineskip}
					%NUMERIC ELEMENTS NEED A HUGH SECOND COLOUMN AND A SMALL FIRST ONE
					\begin{filecontents}{\jobname-afec07}
					\begin{longtable}{lXrrr}
					\toprule
					\textbf{Wert} & \textbf{Label} & \textbf{Häufigkeit} & \textbf{Prozent(gültig)} & \textbf{Prozent} \\
					\endhead
					\midrule
					\multicolumn{5}{l}{\textbf{Gültige Werte}}\\
						%DIFFERENT OBSERVATIONS <=20

					1 &
				% TODO try size/length gt 0; take over for other passages
					\multicolumn{1}{X}{ nein   } &


					%8840 &
					  \num{8840} &
					%--
					  \num[round-mode=places,round-precision=2]{84,25} &
					    \num[round-mode=places,round-precision=2]{84,24} \\
							%????

					2 &
				% TODO try size/length gt 0; take over for other passages
					\multicolumn{1}{X}{ ja, möchte aber nicht   } &


					%101 &
					  \num{101} &
					%--
					  \num[round-mode=places,round-precision=2]{0,96} &
					    \num[round-mode=places,round-precision=2]{0,96} \\
							%????

					3 &
				% TODO try size/length gt 0; take over for other passages
					\multicolumn{1}{X}{ ja, noch nicht begonnen   } &


					%302 &
					  \num{302} &
					%--
					  \num[round-mode=places,round-precision=2]{2,88} &
					    \num[round-mode=places,round-precision=2]{2,88} \\
							%????

					4 &
				% TODO try size/length gt 0; take over for other passages
					\multicolumn{1}{X}{ ja, schon begonnen   } &


					%1087 &
					  \num{1087} &
					%--
					  \num[round-mode=places,round-precision=2]{10,36} &
					    \num[round-mode=places,round-precision=2]{10,36} \\
							%????

					5 &
				% TODO try size/length gt 0; take over for other passages
					\multicolumn{1}{X}{ ja, schon abgeschlossen   } &


					%148 &
					  \num{148} &
					%--
					  \num[round-mode=places,round-precision=2]{1,41} &
					    \num[round-mode=places,round-precision=2]{1,41} \\
							%????

					6 &
				% TODO try size/length gt 0; take over for other passages
					\multicolumn{1}{X}{ ja, aber abgebrochen   } &


					%14 &
					  \num{14} &
					%--
					  \num[round-mode=places,round-precision=2]{0,13} &
					    \num[round-mode=places,round-precision=2]{0,13} \\
							%????
						%DIFFERENT OBSERVATIONS >20
					\midrule
					\multicolumn{2}{l}{Summe (gültig)} &
					  \textbf{\num{10492}} &
					\textbf{100} &
					  \textbf{\num[round-mode=places,round-precision=2]{99,98}} \\
					%--
					\multicolumn{5}{l}{\textbf{Fehlende Werte}}\\
							-998 &
							keine Angabe &
							  \num{2} &
							 - &
							  \num[round-mode=places,round-precision=2]{0,02} \\
					\midrule
					\multicolumn{2}{l}{\textbf{Summe (gesamt)}} &
				      \textbf{\num{10494}} &
				    \textbf{-} &
				    \textbf{100} \\
					\bottomrule
					\end{longtable}
					\end{filecontents}
					\LTXtable{\textwidth}{\jobname-afec07}
				\label{tableValues:afec07}
				\vspace*{-\baselineskip}
                    \begin{noten}
                	    \note{} Deskritive Maßzahlen:
                	    Anzahl unterschiedlicher Beobachtungen: 6%
                	    ; 
                	      Modus ($h$): 1
                     \end{noten}


