%EVERY VARIABLE HAS IT'S OWN PAGE

    \setcounter{footnote}{0}

    %omit vertical space
    \vspace*{-1.8cm}
	\section{bocc302a\_v1 (Branche)}
	\label{section:bocc302a_v1}



	%TABLE FOR VARIABLE DETAILS
    \vspace*{0.5cm}
    \noindent\textbf{Eigenschaften
	% '#' has to be escaped
	\footnote{Detailliertere Informationen zur Variable finden sich unter
		\url{https://metadata.fdz.dzhw.eu/\#!/de/variables/var-gra2009-ds1-bocc302a_v1$}}}\\
	\begin{tabularx}{\hsize}{@{}lX}
	Datentyp: & numerisch \\
	Skalenniveau: & nominal \\
	Zugangswege: &
	  download-cuf, 
	  download-suf, 
	  remote-desktop-suf, 
	  onsite-suf
 \\
    \end{tabularx}



    %TABLE FOR QUESTION DETAILS
    %This has to be tested and has to be improved
    %rausfinden, ob einer Variable mehrere Fragen zugeordnet werden
    %dann evtl. nur die erste verwenden oder etwas anderes tun (Hinweis mehrere Fragen, auflisten mit Link)
				%TABLE FOR QUESTION DETAILS
				\vspace*{0.5cm}
                \noindent\textbf{Frage
	                \footnote{Detailliertere Informationen zur Frage finden sich unter
		              \url{https://metadata.fdz.dzhw.eu/\#!/de/questions/que-gra2009-ins2-4.15$}}}\\
				\begin{tabularx}{\hsize}{@{}lX}
					Fragenummer: &
					  Fragebogen des DZHW-Absolventenpanels 2009 - zweite Welle, Hauptbefragung (PAPI):
					  4.15
 \\
					%--
					Fragetext: & Welchem Wirtschaftsbereich gehört(e) der Betrieb bzw. die Einrichtung schwerpunktmäßig an, in dem/in der Sie arbeite(te)n?\par  Tragen Sie bitte hier die zutreffende Kennziffer aus Liste A ein (siehe hintere Umschlagseite). \\
				\end{tabularx}
				%TABLE FOR QUESTION DETAILS
				\vspace*{0.5cm}
                \noindent\textbf{Frage
	                \footnote{Detailliertere Informationen zur Frage finden sich unter
		              \url{https://metadata.fdz.dzhw.eu/\#!/de/questions/que-gra2009-ins3-30$}}}\\
				\begin{tabularx}{\hsize}{@{}lX}
					Fragenummer: &
					  Fragebogen des DZHW-Absolventenpanels 2009 - zweite Welle, Hauptbefragung (CAWI):
					  30
 \\
					%--
					Fragetext: & Welchem Wirtschaftsbereich gehört(e) der Betrieb bzw. die Einrichtung schwerpunktmäßig an, in dem/der Sie arbeite(te)n? \\
				\end{tabularx}





				%TABLE FOR THE NOMINAL / ORDINAL VALUES
        		\vspace*{0.5cm}
                \noindent\textbf{Häufigkeiten}

                \vspace*{-\baselineskip}
					%NUMERIC ELEMENTS NEED A HUGH SECOND COLOUMN AND A SMALL FIRST ONE
					\begin{filecontents}{\jobname-bocc302a_v1}
					\begin{longtable}{lXrrr}
					\toprule
					\textbf{Wert} & \textbf{Label} & \textbf{Häufigkeit} & \textbf{Prozent(gültig)} & \textbf{Prozent} \\
					\endhead
					\midrule
					\multicolumn{5}{l}{\textbf{Gültige Werte}}\\
						%DIFFERENT OBSERVATIONS <=20
								1 & \multicolumn{1}{X}{Land-/Forstwirtschaft, Fischerei} & %49 &
								  \num{49} &
								%--
								  \num[round-mode=places,round-precision=2]{1,05} &
								  \num[round-mode=places,round-precision=2]{0,47} \\
								2 & \multicolumn{1}{X}{Bergbau} & %8 &
								  \num{8} &
								%--
								  \num[round-mode=places,round-precision=2]{0,17} &
								  \num[round-mode=places,round-precision=2]{0,08} \\
								3 & \multicolumn{1}{X}{Energiewirtschaft} & %64 &
								  \num{64} &
								%--
								  \num[round-mode=places,round-precision=2]{1,37} &
								  \num[round-mode=places,round-precision=2]{0,61} \\
								4 & \multicolumn{1}{X}{Wasser-/Abfallwirtschaft} & %26 &
								  \num{26} &
								%--
								  \num[round-mode=places,round-precision=2]{0,56} &
								  \num[round-mode=places,round-precision=2]{0,25} \\
								5 & \multicolumn{1}{X}{Nahrungs-/Getränke-/Futtermittelindustrie} & %44 &
								  \num{44} &
								%--
								  \num[round-mode=places,round-precision=2]{0,94} &
								  \num[round-mode=places,round-precision=2]{0,42} \\
								6 & \multicolumn{1}{X}{chemische Industrie} & %130 &
								  \num{130} &
								%--
								  \num[round-mode=places,round-precision=2]{2,79} &
								  \num[round-mode=places,round-precision=2]{1,24} \\
								7 & \multicolumn{1}{X}{Maschinen-/Fahrzeugbau} & %235 &
								  \num{235} &
								%--
								  \num[round-mode=places,round-precision=2]{5,04} &
								  \num[round-mode=places,round-precision=2]{2,24} \\
								8 & \multicolumn{1}{X}{Elektrotechnik, Elektronik, EDV-Geräte} & %93 &
								  \num{93} &
								%--
								  \num[round-mode=places,round-precision=2]{1,99} &
								  \num[round-mode=places,round-precision=2]{0,89} \\
								9 & \multicolumn{1}{X}{Metallerzeugung/-verarbeitung} & %37 &
								  \num{37} &
								%--
								  \num[round-mode=places,round-precision=2]{0,79} &
								  \num[round-mode=places,round-precision=2]{0,35} \\
								10 & \multicolumn{1}{X}{Bauunternehmen (Bauhauptgewerbe)} & %55 &
								  \num{55} &
								%--
								  \num[round-mode=places,round-precision=2]{1,18} &
								  \num[round-mode=places,round-precision=2]{0,52} \\
							... & ... & ... & ... & ... \\
								26 & \multicolumn{1}{X}{private Aus- und Weiterbildung} & %63 &
								  \num{63} &
								%--
								  \num[round-mode=places,round-precision=2]{1,35} &
								  \num[round-mode=places,round-precision=2]{0,6} \\

								27 & \multicolumn{1}{X}{Schulen} & %573 &
								  \num{573} &
								%--
								  \num[round-mode=places,round-precision=2]{12,29} &
								  \num[round-mode=places,round-precision=2]{5,46} \\

								28 & \multicolumn{1}{X}{Hochschulen} & %644 &
								  \num{644} &
								%--
								  \num[round-mode=places,round-precision=2]{13,81} &
								  \num[round-mode=places,round-precision=2]{6,14} \\

								29 & \multicolumn{1}{X}{Forschungseinrichtungen} & %167 &
								  \num{167} &
								%--
								  \num[round-mode=places,round-precision=2]{3,58} &
								  \num[round-mode=places,round-precision=2]{1,59} \\

								30 & \multicolumn{1}{X}{Kunst, Kultur} & %76 &
								  \num{76} &
								%--
								  \num[round-mode=places,round-precision=2]{1,63} &
								  \num[round-mode=places,round-precision=2]{0,72} \\

								31 & \multicolumn{1}{X}{Kirchen, Glaubensgemeinschaften} & %58 &
								  \num{58} &
								%--
								  \num[round-mode=places,round-precision=2]{1,24} &
								  \num[round-mode=places,round-precision=2]{0,55} \\

								32 & \multicolumn{1}{X}{Berufs-/Wirtschaftsverbände, Parteien, Vereine, internat. Organisationen} & %111 &
								  \num{111} &
								%--
								  \num[round-mode=places,round-precision=2]{2,38} &
								  \num[round-mode=places,round-precision=2]{1,06} \\

								33 & \multicolumn{1}{X}{allg. öffentliche Verwaltung} & %239 &
								  \num{239} &
								%--
								  \num[round-mode=places,round-precision=2]{5,13} &
								  \num[round-mode=places,round-precision=2]{2,28} \\

								34 & \multicolumn{1}{X}{Stiftungen} & %24 &
								  \num{24} &
								%--
								  \num[round-mode=places,round-precision=2]{0,51} &
								  \num[round-mode=places,round-precision=2]{0,23} \\

								35 & \multicolumn{1}{X}{Sonstiges} & %22 &
								  \num{22} &
								%--
								  \num[round-mode=places,round-precision=2]{0,47} &
								  \num[round-mode=places,round-precision=2]{0,21} \\

					\midrule
					\multicolumn{2}{l}{Summe (gültig)} &
					  \textbf{\num{4662}} &
					\textbf{100} &
					  \textbf{\num[round-mode=places,round-precision=2]{44,43}} \\
					%--
					\multicolumn{5}{l}{\textbf{Fehlende Werte}}\\
							-998 &
							keine Angabe &
							  \num{62} &
							 - &
							  \num[round-mode=places,round-precision=2]{0,59} \\
							-995 &
							keine Teilnahme (Panel) &
							  \num{5739} &
							 - &
							  \num[round-mode=places,round-precision=2]{54,69} \\
							-989 &
							filterbedingt fehlend &
							  \num{31} &
							 - &
							  \num[round-mode=places,round-precision=2]{0,3} \\
					\midrule
					\multicolumn{2}{l}{\textbf{Summe (gesamt)}} &
				      \textbf{\num{10494}} &
				    \textbf{-} &
				    \textbf{100} \\
					\bottomrule
					\end{longtable}
					\end{filecontents}
					\LTXtable{\textwidth}{\jobname-bocc302a_v1}
				\label{tableValues:bocc302a_v1}
				\vspace*{-\baselineskip}
                    \begin{noten}
                	    \note{} Deskritive Maßzahlen:
                	    Anzahl unterschiedlicher Beobachtungen: 35%
                	    ; 
                	      Modus ($h$): 28
                     \end{noten}


