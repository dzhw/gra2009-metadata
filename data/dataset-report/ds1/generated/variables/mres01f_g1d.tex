%EVERY VARIABLE HAS IT'S OWN PAGE

    \setcounter{footnote}{0}

    %omit vertical space
    \vspace*{-1.8cm}
	\section{mres01f\_g1d (Wohnung Studium: Ort (NUTS2))}
	\label{section:mres01f_g1d}



	%TABLE FOR VARIABLE DETAILS
    \vspace*{0.5cm}
    \noindent\textbf{Eigenschaften
	% '#' has to be escaped
	\footnote{Detailliertere Informationen zur Variable finden sich unter
		\url{https://metadata.fdz.dzhw.eu/\#!/de/variables/var-gra2009-ds1-mres01f_g1d$}}}\\
	\begin{tabularx}{\hsize}{@{}lX}
	Datentyp: & string \\
	Skalenniveau: & nominal \\
	Zugangswege: &
	  download-suf, 
	  remote-desktop-suf, 
	  onsite-suf
 \\
    \end{tabularx}



    %TABLE FOR QUESTION DETAILS
    %This has to be tested and has to be improved
    %rausfinden, ob einer Variable mehrere Fragen zugeordnet werden
    %dann evtl. nur die erste verwenden oder etwas anderes tun (Hinweis mehrere Fragen, auflisten mit Link)
				%TABLE FOR QUESTION DETAILS
				\vspace*{0.5cm}
                \noindent\textbf{Frage
	                \footnote{Detailliertere Informationen zur Frage finden sich unter
		              \url{https://metadata.fdz.dzhw.eu/\#!/de/questions/que-gra2009-ins5-07.1$}}}\\
				\begin{tabularx}{\hsize}{@{}lX}
					Fragenummer: &
					  Fragebogen des DZHW-Absolventenpanels 2009 - zweite Welle, Vertiefungsbefragung Mobilität:
					  07.1
 \\
					%--
					Fragetext: & Zunächst bitten wir Sie uns dabei mitzuteilen, wo und wie Sie direkt während Ihres Studienabschlusses 2008/09 gewohnt haben \\
				\end{tabularx}





				%TABLE FOR THE NOMINAL / ORDINAL VALUES
        		\vspace*{0.5cm}
                \noindent\textbf{Häufigkeiten}

                \vspace*{-\baselineskip}
					%STRING ELEMENTS NEEDS A HUGH FIRST COLOUMN AND A SMALL SECOND ONE
					\begin{filecontents}{\jobname-mres01f_g1d}
					\begin{longtable}{Xlrrr}
					\toprule
					\textbf{Wert} & \textbf{Label} & \textbf{Häufigkeit} & \textbf{Prozent (gültig)} & \textbf{Prozent} \\
					\endhead
					\midrule
					\multicolumn{5}{l}{\textbf{Gültige Werte}}\\
						%DIFFERENT OBSERVATIONS <=20
								\multicolumn{1}{X}{DE11 Stuttgart} & - & 108 & 4,76 & 1,03 \\
								\multicolumn{1}{X}{DE12 Karlsruhe} & - & 69 & 3,04 & 0,66 \\
								\multicolumn{1}{X}{DE13 Freiburg} & - & 29 & 1,28 & 0,28 \\
								\multicolumn{1}{X}{DE14 Tübingen} & - & 79 & 3,48 & 0,75 \\
								\multicolumn{1}{X}{DE21 Oberbayern} & - & 187 & 8,24 & 1,78 \\
								\multicolumn{1}{X}{DE22 Niederbayern} & - & 44 & 1,94 & 0,42 \\
								\multicolumn{1}{X}{DE23 Oberpfalz} & - & 33 & 1,45 & 0,31 \\
								\multicolumn{1}{X}{DE24 Oberfranken} & - & 34 & 1,5 & 0,32 \\
								\multicolumn{1}{X}{DE25 Mittelfranken} & - & 58 & 2,56 & 0,55 \\
								\multicolumn{1}{X}{DE26 Unterfranken} & - & 8 & 0,35 & 0,08 \\
							... & ... & ... & ... & ... \\
								\multicolumn{1}{X}{DEB1 Koblenz} & - & 40 & 1,76 & 0,38 \\
								\multicolumn{1}{X}{DEB2 Trier} & - & 28 & 1,23 & 0,27 \\
								\multicolumn{1}{X}{DEB3 Rheinhessen-Pfalz} & - & 33 & 1,45 & 0,31 \\
								\multicolumn{1}{X}{DEC0 Saarland} & - & 14 & 0,62 & 0,13 \\
								\multicolumn{1}{X}{DED2 Dresden} & - & 129 & 5,69 & 1,23 \\
								\multicolumn{1}{X}{DED4 Chemnitz} & - & 55 & 2,42 & 0,52 \\
								\multicolumn{1}{X}{DED5 Leipzig} & - & 43 & 1,9 & 0,41 \\
								\multicolumn{1}{X}{DEE0 Sachsen-Anhalt} & - & 39 & 1,72 & 0,37 \\
								\multicolumn{1}{X}{DEF0 Schleswig-Holstein} & - & 68 & 3 & 0,65 \\
								\multicolumn{1}{X}{DEG0 Thüringen} & - & 176 & 7,76 & 1,68 \\
					\midrule
						\multicolumn{2}{l}{Summe (gültig)} & 2269 &
						\textbf{100} &
					    21,62 \\
					\multicolumn{5}{l}{\textbf{Fehlende Werte}}\\
							-966 & nicht bestimmbar & 3 & - & 0,03 \\

							-968 & unplausibler Wert & 55 & - & 0,52 \\

							-995 & keine Teilnahme (Panel) & 8029 & - & 76,51 \\

							-998 & keine Angabe & 138 & - & 1,32 \\

					\midrule
					\multicolumn{2}{l}{\textbf{Summe (gesamt)}} & \textbf{10494} & \textbf{-} & \textbf{100} \\
					\bottomrule
					\caption{Werte der Variable mres01f\_g1d}
					\end{longtable}
					\end{filecontents}
					\LTXtable{\textwidth}{\jobname-mres01f_g1d}


