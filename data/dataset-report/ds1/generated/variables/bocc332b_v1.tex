%EVERY VARIABLE HAS IT'S OWN PAGE

    \setcounter{footnote}{0}

    %omit vertical space
    \vspace*{-1.8cm}
	\section{bocc332b\_v1 (zusätzl. Gehaltsbestandteile: feste Bestandteile (Summe))}
	\label{section:bocc332b_v1}



	%TABLE FOR VARIABLE DETAILS
    \vspace*{0.5cm}
    \noindent\textbf{Eigenschaften
	% '#' has to be escaped
	\footnote{Detailliertere Informationen zur Variable finden sich unter
		\url{https://metadata.fdz.dzhw.eu/\#!/de/variables/var-gra2009-ds1-bocc332b_v1$}}}\\
	\begin{tabularx}{\hsize}{@{}lX}
	Datentyp: & numerisch \\
	Skalenniveau: & verhältnis \\
	Zugangswege: &
	  download-cuf, 
	  download-suf, 
	  remote-desktop-suf, 
	  onsite-suf
 \\
    \end{tabularx}



    %TABLE FOR QUESTION DETAILS
    %This has to be tested and has to be improved
    %rausfinden, ob einer Variable mehrere Fragen zugeordnet werden
    %dann evtl. nur die erste verwenden oder etwas anderes tun (Hinweis mehrere Fragen, auflisten mit Link)
				%TABLE FOR QUESTION DETAILS
				\vspace*{0.5cm}
                \noindent\textbf{Frage
	                \footnote{Detailliertere Informationen zur Frage finden sich unter
		              \url{https://metadata.fdz.dzhw.eu/\#!/de/questions/que-gra2009-ins2-4.20$}}}\\
				\begin{tabularx}{\hsize}{@{}lX}
					Fragenummer: &
					  Fragebogen des DZHW-Absolventenpanels 2009 - zweite Welle, Hauptbefragung (PAPI):
					  4.20
 \\
					%--
					Fragetext: & Welche zusätzlichen (Brutto-)Gehaltsbestandteile bekommen/bekamen Sie?\par  Feste Gehaltsbestandteile (z. B. Weihnachtsgeld, Urlaubsgeld, 13. Monatsgehalt, Schichtzulage) Euro/ Jahr \\
				\end{tabularx}
				%TABLE FOR QUESTION DETAILS
				\vspace*{0.5cm}
                \noindent\textbf{Frage
	                \footnote{Detailliertere Informationen zur Frage finden sich unter
		              \url{https://metadata.fdz.dzhw.eu/\#!/de/questions/que-gra2009-ins3-35$}}}\\
				\begin{tabularx}{\hsize}{@{}lX}
					Fragenummer: &
					  Fragebogen des DZHW-Absolventenpanels 2009 - zweite Welle, Hauptbefragung (CAWI):
					  35
 \\
					%--
					Fragetext: & Welche zusätzlichen (Brutto-)Gehaltsbestandteile bekommen/bekamen Sie? \\
				\end{tabularx}





				%TABLE FOR THE NOMINAL / ORDINAL VALUES
        		\vspace*{0.5cm}
                \noindent\textbf{Häufigkeiten}

                \vspace*{-\baselineskip}
					%NUMERIC ELEMENTS NEED A HUGH SECOND COLOUMN AND A SMALL FIRST ONE
					\begin{filecontents}{\jobname-bocc332b_v1}
					\begin{longtable}{lXrrr}
					\toprule
					\textbf{Wert} & \textbf{Label} & \textbf{Häufigkeit} & \textbf{Prozent(gültig)} & \textbf{Prozent} \\
					\endhead
					\midrule
					\multicolumn{5}{l}{\textbf{Gültige Werte}}\\
						%DIFFERENT OBSERVATIONS <=20
								1 & \multicolumn{1}{X}{-} & %2 &
								  \num{2} &
								%--
								  \num[round-mode=places,round-precision=2]{0,11} &
								  \num[round-mode=places,round-precision=2]{0,02} \\
								50 & \multicolumn{1}{X}{-} & %1 &
								  \num{1} &
								%--
								  \num[round-mode=places,round-precision=2]{0,05} &
								  \num[round-mode=places,round-precision=2]{0,01} \\
								60 & \multicolumn{1}{X}{-} & %2 &
								  \num{2} &
								%--
								  \num[round-mode=places,round-precision=2]{0,11} &
								  \num[round-mode=places,round-precision=2]{0,02} \\
								80 & \multicolumn{1}{X}{-} & %1 &
								  \num{1} &
								%--
								  \num[round-mode=places,round-precision=2]{0,05} &
								  \num[round-mode=places,round-precision=2]{0,01} \\
								90 & \multicolumn{1}{X}{-} & %1 &
								  \num{1} &
								%--
								  \num[round-mode=places,round-precision=2]{0,05} &
								  \num[round-mode=places,round-precision=2]{0,01} \\
								100 & \multicolumn{1}{X}{-} & %11 &
								  \num{11} &
								%--
								  \num[round-mode=places,round-precision=2]{0,59} &
								  \num[round-mode=places,round-precision=2]{0,1} \\
								130 & \multicolumn{1}{X}{-} & %1 &
								  \num{1} &
								%--
								  \num[round-mode=places,round-precision=2]{0,05} &
								  \num[round-mode=places,round-precision=2]{0,01} \\
								150 & \multicolumn{1}{X}{-} & %7 &
								  \num{7} &
								%--
								  \num[round-mode=places,round-precision=2]{0,38} &
								  \num[round-mode=places,round-precision=2]{0,07} \\
								164 & \multicolumn{1}{X}{-} & %1 &
								  \num{1} &
								%--
								  \num[round-mode=places,round-precision=2]{0,05} &
								  \num[round-mode=places,round-precision=2]{0,01} \\
								180 & \multicolumn{1}{X}{-} & %1 &
								  \num{1} &
								%--
								  \num[round-mode=places,round-precision=2]{0,05} &
								  \num[round-mode=places,round-precision=2]{0,01} \\
							... & ... & ... & ... & ... \\
								66000 & \multicolumn{1}{X}{-} & %3 &
								  \num{3} &
								%--
								  \num[round-mode=places,round-precision=2]{0,16} &
								  \num[round-mode=places,round-precision=2]{0,03} \\

								70000 & \multicolumn{1}{X}{-} & %3 &
								  \num{3} &
								%--
								  \num[round-mode=places,round-precision=2]{0,16} &
								  \num[round-mode=places,round-precision=2]{0,03} \\

								70200 & \multicolumn{1}{X}{-} & %1 &
								  \num{1} &
								%--
								  \num[round-mode=places,round-precision=2]{0,05} &
								  \num[round-mode=places,round-precision=2]{0,01} \\

								71400 & \multicolumn{1}{X}{-} & %1 &
								  \num{1} &
								%--
								  \num[round-mode=places,round-precision=2]{0,05} &
								  \num[round-mode=places,round-precision=2]{0,01} \\

								80000 & \multicolumn{1}{X}{-} & %3 &
								  \num{3} &
								%--
								  \num[round-mode=places,round-precision=2]{0,16} &
								  \num[round-mode=places,round-precision=2]{0,03} \\

								81000 & \multicolumn{1}{X}{-} & %1 &
								  \num{1} &
								%--
								  \num[round-mode=places,round-precision=2]{0,05} &
								  \num[round-mode=places,round-precision=2]{0,01} \\

								85000 & \multicolumn{1}{X}{-} & %1 &
								  \num{1} &
								%--
								  \num[round-mode=places,round-precision=2]{0,05} &
								  \num[round-mode=places,round-precision=2]{0,01} \\

								98000 & \multicolumn{1}{X}{-} & %2 &
								  \num{2} &
								%--
								  \num[round-mode=places,round-precision=2]{0,11} &
								  \num[round-mode=places,round-precision=2]{0,02} \\

								1e+05 & \multicolumn{1}{X}{-} & %2 &
								  \num{2} &
								%--
								  \num[round-mode=places,round-precision=2]{0,11} &
								  \num[round-mode=places,round-precision=2]{0,02} \\

								120000 & \multicolumn{1}{X}{-} & %1 &
								  \num{1} &
								%--
								  \num[round-mode=places,round-precision=2]{0,05} &
								  \num[round-mode=places,round-precision=2]{0,01} \\

					\midrule
					\multicolumn{2}{l}{Summe (gültig)} &
					  \textbf{\num{1859}} &
					\textbf{100} &
					  \textbf{\num[round-mode=places,round-precision=2]{17,71}} \\
					%--
					\multicolumn{5}{l}{\textbf{Fehlende Werte}}\\
							-998 &
							keine Angabe &
							  \num{900} &
							 - &
							  \num[round-mode=places,round-precision=2]{8,58} \\
							-995 &
							keine Teilnahme (Panel) &
							  \num{5739} &
							 - &
							  \num[round-mode=places,round-precision=2]{54,69} \\
							-989 &
							filterbedingt fehlend &
							  \num{31} &
							 - &
							  \num[round-mode=places,round-precision=2]{0,3} \\
							-988 &
							trifft nicht zu &
							  \num{1965} &
							 - &
							  \num[round-mode=places,round-precision=2]{18,72} \\
					\midrule
					\multicolumn{2}{l}{\textbf{Summe (gesamt)}} &
				      \textbf{\num{10494}} &
				    \textbf{-} &
				    \textbf{100} \\
					\bottomrule
					\end{longtable}
					\end{filecontents}
					\LTXtable{\textwidth}{\jobname-bocc332b_v1}
				\label{tableValues:bocc332b_v1}
				\vspace*{-\baselineskip}
                    \begin{noten}
                	    \note{} Deskritive Maßzahlen:
                	    Anzahl unterschiedlicher Beobachtungen: 320%
                	    ; 
                	      Minimum ($min$): 1; 
                	      Maximum ($max$): 120000; 
                	      arithmetisches Mittel ($\bar{x}$): \num[round-mode=places,round-precision=2]{4927,7961}; 
                	      Median ($\tilde{x}$): 2000; 
                	      Modus ($h$): 1000; 
                	      Standardabweichung ($s$): \num[round-mode=places,round-precision=2]{11874,9405}; 
                	      Schiefe ($v$): \num[round-mode=places,round-precision=2]{4,9927}; 
                	      Wölbung ($w$): \num[round-mode=places,round-precision=2]{30,1541}
                     \end{noten}


