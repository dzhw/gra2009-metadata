%EVERY VARIABLE HAS IT'S OWN PAGE

    \setcounter{footnote}{0}

    %omit vertical space
    \vspace*{-1.8cm}
	\section{bfec151h\_g2o (1. weitere akad. Qualifikation: Hochschule (NUTS2))}
	\label{section:bfec151h_g2o}



	%TABLE FOR VARIABLE DETAILS
    \vspace*{0.5cm}
    \noindent\textbf{Eigenschaften
	% '#' has to be escaped
	\footnote{Detailliertere Informationen zur Variable finden sich unter
		\url{https://metadata.fdz.dzhw.eu/\#!/de/variables/var-gra2009-ds1-bfec151h_g2o$}}}\\
	\begin{tabularx}{\hsize}{@{}lX}
	Datentyp: & string \\
	Skalenniveau: & nominal \\
	Zugangswege: &
	  onsite-suf
 \\
    \end{tabularx}



    %TABLE FOR QUESTION DETAILS
    %This has to be tested and has to be improved
    %rausfinden, ob einer Variable mehrere Fragen zugeordnet werden
    %dann evtl. nur die erste verwenden oder etwas anderes tun (Hinweis mehrere Fragen, auflisten mit Link)
				%TABLE FOR QUESTION DETAILS
				\vspace*{0.5cm}
                \noindent\textbf{Frage
	                \footnote{Detailliertere Informationen zur Frage finden sich unter
		              \url{https://metadata.fdz.dzhw.eu/\#!/de/questions/que-gra2009-ins2-5.2$}}}\\
				\begin{tabularx}{\hsize}{@{}lX}
					Fragenummer: &
					  Fragebogen des DZHW-Absolventenpanels 2009 - zweite Welle, Hauptbefragung (PAPI):
					  5.2
 \\
					%--
					Fragetext: & Bitte tragen Sie diese längerfristigen Studienangebote, die Sie nach Ihrem Studienabschluss aus dem Jahr 2008/2009 begonnen, weitergeführt oder abgeschlossen haben (auch abgebrochene oder unterbrochene), in das folgende Tableau ein! \\
				\end{tabularx}





				%TABLE FOR THE NOMINAL / ORDINAL VALUES
        		\vspace*{0.5cm}
                \noindent\textbf{Häufigkeiten}

                \vspace*{-\baselineskip}
					%STRING ELEMENTS NEEDS A HUGH FIRST COLOUMN AND A SMALL SECOND ONE
					\begin{filecontents}{\jobname-bfec151h_g2o}
					\begin{longtable}{Xlrrr}
					\toprule
					\textbf{Wert} & \textbf{Label} & \textbf{Häufigkeit} & \textbf{Prozent (gültig)} & \textbf{Prozent} \\
					\endhead
					\midrule
					\multicolumn{5}{l}{\textbf{Gültige Werte}}\\
						%DIFFERENT OBSERVATIONS <=20
								\multicolumn{1}{X}{DE11 Stuttgart} & - & 69 & 4,35 & 0,66 \\
								\multicolumn{1}{X}{DE12 Karlsruhe} & - & 61 & 3,85 & 0,58 \\
								\multicolumn{1}{X}{DE13 Freiburg} & - & 27 & 1,7 & 0,26 \\
								\multicolumn{1}{X}{DE14 Tübingen} & - & 33 & 2,08 & 0,31 \\
								\multicolumn{1}{X}{DE21 Oberbayern} & - & 100 & 6,31 & 0,95 \\
								\multicolumn{1}{X}{DE22 Niederbayern} & - & 14 & 0,88 & 0,13 \\
								\multicolumn{1}{X}{DE23 Oberpfalz} & - & 43 & 2,71 & 0,41 \\
								\multicolumn{1}{X}{DE24 Oberfranken} & - & 47 & 2,97 & 0,45 \\
								\multicolumn{1}{X}{DE25 Mittelfranken} & - & 35 & 2,21 & 0,33 \\
								\multicolumn{1}{X}{DE26 Unterfranken} & - & 4 & 0,25 & 0,04 \\
							... & ... & ... & ... & ... \\
								\multicolumn{1}{X}{DEB1 Koblenz} & - & 10 & 0,63 & 0,1 \\
								\multicolumn{1}{X}{DEB2 Trier} & - & 9 & 0,57 & 0,09 \\
								\multicolumn{1}{X}{DEB3 Rheinhessen-Pfalz} & - & 25 & 1,58 & 0,24 \\
								\multicolumn{1}{X}{DEC0 Saarland} & - & 6 & 0,38 & 0,06 \\
								\multicolumn{1}{X}{DED2 Dresden} & - & 37 & 2,33 & 0,35 \\
								\multicolumn{1}{X}{DED4 Chemnitz} & - & 42 & 2,65 & 0,4 \\
								\multicolumn{1}{X}{DED5 Leipzig} & - & 36 & 2,27 & 0,34 \\
								\multicolumn{1}{X}{DEE0 Sachsen-Anhalt} & - & 38 & 2,4 & 0,36 \\
								\multicolumn{1}{X}{DEF0 Schleswig-Holstein} & - & 45 & 2,84 & 0,43 \\
								\multicolumn{1}{X}{DEG0 Thüringen} & - & 108 & 6,81 & 1,03 \\
					\midrule
						\multicolumn{2}{l}{Summe (gültig)} & 1585 &
						\textbf{100} &
					    15,1 \\
					\multicolumn{5}{l}{\textbf{Fehlende Werte}}\\
							-966 & nicht bestimmbar & 126 & - & 1,2 \\

							-968 & unplausibler Wert & 5 & - & 0,05 \\

							-989 & filterbedingt fehlend & 2662 & - & 25,37 \\

							-995 & keine Teilnahme (Panel) & 5739 & - & 54,69 \\

							-998 & keine Angabe & 377 & - & 3,59 \\

					\midrule
					\multicolumn{2}{l}{\textbf{Summe (gesamt)}} & \textbf{10494} & \textbf{-} & \textbf{100} \\
					\bottomrule
					\caption{Werte der Variable bfec151h\_g2o}
					\end{longtable}
					\end{filecontents}
					\LTXtable{\textwidth}{\jobname-bfec151h_g2o}


