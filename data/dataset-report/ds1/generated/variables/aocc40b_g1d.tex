%EVERY VARIABLE HAS IT'S OWN PAGE

    \setcounter{footnote}{0}

    %omit vertical space
    \vspace*{-1.8cm}
	\section{aocc40b\_g1d (Ausbildung vor Studienbeginn: Beruf (KldB 1992 3-Steller))}
	\label{section:aocc40b_g1d}



	%TABLE FOR VARIABLE DETAILS
    \vspace*{0.5cm}
    \noindent\textbf{Eigenschaften
	% '#' has to be escaped
	\footnote{Detailliertere Informationen zur Variable finden sich unter
		\url{https://metadata.fdz.dzhw.eu/\#!/de/variables/var-gra2009-ds1-aocc40b_g1d$}}}\\
	\begin{tabularx}{\hsize}{@{}lX}
	Datentyp: & numerisch \\
	Skalenniveau: & nominal \\
	Zugangswege: &
	  download-suf, 
	  remote-desktop-suf, 
	  onsite-suf
 \\
    \end{tabularx}



    %TABLE FOR QUESTION DETAILS
    %This has to be tested and has to be improved
    %rausfinden, ob einer Variable mehrere Fragen zugeordnet werden
    %dann evtl. nur die erste verwenden oder etwas anderes tun (Hinweis mehrere Fragen, auflisten mit Link)
				%TABLE FOR QUESTION DETAILS
				\vspace*{0.5cm}
                \noindent\textbf{Frage
	                \footnote{Detailliertere Informationen zur Frage finden sich unter
		              \url{https://metadata.fdz.dzhw.eu/\#!/de/questions/que-gra2009-ins1-6.6$}}}\\
				\begin{tabularx}{\hsize}{@{}lX}
					Fragenummer: &
					  Fragebogen des DZHW-Absolventenpanels 2009 - erste Welle:
					  6.6
 \\
					%--
					Fragetext: & Haben Sie vor dem Erststudium eine berufliche Ausbildung abgeschlossen?\par  Wenn ja,\par  ... welchen Ausbildungsberuf haben Sie erlernt?\par  (bitte genaue Berufsbezeichung angeben) \\
				\end{tabularx}





				%TABLE FOR THE NOMINAL / ORDINAL VALUES
        		\vspace*{0.5cm}
                \noindent\textbf{Häufigkeiten}

                \vspace*{-\baselineskip}
					%NUMERIC ELEMENTS NEED A HUGH SECOND COLOUMN AND A SMALL FIRST ONE
					\begin{filecontents}{\jobname-aocc40b_g1d}
					\begin{longtable}{lXrrr}
					\toprule
					\textbf{Wert} & \textbf{Label} & \textbf{Häufigkeit} & \textbf{Prozent(gültig)} & \textbf{Prozent} \\
					\endhead
					\midrule
					\multicolumn{5}{l}{\textbf{Gültige Werte}}\\
						%DIFFERENT OBSERVATIONS <=20
								11 & \multicolumn{1}{X}{Landwirte/Landwirtinnen, Pflanzenschützer/Pflanzenschützerinnen} & %8 &
								  \num{8} &
								%--
								  \num[round-mode=places,round-precision=2]{0,32} &
								  \num[round-mode=places,round-precision=2]{0,08} \\
								12 & \multicolumn{1}{X}{Winzer/Winzerinnen} & %2 &
								  \num{2} &
								%--
								  \num[round-mode=places,round-precision=2]{0,08} &
								  \num[round-mode=places,round-precision=2]{0,02} \\
								23 & \multicolumn{1}{X}{Tier-, Pferde-, Fischwirte und -wirtinnen} & %5 &
								  \num{5} &
								%--
								  \num[round-mode=places,round-precision=2]{0,2} &
								  \num[round-mode=places,round-precision=2]{0,05} \\
								24 & \multicolumn{1}{X}{Tierpfleger/Tierpflegerinnen und verwandte Berufe, a.n.g.} & %2 &
								  \num{2} &
								%--
								  \num[round-mode=places,round-precision=2]{0,08} &
								  \num[round-mode=places,round-precision=2]{0,02} \\
								32 & \multicolumn{1}{X}{Land-, Tierwirtschaftsberater und -beraterinnen, Agraringenieur/Agraringenieurinnen, Agrartechniker/Agrartechnikerinnen} & %3 &
								  \num{3} &
								%--
								  \num[round-mode=places,round-precision=2]{0,12} &
								  \num[round-mode=places,round-precision=2]{0,03} \\
								51 & \multicolumn{1}{X}{Gärtner/Gärtnerinnen, Gartenarbeiter/Gartenarbeiterinnen} & %30 &
								  \num{30} &
								%--
								  \num[round-mode=places,round-precision=2]{1,2} &
								  \num[round-mode=places,round-precision=2]{0,29} \\
								52 & \multicolumn{1}{X}{Ingenieure/Ingenieurinnen, Techniker/Technikerinnen in Gartenbau und Landespflege} & %1 &
								  \num{1} &
								%--
								  \num[round-mode=places,round-precision=2]{0,04} &
								  \num[round-mode=places,round-precision=2]{0,01} \\
								53 & \multicolumn{1}{X}{Floristen/Floristinnen} & %6 &
								  \num{6} &
								%--
								  \num[round-mode=places,round-precision=2]{0,24} &
								  \num[round-mode=places,round-precision=2]{0,06} \\
								61 & \multicolumn{1}{X}{Forstverwalter/Forstverwalterinnen, Förster/Försterinnen, Jäger/Jägerinnen} & %2 &
								  \num{2} &
								%--
								  \num[round-mode=places,round-precision=2]{0,08} &
								  \num[round-mode=places,round-precision=2]{0,02} \\
								62 & \multicolumn{1}{X}{Forstwirte/Forstwirtinnen (Waldarbeiter/Waldarbeiterinnen)} & %3 &
								  \num{3} &
								%--
								  \num[round-mode=places,round-precision=2]{0,12} &
								  \num[round-mode=places,round-precision=2]{0,03} \\
							... & ... & ... & ... & ... \\
								885 & \multicolumn{1}{X}{Erziehungswissenschaftler/Erziehungswissenschaftlerinnen, a.n.g.} & %2 &
								  \num{2} &
								%--
								  \num[round-mode=places,round-precision=2]{0,08} &
								  \num[round-mode=places,round-precision=2]{0,02} \\

								886 & \multicolumn{1}{X}{Psychologen/Psychologinnen} & %1 &
								  \num{1} &
								%--
								  \num[round-mode=places,round-precision=2]{0,04} &
								  \num[round-mode=places,round-precision=2]{0,01} \\

								901 & \multicolumn{1}{X}{Friseure/Friseurinnen} & %6 &
								  \num{6} &
								%--
								  \num[round-mode=places,round-precision=2]{0,24} &
								  \num[round-mode=places,round-precision=2]{0,06} \\

								902 & \multicolumn{1}{X}{Kosmetiker/Kosmetikerinnen} & %3 &
								  \num{3} &
								%--
								  \num[round-mode=places,round-precision=2]{0,12} &
								  \num[round-mode=places,round-precision=2]{0,03} \\

								912 & \multicolumn{1}{X}{Restaurantfachleute, Stewards/Stewardessen} & %1 &
								  \num{1} &
								%--
								  \num[round-mode=places,round-precision=2]{0,04} &
								  \num[round-mode=places,round-precision=2]{0,01} \\

								914 & \multicolumn{1}{X}{Hotel-, Gaststättenkaufleute, a.n.g.} & %21 &
								  \num{21} &
								%--
								  \num[round-mode=places,round-precision=2]{0,84} &
								  \num[round-mode=places,round-precision=2]{0,2} \\

								915 & \multicolumn{1}{X}{Sonstige Berufe in der Gästebetreuung} & %1 &
								  \num{1} &
								%--
								  \num[round-mode=places,round-precision=2]{0,04} &
								  \num[round-mode=places,round-precision=2]{0,01} \\

								921 & \multicolumn{1}{X}{Haus- und Ernährungswirtschafter und -wirtschafterinnen} & %6 &
								  \num{6} &
								%--
								  \num[round-mode=places,round-precision=2]{0,24} &
								  \num[round-mode=places,round-precision=2]{0,06} \\

								934 & \multicolumn{1}{X}{Gebäudereiniger/Gebäudereinigerinnen, Raumpfleger/Raumpflegerinnen} & %1 &
								  \num{1} &
								%--
								  \num[round-mode=places,round-precision=2]{0,04} &
								  \num[round-mode=places,round-precision=2]{0,01} \\

								935 & \multicolumn{1}{X}{Städtereiniger/Städtereinigerinnen, Entsorger/Entsorgerinnen} & %3 &
								  \num{3} &
								%--
								  \num[round-mode=places,round-precision=2]{0,12} &
								  \num[round-mode=places,round-precision=2]{0,03} \\

					\midrule
					\multicolumn{2}{l}{Summe (gültig)} &
					  \textbf{\num{2497}} &
					\textbf{100} &
					  \textbf{\num[round-mode=places,round-precision=2]{23,79}} \\
					%--
					\multicolumn{5}{l}{\textbf{Fehlende Werte}}\\
							-998 &
							keine Angabe &
							  \num{104} &
							 - &
							  \num[round-mode=places,round-precision=2]{0,99} \\
							-988 &
							trifft nicht zu &
							  \num{7877} &
							 - &
							  \num[round-mode=places,round-precision=2]{75,06} \\
							-966 &
							nicht bestimmbar &
							  \num{16} &
							 - &
							  \num[round-mode=places,round-precision=2]{0,15} \\
					\midrule
					\multicolumn{2}{l}{\textbf{Summe (gesamt)}} &
				      \textbf{\num{10494}} &
				    \textbf{-} &
				    \textbf{100} \\
					\bottomrule
					\end{longtable}
					\end{filecontents}
					\LTXtable{\textwidth}{\jobname-aocc40b_g1d}
				\label{tableValues:aocc40b_g1d}
				\vspace*{-\baselineskip}
                    \begin{noten}
                	    \note{} Deskritive Maßzahlen:
                	    Anzahl unterschiedlicher Beobachtungen: 186%
                	    ; 
                	      Modus ($h$): 691
                     \end{noten}


