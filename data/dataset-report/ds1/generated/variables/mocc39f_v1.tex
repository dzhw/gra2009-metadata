%EVERY VARIABLE HAS IT'S OWN PAGE

    \setcounter{footnote}{0}

    %omit vertical space
    \vspace*{-1.8cm}
	\section{mocc39f\_v1 (Arbeits-/Lebensziele: politisch engagieren)}
	\label{section:mocc39f_v1}



	%TABLE FOR VARIABLE DETAILS
    \vspace*{0.5cm}
    \noindent\textbf{Eigenschaften
	% '#' has to be escaped
	\footnote{Detailliertere Informationen zur Variable finden sich unter
		\url{https://metadata.fdz.dzhw.eu/\#!/de/variables/var-gra2009-ds1-mocc39f_v1$}}}\\
	\begin{tabularx}{\hsize}{@{}lX}
	Datentyp: & numerisch \\
	Skalenniveau: & ordinal \\
	Zugangswege: &
	  download-cuf, 
	  download-suf, 
	  remote-desktop-suf, 
	  onsite-suf
 \\
    \end{tabularx}



    %TABLE FOR QUESTION DETAILS
    %This has to be tested and has to be improved
    %rausfinden, ob einer Variable mehrere Fragen zugeordnet werden
    %dann evtl. nur die erste verwenden oder etwas anderes tun (Hinweis mehrere Fragen, auflisten mit Link)
				%TABLE FOR QUESTION DETAILS
				\vspace*{0.5cm}
                \noindent\textbf{Frage
	                \footnote{Detailliertere Informationen zur Frage finden sich unter
		              \url{https://metadata.fdz.dzhw.eu/\#!/de/questions/que-gra2009-ins5-01$}}}\\
				\begin{tabularx}{\hsize}{@{}lX}
					Fragenummer: &
					  Fragebogen des DZHW-Absolventenpanels 2009 - zweite Welle, Vertiefungsbefragung Mobilität:
					  01
 \\
					%--
					Fragetext: & Die ersten beiden Fragen beziehen sich ganz allgemein auf Ihre Person.,Zunächst würden wir gerne von Ihnen wissen, wie wichtig Ihnen folgende Arbeits- bzw. Lebensziele sind.,sehr wichtig,überhaupt nicht wichtig,Mich politisch engagieren \\
				\end{tabularx}





				%TABLE FOR THE NOMINAL / ORDINAL VALUES
        		\vspace*{0.5cm}
                \noindent\textbf{Häufigkeiten}

                \vspace*{-\baselineskip}
					%NUMERIC ELEMENTS NEED A HUGH SECOND COLOUMN AND A SMALL FIRST ONE
					\begin{filecontents}{\jobname-mocc39f_v1}
					\begin{longtable}{lXrrr}
					\toprule
					\textbf{Wert} & \textbf{Label} & \textbf{Häufigkeit} & \textbf{Prozent(gültig)} & \textbf{Prozent} \\
					\endhead
					\midrule
					\multicolumn{5}{l}{\textbf{Gültige Werte}}\\
						%DIFFERENT OBSERVATIONS <=20

					1 &
				% TODO try size/length gt 0; take over for other passages
					\multicolumn{1}{X}{ sehr wichtig   } &


					%72 &
					  \num{72} &
					%--
					  \num[round-mode=places,round-precision=2]{2,93} &
					    \num[round-mode=places,round-precision=2]{0,69} \\
							%????

					2 &
				% TODO try size/length gt 0; take over for other passages
					\multicolumn{1}{X}{ 2   } &


					%312 &
					  \num{312} &
					%--
					  \num[round-mode=places,round-precision=2]{12,71} &
					    \num[round-mode=places,round-precision=2]{2,97} \\
							%????

					3 &
				% TODO try size/length gt 0; take over for other passages
					\multicolumn{1}{X}{ 3   } &


					%660 &
					  \num{660} &
					%--
					  \num[round-mode=places,round-precision=2]{26,89} &
					    \num[round-mode=places,round-precision=2]{6,29} \\
							%????

					4 &
				% TODO try size/length gt 0; take over for other passages
					\multicolumn{1}{X}{ 4   } &


					%907 &
					  \num{907} &
					%--
					  \num[round-mode=places,round-precision=2]{36,96} &
					    \num[round-mode=places,round-precision=2]{8,64} \\
							%????

					5 &
				% TODO try size/length gt 0; take over for other passages
					\multicolumn{1}{X}{ überhaupt nicht wichtig   } &


					%503 &
					  \num{503} &
					%--
					  \num[round-mode=places,round-precision=2]{20,5} &
					    \num[round-mode=places,round-precision=2]{4,79} \\
							%????
						%DIFFERENT OBSERVATIONS >20
					\midrule
					\multicolumn{2}{l}{Summe (gültig)} &
					  \textbf{\num{2454}} &
					\textbf{100} &
					  \textbf{\num[round-mode=places,round-precision=2]{23,38}} \\
					%--
					\multicolumn{5}{l}{\textbf{Fehlende Werte}}\\
							-998 &
							keine Angabe &
							  \num{11} &
							 - &
							  \num[round-mode=places,round-precision=2]{0,1} \\
							-995 &
							keine Teilnahme (Panel) &
							  \num{8029} &
							 - &
							  \num[round-mode=places,round-precision=2]{76,51} \\
					\midrule
					\multicolumn{2}{l}{\textbf{Summe (gesamt)}} &
				      \textbf{\num{10494}} &
				    \textbf{-} &
				    \textbf{100} \\
					\bottomrule
					\end{longtable}
					\end{filecontents}
					\LTXtable{\textwidth}{\jobname-mocc39f_v1}
				\label{tableValues:mocc39f_v1}
				\vspace*{-\baselineskip}
                    \begin{noten}
                	    \note{} Deskritive Maßzahlen:
                	    Anzahl unterschiedlicher Beobachtungen: 5%
                	    ; 
                	      Minimum ($min$): 1; 
                	      Maximum ($max$): 5; 
                	      Median ($\tilde{x}$): 4; 
                	      Modus ($h$): 4
                     \end{noten}


