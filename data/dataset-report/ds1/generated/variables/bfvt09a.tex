%EVERY VARIABLE HAS IT'S OWN PAGE

    \setcounter{footnote}{0}

    %omit vertical space
    \vspace*{-1.8cm}
	\section{bfvt09a (Bedarf Weiterbildung: Inhalt 1)}
	\label{section:bfvt09a}



	%TABLE FOR VARIABLE DETAILS
    \vspace*{0.5cm}
    \noindent\textbf{Eigenschaften
	% '#' has to be escaped
	\footnote{Detailliertere Informationen zur Variable finden sich unter
		\url{https://metadata.fdz.dzhw.eu/\#!/de/variables/var-gra2009-ds1-bfvt09a$}}}\\
	\begin{tabularx}{\hsize}{@{}lX}
	Datentyp: & numerisch \\
	Skalenniveau: & nominal \\
	Zugangswege: &
	  download-cuf, 
	  download-suf, 
	  remote-desktop-suf, 
	  onsite-suf
 \\
    \end{tabularx}



    %TABLE FOR QUESTION DETAILS
    %This has to be tested and has to be improved
    %rausfinden, ob einer Variable mehrere Fragen zugeordnet werden
    %dann evtl. nur die erste verwenden oder etwas anderes tun (Hinweis mehrere Fragen, auflisten mit Link)
				%TABLE FOR QUESTION DETAILS
				\vspace*{0.5cm}
                \noindent\textbf{Frage
	                \footnote{Detailliertere Informationen zur Frage finden sich unter
		              \url{https://metadata.fdz.dzhw.eu/\#!/de/questions/que-gra2009-ins2-7.1$}}}\\
				\begin{tabularx}{\hsize}{@{}lX}
					Fragenummer: &
					  Fragebogen des DZHW-Absolventenpanels 2009 - zweite Welle, Hauptbefragung (PAPI):
					  7.1
 \\
					%--
					Fragetext: & Sehen Sie für sich persönlich generell (weiteren) Bedarf zur Teilnahme an Weiterbildung und Qualifizierung?; Wenn ja: Tragen Sie hier bitte die für Sie wichtigsten Themen bzw. Fachgebiete ein.\par  Thema \\
				\end{tabularx}
				%TABLE FOR QUESTION DETAILS
				\vspace*{0.5cm}
                \noindent\textbf{Frage
	                \footnote{Detailliertere Informationen zur Frage finden sich unter
		              \url{https://metadata.fdz.dzhw.eu/\#!/de/questions/que-gra2009-ins3-80$}}}\\
				\begin{tabularx}{\hsize}{@{}lX}
					Fragenummer: &
					  Fragebogen des DZHW-Absolventenpanels 2009 - zweite Welle, Hauptbefragung (CAWI):
					  80
 \\
					%--
					Fragetext: & Wählen Sie bitte die für Sie wichtigsten Themen bzw. Fachgebiete aus \\
				\end{tabularx}





				%TABLE FOR THE NOMINAL / ORDINAL VALUES
        		\vspace*{0.5cm}
                \noindent\textbf{Häufigkeiten}

                \vspace*{-\baselineskip}
					%NUMERIC ELEMENTS NEED A HUGH SECOND COLOUMN AND A SMALL FIRST ONE
					\begin{filecontents}{\jobname-bfvt09a}
					\begin{longtable}{lXrrr}
					\toprule
					\textbf{Wert} & \textbf{Label} & \textbf{Häufigkeit} & \textbf{Prozent(gültig)} & \textbf{Prozent} \\
					\endhead
					\midrule
					\multicolumn{5}{l}{\textbf{Gültige Werte}}\\
						%DIFFERENT OBSERVATIONS <=20
								1 & \multicolumn{1}{X}{ingenieurwissenschaftliche Themen} & %446 &
								  \num{446} &
								%--
								  \num[round-mode=places,round-precision=2]{11,75} &
								  \num[round-mode=places,round-precision=2]{4,25} \\
								2 & \multicolumn{1}{X}{naturwissenschaftliche Themen} & %301 &
								  \num{301} &
								%--
								  \num[round-mode=places,round-precision=2]{7,93} &
								  \num[round-mode=places,round-precision=2]{2,87} \\
								3 & \multicolumn{1}{X}{mathematische Gebiete/Statistik} & %120 &
								  \num{120} &
								%--
								  \num[round-mode=places,round-precision=2]{3,16} &
								  \num[round-mode=places,round-precision=2]{1,14} \\
								4 & \multicolumn{1}{X}{sozialwissenschaftliche Themen} & %223 &
								  \num{223} &
								%--
								  \num[round-mode=places,round-precision=2]{5,87} &
								  \num[round-mode=places,round-precision=2]{2,13} \\
								5 & \multicolumn{1}{X}{geisteswissenschtliche Themen} & %113 &
								  \num{113} &
								%--
								  \num[round-mode=places,round-precision=2]{2,98} &
								  \num[round-mode=places,round-precision=2]{1,08} \\
								6 & \multicolumn{1}{X}{pädagogische/psychologische Themen} & %627 &
								  \num{627} &
								%--
								  \num[round-mode=places,round-precision=2]{16,52} &
								  \num[round-mode=places,round-precision=2]{5,97} \\
								7 & \multicolumn{1}{X}{medizinische Spezialgebiete} & %266 &
								  \num{266} &
								%--
								  \num[round-mode=places,round-precision=2]{7,01} &
								  \num[round-mode=places,round-precision=2]{2,53} \\
								8 & \multicolumn{1}{X}{informationstechnisches Spezialwissen} & %172 &
								  \num{172} &
								%--
								  \num[round-mode=places,round-precision=2]{4,53} &
								  \num[round-mode=places,round-precision=2]{1,64} \\
								9 & \multicolumn{1}{X}{Managementwissen} & %355 &
								  \num{355} &
								%--
								  \num[round-mode=places,round-precision=2]{9,35} &
								  \num[round-mode=places,round-precision=2]{3,38} \\
								10 & \multicolumn{1}{X}{Wirtschaftskenntnisse} & %188 &
								  \num{188} &
								%--
								  \num[round-mode=places,round-precision=2]{4,95} &
								  \num[round-mode=places,round-precision=2]{1,79} \\
							... & ... & ... & ... & ... \\
								15 & \multicolumn{1}{X}{EDV-Anwendungen} & %144 &
								  \num{144} &
								%--
								  \num[round-mode=places,round-precision=2]{3,79} &
								  \num[round-mode=places,round-precision=2]{1,37} \\

								16 & \multicolumn{1}{X}{Fremdsprachen} & %97 &
								  \num{97} &
								%--
								  \num[round-mode=places,round-precision=2]{2,56} &
								  \num[round-mode=places,round-precision=2]{0,92} \\

								17 & \multicolumn{1}{X}{Mitarbeiterführung/Personalentwicklung} & %171 &
								  \num{171} &
								%--
								  \num[round-mode=places,round-precision=2]{4,5} &
								  \num[round-mode=places,round-precision=2]{1,63} \\

								18 & \multicolumn{1}{X}{Kommunikations-/Interaktionstraining} & %154 &
								  \num{154} &
								%--
								  \num[round-mode=places,round-precision=2]{4,06} &
								  \num[round-mode=places,round-precision=2]{1,47} \\

								19 & \multicolumn{1}{X}{internationale Beziehungen, Kulturkenntnisse, Landeskunde} & %20 &
								  \num{20} &
								%--
								  \num[round-mode=places,round-precision=2]{0,53} &
								  \num[round-mode=places,round-precision=2]{0,19} \\

								20 & \multicolumn{1}{X}{ökologische Themen} & %26 &
								  \num{26} &
								%--
								  \num[round-mode=places,round-precision=2]{0,68} &
								  \num[round-mode=places,round-precision=2]{0,25} \\

								21 & \multicolumn{1}{X}{berufsethische Themen} & %10 &
								  \num{10} &
								%--
								  \num[round-mode=places,round-precision=2]{0,26} &
								  \num[round-mode=places,round-precision=2]{0,1} \\

								22 & \multicolumn{1}{X}{Existenzgründung} & %18 &
								  \num{18} &
								%--
								  \num[round-mode=places,round-precision=2]{0,47} &
								  \num[round-mode=places,round-precision=2]{0,17} \\

								23 & \multicolumn{1}{X}{betriebliches Gesundheitswesen, Arbeitssicherheit} & %22 &
								  \num{22} &
								%--
								  \num[round-mode=places,round-precision=2]{0,58} &
								  \num[round-mode=places,round-precision=2]{0,21} \\

								24 & \multicolumn{1}{X}{Sonstige} & %94 &
								  \num{94} &
								%--
								  \num[round-mode=places,round-precision=2]{2,48} &
								  \num[round-mode=places,round-precision=2]{0,9} \\

					\midrule
					\multicolumn{2}{l}{Summe (gültig)} &
					  \textbf{\num{3796}} &
					\textbf{100} &
					  \textbf{\num[round-mode=places,round-precision=2]{36,17}} \\
					%--
					\multicolumn{5}{l}{\textbf{Fehlende Werte}}\\
							-998 &
							keine Angabe &
							  \num{413} &
							 - &
							  \num[round-mode=places,round-precision=2]{3,94} \\
							-995 &
							keine Teilnahme (Panel) &
							  \num{5739} &
							 - &
							  \num[round-mode=places,round-precision=2]{54,69} \\
							-988 &
							trifft nicht zu &
							  \num{546} &
							 - &
							  \num[round-mode=places,round-precision=2]{5,2} \\
					\midrule
					\multicolumn{2}{l}{\textbf{Summe (gesamt)}} &
				      \textbf{\num{10494}} &
				    \textbf{-} &
				    \textbf{100} \\
					\bottomrule
					\end{longtable}
					\end{filecontents}
					\LTXtable{\textwidth}{\jobname-bfvt09a}
				\label{tableValues:bfvt09a}
				\vspace*{-\baselineskip}
                    \begin{noten}
                	    \note{} Deskritive Maßzahlen:
                	    Anzahl unterschiedlicher Beobachtungen: 24%
                	    ; 
                	      Modus ($h$): 6
                     \end{noten}


