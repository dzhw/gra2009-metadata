%EVERY VARIABLE HAS IT'S OWN PAGE

    \setcounter{footnote}{0}

    %omit vertical space
    \vspace*{-1.8cm}
	\section{mres032e\_g1r (2. Wohnung: Ort (Bundesland/Land))}
	\label{section:mres032e_g1r}



	%TABLE FOR VARIABLE DETAILS
    \vspace*{0.5cm}
    \noindent\textbf{Eigenschaften
	% '#' has to be escaped
	\footnote{Detailliertere Informationen zur Variable finden sich unter
		\url{https://metadata.fdz.dzhw.eu/\#!/de/variables/var-gra2009-ds1-mres032e_g1r$}}}\\
	\begin{tabularx}{\hsize}{@{}lX}
	Datentyp: & numerisch \\
	Skalenniveau: & nominal \\
	Zugangswege: &
	  remote-desktop-suf, 
	  onsite-suf
 \\
    \end{tabularx}



    %TABLE FOR QUESTION DETAILS
    %This has to be tested and has to be improved
    %rausfinden, ob einer Variable mehrere Fragen zugeordnet werden
    %dann evtl. nur die erste verwenden oder etwas anderes tun (Hinweis mehrere Fragen, auflisten mit Link)
				%TABLE FOR QUESTION DETAILS
				\vspace*{0.5cm}
                \noindent\textbf{Frage
	                \footnote{Detailliertere Informationen zur Frage finden sich unter
		              \url{https://metadata.fdz.dzhw.eu/\#!/de/questions/que-gra2009-ins5-11.1$}}}\\
				\begin{tabularx}{\hsize}{@{}lX}
					Fragenummer: &
					  Fragebogen des DZHW-Absolventenpanels 2009 - zweite Welle, Vertiefungsbefragung Mobilität:
					  11.1
 \\
					%--
					Fragetext: & Bitte nennen Sie uns nun die nächste Wohnung, in die Sie nach Ihrem Studienabschluss 2008/2009 eingezogen sind.,Zeitraum (Monat/Jahr),Wohnort,Wohnten Sie die meiste Zeit(Mehrfachnennung möglich),Handelte es sich um,Bundesland bzw. Land (bei Ausland) \\
				\end{tabularx}





				%TABLE FOR THE NOMINAL / ORDINAL VALUES
        		\vspace*{0.5cm}
                \noindent\textbf{Häufigkeiten}

                \vspace*{-\baselineskip}
					%NUMERIC ELEMENTS NEED A HUGH SECOND COLOUMN AND A SMALL FIRST ONE
					\begin{filecontents}{\jobname-mres032e_g1r}
					\begin{longtable}{lXrrr}
					\toprule
					\textbf{Wert} & \textbf{Label} & \textbf{Häufigkeit} & \textbf{Prozent(gültig)} & \textbf{Prozent} \\
					\endhead
					\midrule
					\multicolumn{5}{l}{\textbf{Gültige Werte}}\\
						%DIFFERENT OBSERVATIONS <=20
								1 & \multicolumn{1}{X}{Schleswig-Holstein} & %12 &
								  \num{12} &
								%--
								  \num[round-mode=places,round-precision=2]{1,21} &
								  \num[round-mode=places,round-precision=2]{0,11} \\
								2 & \multicolumn{1}{X}{Hamburg} & %43 &
								  \num{43} &
								%--
								  \num[round-mode=places,round-precision=2]{4,34} &
								  \num[round-mode=places,round-precision=2]{0,41} \\
								3 & \multicolumn{1}{X}{Niedersachsen} & %81 &
								  \num{81} &
								%--
								  \num[round-mode=places,round-precision=2]{8,17} &
								  \num[round-mode=places,round-precision=2]{0,77} \\
								4 & \multicolumn{1}{X}{Bremen} & %8 &
								  \num{8} &
								%--
								  \num[round-mode=places,round-precision=2]{0,81} &
								  \num[round-mode=places,round-precision=2]{0,08} \\
								5 & \multicolumn{1}{X}{Nordrhein-Westfalen} & %125 &
								  \num{125} &
								%--
								  \num[round-mode=places,round-precision=2]{12,61} &
								  \num[round-mode=places,round-precision=2]{1,19} \\
								6 & \multicolumn{1}{X}{Hessen} & %59 &
								  \num{59} &
								%--
								  \num[round-mode=places,round-precision=2]{5,95} &
								  \num[round-mode=places,round-precision=2]{0,56} \\
								7 & \multicolumn{1}{X}{Rheinland-Pfalz} & %24 &
								  \num{24} &
								%--
								  \num[round-mode=places,round-precision=2]{2,42} &
								  \num[round-mode=places,round-precision=2]{0,23} \\
								8 & \multicolumn{1}{X}{Baden-Württemberg} & %131 &
								  \num{131} &
								%--
								  \num[round-mode=places,round-precision=2]{13,22} &
								  \num[round-mode=places,round-precision=2]{1,25} \\
								9 & \multicolumn{1}{X}{Bayern} & %148 &
								  \num{148} &
								%--
								  \num[round-mode=places,round-precision=2]{14,93} &
								  \num[round-mode=places,round-precision=2]{1,41} \\
								10 & \multicolumn{1}{X}{Saarland} & %3 &
								  \num{3} &
								%--
								  \num[round-mode=places,round-precision=2]{0,3} &
								  \num[round-mode=places,round-precision=2]{0,03} \\
							... & ... & ... & ... & ... \\
								422 & \multicolumn{1}{X}{Armenien} & %1 &
								  \num{1} &
								%--
								  \num[round-mode=places,round-precision=2]{0,1} &
								  \num[round-mode=places,round-precision=2]{0,01} \\

								432 & \multicolumn{1}{X}{Vietnam} & %1 &
								  \num{1} &
								%--
								  \num[round-mode=places,round-precision=2]{0,1} &
								  \num[round-mode=places,round-precision=2]{0,01} \\

								436 & \multicolumn{1}{X}{Indien, einschl. Sikkim und Gôa} & %2 &
								  \num{2} &
								%--
								  \num[round-mode=places,round-precision=2]{0,2} &
								  \num[round-mode=places,round-precision=2]{0,02} \\

								442 & \multicolumn{1}{X}{Japan} & %2 &
								  \num{2} &
								%--
								  \num[round-mode=places,round-precision=2]{0,2} &
								  \num[round-mode=places,round-precision=2]{0,02} \\

								462 & \multicolumn{1}{X}{Philippinen} & %1 &
								  \num{1} &
								%--
								  \num[round-mode=places,round-precision=2]{0,1} &
								  \num[round-mode=places,round-precision=2]{0,01} \\

								474 & \multicolumn{1}{X}{Singapur} & %1 &
								  \num{1} &
								%--
								  \num[round-mode=places,round-precision=2]{0,1} &
								  \num[round-mode=places,round-precision=2]{0,01} \\

								479 & \multicolumn{1}{X}{China} & %3 &
								  \num{3} &
								%--
								  \num[round-mode=places,round-precision=2]{0,3} &
								  \num[round-mode=places,round-precision=2]{0,03} \\

								482 & \multicolumn{1}{X}{Malaysia} & %1 &
								  \num{1} &
								%--
								  \num[round-mode=places,round-precision=2]{0,1} &
								  \num[round-mode=places,round-precision=2]{0,01} \\

								523 & \multicolumn{1}{X}{Australien} & %4 &
								  \num{4} &
								%--
								  \num[round-mode=places,round-precision=2]{0,4} &
								  \num[round-mode=places,round-precision=2]{0,04} \\

								536 & \multicolumn{1}{X}{Neuseeland} & %3 &
								  \num{3} &
								%--
								  \num[round-mode=places,round-precision=2]{0,3} &
								  \num[round-mode=places,round-precision=2]{0,03} \\

					\midrule
					\multicolumn{2}{l}{Summe (gültig)} &
					  \textbf{\num{991}} &
					\textbf{100} &
					  \textbf{\num[round-mode=places,round-precision=2]{9,44}} \\
					%--
					\multicolumn{5}{l}{\textbf{Fehlende Werte}}\\
							-998 &
							keine Angabe &
							  \num{175} &
							 - &
							  \num[round-mode=places,round-precision=2]{1,67} \\
							-995 &
							keine Teilnahme (Panel) &
							  \num{8029} &
							 - &
							  \num[round-mode=places,round-precision=2]{76,51} \\
							-989 &
							filterbedingt fehlend &
							  \num{1299} &
							 - &
							  \num[round-mode=places,round-precision=2]{12,38} \\
					\midrule
					\multicolumn{2}{l}{\textbf{Summe (gesamt)}} &
				      \textbf{\num{10494}} &
				    \textbf{-} &
				    \textbf{100} \\
					\bottomrule
					\end{longtable}
					\end{filecontents}
					\LTXtable{\textwidth}{\jobname-mres032e_g1r}
				\label{tableValues:mres032e_g1r}
				\vspace*{-\baselineskip}
                    \begin{noten}
                	    \note{} Deskritive Maßzahlen:
                	    Anzahl unterschiedlicher Beobachtungen: 58%
                	    ; 
                	      Modus ($h$): 9
                     \end{noten}


