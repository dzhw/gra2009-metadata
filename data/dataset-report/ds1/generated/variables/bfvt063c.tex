%EVERY VARIABLE HAS IT'S OWN PAGE

    \setcounter{footnote}{0}

    %omit vertical space
    \vspace*{-1.8cm}
	\section{bfvt063c (mehrtägige berufl. Weiterbildung: Inhalt 1)}
	\label{section:bfvt063c}



	%TABLE FOR VARIABLE DETAILS
    \vspace*{0.5cm}
    \noindent\textbf{Eigenschaften
	% '#' has to be escaped
	\footnote{Detailliertere Informationen zur Variable finden sich unter
		\url{https://metadata.fdz.dzhw.eu/\#!/de/variables/var-gra2009-ds1-bfvt063c$}}}\\
	\begin{tabularx}{\hsize}{@{}lX}
	Datentyp: & numerisch \\
	Skalenniveau: & nominal \\
	Zugangswege: &
	  download-cuf, 
	  download-suf, 
	  remote-desktop-suf, 
	  onsite-suf
 \\
    \end{tabularx}



    %TABLE FOR QUESTION DETAILS
    %This has to be tested and has to be improved
    %rausfinden, ob einer Variable mehrere Fragen zugeordnet werden
    %dann evtl. nur die erste verwenden oder etwas anderes tun (Hinweis mehrere Fragen, auflisten mit Link)
				%TABLE FOR QUESTION DETAILS
				\vspace*{0.5cm}
                \noindent\textbf{Frage
	                \footnote{Detailliertere Informationen zur Frage finden sich unter
		              \url{https://metadata.fdz.dzhw.eu/\#!/de/questions/que-gra2009-ins2-6.5$}}}\\
				\begin{tabularx}{\hsize}{@{}lX}
					Fragenummer: &
					  Fragebogen des DZHW-Absolventenpanels 2009 - zweite Welle, Hauptbefragung (PAPI):
					  6.5
 \\
					%--
					Fragetext: & Im Folgenden bitten wir Sie um Angaben zu beruflichen Fort- und Weiterbildungen der letzten 12 Monate. Bitte denken Sie dabei an alle Weiterbildungen, die Sie besucht haben und geben sie diese in der passenden Zeile an.\par  3. Fort- /oder Weiterbildung\par  Themen (Mehrfachnennung möglich)\par  Schlüssel s. Klappliste B) \\
				\end{tabularx}
				%TABLE FOR QUESTION DETAILS
				\vspace*{0.5cm}
                \noindent\textbf{Frage
	                \footnote{Detailliertere Informationen zur Frage finden sich unter
		              \url{https://metadata.fdz.dzhw.eu/\#!/de/questions/que-gra2009-ins3-67$}}}\\
				\begin{tabularx}{\hsize}{@{}lX}
					Fragenummer: &
					  Fragebogen des DZHW-Absolventenpanels 2009 - zweite Welle, Hauptbefragung (CAWI):
					  67
 \\
					%--
					Fragetext: & Bitte tragen Sie hier die für Sie wichtigsten Themen bzw. Fachgebiete dieser Veranstaltungen ein. \\
				\end{tabularx}





				%TABLE FOR THE NOMINAL / ORDINAL VALUES
        		\vspace*{0.5cm}
                \noindent\textbf{Häufigkeiten}

                \vspace*{-\baselineskip}
					%NUMERIC ELEMENTS NEED A HUGH SECOND COLOUMN AND A SMALL FIRST ONE
					\begin{filecontents}{\jobname-bfvt063c}
					\begin{longtable}{lXrrr}
					\toprule
					\textbf{Wert} & \textbf{Label} & \textbf{Häufigkeit} & \textbf{Prozent(gültig)} & \textbf{Prozent} \\
					\endhead
					\midrule
					\multicolumn{5}{l}{\textbf{Gültige Werte}}\\
						%DIFFERENT OBSERVATIONS <=20
								1 & \multicolumn{1}{X}{ingenieurwissenschaftliche Themen} & %146 &
								  \num{146} &
								%--
								  \num[round-mode=places,round-precision=2]{8,98} &
								  \num[round-mode=places,round-precision=2]{1,39} \\
								2 & \multicolumn{1}{X}{naturwissenschaftliche Themen} & %103 &
								  \num{103} &
								%--
								  \num[round-mode=places,round-precision=2]{6,33} &
								  \num[round-mode=places,round-precision=2]{0,98} \\
								3 & \multicolumn{1}{X}{mathematische Gebiete/Statistik} & %21 &
								  \num{21} &
								%--
								  \num[round-mode=places,round-precision=2]{1,29} &
								  \num[round-mode=places,round-precision=2]{0,2} \\
								4 & \multicolumn{1}{X}{sozialwissenschaftliche Themen} & %50 &
								  \num{50} &
								%--
								  \num[round-mode=places,round-precision=2]{3,08} &
								  \num[round-mode=places,round-precision=2]{0,48} \\
								5 & \multicolumn{1}{X}{geisteswissenschtliche Themen} & %53 &
								  \num{53} &
								%--
								  \num[round-mode=places,round-precision=2]{3,26} &
								  \num[round-mode=places,round-precision=2]{0,51} \\
								6 & \multicolumn{1}{X}{pädagogische/psychologische Themen} & %275 &
								  \num{275} &
								%--
								  \num[round-mode=places,round-precision=2]{16,91} &
								  \num[round-mode=places,round-precision=2]{2,62} \\
								7 & \multicolumn{1}{X}{medizinische Spezialgebiete} & %165 &
								  \num{165} &
								%--
								  \num[round-mode=places,round-precision=2]{10,15} &
								  \num[round-mode=places,round-precision=2]{1,57} \\
								8 & \multicolumn{1}{X}{informationstechnisches Spezialwissen} & %75 &
								  \num{75} &
								%--
								  \num[round-mode=places,round-precision=2]{4,61} &
								  \num[round-mode=places,round-precision=2]{0,71} \\
								9 & \multicolumn{1}{X}{Managementwissen} & %116 &
								  \num{116} &
								%--
								  \num[round-mode=places,round-precision=2]{7,13} &
								  \num[round-mode=places,round-precision=2]{1,11} \\
								10 & \multicolumn{1}{X}{Wirtschaftskenntnisse} & %54 &
								  \num{54} &
								%--
								  \num[round-mode=places,round-precision=2]{3,32} &
								  \num[round-mode=places,round-precision=2]{0,51} \\
							... & ... & ... & ... & ... \\
								15 & \multicolumn{1}{X}{EDV-Anwendungen} & %111 &
								  \num{111} &
								%--
								  \num[round-mode=places,round-precision=2]{6,83} &
								  \num[round-mode=places,round-precision=2]{1,06} \\

								16 & \multicolumn{1}{X}{Fremdsprachen} & %22 &
								  \num{22} &
								%--
								  \num[round-mode=places,round-precision=2]{1,35} &
								  \num[round-mode=places,round-precision=2]{0,21} \\

								17 & \multicolumn{1}{X}{Mitarbeiterführung/Personalentwicklung} & %69 &
								  \num{69} &
								%--
								  \num[round-mode=places,round-precision=2]{4,24} &
								  \num[round-mode=places,round-precision=2]{0,66} \\

								18 & \multicolumn{1}{X}{Kommunikations-/Interaktionstraining} & %153 &
								  \num{153} &
								%--
								  \num[round-mode=places,round-precision=2]{9,41} &
								  \num[round-mode=places,round-precision=2]{1,46} \\

								19 & \multicolumn{1}{X}{internationale Beziehungen, Kulturkenntnisse, Landeskunde} & %13 &
								  \num{13} &
								%--
								  \num[round-mode=places,round-precision=2]{0,8} &
								  \num[round-mode=places,round-precision=2]{0,12} \\

								20 & \multicolumn{1}{X}{ökologische Themen} & %10 &
								  \num{10} &
								%--
								  \num[round-mode=places,round-precision=2]{0,62} &
								  \num[round-mode=places,round-precision=2]{0,1} \\

								21 & \multicolumn{1}{X}{berufsethische Themen} & %2 &
								  \num{2} &
								%--
								  \num[round-mode=places,round-precision=2]{0,12} &
								  \num[round-mode=places,round-precision=2]{0,02} \\

								22 & \multicolumn{1}{X}{Existenzgründung} & %8 &
								  \num{8} &
								%--
								  \num[round-mode=places,round-precision=2]{0,49} &
								  \num[round-mode=places,round-precision=2]{0,08} \\

								23 & \multicolumn{1}{X}{betriebliches Gesundheitswesen, Arbeitssicherheit} & %35 &
								  \num{35} &
								%--
								  \num[round-mode=places,round-precision=2]{2,15} &
								  \num[round-mode=places,round-precision=2]{0,33} \\

								24 & \multicolumn{1}{X}{Sonstige} & %35 &
								  \num{35} &
								%--
								  \num[round-mode=places,round-precision=2]{2,15} &
								  \num[round-mode=places,round-precision=2]{0,33} \\

					\midrule
					\multicolumn{2}{l}{Summe (gültig)} &
					  \textbf{\num{1626}} &
					\textbf{100} &
					  \textbf{\num[round-mode=places,round-precision=2]{15,49}} \\
					%--
					\multicolumn{5}{l}{\textbf{Fehlende Werte}}\\
							-998 &
							keine Angabe &
							  \num{3129} &
							 - &
							  \num[round-mode=places,round-precision=2]{29,82} \\
							-995 &
							keine Teilnahme (Panel) &
							  \num{5739} &
							 - &
							  \num[round-mode=places,round-precision=2]{54,69} \\
					\midrule
					\multicolumn{2}{l}{\textbf{Summe (gesamt)}} &
				      \textbf{\num{10494}} &
				    \textbf{-} &
				    \textbf{100} \\
					\bottomrule
					\end{longtable}
					\end{filecontents}
					\LTXtable{\textwidth}{\jobname-bfvt063c}
				\label{tableValues:bfvt063c}
				\vspace*{-\baselineskip}
                    \begin{noten}
                	    \note{} Deskritive Maßzahlen:
                	    Anzahl unterschiedlicher Beobachtungen: 24%
                	    ; 
                	      Modus ($h$): 6
                     \end{noten}


