%EVERY VARIABLE HAS IT'S OWN PAGE

    \setcounter{footnote}{0}

    %omit vertical space
    \vspace*{-1.8cm}
	\section{bfvt09c (Bedarf Weiterbildung: Inhalt 3)}
	\label{section:bfvt09c}



	%TABLE FOR VARIABLE DETAILS
    \vspace*{0.5cm}
    \noindent\textbf{Eigenschaften
	% '#' has to be escaped
	\footnote{Detailliertere Informationen zur Variable finden sich unter
		\url{https://metadata.fdz.dzhw.eu/\#!/de/variables/var-gra2009-ds1-bfvt09c$}}}\\
	\begin{tabularx}{\hsize}{@{}lX}
	Datentyp: & numerisch \\
	Skalenniveau: & nominal \\
	Zugangswege: &
	  download-cuf, 
	  download-suf, 
	  remote-desktop-suf, 
	  onsite-suf
 \\
    \end{tabularx}



    %TABLE FOR QUESTION DETAILS
    %This has to be tested and has to be improved
    %rausfinden, ob einer Variable mehrere Fragen zugeordnet werden
    %dann evtl. nur die erste verwenden oder etwas anderes tun (Hinweis mehrere Fragen, auflisten mit Link)
				%TABLE FOR QUESTION DETAILS
				\vspace*{0.5cm}
                \noindent\textbf{Frage
	                \footnote{Detailliertere Informationen zur Frage finden sich unter
		              \url{https://metadata.fdz.dzhw.eu/\#!/de/questions/que-gra2009-ins2-7.1$}}}\\
				\begin{tabularx}{\hsize}{@{}lX}
					Fragenummer: &
					  Fragebogen des DZHW-Absolventenpanels 2009 - zweite Welle, Hauptbefragung (PAPI):
					  7.1
 \\
					%--
					Fragetext: & Sehen Sie für sich persönlich generell (weiteren) Bedarf zur Teilnahme an Weiterbildung und Qualifizierung?; Wenn ja: Tragen Sie hier bitte die für Sie wichtigsten Themen bzw. Fachgebiete ein.\par  Thema \\
				\end{tabularx}
				%TABLE FOR QUESTION DETAILS
				\vspace*{0.5cm}
                \noindent\textbf{Frage
	                \footnote{Detailliertere Informationen zur Frage finden sich unter
		              \url{https://metadata.fdz.dzhw.eu/\#!/de/questions/que-gra2009-ins3-80$}}}\\
				\begin{tabularx}{\hsize}{@{}lX}
					Fragenummer: &
					  Fragebogen des DZHW-Absolventenpanels 2009 - zweite Welle, Hauptbefragung (CAWI):
					  80
 \\
					%--
					Fragetext: & Wählen Sie bitte die für Sie wichtigsten Themen bzw. Fachgebiete aus \\
				\end{tabularx}





				%TABLE FOR THE NOMINAL / ORDINAL VALUES
        		\vspace*{0.5cm}
                \noindent\textbf{Häufigkeiten}

                \vspace*{-\baselineskip}
					%NUMERIC ELEMENTS NEED A HUGH SECOND COLOUMN AND A SMALL FIRST ONE
					\begin{filecontents}{\jobname-bfvt09c}
					\begin{longtable}{lXrrr}
					\toprule
					\textbf{Wert} & \textbf{Label} & \textbf{Häufigkeit} & \textbf{Prozent(gültig)} & \textbf{Prozent} \\
					\endhead
					\midrule
					\multicolumn{5}{l}{\textbf{Gültige Werte}}\\
						%DIFFERENT OBSERVATIONS <=20
								1 & \multicolumn{1}{X}{ingenieurwissenschaftliche Themen} & %26 &
								  \num{26} &
								%--
								  \num[round-mode=places,round-precision=2]{1,27} &
								  \num[round-mode=places,round-precision=2]{0,25} \\
								2 & \multicolumn{1}{X}{naturwissenschaftliche Themen} & %35 &
								  \num{35} &
								%--
								  \num[round-mode=places,round-precision=2]{1,71} &
								  \num[round-mode=places,round-precision=2]{0,33} \\
								3 & \multicolumn{1}{X}{mathematische Gebiete/Statistik} & %41 &
								  \num{41} &
								%--
								  \num[round-mode=places,round-precision=2]{2} &
								  \num[round-mode=places,round-precision=2]{0,39} \\
								4 & \multicolumn{1}{X}{sozialwissenschaftliche Themen} & %45 &
								  \num{45} &
								%--
								  \num[round-mode=places,round-precision=2]{2,2} &
								  \num[round-mode=places,round-precision=2]{0,43} \\
								5 & \multicolumn{1}{X}{geisteswissenschtliche Themen} & %49 &
								  \num{49} &
								%--
								  \num[round-mode=places,round-precision=2]{2,39} &
								  \num[round-mode=places,round-precision=2]{0,47} \\
								6 & \multicolumn{1}{X}{pädagogische/psychologische Themen} & %115 &
								  \num{115} &
								%--
								  \num[round-mode=places,round-precision=2]{5,61} &
								  \num[round-mode=places,round-precision=2]{1,1} \\
								7 & \multicolumn{1}{X}{medizinische Spezialgebiete} & %42 &
								  \num{42} &
								%--
								  \num[round-mode=places,round-precision=2]{2,05} &
								  \num[round-mode=places,round-precision=2]{0,4} \\
								8 & \multicolumn{1}{X}{informationstechnisches Spezialwissen} & %43 &
								  \num{43} &
								%--
								  \num[round-mode=places,round-precision=2]{2,1} &
								  \num[round-mode=places,round-precision=2]{0,41} \\
								9 & \multicolumn{1}{X}{Managementwissen} & %144 &
								  \num{144} &
								%--
								  \num[round-mode=places,round-precision=2]{7,03} &
								  \num[round-mode=places,round-precision=2]{1,37} \\
								10 & \multicolumn{1}{X}{Wirtschaftskenntnisse} & %125 &
								  \num{125} &
								%--
								  \num[round-mode=places,round-precision=2]{6,1} &
								  \num[round-mode=places,round-precision=2]{1,19} \\
							... & ... & ... & ... & ... \\
								15 & \multicolumn{1}{X}{EDV-Anwendungen} & %184 &
								  \num{184} &
								%--
								  \num[round-mode=places,round-precision=2]{8,98} &
								  \num[round-mode=places,round-precision=2]{1,75} \\

								16 & \multicolumn{1}{X}{Fremdsprachen} & %200 &
								  \num{200} &
								%--
								  \num[round-mode=places,round-precision=2]{9,76} &
								  \num[round-mode=places,round-precision=2]{1,91} \\

								17 & \multicolumn{1}{X}{Mitarbeiterführung/Personalentwicklung} & %236 &
								  \num{236} &
								%--
								  \num[round-mode=places,round-precision=2]{11,52} &
								  \num[round-mode=places,round-precision=2]{2,25} \\

								18 & \multicolumn{1}{X}{Kommunikations-/Interaktionstraining} & %290 &
								  \num{290} &
								%--
								  \num[round-mode=places,round-precision=2]{14,15} &
								  \num[round-mode=places,round-precision=2]{2,76} \\

								19 & \multicolumn{1}{X}{internationale Beziehungen, Kulturkenntnisse, Landeskunde} & %53 &
								  \num{53} &
								%--
								  \num[round-mode=places,round-precision=2]{2,59} &
								  \num[round-mode=places,round-precision=2]{0,51} \\

								20 & \multicolumn{1}{X}{ökologische Themen} & %30 &
								  \num{30} &
								%--
								  \num[round-mode=places,round-precision=2]{1,46} &
								  \num[round-mode=places,round-precision=2]{0,29} \\

								21 & \multicolumn{1}{X}{berufsethische Themen} & %33 &
								  \num{33} &
								%--
								  \num[round-mode=places,round-precision=2]{1,61} &
								  \num[round-mode=places,round-precision=2]{0,31} \\

								22 & \multicolumn{1}{X}{Existenzgründung} & %36 &
								  \num{36} &
								%--
								  \num[round-mode=places,round-precision=2]{1,76} &
								  \num[round-mode=places,round-precision=2]{0,34} \\

								23 & \multicolumn{1}{X}{betriebliches Gesundheitswesen, Arbeitssicherheit} & %27 &
								  \num{27} &
								%--
								  \num[round-mode=places,round-precision=2]{1,32} &
								  \num[round-mode=places,round-precision=2]{0,26} \\

								24 & \multicolumn{1}{X}{Sonstige} & %71 &
								  \num{71} &
								%--
								  \num[round-mode=places,round-precision=2]{3,47} &
								  \num[round-mode=places,round-precision=2]{0,68} \\

					\midrule
					\multicolumn{2}{l}{Summe (gültig)} &
					  \textbf{\num{2049}} &
					\textbf{100} &
					  \textbf{\num[round-mode=places,round-precision=2]{19,53}} \\
					%--
					\multicolumn{5}{l}{\textbf{Fehlende Werte}}\\
							-998 &
							keine Angabe &
							  \num{2160} &
							 - &
							  \num[round-mode=places,round-precision=2]{20,58} \\
							-995 &
							keine Teilnahme (Panel) &
							  \num{5739} &
							 - &
							  \num[round-mode=places,round-precision=2]{54,69} \\
							-988 &
							trifft nicht zu &
							  \num{546} &
							 - &
							  \num[round-mode=places,round-precision=2]{5,2} \\
					\midrule
					\multicolumn{2}{l}{\textbf{Summe (gesamt)}} &
				      \textbf{\num{10494}} &
				    \textbf{-} &
				    \textbf{100} \\
					\bottomrule
					\end{longtable}
					\end{filecontents}
					\LTXtable{\textwidth}{\jobname-bfvt09c}
				\label{tableValues:bfvt09c}
				\vspace*{-\baselineskip}
                    \begin{noten}
                	    \note{} Deskritive Maßzahlen:
                	    Anzahl unterschiedlicher Beobachtungen: 24%
                	    ; 
                	      Modus ($h$): 18
                     \end{noten}


