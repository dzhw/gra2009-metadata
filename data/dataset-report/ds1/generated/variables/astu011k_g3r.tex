%EVERY VARIABLE HAS IT'S OWN PAGE

    \setcounter{footnote}{0}

    %omit vertical space
    \vspace*{-1.8cm}
	\section{astu011k\_g3r (1. Studium: Hochschule (Bundes-/Ausland))}
	\label{section:astu011k_g3r}



	%TABLE FOR VARIABLE DETAILS
    \vspace*{0.5cm}
    \noindent\textbf{Eigenschaften
	% '#' has to be escaped
	\footnote{Detailliertere Informationen zur Variable finden sich unter
		\url{https://metadata.fdz.dzhw.eu/\#!/de/variables/var-gra2009-ds1-astu011k_g3r$}}}\\
	\begin{tabularx}{\hsize}{@{}lX}
	Datentyp: & numerisch \\
	Skalenniveau: & nominal \\
	Zugangswege: &
	  remote-desktop-suf, 
	  onsite-suf
 \\
    \end{tabularx}



    %TABLE FOR QUESTION DETAILS
    %This has to be tested and has to be improved
    %rausfinden, ob einer Variable mehrere Fragen zugeordnet werden
    %dann evtl. nur die erste verwenden oder etwas anderes tun (Hinweis mehrere Fragen, auflisten mit Link)
				%TABLE FOR QUESTION DETAILS
				\vspace*{0.5cm}
                \noindent\textbf{Frage
	                \footnote{Detailliertere Informationen zur Frage finden sich unter
		              \url{https://metadata.fdz.dzhw.eu/\#!/de/questions/que-gra2009-ins1-1.1$}}}\\
				\begin{tabularx}{\hsize}{@{}lX}
					Fragenummer: &
					  Fragebogen des DZHW-Absolventenpanels 2009 - erste Welle:
					  1.1
 \\
					%--
					Fragetext: & Bitte tragen Sie in das folgende Tableau Ihren Studienverlauf ein. \\
				\end{tabularx}





				%TABLE FOR THE NOMINAL / ORDINAL VALUES
        		\vspace*{0.5cm}
                \noindent\textbf{Häufigkeiten}

                \vspace*{-\baselineskip}
					%NUMERIC ELEMENTS NEED A HUGH SECOND COLOUMN AND A SMALL FIRST ONE
					\begin{filecontents}{\jobname-astu011k_g3r}
					\begin{longtable}{lXrrr}
					\toprule
					\textbf{Wert} & \textbf{Label} & \textbf{Häufigkeit} & \textbf{Prozent(gültig)} & \textbf{Prozent} \\
					\endhead
					\midrule
					\multicolumn{5}{l}{\textbf{Gültige Werte}}\\
						%DIFFERENT OBSERVATIONS <=20

					1 &
				% TODO try size/length gt 0; take over for other passages
					\multicolumn{1}{X}{ Schleswig-Holstein   } &


					%270 &
					  \num{270} &
					%--
					  \num[round-mode=places,round-precision=2]{2,58} &
					    \num[round-mode=places,round-precision=2]{2,57} \\
							%????

					2 &
				% TODO try size/length gt 0; take over for other passages
					\multicolumn{1}{X}{ Hamburg   } &


					%317 &
					  \num{317} &
					%--
					  \num[round-mode=places,round-precision=2]{3,03} &
					    \num[round-mode=places,round-precision=2]{3,02} \\
							%????

					3 &
				% TODO try size/length gt 0; take over for other passages
					\multicolumn{1}{X}{ Niedersachsen   } &


					%933 &
					  \num{933} &
					%--
					  \num[round-mode=places,round-precision=2]{8,91} &
					    \num[round-mode=places,round-precision=2]{8,89} \\
							%????

					4 &
				% TODO try size/length gt 0; take over for other passages
					\multicolumn{1}{X}{ Bremen   } &


					%109 &
					  \num{109} &
					%--
					  \num[round-mode=places,round-precision=2]{1,04} &
					    \num[round-mode=places,round-precision=2]{1,04} \\
							%????

					5 &
				% TODO try size/length gt 0; take over for other passages
					\multicolumn{1}{X}{ Nordrhein-Westfalen   } &


					%1675 &
					  \num{1675} &
					%--
					  \num[round-mode=places,round-precision=2]{15,99} &
					    \num[round-mode=places,round-precision=2]{15,96} \\
							%????

					6 &
				% TODO try size/length gt 0; take over for other passages
					\multicolumn{1}{X}{ Hessen   } &


					%577 &
					  \num{577} &
					%--
					  \num[round-mode=places,round-precision=2]{5,51} &
					    \num[round-mode=places,round-precision=2]{5,5} \\
							%????

					7 &
				% TODO try size/length gt 0; take over for other passages
					\multicolumn{1}{X}{ Rheinland-Pfalz   } &


					%475 &
					  \num{475} &
					%--
					  \num[round-mode=places,round-precision=2]{4,53} &
					    \num[round-mode=places,round-precision=2]{4,53} \\
							%????

					8 &
				% TODO try size/length gt 0; take over for other passages
					\multicolumn{1}{X}{ Baden-Württemberg   } &


					%1540 &
					  \num{1540} &
					%--
					  \num[round-mode=places,round-precision=2]{14,7} &
					    \num[round-mode=places,round-precision=2]{14,68} \\
							%????

					9 &
				% TODO try size/length gt 0; take over for other passages
					\multicolumn{1}{X}{ Bayern   } &


					%1664 &
					  \num{1664} &
					%--
					  \num[round-mode=places,round-precision=2]{15,88} &
					    \num[round-mode=places,round-precision=2]{15,86} \\
							%????

					10 &
				% TODO try size/length gt 0; take over for other passages
					\multicolumn{1}{X}{ Saarland   } &


					%73 &
					  \num{73} &
					%--
					  \num[round-mode=places,round-precision=2]{0,7} &
					    \num[round-mode=places,round-precision=2]{0,7} \\
							%????

					11 &
				% TODO try size/length gt 0; take over for other passages
					\multicolumn{1}{X}{ Berlin   } &


					%634 &
					  \num{634} &
					%--
					  \num[round-mode=places,round-precision=2]{6,05} &
					    \num[round-mode=places,round-precision=2]{6,04} \\
							%????

					12 &
				% TODO try size/length gt 0; take over for other passages
					\multicolumn{1}{X}{ Brandenburg   } &


					%233 &
					  \num{233} &
					%--
					  \num[round-mode=places,round-precision=2]{2,22} &
					    \num[round-mode=places,round-precision=2]{2,22} \\
							%????

					13 &
				% TODO try size/length gt 0; take over for other passages
					\multicolumn{1}{X}{ Mecklenburg-Vorpommern   } &


					%240 &
					  \num{240} &
					%--
					  \num[round-mode=places,round-precision=2]{2,29} &
					    \num[round-mode=places,round-precision=2]{2,29} \\
							%????

					14 &
				% TODO try size/length gt 0; take over for other passages
					\multicolumn{1}{X}{ Sachsen   } &


					%838 &
					  \num{838} &
					%--
					  \num[round-mode=places,round-precision=2]{8} &
					    \num[round-mode=places,round-precision=2]{7,99} \\
							%????

					15 &
				% TODO try size/length gt 0; take over for other passages
					\multicolumn{1}{X}{ Sachsen-Anhalt   } &


					%204 &
					  \num{204} &
					%--
					  \num[round-mode=places,round-precision=2]{1,95} &
					    \num[round-mode=places,round-precision=2]{1,94} \\
							%????

					16 &
				% TODO try size/length gt 0; take over for other passages
					\multicolumn{1}{X}{ Thüringen   } &


					%619 &
					  \num{619} &
					%--
					  \num[round-mode=places,round-precision=2]{5,91} &
					    \num[round-mode=places,round-precision=2]{5,9} \\
							%????

					21 &
				% TODO try size/length gt 0; take over for other passages
					\multicolumn{1}{X}{ Deutschland ohne nähere Angabe   } &


					%6 &
					  \num{6} &
					%--
					  \num[round-mode=places,round-precision=2]{0,06} &
					    \num[round-mode=places,round-precision=2]{0,06} \\
							%????

					22 &
				% TODO try size/length gt 0; take over for other passages
					\multicolumn{1}{X}{ Ausland   } &


					%70 &
					  \num{70} &
					%--
					  \num[round-mode=places,round-precision=2]{0,67} &
					    \num[round-mode=places,round-precision=2]{0,67} \\
							%????
						%DIFFERENT OBSERVATIONS >20
					\midrule
					\multicolumn{2}{l}{Summe (gültig)} &
					  \textbf{\num{10477}} &
					\textbf{100} &
					  \textbf{\num[round-mode=places,round-precision=2]{99,84}} \\
					%--
					\multicolumn{5}{l}{\textbf{Fehlende Werte}}\\
							-998 &
							keine Angabe &
							  \num{15} &
							 - &
							  \num[round-mode=places,round-precision=2]{0,14} \\
							-966 &
							nicht bestimmbar &
							  \num{2} &
							 - &
							  \num[round-mode=places,round-precision=2]{0,02} \\
					\midrule
					\multicolumn{2}{l}{\textbf{Summe (gesamt)}} &
				      \textbf{\num{10494}} &
				    \textbf{-} &
				    \textbf{100} \\
					\bottomrule
					\end{longtable}
					\end{filecontents}
					\LTXtable{\textwidth}{\jobname-astu011k_g3r}
				\label{tableValues:astu011k_g3r}
				\vspace*{-\baselineskip}
                    \begin{noten}
                	    \note{} Deskritive Maßzahlen:
                	    Anzahl unterschiedlicher Beobachtungen: 18%
                	    ; 
                	      Modus ($h$): 5
                     \end{noten}


