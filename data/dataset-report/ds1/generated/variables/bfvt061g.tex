%EVERY VARIABLE HAS IT'S OWN PAGE

    \setcounter{footnote}{0}

    %omit vertical space
    \vspace*{-1.8cm}
	\section{bfvt061g (mehrmonatige berufl. Weiterbildung: Inhalt 5)}
	\label{section:bfvt061g}



	%TABLE FOR VARIABLE DETAILS
    \vspace*{0.5cm}
    \noindent\textbf{Eigenschaften
	% '#' has to be escaped
	\footnote{Detailliertere Informationen zur Variable finden sich unter
		\url{https://metadata.fdz.dzhw.eu/\#!/de/variables/var-gra2009-ds1-bfvt061g$}}}\\
	\begin{tabularx}{\hsize}{@{}lX}
	Datentyp: & numerisch \\
	Skalenniveau: & nominal \\
	Zugangswege: &
	  download-cuf, 
	  download-suf, 
	  remote-desktop-suf, 
	  onsite-suf
 \\
    \end{tabularx}



    %TABLE FOR QUESTION DETAILS
    %This has to be tested and has to be improved
    %rausfinden, ob einer Variable mehrere Fragen zugeordnet werden
    %dann evtl. nur die erste verwenden oder etwas anderes tun (Hinweis mehrere Fragen, auflisten mit Link)
				%TABLE FOR QUESTION DETAILS
				\vspace*{0.5cm}
                \noindent\textbf{Frage
	                \footnote{Detailliertere Informationen zur Frage finden sich unter
		              \url{https://metadata.fdz.dzhw.eu/\#!/de/questions/que-gra2009-ins2-6.5$}}}\\
				\begin{tabularx}{\hsize}{@{}lX}
					Fragenummer: &
					  Fragebogen des DZHW-Absolventenpanels 2009 - zweite Welle, Hauptbefragung (PAPI):
					  6.5
 \\
					%--
					Fragetext: & Im Folgenden bitten wir Sie um Angaben zu beruflichen Fort- und Weiterbildungen der letzten 12 Monate. Bitte denken Sie dabei an alle Weiterbildungen, die Sie besucht haben und geben sie diese in der passenden Zeile an.\par  1. Fort- /oder Weiterbildung\par  Themen (Mehrfachnennung möglich)\par  Schlüssel s. Klappliste B) \\
				\end{tabularx}
				%TABLE FOR QUESTION DETAILS
				\vspace*{0.5cm}
                \noindent\textbf{Frage
	                \footnote{Detailliertere Informationen zur Frage finden sich unter
		              \url{https://metadata.fdz.dzhw.eu/\#!/de/questions/que-gra2009-ins3-59$}}}\\
				\begin{tabularx}{\hsize}{@{}lX}
					Fragenummer: &
					  Fragebogen des DZHW-Absolventenpanels 2009 - zweite Welle, Hauptbefragung (CAWI):
					  59
 \\
					%--
					Fragetext: & Bitte tragen Sie hier die für Sie wichtigsten Themen bzw. Fachgebiete dieser Veranstaltungen ein. \\
				\end{tabularx}





				%TABLE FOR THE NOMINAL / ORDINAL VALUES
        		\vspace*{0.5cm}
                \noindent\textbf{Häufigkeiten}

                \vspace*{-\baselineskip}
					%NUMERIC ELEMENTS NEED A HUGH SECOND COLOUMN AND A SMALL FIRST ONE
					\begin{filecontents}{\jobname-bfvt061g}
					\begin{longtable}{lXrrr}
					\toprule
					\textbf{Wert} & \textbf{Label} & \textbf{Häufigkeit} & \textbf{Prozent(gültig)} & \textbf{Prozent} \\
					\endhead
					\midrule
					\multicolumn{5}{l}{\textbf{Gültige Werte}}\\
						%DIFFERENT OBSERVATIONS <=20

					3 &
				% TODO try size/length gt 0; take over for other passages
					\multicolumn{1}{X}{ mathematische Gebiete/Statistik   } &


					%1 &
					  \num{1} &
					%--
					  \num[round-mode=places,round-precision=2]{2,63} &
					    \num[round-mode=places,round-precision=2]{0,01} \\
							%????

					4 &
				% TODO try size/length gt 0; take over for other passages
					\multicolumn{1}{X}{ sozialwissenschaftliche Themen   } &


					%1 &
					  \num{1} &
					%--
					  \num[round-mode=places,round-precision=2]{2,63} &
					    \num[round-mode=places,round-precision=2]{0,01} \\
							%????

					6 &
				% TODO try size/length gt 0; take over for other passages
					\multicolumn{1}{X}{ pädagogische/psychologische Themen   } &


					%4 &
					  \num{4} &
					%--
					  \num[round-mode=places,round-precision=2]{10,53} &
					    \num[round-mode=places,round-precision=2]{0,04} \\
							%????

					7 &
				% TODO try size/length gt 0; take over for other passages
					\multicolumn{1}{X}{ medizinische Spezialgebiete   } &


					%1 &
					  \num{1} &
					%--
					  \num[round-mode=places,round-precision=2]{2,63} &
					    \num[round-mode=places,round-precision=2]{0,01} \\
							%????

					11 &
				% TODO try size/length gt 0; take over for other passages
					\multicolumn{1}{X}{ nationales Recht   } &


					%2 &
					  \num{2} &
					%--
					  \num[round-mode=places,round-precision=2]{5,26} &
					    \num[round-mode=places,round-precision=2]{0,02} \\
							%????

					13 &
				% TODO try size/length gt 0; take over for other passages
					\multicolumn{1}{X}{ Verwaltung, Organisation   } &


					%4 &
					  \num{4} &
					%--
					  \num[round-mode=places,round-precision=2]{10,53} &
					    \num[round-mode=places,round-precision=2]{0,04} \\
							%????

					14 &
				% TODO try size/length gt 0; take over for other passages
					\multicolumn{1}{X}{ Vetriebsschulungen   } &


					%1 &
					  \num{1} &
					%--
					  \num[round-mode=places,round-precision=2]{2,63} &
					    \num[round-mode=places,round-precision=2]{0,01} \\
							%????

					15 &
				% TODO try size/length gt 0; take over for other passages
					\multicolumn{1}{X}{ EDV-Anwendungen   } &


					%4 &
					  \num{4} &
					%--
					  \num[round-mode=places,round-precision=2]{10,53} &
					    \num[round-mode=places,round-precision=2]{0,04} \\
							%????

					17 &
				% TODO try size/length gt 0; take over for other passages
					\multicolumn{1}{X}{ Mitarbeiterführung/Personalentwicklung   } &


					%7 &
					  \num{7} &
					%--
					  \num[round-mode=places,round-precision=2]{18,42} &
					    \num[round-mode=places,round-precision=2]{0,07} \\
							%????

					18 &
				% TODO try size/length gt 0; take over for other passages
					\multicolumn{1}{X}{ Kommunikations-/Interaktionstraining   } &


					%5 &
					  \num{5} &
					%--
					  \num[round-mode=places,round-precision=2]{13,16} &
					    \num[round-mode=places,round-precision=2]{0,05} \\
							%????

					19 &
				% TODO try size/length gt 0; take over for other passages
					\multicolumn{1}{X}{ internationale Beziehungen, Kulturkenntnisse, Landeskunde   } &


					%1 &
					  \num{1} &
					%--
					  \num[round-mode=places,round-precision=2]{2,63} &
					    \num[round-mode=places,round-precision=2]{0,01} \\
							%????

					20 &
				% TODO try size/length gt 0; take over for other passages
					\multicolumn{1}{X}{ ökologische Themen   } &


					%1 &
					  \num{1} &
					%--
					  \num[round-mode=places,round-precision=2]{2,63} &
					    \num[round-mode=places,round-precision=2]{0,01} \\
							%????

					21 &
				% TODO try size/length gt 0; take over for other passages
					\multicolumn{1}{X}{ berufsethische Themen   } &


					%3 &
					  \num{3} &
					%--
					  \num[round-mode=places,round-precision=2]{7,89} &
					    \num[round-mode=places,round-precision=2]{0,03} \\
							%????

					22 &
				% TODO try size/length gt 0; take over for other passages
					\multicolumn{1}{X}{ Existenzgründung   } &


					%1 &
					  \num{1} &
					%--
					  \num[round-mode=places,round-precision=2]{2,63} &
					    \num[round-mode=places,round-precision=2]{0,01} \\
							%????

					24 &
				% TODO try size/length gt 0; take over for other passages
					\multicolumn{1}{X}{ Sonstige   } &


					%2 &
					  \num{2} &
					%--
					  \num[round-mode=places,round-precision=2]{5,26} &
					    \num[round-mode=places,round-precision=2]{0,02} \\
							%????
						%DIFFERENT OBSERVATIONS >20
					\midrule
					\multicolumn{2}{l}{Summe (gültig)} &
					  \textbf{\num{38}} &
					\textbf{100} &
					  \textbf{\num[round-mode=places,round-precision=2]{0,36}} \\
					%--
					\multicolumn{5}{l}{\textbf{Fehlende Werte}}\\
							-998 &
							keine Angabe &
							  \num{4717} &
							 - &
							  \num[round-mode=places,round-precision=2]{44,95} \\
							-995 &
							keine Teilnahme (Panel) &
							  \num{5739} &
							 - &
							  \num[round-mode=places,round-precision=2]{54,69} \\
					\midrule
					\multicolumn{2}{l}{\textbf{Summe (gesamt)}} &
				      \textbf{\num{10494}} &
				    \textbf{-} &
				    \textbf{100} \\
					\bottomrule
					\end{longtable}
					\end{filecontents}
					\LTXtable{\textwidth}{\jobname-bfvt061g}
				\label{tableValues:bfvt061g}
				\vspace*{-\baselineskip}
                    \begin{noten}
                	    \note{} Deskritive Maßzahlen:
                	    Anzahl unterschiedlicher Beobachtungen: 15%
                	    ; 
                	      Modus ($h$): 17
                     \end{noten}


