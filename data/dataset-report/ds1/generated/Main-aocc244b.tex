%% LaTeX2e file `Main-aocc244b'
%% generated by the `filecontents' environment
%% from source `Main' on 2018/11/07.
%%
     \begin{longtable}{lXrrr}
     \toprule
     \textbf{Wert} & \textbf{Label} & \textbf{Häufigkeit} & \textbf{Prozent(gültig)} & \textbf{Prozent} \\
     \endhead
     \midrule
     \multicolumn{5}{l}{\textbf{Gültige Werte}}\\
      %DIFFERENT OBSERVATIONS <=20

     2008 &
    % TODO try size/length gt 0; take over for other passages
     \multicolumn{1}{X}{ -  } &


     %2 &
       \num{2} &
     %--
       \num[round-mode=places,round-precision=2]{1,18} &
         \num[round-mode=places,round-precision=2]{0,02} \\
       %????

     2009 &
    % TODO try size/length gt 0; take over for other passages
     \multicolumn{1}{X}{ -  } &


     %47 &
       \num{47} &
     %--
       \num[round-mode=places,round-precision=2]{27,81} &
         \num[round-mode=places,round-precision=2]{0,45} \\
       %????

     2010 &
    % TODO try size/length gt 0; take over for other passages
     \multicolumn{1}{X}{ -  } &


     %120 &
       \num{120} &
     %--
       \num[round-mode=places,round-precision=2]{71,01} &
         \num[round-mode=places,round-precision=2]{1,14} \\
       %????
      %DIFFERENT OBSERVATIONS >20
     \midrule
     \multicolumn{2}{l}{Summe (gültig)} &
       \textbf{\num{169}} &
     \textbf{100} &
       \textbf{\num[round-mode=places,round-precision=2]{1,61}} \\
     %--
     \multicolumn{5}{l}{\textbf{Fehlende Werte}}\\
       -998 &
       keine Angabe &
         \num{8237} &
        - &
         \num[round-mode=places,round-precision=2]{78,49} \\
       -989 &
       filterbedingt fehlend &
         \num{2088} &
        - &
         \num[round-mode=places,round-precision=2]{19,9} \\
     \midrule
     \multicolumn{2}{l}{\textbf{Summe (gesamt)}} &
          \textbf{\num{10494}} &
        \textbf{-} &
        \textbf{100} \\
     \bottomrule
     \end{longtable}
     
