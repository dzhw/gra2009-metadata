%EVERY VARIABLE HAS IT'S OWN PAGE

    \setcounter{footnote}{0}

    %omit vertical space
    \vspace*{-1.8cm}
	\section{end\_m (Endzeitpunkt Episode (Monat))}
	\label{section:end_m}



	% TABLE FOR VARIABLE DETAILS
  % '#' has to be escaped
    \vspace*{0.5cm}
    \noindent\textbf{Eigenschaften\footnote{Detailliertere Informationen zur Variable finden sich unter
		\url{https://metadata.fdz.dzhw.eu/\#!/de/variables/var-gra2009-ds2-end_m$}}}\\
	\begin{tabularx}{\hsize}{@{}lX}
	Datentyp: & numerisch \\
	Skalenniveau: & ordinal \\
	Zugangswege: &
	  download-cuf, 
	  download-suf, 
	  remote-desktop-suf, 
	  onsite-suf
 \\
    \end{tabularx}



    %TABLE FOR QUESTION DETAILS
    %This has to be tested and has to be improved
    %rausfinden, ob einer Variable mehrere Fragen zugeordnet werden
    %dann evtl. nur die erste verwenden oder etwas anderes tun (Hinweis mehrere Fragen, auflisten mit Link)
				%TABLE FOR QUESTION DETAILS
				\vspace*{0.5cm}
                \noindent\textbf{Frage\footnote{Detailliertere Informationen zur Frage finden sich unter
		              \url{https://metadata.fdz.dzhw.eu/\#!/de/questions/que-gra2009-ins1-4.1$}}}\\
				\begin{tabularx}{\hsize}{@{}lX}
					Fragenummer: &
					  Fragebogen des DZHW-Absolventenpanels 2009 - erste Welle:
					  4.1
 \\
					%--
					Fragetext: & Um die Wege beim Übergang aus dem Studium in das Berufsleben und in andere Lebensbereiche besser verstehen zu können, bitten wir Sie, Ihre seit dem Studienabschluss ausgeübten Tätigkeiten in den folgenden Kalender einzutragen Bitte kennzeichnen Sie den Monat, in dem Sie die letzte Prüfungsleistung Ihres abgeschlossenen Studiums erbracht haben, mit einem X und tragen Sie für die Zeit vom Studienabschluss bis heute Ihre Tätigkeiten anhand der aufgeführten Kennbuchstaben in den Kalender ein. Haben Sie mehrere Tätigkeiten gleichzeitig ausgeübt, können Sie diese untereinander aufführen. Wichtig ist, dass es keine zeitlichen Lücken gibt. \\
				\end{tabularx}
				%TABLE FOR QUESTION DETAILS
				\vspace*{0.5cm}
                \noindent\textbf{Frage\footnote{Detailliertere Informationen zur Frage finden sich unter
		              \url{https://metadata.fdz.dzhw.eu/\#!/de/questions/que-gra2009-ins2-1.5$}}}\\
				\begin{tabularx}{\hsize}{@{}lX}
					Fragenummer: &
					  Fragebogen des DZHW-Absolventenpanels 2009 - zweite Welle, Hauptbefragung (PAPI):
					  1.5
 \\
					%--
					Fragetext: & Um die Wege beim Übergang aus dem Studium in das Berufsleben und in andere Lebensbereiche besser verstehen zu können, bitten wir Sie, Ihre seit Anfang 2010 ausgeübten Tätigkeiten in den folgenden Kalender einzutragen. Bitte tragen Sie für die Zeit von Januar 2010 bis heute Ihre Tätigkeiten anhand der aufgeführten Kennbuchstaben in Ihren persönlichen Kalender ein. Haben Sie mehrere Tätigkeiten gleichzeitig ausgeübt, können Sie diese untereinander aufführen. Wichtig ist, dass es keine zeitlichen Lücken gibt. \\
				\end{tabularx}
				%TABLE FOR QUESTION DETAILS
				\vspace*{0.5cm}
                \noindent\textbf{Frage\footnote{Detailliertere Informationen zur Frage finden sich unter
		              \url{https://metadata.fdz.dzhw.eu/\#!/de/questions/que-gra2009-ins3-05$}}}\\
				\begin{tabularx}{\hsize}{@{}lX}
					Fragenummer: &
					  Fragebogen des DZHW-Absolventenpanels 2009 - zweite Welle, Hauptbefragung (CAWI):
					  05
 \\
					%--
					Fragetext: & Um die Wege beim Übergang aus dem Studium in das Berufsleben und in andere Lebensbereiche besser verstehen zu können, bitten wir Sie, Ihre seit Anfang 2010 ausgeübten Tätigkeiten in den folgenden Kalender einzutragen. \\
				\end{tabularx}





				%TABLE FOR THE NOMINAL / ORDINAL VALUES
        		\vspace*{0.5cm}
                \noindent\textbf{Häufigkeiten}

                \vspace*{-\baselineskip}
					%NUMERIC ELEMENTS NEED A HUGH SECOND COLOUMN AND A SMALL FIRST ONE
					\begin{filecontents}{\jobname-end_m}
					\begin{longtable}{lXrrr}
					\toprule
					\textbf{Wert} & \textbf{Label} & \textbf{Häufigkeit} & \textbf{Prozent(gültig)} & \textbf{Prozent} \\
					\endhead
					\midrule
					\multicolumn{5}{l}{\textbf{Gültige Werte}}\\
						%DIFFERENT OBSERVATIONS <=20

					1 &
				% TODO try size/length gt 0; take over for other passages
					\multicolumn{1}{X}{ Januar   } &


					%1889 &
					  \num{1889} &
					%--
					  \num[round-mode=places,round-precision=2]{5.13} &
					    \num[round-mode=places,round-precision=2]{5.13} \\
							%????

					2 &
				% TODO try size/length gt 0; take over for other passages
					\multicolumn{1}{X}{ Februar   } &


					%5015 &
					  \num{5015} &
					%--
					  \num[round-mode=places,round-precision=2]{13.61} &
					    \num[round-mode=places,round-precision=2]{13.61} \\
							%????

					3 &
				% TODO try size/length gt 0; take over for other passages
					\multicolumn{1}{X}{ März   } &


					%5119 &
					  \num{5119} &
					%--
					  \num[round-mode=places,round-precision=2]{13.89} &
					    \num[round-mode=places,round-precision=2]{13.89} \\
							%????

					4 &
				% TODO try size/length gt 0; take over for other passages
					\multicolumn{1}{X}{ April   } &


					%2734 &
					  \num{2734} &
					%--
					  \num[round-mode=places,round-precision=2]{7.42} &
					    \num[round-mode=places,round-precision=2]{7.42} \\
							%????

					5 &
				% TODO try size/length gt 0; take over for other passages
					\multicolumn{1}{X}{ Mai   } &


					%3191 &
					  \num{3191} &
					%--
					  \num[round-mode=places,round-precision=2]{8.66} &
					    \num[round-mode=places,round-precision=2]{8.66} \\
							%????

					6 &
				% TODO try size/length gt 0; take over for other passages
					\multicolumn{1}{X}{ Juni   } &


					%3319 &
					  \num{3319} &
					%--
					  \num[round-mode=places,round-precision=2]{9.01} &
					    \num[round-mode=places,round-precision=2]{9.01} \\
							%????

					7 &
				% TODO try size/length gt 0; take over for other passages
					\multicolumn{1}{X}{ Juli   } &


					%3525 &
					  \num{3525} &
					%--
					  \num[round-mode=places,round-precision=2]{9.57} &
					    \num[round-mode=places,round-precision=2]{9.57} \\
							%????

					8 &
				% TODO try size/length gt 0; take over for other passages
					\multicolumn{1}{X}{ August   } &


					%3423 &
					  \num{3423} &
					%--
					  \num[round-mode=places,round-precision=2]{9.29} &
					    \num[round-mode=places,round-precision=2]{9.29} \\
							%????

					9 &
				% TODO try size/length gt 0; take over for other passages
					\multicolumn{1}{X}{ September   } &


					%3645 &
					  \num{3645} &
					%--
					  \num[round-mode=places,round-precision=2]{9.89} &
					    \num[round-mode=places,round-precision=2]{9.89} \\
							%????

					10 &
				% TODO try size/length gt 0; take over for other passages
					\multicolumn{1}{X}{ Oktober   } &


					%1649 &
					  \num{1649} &
					%--
					  \num[round-mode=places,round-precision=2]{4.48} &
					    \num[round-mode=places,round-precision=2]{4.48} \\
							%????

					11 &
				% TODO try size/length gt 0; take over for other passages
					\multicolumn{1}{X}{ November   } &


					%1043 &
					  \num{1043} &
					%--
					  \num[round-mode=places,round-precision=2]{2.83} &
					    \num[round-mode=places,round-precision=2]{2.83} \\
							%????

					12 &
				% TODO try size/length gt 0; take over for other passages
					\multicolumn{1}{X}{ Dezember   } &


					%2289 &
					  \num{2289} &
					%--
					  \num[round-mode=places,round-precision=2]{6.21} &
					    \num[round-mode=places,round-precision=2]{6.21} \\
							%????
						%DIFFERENT OBSERVATIONS >20
					\midrule
					\multicolumn{2}{l}{Summe (gültig)} &
					  \textbf{\num{36841}} &
					\textbf{\num{100}} &
					  \textbf{\num[round-mode=places,round-precision=2]{100}} \\
					%--
					\multicolumn{5}{l}{\textbf{Fehlende Werte}}\\
						& & 0 & 0 & 0 \\
					\midrule
					\multicolumn{2}{l}{\textbf{Summe (gesamt)}} &
				      \textbf{\num{36841}} &
				    \textbf{-} &
				    \textbf{\num{100}} \\
					\bottomrule
					\end{longtable}
					\end{filecontents}
					\LTXtable{\textwidth}{\jobname-end_m}
				\label{tableValues:end_m}
				\vspace*{-\baselineskip}
                    \begin{noten}
                	    \note{} Deskriptive Maßzahlen:
                	    Anzahl unterschiedlicher Beobachtungen: 12%
                	    ; 
                	      Minimum ($min$): 1; 
                	      Maximum ($max$): 12; 
                	      Median ($\tilde{x}$): 6; 
                	      Modus ($h$): 3
                     \end{noten}

